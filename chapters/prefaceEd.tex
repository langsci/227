\documentclass[output=paper]{langsci/langscibook} 
% \title{\textit{Definiteness across languages}: An overview } 

\title{\texorpdfstring{\textit{Definiteness across languages}: An overview}{Definiteness across languages: An overview}}

\rohead{Definiteness across languages: An overview} 
\renewcommand{\lsCollectionPaperFooterTitle}{Definiteness across languages: An overview}

\author{%
 Ana Aguilar-Guevara\affiliation{Universidad Nacional Autónoma de México}\and Julia {Pozas Loyo}\affiliation{El Colegio de México}\lastand Violeta {Vázquez-Rojas Maldonado}\affiliation{El Colegio de México}
}

% \chapterDOI{} %will be filled in at production
%\epigram{}

\begin{document}
\abstract{\noabstract}
\maketitle

\section{The meaning and expression of definiteness}

Definiteness has been a central topic in theoretical semantics since its modern foundation. Two main lines of thought have classically debated about the proper analysis of \is{definite noun phrases}definite noun phrases. One of them, initiated by \citet{Frege1892}, \citet{Russell1905}, and \citet{Strawson1950}, argues that definite descriptions crucially involve the condition -- be it asserted or presupposed -- that their descriptive content is satisfied by a \is{uniqueness}unique entity (in the relevant context of use). The other line of thought, originally proposed by \citet{Christophersen1939}, but elaborated by \citet{Heim1982} and \citet{Kamp1981}, claims that the core of definiteness depends on the existence of a referent in the common ground known by the speaker and the hearer. Most of the contemporary approaches to definiteness opt for either \isi{uniqueness} \citep[e.g.][]{Hawkins1978,Kadmon1990,Hawkins1991,Abbott1999} or \isi{familiarity} \citep[e.g.][]{Green1996,Chafe1996}, although there are other studies that point out that neither approach by itself provides a satisfactory explanation for all the empirical data concerning the use of definite descriptions in \ili{English} \citep[e.g.][]{BirnerWard1994}. These findings direct to a third standpoint that defends that the semantic basis of definiteness lies in a different characteristic, such as salience \citep{Lewis1979} or identifiability \citep{BirnerWard1994}. Another stance combines the two first “classical” approaches and claims that both uniqueness and familiarity are needed to explain the empirical behavior of the English definite article \citep{Farkas2002,Roberts2003}.

The theoretical discussion on definiteness has been revisited more recently by \citet{Schwarz2009,Schwarz2013} and \citet{CoppockBeaver2015}. In investigating the expression of definiteness in different languages, Schwarz proposes that, in order to account for the semantic value of definite descriptions crosslinguistically, both familiarity and uniqueness are needed. In some languages, moreover, they even correspond to different forms of definite markers that can be dubbed, respectively, \is{strong definite articles}“strong” and \is{weak definite articles}“weak” definite articles. When such semantic division of labor is explicit, the \isi{uniqueness} component is often encoded by a \is{bare noun phrases}bare noun phrase or by a \is{null determiners}silent determiner \citep{ArkohMatthewson2013}. \citet{CoppockBeaver2015} also analyze definiteness into two main components: \isi{uniqueness} and determinacy. \is{definiteness marking}Definiteness marking is seen as a morphological category that triggers a uniqueness presupposition, while determinacy consists in referring to an individual (i.e. having a type $e$ denotation). Definite descriptions are argued to be fundamentally predicative, presupposing uniqueness but not existence, and to acquire existential import through general type-shifting operations \citep{Partee1987}. \is{type shifting}Type-shifters enable argumental definite descriptions to become either determinate (and thus denote an individual) or indeterminate (and thus function as an \isi{existential quantifier}). 

The study of the meaning and expression of definiteness has not only advanced our understanding of regular \isi{definite noun phrases}, that is to say, constituents that refer to ordinary individuals, like the one exemplified in \REF{ex:aguilar:1a}. Other interpretations, like \isi{generic definites} \REF{ex:aguilar:1b}, \isi{weak definites} \REF{ex:aguilar:1c} and \isi{superlatives} \REF{ex:aguilar:1d}, allegedly involve reference to non-ordinary objects or individuals, and yet in languages like \ili{English} they are associated with the presence of a \is{definiteness marking}definiteness marker.

\ea
	\ea \label{ex:aguilar:1a}
	\textit{Hopefully, people will go out and start looking at \textbf{the moon} today.} %\jambox{Regular definite}
	
	\ex \label{ex:aguilar:1b}
	\textit{\textbf{The potato genome} contains twelve chromosomes.} %\jambox{Generic definite}
	
	\ex \label{ex:aguilar:1c}
	\textit{When do babies go to \textbf{the dentist} for their first visit?} %\jambox{Weak definite}
	
	\ex \label{ex:aguilar:1d}
	\textit{Donald owns \textbf{the highest building} in New York.} %\jambox{Superlative definite}
	
	\z  
\z \newpage

These “non-ordinary” definite descriptions have been discussed in the literature, for example: generic definites are analyzed in \citet{Chierchia1998}, \citet{Dayal2004}, \citet{Krifka2003}, \citet{Farkas2007} and \citet{Borik2012a}; weak definites have been the main topic in \citet{CarlsonSussman2005}, \citet{Aguilar-GuevaraZwarsts2011,Aguilar-GuevaraZwarts2013}, \citet{Schwarz2014} and \citet{Zwarts2014}; while superlatives have been treated by \citet{Szabolcsi1986}, \citet{Hackl2009}, \citet{SharvitStateva2002}, \citet{Krasikova2012} and \citet{CoppockBeaver2014}.

Definiteness has also awakened the interest of generative syntacticians. The common assumption for languages with \isi{articles} is that these correspond to the heads of determiner projections \is{determiner phrases}(DP). In contrast, the opinions about \isi{article-less  languages} are divided. Some authors, following the \is{Universal DP (Approach)}Universal DP approach, assume that a DP is present in all languages, regardless of whether or not they have an overt \is{definite articles}definite article (e.g. \citealt{Cinque1994}; \citealt{Longobardi1994}). This means that \isi{bare nouns} with a definite interpretation in article-less languages have a definite article, the D-head, which is \is{null determiners}unpronounced. Other authors, following the \is{DP/NP Approach}DP/NP approach, propose that not all nominal arguments correspond to DPs and that some languages might lack the category D altogether. On this view, the lack of an article indicates the absence of a DP (e.g. \citealt{Baker2003}; \citealt{Boskovic2008}); therefore, a basically predicative category like NP is capable of referring to individuals by means of \is{type shifting}type-shifting operations. There is a particular type-shifter, \is{iota operator}$\iota$, which would be responsible for the definite interpretation of noun phrases with no articles or overt markers for definiteness (\citealt{Chierchia1998}; \citealt{Dayal2004}).

Moreover, \isi{definiteness marking}, although usually encoded by determiners or particles in the adnominal domain, might be expressed in different syntactic projections, for instance, in bare \isi{classifier phrases}. \citet{ChengSybesma1999} claim that in languages like \ili{Cantonese} and \ili{Mandarin} Chinese the \is{classifiers}classifier head provides the definiteness meaning -- when no \is{numerals}numeral is present. \citet{SimpsonEtAlii2011} study bare classifier definites in other languages (\ili{Vietnamese}, \ili{Hmong}, \ili{Bangla}) and confirm the presence of this pattern, although the fact that also \isi{bare nouns} may receive definite interpretations calls into question that classifiers have incorporated the definiteness feature into their meaning in all such languages. The whole extent of this panorama of definiteness marking in categories other than D has not yet been acknowledged.

Despite its theoretical significance, there has been surprisingly scarce research on the cross-linguistic  expression of definiteness. One of the few examples of this kind of approach are the works of \citet{Dryer2005definiteart,Dryer2013,Dryer2014}, which register the different patterns that languages show regarding the occurrence of \isi{definite articles} and their formal similarity with \isi{demonstratives}. Another example is \citet{Givon1978}, who discusses how the contrast between definiteness and \isi{indefiniteness}, on the one hand, and \is{reference}referentiality vs. non referentiality (\isi{genericity}), on the other hand, are mapped crosslinguistically. Even with the valuable contribution of these studies, our knowledge on definiteness across languages still calls for a deeper typological understanding of the syntax of \isi{definite noun phrases} as well as of the whole range of their possible interpretations. 

With the purpose of contributing to filling this gap, the present volume gathers a collection of studies exploiting insights from formal semantics and syntax, typological and language specific studies, and, crucially, semantic fieldwork and cross-linguistic semantics, in order to address the expression and interpretation of definiteness in a diverse group of languages, most of them understudied. 

The papers presented in this volume aim to establish a dialogue between theory and data. In doing so, they adhere to a general guideline:  theories are used to make predictions about how definiteness is expressed in particular languages and what kind of semantic components it is expected to display. Theoretical predictions determine -- among other things -- in which contexts of use a purported definite expression will be acceptable and in which contexts it is likely to be rejected. These predictions are confronted with empirical data not only to test the adequacy of current theories, but also to bring along more questions about the possible diversity of meanings attested and their corresponding forms of expression. 

One of the goals of cross-linguistic comparison is to find patterns that are constant across languages and to identify those that are subject to variation. This is what, ultimately, brings together the interests of linguists willing to contribute to a comprehensive panorama of a particular phenomenon explored in a diverse pool of particular language systems. This practice has a long and reputable tradition in practically all fields of linguistics, but studies in the semantics of the nominal domain, especially from the formal perspective, only recently turned into this direction, starting with the seminal work of \citet{Bach1995} on quantification. More research from this standpoint has followed, like the works collected in \citet{Matthewson2008a}, \citet{Keenan2012}, to mention only some of the most emblematic. It is to this line of work that the present volume seeks to contribute. Given that we can safely assume that all languages are capable of making definite \isi{reference} and that, therefore, there must be a way in every language to refer to particular individuals which are assumed to be known to speaker and hearer, or which are assumed to be \is{uniqueness}unique in the relevant context of a speech-act, the task is to determine how they do it and which other semantic phenomena are associated with \isi{definiteness marking}. \newpage

With these antecedents in mind, we can now sum up the main questions that tie together the papers in this volume: What formal strategies do natural languages employ to \is{definiteness marking}encode definiteness? What are the possible meanings associated to this notion across languages? Are there different types of definite \isi{reference}? Which other functions (besides marking definite reference) are associated with definite descriptions? In this spirit, each of the papers contained in this volume addresses at least one of these questions and, in doing so, we believe they enrich our understanding of definiteness and with it, they contribute to our knowledge of the human capacity of language in general. 

\section{Overview of the volume}

This volume is composed of thirteen papers plus the editors’ introduction. As mentioned above, the unifying factor among them is, on the one hand, the aim to contribute to a better understanding of how definiteness is expressed and how definite descriptions are interpreted in natural languages and, on the other hand, the fact that authors combine theory and first-hand data in order to arrive to new insights about this classical subject. 

The contributions are organized around three main overarching topics or questions. The first group of papers (Schwarz, Cisneros, Šereikaitė, Irani, Pico, and Le Bruyn) addresses the topic of how definiteness is encoded in natural languages and which basic semantic features are involved in its expression. The second group of papers focuses on what is the syntactic locus of definiteness and what is the relation between definiteness marking and other projections (besides D) in the nominal domain. This question brings together the works of Hall, Despić and Borik \& Espinal.  Finally, the third group of papers (which include Williams, de Sá et al., Coppock \& Strand, and Etxeberria \& Giannakidou) deals with constructions in which definiteness markers seem to be associated to functions or meanings beyond canonical definite \isi{reference}. In the next paragraphs, we present a brief overview of each of the aforementioned contributions.

Florian Schwarz’s paper “Weak and strong definite articles: Meaning and form across languages” revisits the contrast between two types of definite descriptions on the light of new data drawn from a number of different languages (\ili{Hausa}, \ili{Lakhota}, \ili{Mauritian Creole}, \ili{Haitian Creole}, among others). According to his previous findings \citep{Schwarz2009}, some languages differentiate overtly between definite descriptions referring to entities that are \is{uniqueness}unique -- relative to some domain -- and definites that refer to entities that have been previously mentioned in discourse. \is{unique definites}Unique definites are called \textit{weak}, while \is{familiar definites}familiar (\is{anaphoric definites}anaphoric) definites are considered \textit{strong}. There is an interesting pattern found across languages that show this distinction: “weak” definites may be overtly marked or not marked at all, but in any case, their marker is morphophonologically less robust than the “strong” marker. The new data examined in this paper shows that, along with variations in form, strong and weak definites may also show some variations in meaning. For instance, in \ili{Icelandic}, a \is{strong definite articles}strong article might be used for first time anaphoric references, but then in subsequent discourse, the \is{weak definite articles}weak form can be used to pick up the same referent. Another semantic distinction relates to which article is chosen when a referent meets both conditions (\isi{uniqueness} and \isi{familiarity}) -- e.g. when referring to the family dog. \ili{German} might choose the strong article for this, while \ili{Akan} apparently the weak form (no article) for the same situation. A central question present throughout this paper is whether the patterns of semantic variation found across languages still fit within the strong/weak contrast, as though they are different points within a continuum that has uniqueness and familiarity as endpoints, or if they are orthogonal to it. 

The weak vs. strong definite distinction is also the topic of three other papers in this volume. Carlos Cisneros’s paper, “Definiteness in \il{Mixtec!Cuevas Mixtec}Cuevas Mixtec”, shows that this \ili{Otomanguean} language has two means for \is{definiteness marking}marking definiteness: \isi{bare nouns}, which are used to refer to entities that \is{uniqueness}uniquely satisfy a \is{nouns}noun’s description, and \isi{definite articles}  -- derived from \isi{noun classifiers} --, which are used for \isi{anaphoric definites}. However, not all nouns resort to the same markers to formalize this distinction. Thus, according to their strategies for encoding uniqueness or familiarity, the author recognizes three types of nouns: (a) those that express uniqueness with a \is{bare nominals}bare nominal and \isi{anaphoricity} with the \is{classifiers}classifier-like article; (b) those that use overt marking for both types of definiteness (“irregular nominals”); and (c) those which cannot combine with definite articles at all. Nouns in the (b) type are usually animate, so animacy seems to drive the patterns by which nouns select their definiteness markers. The paper contributes to the discussion put forth by Schwarz’s work by underlining the possibility of variation between different types of definiteness-marking strategies, not only across languages, but within a \is{language internal variation}single language, likely driven by lexical classes (particularly by animacy features). Also, it brings up the topic of what formal devices are involved in marking definiteness. While definiteness markers are commonly related to \isi{demonstratives} or other types of \isi{determiners}, little has been said about their relation with other syntactic categories, like nominal classifiers~-- in Mixtec~--, or \isi{adjectives}, as in Lithuanian, a phenomenon discussed in Šereikaitė’s work.\pagebreak 

Milena Šereikaitė’s paper “Strong vs. weak definites: Evidence from \ili{Lithuanian} adjectives” presents an analysis of the contrast between long and short adjectives in Lithuanian. As the author shows, in \ili{Lithuanian} -- a \is{article-less languages}language without articles -- definiteness can be encoded in a system of two forms of \isi{adjectives} that mirrors the strong/weak distinction for definite descriptions: the long adjective form, marked with the morpheme \textit{–ji(s)}, behaves like a \is{strong definite articles}strong article, while the bare form, in addition to being \is{indefiniteness}indefinite, is licensed by \isi{uniqueness} of reference, and thus semantically resembles \isi{weak definite articles}. More precisely, by examining the behavior of \isi{nouns} with long and short adjectives in different contexts, the author shows that long adjectives are felicitous in \is{anaphora}anaphoric uses with identical and not identical antecedents, while the bare form of adjectives is not only compatible with indefinite contexts -- such as existentials and the introduction of new referents into \isi{discourse} --, but, crucially, bare adjectives can also trigger a definite reading in contexts that require uniqueness, such as \isi{larger situation} uses and \is{part-whole relationship}part-whole \isi{bridging}. In sum, Šereikaitė’s chapter provides further support for the distinction between strong versus weak definites, and underlines the fact that this distinction is not necessarily \is{definiteness marking}encoded in \isi{determiners} or \isi{bare nouns}. 

The third language-specific study in this volume directly based on Schwarz’s strong/weak distinction for definite descriptions is Ava Irani’s “On (in)definite expressions in \ili{American Sign Language}”, which inquires on the nature of the pointing sign \textsc{ix} and concludes that, contrary to what previous studies had proposed \citep{KoulidobrovaLilloMartin2016}, it does not correspond to a \is{demonstratives}demonstrative. The claim is based on the fact that \textsc{ix} is not compatible with two contexts in which demonstratives are expected to appear: it does not allow contrastive readings, and it cannot point out to salient out-of-the-blue referents in a neutral location. Therefore, Irani argues that when a \is{noun phrases}NP referring to a previously established locus follows \textsc{ix}, it behaves as a strong definite article: it can be used in anaphora, and in \is{product-producer relationship}producer-product \isi{bridging}. By contrast, weak definite descriptions are expressed with \is{bare noun phrases}bare NPs, similarly to what has been observed in \is{classifiers}classifier languages (as in Cisneros’s work in this volume). In ASL, Irani argues, both bare NPs and \textsc{ix}+NPs can be definite or \is{indefiniteness}indefinite, depending on the specification of a locus feature, which, according to the author, suggests that in ASL definiteness is not semantically encoded. In conclusion, Irani’s work sums more evidence to the growing body of data showing that, at least for some languages, standard semantic approaches to definiteness such as familiarity and uniqueness, might not be sufficient to explain how a given NP gets it definite or its indefinite interpretation.\pagebreak

Another language-specific study included in this volume is Maurice Pico’s contribution “A nascent \is{definiteness marking}definiteness marker in \ili{Yokot'an} Maya”, which discusses the meaning of the particle \textit{ni}, a reduction of the distal \is{demonstratives}demonstrative \textit{jini} in this \ili{Mayan} language. In the previous literature, the particle \textit{ni} has been treated as a definite determiner, despite the fact that neither \isi{uniqueness} or \isi{familiarity} seem to be natural choices to account for the motivation behind its use. To better understand the presence of \textit{ni}, Pico carries out a detailed text analysis in terms of \isi{Centering Theory}, a framework specialized in modeling the way in which the changing salience of referring expressions helps to manage attention and attention shifts throughout the \isi{discourse} progression. From this analysis, Pico concludes that \textit{ni} is an attentional transition marker, that is, an indicator of change in the discourse status of the entity evoked by an NP, and it is thus particularly used to perform topicality shifts. This proposal accounts for the different uses of \textit{ni}, for its low frequency and relative optionality, and for its co-presence  with the \isi{topic} marker \textit{ba}. Furthermore, the proposal is compatible with the early stage of \isi{grammaticalization} at which the particle should stand according to the grammaticalization paths proposed in the literature for the development of \isi{definite articles} from \isi{demonstratives} \citep{Greenberg1978howgender,Hawkins2004}.

The next paper in the volume explores the meaning relations between members of different \is{articles}article systems. In “Definiteness across languages and in \is{transfer (from L1 to L2)}\is{second language acquisition}L2 acquisition”, Bert Le Bruyn claims that \is{article-less languages}languages with no articles are not all equal, and their subjacent differences come to light when their speakers acquire \ili{English} as a second language. According to a previous study by \citet{IoninKoWexler2004}, speakers of \ili{Korean}, \ili{Russian} and \ili{Japanese} as L1 overproduce definite articles in English when referring to specific entities, that is, to referents that are \is{familiarity}familiar and salient for speakers, but unknown to the hearer. Thus, overproduction of definite articles by speakers of these languages is seemingly triggered by this particular type of \isi{specificity}. These results are interpreted as though speakers of such languages ``fluctuate'' between two types of definite article systems: in one system (like \ili{English}), definite articles are used for definite reference, irrespective of specificity. In other systems, like \ili{Samoan}, definite articles are used for \isi{specific reference}, whether definite or \is{indefiniteness}indefinite, as well as foe non-specific definites. The explanation thus provided for the overproduction of definite articles under specificity conditions is called “the \isi{Fluctuation Hypothesis}”. Le Bruyn shows that L1 speakers of \ili{Mandarin}, however, do not comply with the predictions of the Fluctuation Hypothesis. Speakers did not produce definite articles for specific indefinites more than they did for the non-specific ones. Therefore, their choice did not seem to be driven by specificity, at least not the type of specificity tested by the previous study. The author designed a second test in which specificity was reflected on the referent being foregrounded and noteworthy (but, crucially, not unique or familiar), while non-specific referents were deemed such for their being backgrounded and not noteworthy. This contrast revealed that, when overproducing definite articles, \ili{Mandarin} L1 speakers were more likely to use them for non-specific (backgrounded) referents than for foregrounded (i.e. specific) referents. The findings point to the need for designing a research program that compares multiple L1 and their whole definiteness marking resources in order to respond to the question of how L1 influences L2 acquisition. 
 
The next three papers focus on determining the syntactic locus of \is{definiteness marking}definiteness markers and on assessing the relation between definiteness marking and other projections in the nominal domain. “Licensing D in classifier languages and ``numeral blocking''” by David Hall deals with definiteness in \isi{numeral classifier languages}. The paper proposes an alternative analysis to standard accounts of definiteness in this type of systems \citep{ChengSybesma1999,Simpson2005}. In \il{Wu!Wenzhou Wu}Wenzhou Wu and \ili{Weining Ahmao}, bare \isi{classifier phrases} can express definiteness, but the definite interpretation is \is{numeral blocking}blocked under the presence of a \is{numerals}numeral. The standard explanation for this fact is that the \is{classifiers}classifier may express definiteness if it moves up to a \is{determiners}Determiner head, but the presence of a numeral in the Specifier of an intervening \is{number}Number head blocks this movement \citep{Simpson2005}. By contrast, the proposal put forth by Hall argues that in this language there are two separate syntactic structures for Cl-N and \#-Cl-N. phrases in this language. Crucially, in the later case where the numeral is required, the numeral and the classifier form a constituent, to the exclusion of the \is{nouns}noun. In sum, Hall’s paper aims to contribute to a better understanding of the relation between the interaction of functional heads in the nominal domain and definiteness, specifically, in numeral classifier languages.

The second paper addressing the interaction of nominal functional projections in the expression of definiteness is Miloje Despić’s contribution, “On kinds and \isi{anaphoricity} in \is{article-less languages}languages without definite articles”. This paper studies the availability of anaphoric readings for bare nouns in languages that do not have definite articles, specifically, \il{Serbo-Croatian!Serbian}Serbian, \ili{Turkish}, \ili{Japanese}, \ili{Mandarin}, and \ili{Hindi}. Some of these languages have \isi{number} marking and others do not. Following the proposal that these languages do not project \is{NP/DP Approach}DPs (\citealt{Baker2003}; \citealt{Boskovic2008,BoskovicGajewski2011}; \citealt{Despic2011,Despic2013,Despic2015}), their anaphoric interpretations represent a theoretical problem, since it is standardly assumed that DP is the projection responsible for anaphoric readings, as it happens with the \ili{English} example \textit{I have an apple and a pear. I gave you \textbf{the apple}}. This suggests that there must be some other mechanism for anaphoricity. The main empirical contribution of the paper is a typology of interpretations for \isi{bare nouns} in the studied languages, which highlights the correlation between the presence of number marking and the availability of anaphoric readings in bare nouns that refer to kinds, while its explanatory import is to account for all these possibilities based on \citegen{Dayal2004} system of \is{type shifting}type-shifting operations. The proposal, in a nutshell, is that \is{kind reference}kind-referring \isi{noun phrases} can only obtain anaphoric readings in languages with \isi{number} marking and that this is due to the fact that these languages derive kind reference by means of a mechanism that introduces the $\iota$ type-shifter and enables definiteness.

Another contribution dealing with the syntax and semantics of kind-referring \isi{bare nouns} is Olga Borik \& María Teresa Espinal’s paper, “Definiteness in \ili{Russian} \is{bare nominals}bare nominal \isi{kinds}”. According to the authors, Russian \is{bare singulars}bare singular nouns in argument position with \isi{kind-level predicates} are interpreted as \isi{definite kinds}. The general hypothesis is that definite kinds, even in a \is{article-less languages}language without articles such as Russian, encode definiteness semantically and syntactically. In the case of Russian, definiteness is provided by a \is{null determiners}null D interpreted as \is{iota operator}$\iota$. In the spirit of emphasizing the dialogue between theory and data, the authors provide independent empirical semantic and syntactic data to support their claims. Thus, in order to demonstrate that Russian bare singular nouns are interpreted as definites, Borik \& Espinal show that they are acceptable in kind-level predicates of the ``extinct''-type. Given that these contexts require their subject to be definite, it follows that, semantically, Russian bare singulars are definites. As for the syntactic evidence for a null D, the authors compare the behavior of bare plurals with kind reference and small nominals (which are arguably not \is{determiner phrases}DPs) in some of the contexts analyzed in \citet{Pereltsvaig2006} -- i.e. control of PRO, the possibility of being antecedents of reflexive pronouns, pronominal substitution, and the distribution of \isi{relative clauses} -- to show that Russian bare singulars behave as one would expect from a DP. In conclusion, Borik \& Espinal’s paper deals with two of the subjects that has long interested linguist working on definiteness: \is{kind reference}reference to kinds and its links to definiteness and the locus of definiteness in article-less languages. 

The last four papers in the volume focus on non-canonical uses of \isi{definite noun phrases}. The next two contributions deal with so-called “\isi{weak definites}”, an interpretation of definite descriptions that does not comply with the requirement of referring to a unique or familiar entity. Adina Williams’s chapter, “A \is{morphosemantics}morpho-semantic account of weak definites and \isi{bare institutional singulars} in English”, analyzes \ili{English} weak definites (like in \textit{going to \textbf{the store}}) and bare institutional singulars (BIS; like in \textit{going to \textbf{school}}), which are analogous in meaning and distribution and in this respect differ from regular \isi{definites} (like in \textit{going to \textbf{the castle}}), which the author calls \is{strong definites}\textit{strong definites}.\footnote{Notice that this means that the weak/strong distinction Williams refers to is not the same one adopted by \citet{Schwarz2009}.}  The main concern of the study is the role that NumP plays in their interpretation, along with the denotation of their head \is{nouns}noun. The author provides a morpho-semantic account of the phenomenon, according to which the particular behavior of these constructions is a consequence of the lexical nature of their head noun. Williams recognizes three lexical classes of \isi{nominal roots}, each of them with different capacities regarding the weak/strong distinction: (i) strong-only roots, which are of type $ \<n, \<e, t\>\>$, have a count interpretation and can combine with NumP and with a regular, strong, \is{definite determiners}definite determiner; (ii) strong-weak ambiguous roots, which can be of type $ \<n, \<e, t\>\>$, are countable and combine with NumP and with a regular determiner, or, alternatively, are of type $ \<e, t\>$, not number specific, and may combine with a weak determiner; (iii) BIS roots, which can be of type $ \<n, \<e, t\> \>$ and behave as class (i), or of type $ \<k, t\>$, in which case they are incompatible with a \is{determiners}determiner but can semantically incorporate. The syntactic consequence of the lexical differences between regular and weak definites and bare institutional singulars is that, whereas the first type projects both NumP and DP, the second type projects only DP, and the third type does not project either of them. As a semantic consequence, there are three different types of compositional derivations of definite noun phrases: one for regular definites, one for \isi{weak definites} and one for \isi{bare institutional singulars}.\largerpage[2]

The second paper devoted to weak definites is “Is the weak definite a generic? An experimental investigation”, a paper coauthored by Thaís de Sá, Greg N. Carlson, Maria Luiza Cunha Lima and Michael K. Tanenhaus. The authors present data from a \isi{corpus study} and \is{experimental study}four experiments aiming to examine the different interpretative properties of weak definites in comparison with regular and \isi{generic definites}. This comparison turns relevant given that some of the existing semantic accounts of weak definites, in particular, \citet{Aguilar-GuevaraZwarsts2011,Aguilar-GuevaraZwarts2013}, assume that they are completely different from regular definites and closer to generic definites. The results of the studies offered in this paper show that weak definites do not behave as regular strong definites nor as generic definites (like in \textit{\textbf{The hospital} is not my favorite place}). The corpus study revealed that weak definites and generics are not in complementary distribution in any of the syntactic environments in which they appear. Moreover, the majority of weak definites occurred in clauses with activity and telic predicates, while generic definites occurred more in clauses with stative and activity predicates. Experiment 1 showed that, whereas regular definites were judged as denoting an individual, generic definites were judged to be about a category, and in this respect, weak definites behaved more similarly to the former than to the latter. Experiment 2 attested regular definites licensing more continuations containing corefering anaphoric \isi{noun phrases} than \isi{generic definites}, which encourage more interpretations introducing new events; in this respect, \isi{weak definites} again showed more similarity with regular definites than with \isi{generic definites}. Experiment 3 revealed analogous results in a free completion task. Finally, Experiment 4 required participants to repeat the target noun phrases in their completions; the completions triggered by each condition suggest that generics behave differently from both regular and weak definites.

Just as weak definites deviate from the canonical semantic reference of definite descriptions, \isi{definite determiners} also occur in constructions where a simple account based on \isi{familiarity} or \isi{uniqueness} is not sufficient. One of these non-canonical type of definiteness is the one observed in \is{superlatives}superlative constructions composed of a definite marker plus a \is{comparatives}comparative one, like in \textit{Este libro es \textbf{el más interesante}} (literally, ‘This book is the more interesting’) in \ili{Spanish}. In their chapter, “\textit{Most} vs. \textit{the most} in languages where \textit{the more} means \textit{most}”, Elizabeth Coppock and Linnea Strand study the expression of superlativity in \ili{French}, \ili{Spanish}, \ili{Italian}, \ili{Romanian}, and \ili{Greek}, in the illustrated construction is allowed. The authors provide a classification of superlative constructions based on a number of distributional and interpretative criteria, such as prenominal vs. postnominal position, adjectival vs. adverbial domain, qualitative vs. quantitative reading, absolute vs. relative reading, and relative vs. proportional reading. Among the different subtypes of constructions, the presence/absence of \is{definiteness marking}definiteness markers varies from language to language. The chapter makes two explanatory contributions. First, it argues that the variety of patterns found in the studied languages regarding the presence/absence of a definite marker is due to the interaction of two competing pressures within the grammar. One of them is the pressure to mark \isi{uniqueness} overtly. The other is the pressure to avoid combining a definite determiner with a predicate of entities other than individuals, such as events or degrees. In conjunction with some assumptions regarding the semantics of various types of superlatives, these pressures result in a disinclination for certain patterns. The second explanatory takeaway of this chapter is a compositional analysis of the described superlative constructions, based on standard and in more recent mechanisms proposed in formal semantics (\isi{Functional Application}, \isi{Definite Null Instantiation}, and \isi{Measure Identification}).

The volume closes with another study of a non-canonical use of \isi{definite determiners}. Urtzi Etxeberria and Anastasia Giannakidou’s paper, “Definiteness, partitivity and \isi{domain restriction}: A fresh look at \isi{definite reduplication}” tries to find a link between two phenomena that up to now had been considered independent: definite reduplication in \ili{Greek} and overt domain restriction in \isi{quantifier phrases} in \ili{Basque}, Greek, \ili{Bulgarian} and \ili{Hungarian}. Based on judgments about the interpretation of doubly-marked definites (like the fact that they are infelicitous when only one entity in the context satisfies the predicate provided by the \is{adjectives}adjective) they argue that Greek definite reduplication has a \is{partitives}partitive-like interpretation, and thus, the second definite marker (the one that precedes the adjective) is in fact a domain restrictor. The paper thus explores the possibility that D performs two different types of functions cross-linguistically: a saturating and a non-saturating type. Saturating D yields $e$-type expressions after combining with a predicate $ \<e, t\> $. That is the common case of definiteness markers, like the ones that have been discussed through most of the papers in this volume, where the resulting DP refers to a unique, salient or familiar individual. The non-saturating D, in contrast, combines with a given expression only to yield another expression of the same semantic type. If it combines with a predicate, as in Greek \isi{polydefinites}, it yields a predicate-like expression (as in \ili{Greek} \isi{definite reduplication}), and if it combines with a \is{generalized quantifiers}generalized quantifier, it yields a \is{domain restriction}domain-restricted quantifier, as in quantifier expressions in the languages analyzed.

\section{Acknowledgements}

The papers that compose this volume were presented in a preliminary version at the \textit{Definiteness across languages} Workshop, held in Mexico City in June 23-25, 2016. Since then, a long process of editing, reviewing, revising and editing again has taken place. We appreciate the help of every person who was involved at any stage of the way, starting with all the participants who enlivened the discussion and enriched the DAL Workshop with their presence. We sincerely thank our sponsoring institutions: El Colegio de México (Colmex) -- particularly the Centro de Estudios Lingüísticos y Literarios -- and Universidad Nacional Autónoma de México (UNAM) -- through the Facultad de Filosofía y Letras, Instituto de Investigaciones Antropológicas and Unidad de Posgrado. A majority of the funding for this workshop came from the project PAPIIT IA401116 \textit{Definitud regular y defectiva en la lengua natural} (UNAM) and the Cátedra Jaime Torres Bodet (Colmex). We particularly thank our colleague Samuel Herrera Castro for his hard work co-organizing this workshop, as well as the students who collaborated as staff: José Luis Brito Olvera, Mayra Gabriela García Rodríguez, Héctor Hernández Pérez,\largerpage Rafael Herrera Jiménez, and María Antonieta Vergara.\newpage 

As for the editing process proper, we thank all the authors in this volume for generously reviewing their colleague’s papers, and thanks also to the following external reviewers: Gemma Barberá, Cristina Buenrostro, Lisa Bylina, Lucas Champollion, Henri\"ette de Swart, Tom Leu, Suzi Lima, Cristina Schmitt, Andrew Simpson, Rint Sybesma and Ryan Sullivant. Many thanks to the series editors Martin Haspelmath and Sebastian Nordhoff for carefully guiding us through from the submission to the completion of this volume. Finally, special thanks to our editorial assistant, Jordi Martínez Martínez, for his invaluable help in formatting, proofreading, and building the language and subject indexes.

{\sloppy
\printbibliography[heading=subbibliography,notkeyword=this]
}
\end{document}
