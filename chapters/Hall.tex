\documentclass[output=paper
,modfonts
,nonflat]{langsci/langscibook} 

\title{Licensing D in classifier languages and ``numeral blocking''} 
\author{David Hall\affiliation{Queen Mary University of London}}
\ChapterDOI{10.5281/zenodo.3252022}

% \epigram{}

\abstract{Since \citet{ChengSybesma1999}, there has been much discussion of how the interaction of functional heads in the extended nominal projection in \isi{numeral classifier languages} gives rise to a definite interpretation. An important observation that came out of this discussion is that there appears to be some kind of interaction between a classifier head (call it Cl) and definiteness, where either Cl and D interact through \isi{head movement} \citep{Simpson2005}, or the Cl head itself introduces an \is{iota operator}$\iota$-operator. \citeauthor{ChengSybesma1999} note that in \ili{Cantonese}, which exhibits bare Cl-N sequences with a definite interpretation, the addition of a numeral has the effect of ``undoing the definiteness". The standard approach to accounting for this blocking of definiteness is that of \citet{Simpson2005}, where it is suggested that for a definite interpretation to arise in classifier languages, the Cl head has to move to D \citep[in the spirit of][]{Longobardi1994}. The blocking of a definite interpretation in \ili{Cantonese} is the result of a \isi{Head Movement Constraint} violation; Cl cannot move to D over the numeral. I show that this \isi{numeral blocking} effect extends to other languages too, and I argue based on data from those languages that a Head Movement Constraint based account of definiteness in classifier languages cannot capture the facts, and that we require an alternative. I put forward a proposal which has the consequence that the classifier and numeral form a constituent to the exclusion of the \is{nouns}noun, and then discuss some suggestive evidence in favour of such a structural configuration.}

\begin{document}
\maketitle\settowidth\jamwidth{}

\section{Introduction}\is{classifiers|(}\is{numerals|(}

A much discussed question related to numeral classifier languages\footnote{Throughout I use the term \textit{classifier languages} to mean \textit{numeral classifier languages}.} is how they \is{definiteness marking}encode definiteness, and whether there are differences among classifier lan-\linebreak guages with respect to this property. \citet{ChengSybesma1999} was an early attempt to systematically provide a syntactico-semantic explanation for differen\-ces observed between \ili{Mandarin} Chinese (henceforth \il{Mandarin}MC) and \ili{Cantonese}, with respect to the \is{noun phrases}noun phrase configurations which give rise to a definite interpretation. \ili{Cantonese} exhibits noun phrases composed of a bare classifier\footnote{\textit{Bare} here is intended to indicate the absence of a numeral. Many \is{numeral classifier languages}classifier languages, such as \ili{Japanese}, disallow classifiers where no numeral is present.} followed by a \is{nouns}noun (Cl--N phrases), which can be interpreted as a \is{definite noun phrases}definite noun phrase, whereas \il{Mandarin}MC only allows an \is{indefinites|(}indefinite interpretation for Cl--N phrases. Furthermore, in both languages, the presence of a numeral always forces an indefinite interpretation, regardless of whether Cl--N can be definite in that language. 

In this paper I discuss the standard explanation for the definite interpretation associated with bare classifiers in \ili{Cantonese}, and the related explanation for the ``blocking'' effect that the numeral has on definiteness, which has previously been tied to the \isi{Head Movement Constraint} (HMC). I show that the \isi{numeral blocking} effect extends to other classifier languages, including two languages where there is an overt morphological \is{definiteness marking}instantiation of definiteness on the classifier. I then argue that the standard HMC explanation of numeral blocking does not work in light of morphological facts from one of these languages, under a certain set of well-motivated assumptions about the structure of the \is{determiner phrases}DP. I ultimately conclude that a revised analysis, involving two separate structures for Cl-N phrases and phrases with a numeral is required, and that a consequence of this analysis, that numerals form a constituent with the classifier to the exclusion of the noun, is supported by typological evidence related to word order in classifier languages.

In the next section I introduce the relevant data from \il{Mandarin}MC and \ili{Cantonese}, before introducing the analyses in \citet{ChengSybesma1999} and \citet{Simpson2005}.\footnote{Much of the paper is a revised version of parts of \S4 and \S5 of \citet{Hall2015}.}

\section{Definiteness in Mandarin Chinese and Cantonese} \il{Mandarin}

Both \ili{Mandarin} Chinese (\il{Mandarin}MC) and \ili{Cantonese} are what I will refer to as \textit{classifier languages}, that is, languages which employ a set of morphemes to categorize or classify the noun that they co-occur with. The classifiers discussed here are sometimes referred to as \is{numeral classifiers}Numeral Classifiers \citep{Aikhenvald2000}, particularly given that they obligatorily appear when a numeral is present. Both languages allow \isi{bare nouns}, noun phrases composed of a classifier-noun sequence (Cl--N phrases) and noun phrases composed of a numeral-classifier-noun sequence (\mbox{\#--Cl--N}\footnote{Throughout, I will use \# as an abbreviation for \textit{numeral}.} phrases) in argument position. However, there are a number of interesting constraints on where each type of noun phrase can appear. Furthermore, these constraints differ between the two languages, as discussed in depth in \citet{ChengSybesma1999}.


Overall, the possible interpretations available to different \isi{noun phrases} in \il{Mandarin}MC and \ili{Cantonese} depend on the shape of the noun phrase: in particular, whether it is a \is{bare nouns}bare N, a Cl--N, or a \#--Cl--N. \citet{Jenks2012} points out that the difference between \il{Mandarin}MC and \ili{Cantonese} noun phrase distribution and interpretation can be subsumed under a larger generalization that appears to hold quite robustly across a number of \ili{Sino-Tibetan} and \ili{Austroasiatic} classifier languages, including \ili{Hmong}, \ili{Cantonese}, \il{Mandarin}MC, \ili{Min}, and \ili{Vietnamese}.\footnote{Note that \citet{Trinh2011} claims that bare nouns cannot be definite in \ili{Vietnamese}, but \citet{Nguyen2004} and Jenks claim otherwise. See also \citet{SimpsonEtAlii2011} for a challenge to the complementarity of definite bare Ns and definite Cl--N phrases.} The generalization takes the form of two one-way entailments: if a classifier language has bare nouns which can be interpreted as definite, then Cl--N phrases will not be interpreted as \is{definites}definite; if a classifier language has Cl--N phrases which can be interpreted as definite, then bare nouns will not be interpreted as definite.\footnote{We will see an example of a language in \sectref{sec:hall:4.1}, \il{Wu!Wenzhou Wu}Wenzhou Wu, which is a counter-example to this generalization.} 

\ea \label{ex:hall:1}
Noun phrase interpretation in \is{numeral classifier languages}classifier languages
\ea Bare N [$\pm$def] $\rightarrow$ Cl--N [$-$def] \jambox{Type A language}
\ex Cl--N [$\pm$def] $\rightarrow$ Bare N [$-$def] \jambox{Type B language}
\z
\z

\il{Mandarin}MC is a Type A language: it exhibits definite bare nouns and Cl--N phrases which are obligatorily indefinite. \ili{Cantonese} is a Type B language: it has definite Cl--N phrases and obligatorily indefinite bare nouns.
Another generalization that can be added to the above is that, regardless of the availability of a definite interpretation for a Cl--N phrase, the presence of a numeral always blocks a definite interpretation.

\ea \label{ex:hall:2}
\#--Cl--N [$-$def] \jambox{Type A\&B languages} 
\z

My focus in this paper is on Type B languages; in particular on the definite interpretation associated with Cl--N phrases, and the reasons why \REF{ex:hall:2} holds in those languages. In the next subsection I lay out the full set of facts related to \il{Mandarin}MC and \ili{Cantonese}, before introducing two previous analyses of the differences between the two languages.

\subsection{Mandarin Chinese -- a Type A classifier language} \il{Mandarin}

\il{Mandarin}MC is a Type A classifier language (following the generalization in \ref{ex:hall:1}).\footnote{Note that throughout I discuss sortal classifiers, and not mensural classifiers, or ``massifiers'' to use \citegen{ChengSybesma1998} term. I believe that massifiers have a different structure, which is evidenced by their different properties (a modifier can appear between the massifier and the \is{nouns}noun, a modification marker \textit{de} is optionally present). See \citet{ChengSybesma1998} and \citet{ChengSybesma1999} for discussion.} In postverbal object position, bare nouns can have either definite or indefinite interpretation whereas in preverbal subject position (or \isi{topic} position), bare nouns cannot be interpreted as indefinite \REF{ex:hall:3a}, because of a general restriction on the preverbal subject position which means that \is{indefinites}indefinite noun phrases cannot appear there (\citealt[288]{HuangEtAlii2009} and references cited therein). \is{noun phrases}Noun phrases with a \is{demonstratives}demonstrative are also acceptable in preverbal subject position \REF{ex:hall:3b}, and can take on an \is{anaphoric definites}anaphoric definite interpretation (in the sense of \citealt{Schwarz2009}; see \citealt{Jenks2015}).\footnote{Judgements on example sentences are taken directly from the literature, unless otherwise stated.}

\ea \label{ex:hall:3}
\ea \label{ex:hall:3a}
\gll Gou chi-le dangao. \\
 dog eat-{\sc{prf}} cake \\
\glt `The dog ate the cake/a cake.' NOT `a dog ... ' \\
\ex \label{ex:hall:3b}
\gll Nei-zhi gou chi-le dangao. \\
that-{\sc{cl}} dog eat-{\sc{prf}} cake \\
\glt `That/the dog ate the cake/a cake.' \\
\z
\z

\is{bare nouns}Bare \isi{count nouns} are \is{number neutrality}number neutral, and thus can refer to either singular objects or pluralities. Bare nouns can also refer to mass objects (examples taken from \citealt{ChengSybesma1999}, with some modification):\footnote{I focus here on definite and indefinite interpretations, and put aside kind and \is{genericity}generic interpretations, which bare nouns can also take on. For discussion of \is{kinds}kind and generic interpretations in \il{Mandarin}MC, see \citet{Krifka1995}.}

\ea \label{ex:hall:4}
\ea
\gll Hufei mai shu qu le.\\
 Hufei buy book go {\sc{SFP}}\\
\glt `Hufei went to buy a book/books/the book(s).'\\

\ex
\gll Hufei he-wan-le tang.\\
 Hufei drink-finish-{\sc{prf}} soup\\
\glt `Hufei drank the soup/some soup.'\\
\z
\z


Where a \is{nouns}noun is accompanied by a numeral, a classifier is obligatorily present \REF{ex:hall:5},\footnote{Although see \citet{Tao2006} for a discussion of the phenomenon of classifier reduction (of the general classifier \textit{ge}) in spoken Beijing \ili{Mandarin} Chinese.} and the \#--Cl--N phrase is obligatorily \is{indefinites}indefinite. Cl--N phrases are also possible without a numeral, and are obligatorily \is{indefinites}indefinite and singular \REF{ex:hall:6}.\footnote{A possible exception is the classifier-like plural marking element \textit{xie}, which I put aside here. See \citet[\S4.2.3]{Hall2015} for discussion.} Because of the ``definiteness constraint'' on preverbal subject position, Cl--N and \#--Cl--N phrases are degraded in this position \REF{ex:hall:7}. 

\ea \label{ex:hall:5}
\gll 
Wo xiang mai liang \textnormal{*}{\op}ben{\cp} shu.\\
I want buy two \phantom{*(}{\sc{cl}} book\\
\glt 
`I want to buy two books.'\\
\z\vspace{-.5\baselineskip}

\ea \label{ex:hall:6}
\gll 
Wo xiang mai ben shu.\\
I want buy {\sc{cl}} book\\
\glt 
`I want to buy \textbf{a} book.' NOT `I want to buy (some) books.'\\
\z\vspace{-.5\baselineskip}\largerpage

\ea \label{ex:hall:7}
\ea[??]{
\gll San-ge xuesheng chi-le dangao.\\
three-{\sc{cl}} student eat-{\sc{prf}} cake\\
\glt 
Intended: `Three students ate the cake.'\\}
\ex[*]{
\gll 
Ge xuesheng chi-le dangao.\\
{\sc{cl}} student eat-{\sc{prf}} cake\\
\glt
Intended: `A student ate the cake.'\\}
\z
\z

\subsection{Cantonese -- a Type B classifier language}

\ili{Cantonese} is a Type B classifier language (following the generalization in \ref{ex:hall:1}). In postverbal object position, Cl--N phrases can have either definite or indefinite interpretation \REF{ex:hall:8} whereas in preverbal subject position (or \isi{topic} position), Cl--N phrases can only be definite \REF{ex:hall:9}. As with \il{Mandarin}MC, Cl--N phrases are always singular.\footnote{Again, this is with the exception of \isi{nouns} that appear with the ``plural classifier'' \textit{di\tss{1}}, which I discuss in \citet[\S4.2.3]{Hall2015}.} \is{bare nouns}Bare nouns, on the other hand, are obligatorily indefinite (thus being unacceptable in preverbal subject position, \ref{ex:hall:9a}), and are \is{number neutrality}number neutral. Examples here are again taken from \citet{ChengSybesma1999}.\footnote{Superscript numbers on \ili{Cantonese} examples indicate tone.}

\ea \label{ex:hall:8}
\gll
Ngo\tss{5} soeng\tss{2} maai\tss{5} bun\tss{2} syu\tss{1} {\op}lei\tss{4} tai\tss{2}{\cp}. \\
I want buy {\sc{cl}} book \phantom{(}come read\\
\glt 
`I want to buy a book (to read).'\\
\z\vspace{-.5\baselineskip}

\ea \label{ex:hall:9}
\ea[*]{ \label{ex:hall:9a}
\gll 
Gau\tss{2} soeng\tss{2} gwo\tss{3} maa\tss{5}lou\tss{6}.\\
dog want cross road \\
\glt
Intended: `The dog wants to cross the road.'}
\ex[]{ \label{ex:hall:9b}
\gll Zek\tss{3} gau\tss{2} soeng\tss{2} gwo\tss{3} maa\tss{5}lou\tss{6}.\\
{\sc{cl}} dog want cross road\\
\glt `The dog wants to cross the road.', NOT `a dog ... '\\}
\z
\z\vspace{-.5\baselineskip}

\ea
\gll Wufei heoi\tss{3} maai\tss{5} syu\tss{1}.\\
Wufei go buy book\\
\glt `Wufei went to buy a book/books.'\\
\z

As with \il{Mandarin}MC, \#--Cl--N phrases are always interpreted as \is{indefinites}indefinite, and thus are infelicitous in preverbal subject or \isi{topic} position (examples elicited from a native \ili{Cantonese} speaking informant). Here I include a Cl--N phrase (which gets a definite interpretation) for contrast.

\ea
\ea[]{
\gll
Zek\tss{3} gau\tss{2} sik\tss{6}-gan\tss{2} juk\tss{6}.\\
{\sc{cl}} dog eat-{\sc{prog}} meat\\
\glt
`The dog is eating meat.'\\}

\ex[*]{
\gll
Loeng\tss{5}-zek\tss{3} gau\tss{2} sik\tss{6}-gan\tss{2} juk\tss{6}.\\
two-{\sc{cl}} dog eat-{\sc{prog}} meat\\
\glt Intended: `The two dogs are eating meat.'\\}
\z
\z

\subsection{Summary}

In summary, we have the set of interpretations in \tabref{tab:hall:1}, associated with particular \is{noun phrases}noun phrase configurations, available in the two languages.

%\newpage
What is important here is that we have a language, i.e. \ili{Cantonese}, where a definite interpretation is possible in a \is{noun phrases}noun phrase composed of a bare classifier followed by a \is{nouns}noun, but where the introduction of a numeral always blocks a definite interpretation. An account of the interpretive differences in noun phrases between the two languages will focus on two facts:\largerpage[2]

\begin{enumerate}
\item Cl--N can be definite in \ili{Cantonese}, but not in \il{Mandarin}MC.
\item \#--Cl--N is always indefinite in both languages.
\end{enumerate}

\noindent In the next section I introduce two previous accounts of these facts.

\begin{table}[H]
\caption{Summary of \S2}
\begin{tabularx}{\textwidth}{lccl}
\lsptoprule
Noun phrase config. & Definite & Indefinite & Number\\
\midrule
\il{Mandarin}MC &&\\
\midrule
N & $\checkmark$ & $\checkmark$ & Neutral\\
Cl--N & * & $\checkmark$ & Sg\\
\#--Cl--N & * & $\checkmark$ & Sg/Pl (\# dependent)\\
\midrule
\ili{Cantonese} &&\\
\midrule
N &*&$\checkmark$ & Neutral\\
Cl--N &$\checkmark$&$\checkmark$ & Sg\\
\#--Cl--N &*&$\checkmark$ & Sg/Pl (\# dependent)\\
\lspbottomrule
\end{tabularx} \label{tab:hall:1}
\end{table}

\section{Previous accounts}

\subsection{Cheng \& Sybesma (1999)}\largerpage[2]

\citet{ChengSybesma1999} offered the first account of the above distribution of interpretations across different noun phrase configurations. They argue that the Cl head in \il{Mandarin}MC and \ili{Cantonese} plays the (semantic) role that D does in \ili{English}, that of introducing a definite interpretation through an \is{iota operator}iota operator. Following \citet{Chierchia1998}, this is introduced either directly as a definite classifier, as in \ili{Cantonese}, or as a type-shifting last resort operator where no definite lexical item is available, as in \il{Mandarin}MC. Cheng \& Sybesma also propose that a necessary step for the last resort type-shifting in \il{Mandarin}MC is N-to-Cl movement, which is why \is{bare nouns}bare Ns can have a definite interpretation in that language. So, in \ili{Cantonese}, the classifier is an overt \is{definite articles}definite article, giving definite Cl--N phrases, and in \il{Mandarin}MC, N moves to the empty Cl projection, giving definite bare nouns.{\interfootnotelinepenalty=10000\footnote{\citeauthor{ChengSybesma1999} accept that this movement would result in an illicit ordering of the \is{adjectives}adjective and \is{nouns}noun, if the adjective merges lower than Cl, and the noun moves up to Cl:\is{classifier phrases|(}

\ea Predicted order: N$\succ$Adj\\
\begin{forest}sn edges
[ClP [Cl\sub{∅}\sub{[+def]},name=class] [NP [AP] [N\sub{[+def]},name=noun]] ]
\draw[->] (noun) to[out=north east,in=north east,looseness=1.2,overlay] (class);
\end{forest}
\z
They therefore claim that the movement has to be covert.}}

\noindent\begin{minipage}[t]{0.5\textwidth}
\ea \il{Mandarin}MC

\begin{forest}sn edges
[ClP [Cl\sub{∅}\sub{[+def]},name=class] [N,name=noun] ]
\draw[->] (noun) to[out=south,in=south,looseness=1] (class);
\end{forest}

\z
\end{minipage}%
\begin{minipage}[t]{0.5\textwidth}
\ea \ili{Cantonese}

\begin{forest}sn edges
[ClP [Cl\sub{[+def]}] [N] ]
\end{forest}

\z
\end{minipage}\\

Simply put then, the difference between \il{Mandarin}MC and \ili{Cantonese} lies in how the \is{definiteness marking}definiteness ``feature'' encoded in the Cl head is licensed.
The fact that numerals block definiteness in both languages is argued to arise from the fact that all indefinite Cl--N phrases involve the projection of a Numeral head above ClP, as in \REF{ex:hall:14}.

\ea \label{ex:hall:14}
\is{indefinites}Indefinite Cl--N phrase \\
\begin{forest}sn edges
[NumeralP [Numeral] [ClP [Cl] [NP [N]]]]
\end{forest}
\z

Numerals are claimed to fundamentally involve \is{existential quantifier}existential quantification, and therefore the merger of a Numeral head has the effect of ``undoing the definiteness" \citep[528]{ChengSybesma1999}. From the perspective of compositional semantics, however, this doesn't entirely make sense. In the system proposed in \citet{Chierchia1998} (based ultimately on Partee's \citeyear{Partee1987} set of type-shifters), the \is{iota operator}iota-operator takes a property and returns a \is{uniqueness}unique individual (of type \type{$e$}), whereas the existential operator takes a property and returns a \is{generalized quantifiers}generalized quantifier (of type \type{\type{$e$, $t$}, $e$}). If we compose the property introduced by N with the iota operator first at Cl, then an \isi{existential quantifier} introduced at Numeral would not be able to compose with the resultant individual (of type $e$).\pagebreak  

\ea
\begin{forest}sn edges
[NumeralP\sub{\type{??}} [Numeral\sub{$\exists$}] [ClP\sub{\type{$e$} ($\iota x$)} [Cl\sub{(\type{\type{$e$, $t$}, $e$})}] [NP\sub{\type{$e$, $t$}} [N]]]]
\end{forest}
\z

The individual is bound by the iota operator at the ClP level, meaning that it can no longer be \is{quantifiers}quantified over in the way suggested by \citeauthor{ChengSybesma1999}.\footnote{It is possible to introduce a covert type-shifter (``IDENT'' or ``Id'' in Partee's terms) to take ClP from \type{$e$} to \type{$e,t$} so that it could combine with the numeral. This would put us in the position of saying that the iota operator applies only to have the \is{type shifting}type shifted back by the covert partial inverse of iota, which is hardly satisfying. It would again in effect be the same as saying that ``numerals undo definiteness'', or that the merger of a numeral must be preceded by composition of ClP with a covert operator that undoes definiteness.} If, on the other hand, the notion of ``undoing'' of definiteness is intended to mean that an iota operator is never present in Cl when a numeral is merged, then this becomes a simple stipulation, and a restatement of the facts. Because of the inexplicit nature of the explanation, I put aside \citeauthor{ChengSybesma1999}'s approach to \is{numeral blocking}Numeral Blocking, and instead focus on a related proposal that builds on \citeauthor{ChengSybesma1999}'s initial insights. The standard account which avoids the problems discussed immediately above is developed in \citet{Simpson2005}, where the locus of definiteness is not Cl, but D, \is{Universal DP (Approach)}assuming that DPs are universal, even where a language does not exhibit overt articles.

\subsection{The DP account}\is{determiner phrases|(}

The DP account of the \il{Mandarin}MC and \ili{Cantonese} facts is proposed by \citet{Simpson2005}, (and defended by \citealt{WuBodomo2009}). Simpson builds on the ideas in \citet{ChengSybesma1999}, but crucially the account differs in that it takes D to be the locus of definiteness, following \citet{Longobardi1994}. The central idea is that it is \isi{head movement} of Cl to D in \ili{Cantonese} that gives rise to the definite interpretation of Cl--N phrases. Definite D must be overtly instantiated by some lexical element to be licensed, and so a lack of movement of the classifier to the D head results in an indefinite Cl--N configuration. 

\noindent\begin{minipage}[t]{.5\linewidth}
\ea \ili{Cantonese} Cl--N [+def]
\begin{forest}sn edges
[DP [D\sub{[+def]},name=DDef] [ClP [Cl\sub{[+def]},name=ClDef] [NP] ] ] 
\draw[->] (ClDef) to[out=west,in=south] (DDef);
\end{forest}
\z
\end{minipage}%
\begin{minipage}[t]{.5\linewidth}
\ea \ili{Cantonese} Cl--N [$-$def]
\begin{forest}sn edges
[DP [D] [ClP [Cl] [NP] ] ] 
\end{forest}
\z
\end{minipage}\vspace{\baselineskip}

In \il{Mandarin}MC, this movement is not available, presumably because the Cl does not come with a \is{definiteness marking}definiteness feature. This means that a bare Cl--N phrase never receives a definite interpretation.\footnote{There is no discussion of how \isi{bare nouns} get a definite interpretation under this analysis: however it has been suggested that it involves N-to-D movement of the type discussed in \citet{Longobardi1994}, although with \isi{common nouns}, not just \is{proper names}proper nouns. Such an analysis has problems of its own, but I will not discuss them here for reasons of space. See footnote \ref{n-to-d} for further discussion.}

 An advantage of this \isi{head movement} approach is that it can straightforwardly account for the fact that numerals block definiteness in \ili{Cantonese}, without any awkward stipulations. Although the exact syntactic position of the numeral is not explicitly discussed in \citet{Simpson2005}, the discussion suggests that the numeral is introduced as a head above ClP. This means that the Numeral head will act as an intervenor for Cl-to-D movement, as per the \isi{Head Movement Constraint} of \citet{Travis1984}, and will therefore block a definite interpretation.

\ea
\textbf{The \isi{Head Movement Constraint} (HMC)}\\
An X\tss{0} may only move into the Y\tss{0} which properly governs it.
\z 

\ea \begin{forest}sn edges
[DP, s sep=10mm [D\sub{[+def]},name=Hi]  [NumeralP [\textbf{Numeral}\\(\textit{intervenor})] [ClP [{Cl}\sub{[+def]},name=Lo] [NP [N] ] ] ] ] 
\draw[->](Lo) to[out=south west,in=south west,looseness=1] node[midway]{\Huge$\ast$} (Hi);
\end{forest}
\z \is{classifier phrases|)}

This is a simple and elegant explanation of the \isi{numeral blocking} effect. No stipulation of the ``undoing of definiteness'' is required, and we have a straightforward explanation in terms of locality and the interaction of syntactic features and interpretation. However, I intend to argue that it is not the simplest account, based on certain well-motivated assumptions about the structure of the \is{determiner phrases|)}DP, and facts from other \is{numeral classifier languages}classifier languages. 

In the next section I will show that numerals blocking definiteness is not a peculiarity of \ili{Cantonese}, and in fact extends to other classifier languages. Furthermore, morphological facts from one language in particular, \ili{Weining Ahmao}, suggest that the simple \is{Head Movement Constraint}HMC explanation of the Numeral Blocking effect proposed by \citeauthor{Simpson2005} could not be correct, and in order to explain the full set of typological facts, two different structures will be proposed for \#--Cl--N and bare Cl--N phrases.

\section{Numerals block definiteness: Cross-linguistic considerations}\is{numeral blocking|(}

The blocking effect of numerals is a general effect that can be seen in other classifier languages. \ili{Cantonese} classifiers are able to \is{definiteness marking}signal definiteness without any difference in the morphological shape of the classifier. That is to say, a Cl--N sequence is interpreted as either definite or indefinite depending on context, rather than the shape of the classifier which accompanies the \is{nouns}noun. This is also true of other classifier languages, including \ili{Vietnamese} and \ili{Nung}. However, there are classifier languages spoken in China which exhibit ``inflecting'' classifiers; that is, classifiers whose morphology encodes different interpretive features of the \is{noun phrases}noun phrase. The striking fact about those languages is that, even though definiteness can be overtly \is{definiteness marking}marked on the classifier, the presence of a numeral always blocks definiteness, and prevents the definite form of the classifier from being used. I give a description of the classifier morphology of two languages which exhibit inflecting classifiers in the following subsections, and show that these languages also appear to exhibit the same numeral blocking effect as \ili{Cantonese}.

\subsection{Wenzhou Wu}\il{Wu!Wenzhou Wu} \label{sec:hall:4.1}

The southern \ili{Wu} variety spoken in Wenzhou is a local dialect of one of the ten major varieties of \ili{Chinese}, \ili{Wu}. \citet{ChengSybesma2005} discuss the different interpretive possibilities for different \is{noun phrases}noun phrase configurations in four varieties of \ili{Chinese}, including \il{Wu!Wenzhou Wu}Wenzhou Wu (WW). They note that \il{Wu!Wenzhou Wu}WW \isi{bare nouns} have the same distribution as \il{Mandarin}MC bare nouns, in that they can be either definite or indefinite in object position, and can only be interpreted as definite in subject position.

Cl--N phrases, however, differ from both \il{Mandarin}MC and \ili{Cantonese}. While \il{Wu!Wenzhou Wu}WW is similar to \ili{Cantonese} in allowing a definite interpretation for Cl--N phrases, it differs from \ili{Cantonese} in that a definite interpretation for a Cl--N phrase is signalled by a shift in the tone of the classifier. As \citet{ChengSybesma2005} discuss in detail, the eight lexical tones of the language can be divided into four subgroups (A, B, C, and D), each subgroup containing two register subclasses, `hi' and `lo'. I reproduce \tabref{tab:Hall:ToneV} presenting the tone values for each lexical tone here (contour values taken from \citealt{Norman1988}).

\begin{table}[h] \small
\caption{Lexical tones of \il{Wu!Wenzhou Wu}Wenzhou Wu\label{tab:Hall:ToneV}}
\begin{tabularx}{\textwidth}{cccccccc}
\lsptoprule
1: hi-A&2: lo-A&3: hi-B&4: lo-B&5: hi-C&6: lo-C&7: hi-D&8: lo-D\\
\midrule
44&31&45 (abrupt)&24 (abrupt)&42&11&23&12\\
\lspbottomrule
\end{tabularx}
\end{table}

In an \is{indefinites}\is{noun phrases}indefinite noun phrase containing a classifier, the classifier carries its underlying, lexically specified tone. However, when the tone of the classifier shifts to a D tone (no matter what the underlying lexical tone of that particular classifier is), the Cl--N phrase is interpreted as definite. Thus, when definite, hi-A (tone 1), hi-B (tone 3), hi-C (tone 5) all shift to hi-D (tone 7), and hi-D (tone 8) also surfaces as hi-D. Lo-A (tone 2), lo-B (tone 4), lo-C (tone 6) and lo-D (tone 8) all surface as lo-D. A change in the morphology of the classifier gives rise to a change in interpretation. A minimal pair can be shown for a Cl--N phrase in object position \REF{ex:hall:21}, where a Cl--N phrase is acceptable under both a definite and an indefinite reading, the difference in meaning being indicated only by the tone on the classifier.\largerpage

\ea \label{ex:hall:21}
\ea
\gll 
{\`ŋ}\tss{4} ɕi\tss{3} ma\tss{4} \textbf{paŋ\tss{3}} sɨ\tss{1}\\
I want buy {\sc{cl}}\sub{B-tone} book\\
\glt
`I want to buy \textbf{a} book'\\
\ex
\gll
{\`ŋ}\tss{4} ɕi\tss{3} ma\tss{4} \textbf{paŋ\tss{7}} sɨ\tss{1}\\
I want buy {\sc{cl}}\sub{D-tone} book\\
\glt
`I want to buy \textbf{the} book'\\
\z
\z

Because of a ban on indefinite preverbal subjects (similar to that of \il{Mandarin}MC and \ili{Cantonese}), Cl--N phrases in subject position with an underlying ``indefinite'' classifier tone (i.e. any non-D tone) are unacceptable:

\ea \label{ex:hall:22}
\ea[*]{ \label{ex:hall:22a}
\gll 
dʏu\tss{2} kau\tss{8} i\tss{5} tsau\tss{3}-ku\tss{5} ka\tss{1}løy\tss{6}\\
\textsc{cl}\sub{A-tone} dog want walk-cross street\\
\glt
Intended: `A dog wants to cross the street.'\\}
\ex[]{ \label{ex:hall:22b}
\gll
dʏu\tss{8} kau\tss{8} i\tss{5} tsau\tss{3}-ku\tss{5} ka\tss{1}løy\tss{6}\\
\textsc{cl}\sub{D-tone} dog want walk-cross street\\
\glt
`The dog wants to cross the street.'\\}
\z 
\z 

As shown by the example in \REF{ex:hall:22b}, a D-tone alternative is well formed, but produces a definite interpretation. 

What about when numerals are combined with Cl--N phrases? \citet{ChengSybesma2005} point out that classifiers preceded by numerals keep their underlying tone, and \#--Cl--N phrases are necessarily interpreted as indefinite. That is, definite morphology on the classifier is blocked when a numeral merges, and a \#--Cl--N phrase cannot have a definite interpretation.

\ea 
\gll
{\`ŋ}\tss{4} ɕi\tss{3} ma\tss{4} ŋ\tss{4} \textbf{paŋ\tss{3}} sɨ\tss{1} le\tss{2} tshɨ\tss{5}\\
I want buy four {\sc{cl}}\sub{B-tone} book come read\\
\glt
`I want to buy four books to read.'\\
\z 

This is another example of a case where the ability of a classifier to \is{definiteness marking|(}encode definiteness is blocked by a numeral, but where there is an overt morphological reflex of definiteness. 

\subsection{Weining Ahmao}\il{Weining Ahmao} \label{sec:hall:4.2}\largerpage[2]

A second, and here crucial example of ``inflecting'' classifiers is the fascinating case of \ili{Weining Ahmao} \citep{GernerBisang2008,GernerBisang2010}. A \ili{Miao-Yao} language spoken in western Guizhou province, \ili{Weining Ahmao} (WA) \is{definiteness marking|)}encodes not only definiteness, but also number and `size' (diminutive, medial and augmentative) on the classifier. The function of the `size' inflection goes beyond encoding literal size; it mainly carries a socio-pragmatic function whereby the particular choice of classifier form indexes the gender and age of the speaker.{\interfootnotelinepenalty=10000\footnote{The only other vaguely similar socio-pragmatic classifier function that I am aware of is exhibited in Assamese\il{Assamese}, where there are four separate classifiers for humans, but which differ with respect to the status of the human that is being referred to \citep[102--103]{Aikhenvald2000}:

\vspace{.5\baselineskip}
{\centering Table i: Assamese\il{Assamese} classifiers for humans\\
\begin{tabularx}{0.96\textwidth}{QQQQ}
\lsptoprule
{\scriptsize Human males of normal rank (respectful)} & {\scriptsize Female animals; human females (disrespectful)} & {\scriptsize High-status humans of any sex} &{\scriptsize Humans of either sex (respectful)}\\
\midrule
\textit{zɔn}&\textit{zɔni}&\textit{zɔna}&\textit{gɔraki}\\
\lspbottomrule
\end{tabularx}}}}\pagebreak

Male speakers typically use augmentative forms of the classifier, female speakers the medial form, and children the diminutive form. Although this third aspect of classifiers in the language is particularly rare and interesting, I put aside discussion of the socio-pragmatic facts here, and concentrate instead on number and definiteness; I direct the reader to \citet{GernerBisang2008,GernerBisang2010} for an in-depth discussion of the socio-pragmatic nuances of classifier use in the language.

\tabref{tab:hall:2} gives the abstract summary of the forms of classifiers in \ili{Weining Ahmao} that \citet[721]{GernerBisang2008} produce.

\begin{table}[h]
\caption{Summary of the forms of classifiers in \ili{Weining Ahmao}\label{tab:hall:2}}
\resizebox{\textwidth}{!}{\begin{tabular}{llllll}
\lsptoprule
&&\multicolumn{2}{c}{Singular}&\multicolumn{2}{c}{Plural}\\\cmidrule(lr){3-4}\cmidrule(lr){5-6}
Gender/Age&Size&Definite&Indefinite&Definite&Indefinite\\
\midrule
Male&Augmentative&CVT&C*VT&\textit{ti\tss{55}a\tss{11}}CVT\tss{$\prime$}&\textit{di\tss{31}a\tss{11}}C*VT\tss{$\prime$}\\
Female&Medial&C\textit{ai\tss{55}}&C*\textit{ai\tss{213}}&\textit{tiai\tss{55}a\tss{11}}CVT\tss{$\prime$}&\textit{diai\tss{213}a\tss{11}}C*VT\tss{$\prime$}\\
Children&Diminutive&C\textit{a\tss{53}}&C*\textit{a\tss{35}}&\textit{tia\tss{55}a\tss{11}}CVT\tss{$\prime$}&\textit{dia\tss{55}a\tss{11}}C*VT\tss{$\prime$}\\
\lspbottomrule
\end{tabular}}
\end{table}

Taking the augmentative (male) form to be the base form, C stands for simple, double or affricated consonant, V stands for simple or double vowel, T stands for tone, and the superscript numbers represent relative pitch on a scale from 1 (lowest) to 5 (highest). T\tss{$\prime$} indicates an altered tone from T, and * indicates a suprasegmental change in the consonant, such as aspiration or devoicing, although there is also sometimes an absence of sound changes. To illustrate the application of this abstract schema with a concrete example from the language, we take the classifier for animacy, \textit{tu\tss{44}} \citep[722]{GernerBisang2008}, shown in \tabref{WAClPara}.


\begin{table}[h]
\caption{Inflection of \textit{tu\tss{44}}\label{WAClPara}}
\resizebox{\textwidth}{!}{\begin{tabular}{llllll}
\lsptoprule
&&\multicolumn{2}{c}{Singular}&\multicolumn{2}{c}{Plural}\\\cmidrule(lr){3-4}\cmidrule(lr){5-6}
Gender/Age&Size&Definite&Indefinite&Definite&Indefinite\\
\midrule
Male&Augmentative&\textit{tu\tss{44}}&\textit{du\tss{31}}&\textit{ti\tss{55}{a}\tss{11}tu\tss{44}}&\textit{di\tss{31}{a}\tss{11}tu\tss{44}}\\
Female&Medial&\textit{{tai}\tss{44}}&\textit{{dai}\tss{213}}&\textit{ti{ai}\tss{55}{a}\tss{11}tu\tss{44}}&\textit{di{ai}\tss{213}{a}\tss{11}tu\tss{44}}\\
Children&Diminutive&\textit{{ta}\tss{44}}&\textit{{da}\tss{35}}&\textit{ti{a}\tss{55}{a}\tss{11}tu\tss{44}}&\textit{di{a}\tss{55}{a}\tss{11}tu\tss{44}}\\
\lspbottomrule
\end{tabular}}
\end{table}

As an example, \REF{ex:hall:24} shows the four ways a male (adult) speaker can  refer to oxen, with differences in number and \is{definiteness marking}definiteness being encoded solely on the classifier.

\ea \label{ex:hall:24}

\ea
\gll
tu\tss{44} ɲɦu\tss{35}\\
{\sc{cl.aug.sg.def}} ox\\
\glt
`the ox'\\

\ex
\gll
du\tss{31} ɲɦu\tss{35}\\
{\sc{cl.aug.sg.indef}} ox\\
\glt
`an ox'\\

\ex 
\gll
ti\tss{55}a\tss{11}tu\tss{44} ɲɦu\tss{35}\\
{\sc{cl.aug.pl.def}} ox\\
\glt
`the oxen'\\

\ex 
\gll
di\tss{31}a\tss{11}tu\tss{44} ɲɦu\tss{35}\\
{\sc{cl.aug.pl.indef}} ox\\
\glt
`(some) oxen'\\
\z
\z

Interestingly, constructions involving numerals are always interpreted as \is{indefinites}indefinite, and when a numeral (including numerals greater than \is{numeral `one'}`one') is present, both definite forms and plural forms of the classifier are ungrammatical. A numeral therefore must occur only with an indefinite singular classifier (regardless of `size'): all other combinations are ungrammatical \citep[588]{GernerBisang2010}.

\ea 
\ea[*]{
\gll
i\tss{55} tai\tss{44} ɲɦu\tss{35} \\
one {\sc{cl.med.sg.def}} ox \\
\glt 
Intended: `the one (sole) ox' \\}

\ex[]{ 
\gll
i\tss{55} dai\tss{213} ɲɦu\tss{35} \\
one {\sc{cl.med.sg.indef}} ox \\
\glt `one ox' \\}
\z
\z 

\ea 
\ea[*]{ 
\gll 
tsɨ\tss{55} la\tss{53} tau\tss{55} \\
three {\sc{cl.dim.sg.def}} hill \\
\glt
Intended: `the three hills' \\}
\ex[]{ 
\gll 
tsɨ\tss{55} la\tss{35} tau\tss{55} \\
three {\sc{cl.dim.sg.indef}} hill \\
\glt
`three hills' \\}
\z 
\z 

\ea 
\ea[*]{
\gll 
tsɨ\tss{55} ti\tss{55}a\tss{11}lu\tss{55} ɕey\tss{55}\\
three {\sc{cl.aug.pl.def}} valley\\
\glt
Intended: `the three valleys'\\}

\ex[*]{
\gll 
tsɨ\tss{55} diai\tss{213}a\tss{11}lu\tss{55} ɕey\tss{55}\\
three {\sc{cl.med.pl.indef}} valley\\
\glt
Intended: `three valleys'\\}
\z
\z

The same is true for the \is{quantifiers}quantifier \textit{pi\tss{55}dʐau\tss{53}} `several': it can only occur with a singular indefinite classifier:

\ea 
\ea[*]{
\gll
pi\tss{55}dʐau\tss{53} dʑai\tss{53} tɕi\tss{55}\\
several {\sc{cl.med.sg.def}} road\\
\glt Intended: `the several roads'\\}

\ex[]{
\gll
pi\tss{55}dʐau\tss{53} dʑɦai\tss{213} tɕi\tss{55} \\
several {\sc{cl.med.sg.indef}} road \\
\glt `several roads' \\}
\z 
\z 

\is{noun phrases}Noun phrases with a \is{demonstratives}demonstrative and a Cl--N constituent, on the other hand, always take a definite classifier. 

\ea 
\ea[]{
\gll
lu\tss{55} a\tss{55}və\tss{55} vɦai\tss{35}\\
{\sc{cl.aug.sg.def}} stone {\sc{dem:med}}\\
\glt
`that stone (at medial distance from me)'\\}

\ex[*]{
\gll
lu\tss{33} a\tss{55}və\tss{55} vɦai\tss{35}\\
{\sc{cl.aug.sg.indef}} stone {\sc{dem:med}}\\
\glt
Intended: `that stone (at medial distance from me)'\\}
\z 
\z 

This is another example of a classifier language where the \is{definiteness marking}coding of definiteness on the classifier is blocked by the presence of a numeral. I now show how the facts from \ili{Weining Ahmao} are problematic for the \is{Head Movement Constraint}HMC account of numeral blocking, and propose a revised account which can capture all of the relevant facts.\is{numeral blocking|)}

\section{Revising the HMC account}\is{Head Movement Constraint}

Recall from the previous discussion that we have the following facts to account for:

\begin{enumerate}
\item Cl--N phrases can have a definite interpretation in some languages, but \#--Cl--N phrases never can.
\item Classifiers in \il{Wu!Wenzhou Wu}WW can have overt definiteness morphology.
\item Classifiers in \il{Weining Ahmao}WA can have overt number and definiteness morphology.
\item Classifiers cannot take definite form when a numeral is present in \il{Wu!Wenzhou Wu}WW and \il{Weining Ahmao}WA.
\item Classifiers in \il{Weining Ahmao}WA are singular in form when a numeral is present.
\end{enumerate}

Let us assume that number marking is the morphological realisation of a head, Num, and that \isi{definiteness marking} is the morphological realisation of a head, D. I further assume here, against the proposal in \citet{Simpson2005}, and following a number of recent proposals, that numerals merge as specifiers, not as heads \citep{Cinque2005,Borer2005,Ionin2006,Ouwayda2014}.\footnote{The motivations for this assumption come from various facts about complex numerals, and number marking related to numerals across languages. I do not have space to go through each of the arguments here, and instead simply direct the reader to these references.}  

Further, I assume a standard approach to morphological word formation where syntactic operations feed morphological word formation (e.g. \citealt{Travis1984,Baker1988,HalleMarantz1993} among many others),\footnote{I put aside here the fact that in recent years the status of \isi{head movement} as a word formation operation has been questioned widely in the literature. See \citet{Brody2000}, \citet{Abels2003}, \citet{Matushansky2006}, \citet{Roberts2010}, \citet{Svenonius2012}, \citet{Adger2013}, \citet{Hall2015}, among others. Also see \citet{Hall2015} for a similar argument about the \is{Head Movement Constraint}HMC account of \isi{numeral blocking}, but with a revised account of the facts couched in the language of Brody's \isi{Mirror Theory}.} such that roll-up \isi{head movement} and adjunction creates complex heads with complex morphology. Now, if we follow  \citet{Simpson2005} in assuming that definiteness is licensed in Cl--N phrases through the movement of Cl to D, then definiteness morphology on classifiers in \il{Wu!Wenzhou Wu}WW, and number and \is{definiteness marking}definiteness marking on bare classifiers in \il{Weining Ahmao}WA means that successive cyclic \isi{head movement} of Cl through Num up to D must be possible, with the complex head being realised in D.\footnote{\label{n-to-d}An anonymous reviewer asks why it has to be Cl that moves to D, and not, say, N, as in \ili{Italian}. This is a really a deep question about how to account for parametric variation, and I do not have space to go in to detail here, but for concreteness' sake I am adopting the position that feature specifications on functional elements are the locus of variation. This means that there is a feature on the classifier (say, \textit{u}def) which is a goal for Agree with [def] of D, and this Agree relation forces the subsequent \isi{head movement}. N does not move because there is no feature on N which forces movement. The question then arises about \ili{Mandarin}, and N-to-D movement. All I can say about this is that I do not adopt the position that definite \isi{bare nouns} in \ili{Mandarin} involve N-to-D movement \citep{ChengSybesma1999}, and in fact think that this is a position which has various problems associated with it. See \citet[\S4]{Hall2015} for further discussion.} This is illustrated in \REF{ex:hall:30}.\footnote{I leave aside how the relative ordering of the morphemes (Cl, Num and D) is achieved here.}

\ea \label{ex:hall:30}\is{classifier phrases|(}
\begin{forest}
[DP[D [Num,name=d [Cl] [Num]] [D]] [NumP[\textit{t\sub{Num}},name=num] [ClP[\textit{t\sub{Cl}},name=cl] [NP]]]]
\draw[->](cl) to[out=south,in=south,looseness=1] (num);
\draw[->](num) to[out=south west,in=south east,looseness=1] (d);
\end{forest}
\z

We are left with evidence in the morphology that \isi{head movement} through these positions is possible. If Cl can move to Num as the morphology suggests, and if numerals merge in the specifier of Num, then it should also be possible to raise the complex classifier head to D. This movement past the numeral in the specifier position would not constitute an \is{Head Movement Constraint}HMC violation, as there are no intervening heads in the same extended projection. This is shown in \REF{ex:hall:31}.\footnote{Note that, if this movement of Cl to D over the numeral were a possibility, we would also expect to see classifiers preceding numerals where the \is{determiner phrases}DP is \is{definites}definite, and following the numeral when the DP is indefinite, and this is never the case.}\is{indefinites|)}

\ea \label{ex:hall:31}
\begin{forest}
[DP[D [Num,name=d [Cl] [Num]] [D]] [NumP [\#P] [NumP [\textit{t\sub{Num}},name=num] [ClP[\textit{t\sub{Cl}},name=cl] [NP]]]]]
\draw[->](cl) to[out=south,in=south,looseness=1] (num);
\draw[->](num) to[out=south west,in=south east,looseness=1] (d);
\end{forest}
\z

As we have seen, however, this is not the case. The ability to move over the numeral should furthermore naturally extend to \ili{Cantonese}, but again, it clearly does not. We know that the presence of a numeral robustly blocks a definite interpretation across all \is{numeral classifier languages}classifier languages, and also definite morphology in those languages where it exists. This means that an \is{Head Movement Constraint}HMC account of the blocking effect could not be right.\footnote{Of course it is possible that \ili{Cantonese} and \il{Wu!Wenzhou Wu}WW and \il{Weining Ahmao}WA are all just different, and that the HMC account does work for \ili{Cantonese}, and something else is at work in \il{Wu!Wenzhou Wu}WW and \il{Weining Ahmao}WA. However, we are aiming for an explanation that can cover \textit{all} of the facts in the simplest way, avoiding language specific stipulations where possible. I show in \sectref{sec:hall:5.1} that this is possible if we abandon the HMC account.}

\subsection{A new approach}\label{sec:hall:5.1}

To capture the facts, I maintain the core assumption of \citet{Simpson2005} that it is indeed the interaction of Cl and D which gives rise to definite interpretations in Cl--N configurations, but I further propose that Cl--N phrases and \#--Cl--N phrases have different syntactic structures. In a bare Cl--N configuration, the full \is{determiner phrases}DP takes roughly the same form as that proposed by \citeauthor{Simpson2005}: D takes a NumP complement which takes a ClP complement which takes an NP complement. Definite classifiers are the result of movement of the Cl head to D (through Num): I implement this through Agree between Def features on the heads, followed by roll-up movement \citep{Chomsky1995}.

\ea
\begin{forest}
[DP[D,name=D] [NumP[Num,name=Num] [ClP[Cl,name=Cl] [NP]]]]
\draw[->](Cl) to[out=south ,in=south ,looseness=1.5]  (Num);
\draw[->](Num) to[out=south west,in=south west,looseness=1]  (D);
\end{forest}
\z

Where the def feature is not present, no movement takes place and the result is \isi{indefiniteness}.

Where my analysis parts from \citet{Simpson2005} is in the structure of \#--Cl--N phrases. When a numeral is present, I assume that the classifier forms a constituent with it, and this constituent merges in the specifier of Num. I assume that the numeral is phrasal, and is either a specifier of Cl, or an adjunct to it. 

\ea 
\begin{forest}
[DP [D,name=D] [NumP[ClP [\#] [Cl,name=Cl]] [Num [Num] [NP]]]]
\end{forest}
\z\largerpage[1]

In this configuration, Agree between D and Cl is possible, but movement of Cl is blocked because of an independently motivated ban on \is{head movement}Head Movement out of a specifier \citep[see e.g.][]{Roberts2010}, as illustrated in \REF{ex:hall:34}.

\ea \label{ex:hall:34}
\begin{forest}
[DP [D,name=D] [NumP[ClP [\#P] [Cl,name=Cl]] [NumP [Num] [NP]]]]
\draw[overlay,->](Cl) to[out=north east,in=south east,looseness=1.2]  node[midway]{\Huge$\ast$}(D);
\end{forest}
\z\is{determiner phrases}

The blocking effect is therefore not a result of the \is{Head Movement Constraint}HMC, and definite plural classifiers are therefore fully possible where Cl moves through Num to D, so long as a numeral is not present. A further benefit of this approach is that a ban on head movement \textit{into} a specifier also prevents Num from moving into the ClP and being realised on Cl. This explains why the classifier appears singular with numerals in \il{Weining Ahmao}WA. The Num head has a null spell-out when it does not form a complex head with Cl, and the Cl takes a default (singular) spell-out.{\interfootnotelinepenalty=10000\footnote{Amy Rose Deal (p.c.) asks whether this \is{numeral blocking}blocking of definiteness by a numeral might simply be the result of the numeral always having existential force, in a similar as way suggested by Cheng \& Sybesma, and hence that there is no need for a syntactic explanation. A D head merged above Num would not be able to pick out a \is{maximality}maximal individual because it would have already been bound off by the \isi{existential quantifier}. I note that this could not be the case, as \#--Cl--N sequences can in fact have definite interpretations associated with them with the addition of certain other elements higher in the phrase. High \is{adjectives}adjectival modifiers can give rise to definiteness (Adj--\#--Cl--N sequences), as can the introduction of a \is{demonstratives}demonstrative above the numeral. An anonymous reviewer also points out that the quantifier \textit{dou} added to \#--Cl--N in subject position gives rise to a definite interpretation \citep{Cheng2009}. This suggests that the introduction of the numeral does not semantically block the possibility of a definite interpretation. See \citet[\S4]{Hall2015} for discussion.}}

\subsection{Summary}

Again, I restate the empirical facts which were to be explained:

\begin{enumerate}
\item Cl--N phrases can have a definite interpretation in some languages, but \#--Cl--N phrases never can.
\item Bare classifiers in \il{Wu!Wenzhou Wu}WW have overt \is{definiteness marking}definiteness morphology.
\item Bare classifiers in \il{Weining Ahmao}WA have overt number and definiteness morphology.
\item Classifiers cannot take definite form when a numeral is present in \il{Wu!Wenzhou Wu}WW and \il{Weining Ahmao}WA.
\item Classifiers in \il{Weining Ahmao}WA are singular in form when a numeral is present.
\end{enumerate}

Each is now explained under the dual-structure account: Cl can move through Num and D creating a complex definite head with complex morphology, if the language has overt morphological content associated with these heads. The \#--Cl--N structure containing \# and Cl as a constituent means that Cl can't move to D, following a ban on \isi{head movement} out of a specifier, which blocks a definite interpretation. Num can't move to Cl, following a ban on head movement into a specifier, which blocks plural morphology. Each follows from the dual structure proposed, and appealing to these two structures means that the apparent gaps left by the \is{Head Movement Constraint}HMC approach are filled.\largerpage 

The two distinct structures for Cl--N and \#-Cl--N are repeated here in (\ref{ex:hall:35}--\ref{ex:hall:36}).{\interfootnotelinepenalty=10000\footnote{An anonymous reviewer suggests that we might expect there to be further syntactic evidence that the structures are different in these cases. Currently I have not been able to identify any very clear differences aside from those already outlined at the beginning of the paper (i.e. that \#--Cl--N phrases and Cl--N phrases have a different distribution with respect to availability in subject/\isi{topic} and object position). One hint at another potential difference comes from another comment by the same reviewer. \citet{Li2011} points out that for some \il{Mandarin}MC speakers, it is possible to get an \is{adjectives}adjective to intervene between a numeral and a classifier, in a very restricted set of cases:
	
\ea
\gll
Tou shang dai le liang da duo hua.\\
{head} {on} {wear} \textsc{perf} {two} {big} \textsc{cl} {flower}\\
\glt
`(She) wore two big flowers on her head.'\\
\z

For the two speakers that I could get to accept the above example as possible, neither could do the same with a bare Cl--N sequence \textit{da duo hua}. This is potentially another syntactic difference: an \is{adjectives}adjective can merge in between the numeral and classifier in the structure in \REF{ex:hall:36}, but it cannot appear in the bare Cl--N structure in \REF{ex:hall:35}. I accept that this is not knock-down evidence of a major syntactic difference, but is at least suggestive. I leave an investigation of further differences between the two to future research.}} 

\noindent\begin{minipage}{.5\linewidth}
\ea \label{ex:hall:35} Cl--N

\begin{forest}
[DP[D,name=D] [NumP[Num,name=Num] [ClP[Cl,name=Cl] [NP]]]]
\end{forest}
\z
\end{minipage}%
\begin{minipage}{.5\linewidth}
\ea \label{ex:hall:36} \#--Cl--N

\begin{forest}
[DP [D,name=D] [NumP[ClP [\#] [Cl,name=Cl]] [Num [Num] [NP]]]]
\end{forest}
\z
\end{minipage}\vspace{\baselineskip}\is{classifier phrases|)}

A consequence of this analysis is that numerals form a constituent with the classifier to the exclusion of the \is{nouns}noun in \is{numeral classifier languages}classifier languages, when a numeral is present. This could be seen as a counter-intuitive proposal, and in order to fully motivate this approach it is necessary to provide some motivation for the existence of the two structures beyond just the facts discussed above. In the next section I offer some independent support for the proposed \#+Cl constituency.

\section{Classifier and numeral constituency}

There is some debate in the literature on classifiers over whether the classifier and numeral form a constituent, and whether this is consistent across all classifier languages. The variety of positions can be summarized as follows:

\ea \label{ex:hall:37}
\ea Classifier and numeral are a complex head \citep{Kawashima1998}.
\ex Classifier is a head in the extended nominal projection (xNP), Numeral is a specifier of Cl (\citealt{Tang1990}; or Cl is Num, numeral is specifier: \citealt{Watanabe2006}).
\ex Classifier is a head in the xNP, Numeral is a head of NumP \citep{ChengSybesma1999,Simpson2005}.
\ex Classifier is a head in the xNP, Numeral is a specifier of \#P \citep{Borer2005,Ouwayda2014}. 
\ex Classifier and Numeral form a constituent (\citealt{FukuiTakano2000}; also \citealt{Ionin2006}).
\ex Different classifier languages have different structures depending on whether the classifier appears independently \citep{SaitoEtAlii2008,Jenks2010,Hall2015}.
\z
\z 

Most arguments in favour of a complement relation existing between the classifier and the \is{nouns}noun attempt to show that the classifier behaves as a functional head, and therefore that it cannot be part of a single functional unit with the numeral. This does not, however, suggest that the two cannot be a constituent. The only clear argument claiming that the two could not be a constituent, at least in \il{Mandarin}MC, is proposed by \citet{SaitoEtAlii2008}. They show that the numeral and classifier can float to the left in \ili{Japanese}, stranding the noun \REF{ex:hall:38}, but that the same does not hold in \il{Mandarin}MC \REF{ex:hall:39}. 

\ea \label{ex:hall:38}
\ea[]{
\gll
Taroo-wa san-satu no hon-o katta.\\
Taro-\textsc{top} three-\textsc{cl} \textit{no} book-\textsc{acc} bought\\
\glt `Taro bought three books.'\\}

\ex[]{
\gll
San-satu, Taroo-wa hon-o katta.\\
three-\textsc{cl} Taro-\textsc{top} book-\textsc{acc} bought\\}
\z 
\z 

\ea \label{ex:hall:39}
\ea[]{
\gll
Zhangsan mai-le san-ben shu.\\
Zhangsan buy-\textsc{perf} three-\textsc{cl} book\\
\glt 
`Zhangsan bought three books.'\\}

\ex[*]{
\gll
San-ben, Zhangsan mai-le shu.\\
three-\textsc{cl} Zhangsan buy-\textsc{perf} book\\}
\z
\z

They posit an adjunction structure for the numeral and classifier in \ili{Japanese}, where they form a constituent. For \il{Mandarin}MC they suggest that the classifier is a functional head which takes an NP complement, and which projects a numeral in its specifier. This represents the conclusion that the lack of availability of movement of the numeral and classifier in \il{Mandarin}MC means that the numeral and classifier are not a constituent. This is not a particularly strong argument, however, as the lack of movement could just be an independent fact about the language, and this is not ruled out as a possibility in their paper. I therefore continue in the assumption that my proposal is not directly falsified by the Q-Float facts.

Given the controversy and diverse opinions related to the constituency of the numeral, classifier, and \is{nouns}noun, it is necessary to provide some further motivating evidence for the constituency that I propose above. Therefore, in this section, I present some supporting evidence for the claim that the numeral and classifier form a constituent to the exclusion of the noun. First, I briefly argue against the claim that there is a strong selectional relation between the classifier and the noun, and also show that some cross-linguistic evidence supports a view where the classifier and the numeral have a closer relation than the classifier and noun (when both are present). I then move on to my main typological evidence that the numeral and classifier form a constituent to the exclusion of the noun, which involves an argument from word order: if numeral and classifier did not form a separate constituent from the noun then we would expect much more variation in word order within the \is{noun phrases}noun phrase in \is{numeral classifier languages}classifier languages than we actually see.
 
\subsection{Close relationship between classifier and noun} 

The main observation that I want to take into consideration here is that there appears to be something like a selectional or agreement relation between the classifier and the \is{nouns}noun, as the following examples illustrate.

\ea \label{ex:hall:40} 
\textit{gen}: classifier for thin, slender objects
\ea[]{
\gll
yi-gen xiangjiao \\
one-{\sc{cl}} banana \\
\glt
`one banana' \\}

\ex[*]{
\gll 
yi-gen gou \\
one-{\sc{cl}} dog \\
\glt
Intended: `one dog' \\}
\z
\z

\ea \label{ex:hall:41} 
\textit{zhi}: classifier for (certain) animals
\ea[*]{ \label{ex:hall:41a}
\gll yi-zhi xiangjiao \\
one-{\sc{cl}} banana \\
\glt
Intended: `one banana' \\}

\ex[]{
\gll
yi-zhi gou\\
one-{\sc{cl}} dog \\
\glt
`one dog'\\}
\z 
\z

In \REF{ex:hall:40}, the classifier \textit{gen} can only cooccur with a certain set of objects (namely those which are thin and long), and there is something of a clash when the classifier appears with a noun from outside of that class (such as `dog'). `Dog' has to appear with a different classifier, \textit{zhi}, as illustrated in \REF{ex:hall:41}. An anonymous reviewer questions how such a relationship between a classifier and a noun can possibly be set up in a structure such as that proposed in \REF{ex:hall:36}. To this I have two answers. First, I do not think that this ``agreement'' relationship necessarily has to do with Agree or selection or some such purely syntactic relation between two heads. Rather, I think that the relationship is semantic, and results from the lexical entries for the classifiers. One illustration of this comes from an effect seen with some speakers where \isi{nouns} can be coerced into the appropriate group under some circumstances. Two informants fully accept \REF{ex:hall:41a}, under a special kind of interpretation where the banana is assumed to be particularly cute (and possibly have pet like characteristics). I assume here that this means that perhaps the example should not be marked as ungrammatical, but instead as having a strong semantic implausibility associated with it. Further, it seems possible that classifiers are able to shift noun interpretation. Some nouns can appear with various different classifiers, but with different interpretations.

\ea \label{ex:hall:42}
\ea 
\gll
yi-bu dianhua \\
one-{\sc{cl}} telephone \\
\glt 
`one telephone' \\

\ex 
\gll
yi-tong dianhua \\
one-{\sc{cl}} telephone \\
\glt
`one phone call' \\
\z 
\z 

\ea \label{ex:hall:43}
\ea 
\gll
san-zhi hua \\
three-\textsc{cl} flower \\
\glt
`three flowers' (long on their stalks) \\

\ex 
\gll
san-duo hua \\
three-\textsc{cl} flower \\
\glt
`three flowers' (round, with a focus on floweryness) \\
\z 
\z

I take this to mean that the \is{nouns}noun denotes a nebulous property which includes each of the different possible interpretations included in the above examples (`telephone' includes telephone objects as well as calls), and then the semantics of the classifier includes a presupposition that the object being counted is one of a particular set. 

\subsubsection{Classifiers in Mi'gmaq and Chol}\il{Mi'gmaq}\il{Chol}

Some separate supporting evidence that the numeral and classifier are more closely associated comes from \citet{BaleCoon2014}.\footnote{The idea that classifiers are ``for'' numerals, as far as the semantics is concerned, goes back to \citet{Krifka1995}.} They note that \ili{Mi'gmaq} and \ili{Chol} both have a surprising distribution of classifiers if it's assumed that the classifier is semantically more closely related to the noun than the numeral. The facts are as follows.

In \ili{Mi'gmaq}, the numerals 1--5 cannot appear with classifiers, but 6 and higher must. 

	\ea
		\ea[]{
		\gll
		na'n-ijig ji'nm-ug \\
		five-\textsc{agr} man-\textsc{pl} \\}
		
		\ex[*]{
		\gll
		na'n te's-ijig ji'nm-ug \\
		five \textsc{cl}-\textsc{agr} man-\textsc{pl} \\
		\glt `five men' \\}
		\z
	\z
	
	\ea
		\ea[*]{
		\gll 
		asugom-ijig ji'nm-ug \\
		six-\textsc{agr} man-\textsc{pl} \\}
		
		\ex[]{
		\gll
		asugom te's--ijig ji'nm-ug\\
		six \textsc{cl}-\textsc{agr} man-\textsc{pl}\\
		\glt 
		`six men'\\}
		\z
	\z 


In \ili{Chol}, there is a vestigal \ili{Mayan} base-20 number system: speakers only use \ili{Mayan} numerals for 1--6, 10, 20, 40, 60 ..., and otherwise, they use \ili{Spanish} loan numerals. What is important is that classifiers obligatorily appear with \ili{Mayan} numerals \REF{ex:hall:46}, but are obligatorily absent with \ili{Spanish} numerals \REF{ex:hall:47}:\\

\noindent\begin{minipage}{.5\linewidth}

\ea \label{ex:hall:46}
\ea[]{
\gll
ux-p'ej tyumuty \\
three-\textsc{cl} egg \\}

\ex[*]{ 
\gll 
ux tyumuty \\
three egg \\
\glt `three eggs' \\}
\z
\z

\end{minipage}%
\begin{minipage}{.5\linewidth}

\ea \label{ex:hall:47}
\ea[*]{ 
\gll 
nuebe-p'ej tyumuty \\
nine-\textsc{cl} egg \\}

\ex[]{
\gll
nuebe tyumuty \\
nine egg \\
\glt `nine eggs' \\}
\z
\z
\end{minipage} \vspace{\baselineskip}

Note that this is true no matter what noun we use (including Spanish loan \isi{nouns}), and no matter what classifier the numeral combines with. 

Under an account where the numeral and classifier have a closer relationship, these facts immediately make sense. Under a Chierchian account where the classifier acts as an individualizer that ``portions out'' chunks of the mass that nouns denote \citep{Chierchia1998b}, the idiosyncratic behaviour of the numerals receives no explanation. This provides evidence that composition of the classifier and the numeral is required for the numeral to then be able to compose with the noun: this would make sense if \# and Cl form a constituent to the exclusion of the noun.

Of course \ili{Mi'gmaq} and \ili{Chol} are not related to the languages under discussion, but, on the assumption that there is some shared syntactic category of classifier in the \is{determiner phrases}DP of all of these languages, I take this to at least be suggestive evidence that there is a closer relation between the classifier and the numeral than the classifier and the noun. 

In the next subsection I move on to some typological evidence for this close relation between numeral and classifier. 

\subsection{Typology}

So far we have been focusing on languages where the numeral precedes the classifier, and the classifier precedes the \is{nouns}noun, giving the overall order in \REF{ex:hall:48}, illustrated with examples in \REF{ex:hall:49} and \REF{ex:hall:50}.

\noindent\begin{minipage}[t]{.5\linewidth}
\ea \label{ex:hall:48}
\# $\succ$ Cl $\succ$ N
\z
\end{minipage} \\

\noindent\begin{minipage}[t]{.49\linewidth}
\ea \label{ex:hall:49}
\gll liang gen xiangjiao \\
two \textsc{cl}\sub{thin/pole} banana \\
\glt `two bananas'\jambox{(\il{Mandarin}MC: \#$\succ$Cl$\succ$N)}
\z 
\end{minipage}%
\begin{minipage}[t]{.51\linewidth}
\ea \label{ex:hall:50}
\gll
ib-tus tub.txib \\
one-\textsc{cl}\sub{person/animal} messenger \\
\glt `one messenger'\jambox{(\ili{Hmong}: \#$\succ$Cl$\succ$N)}
\z
\end{minipage}\\

Unsurprisingly, we see cross-linguistic variation in the ordering of these elements, and there are languages where the numeral and classifier follow the noun \REF{ex:hall:51}, \REF{ex:hall:52}.\\

\noindent\begin{minipage}{.5\linewidth}
\ea \label{ex:hall:51}
\gll
hon san-satsu \\
book three-\textsc{cl}\sub{bound/printed} \\
\glt
`three books'\jambox{(\ili{Japanese}: N$\succ$\#$\succ$Cl)}
\z
\end{minipage}%
\begin{minipage}{.5\linewidth}
\ea \label{ex:hall:52}
\gll
phya tə Chaʔ \\
mat one \textsc{cl}\sub{flat/thin} \\
\glt 
`one mat'\jambox{(\ili{Burmese}: N$\succ$\#$\succ$Cl)}
\z
\end{minipage}\\

When we look at a full typology of \is{numeral classifier languages}classifier languages, however, it becomes clear that the order of the numeral, classifier and noun is quite constrained. In \citet{Hall2015} I discuss three word order surveys, which produce the following word order typology for classifier languages:\largerpage[1.5]

\ea \label{ex:hall:53}
Order of numeral, classifier and noun (following \citealt{Jones1970}, \citealt{Greenberg1972}, \citealt{Aikhenvald2000}):
\ea \# $\succ$ Cl $\succ$ N: very common (\il{Mandarin}MC, \ili{Vietnamese}, \ili{Cantonese}, ...)
\ex N $\succ$ \# $\succ$ Cl: very common (\ili{Thai}, \ili{Khmer}, \ili{Loniu}, ...)
\ex Cl $\succ$ \# $\succ$ N: very rare (\ili{Ibibio} only)
\ex N $\succ$ Cl $\succ$ \#: very rare/maybe no languages (possibly Bodo\il{Bodo} only)
\ex Cl $\succ$ N $\succ$ \#: very rare (\ili{Ejagham} only)
\ex \# $\succ$ N $\succ$ Cl: not attested
\z
\z

A closer look at the two extremely rare cases, i.e. \ili{Ibibio} (Cl$\succ$\#$\succ$N) and \ili{Ejagham} (Cl$\succ$N$\succ$\#), shows that they should in fact be removed from the typology. \ili{Ibibio} doesn't have classifiers at all \citep{Essien1990}. \ili{Ejagham} does not have obligatory classifiers, and examples involving classifier-like elements discussed in \citet{Greenberg1972} look more like a measure phrase (see \citealt{Watters1981} and \citealt{Hall2015} for discussion). If we remove these languages, then we have the following typology:\footnote{I have also included some additional N $\succ$ Cl $\succ$ \# languages (\ili{Tani} and \ili{Chin} languages) which are not included in the typological studies referenced above.}

\ea \label{ex:hall:54}
\ea \# $\succ$ Cl $\succ$ N: very common (\il{Mandarin}MC, \ili{Vietnamese}, \ili{Cantonese}, ...)
\ex N $\succ$ \# $\succ$ Cl: very common (\ili{Thai}, \ili{Burmese}, \ili{Khmer}, \ili{Loniu}, ...)
\ex Cl $\succ$ \# $\succ$ N: not attested
\ex N $\succ$ Cl $\succ$ \#: rare (a few \ili{Bodo-Garo}, \ili{Tani} and \ili{Chin} languages)
\ex Cl $\succ$ N $\succ$ \#: not attested
\ex \# $\succ$ N $\succ$ Cl: not attested
\z
\z

What is striking in this typology is that there are no attested orders where the numeral and the classifier are separated by the \is{nouns}noun.\footnote{For completeness' sake, I give a full list of all attested word orders in \is{numeral classifier languages}classifier languages in Table i. Note that the ``example languages'' column is not intended as an exhaustive list of all of the languages that exhibit that order.\il{Vietnamese}\il{Nung}\il{Malay}\il{Thai}\il{Khmer}\il{Javanese}\il{Burmese}\il{Maru}\il{Mandarin}\il{Cantonese}\il{Yao}\il{Coast Tsimshian}\il{Newar}\il{Dulong}\il{Nuosu Yi}\il{Lahu}\il{Akha}\il{Kokborok}\il{Apatani}\il{Mizo}\il{Mising}\il{Nishi}\vspace{.5\baselineskip}

{\centering Table i: All \is{determiner phrases}DP internal elements\\
\hspace{0.10\textwidth} \begin{tabularx}{0.75\textwidth}{lXX}
	\lsptoprule
	&Word order&Example languages\\
	\midrule
	1.&Num $\succ$ Cl $\succ$ N $\succ$ A $\succ$ Dem& {Vietnamese}, {Nung}, {Malay}\\
	2.&N $\succ$ A $\succ$ Num $\succ$ Cl $\succ$ Dem& {Thai}, {Khmer}, {Javanese}\\
	3.&Dem $\succ$ N $\succ$ A $\succ$ Num $\succ$ Cl& {Burmese}, {Maru}\\
	4.&Dem $\succ$ Num $\succ$ Cl $\succ$ A $\succ$ N& MC, {Cantonese}\\
	5.&Dem $\succ$ Num $\succ$ Cl $\succ$ N $\succ$ A& {Yao}\\
	6.&Num $\succ$ Cl $\succ$ A $\succ$ N $\succ$ Dem& {Coast Tsimshian}\\
	7.&Dem $\succ$ A $\succ$ N $\succ$ Num $\succ$ Cl&{Newari}, {Dulong}\\
	8.&N $\succ$ A $\succ$ Dem $\succ$ Num $\succ$ Cl& {Nuosu Yi}, {Lahu}, Akha\\
	9.&Dem $\succ$ N $\succ$ Adj $\succ$ Cl $\succ$ Num & {Kokborok}, Apatani, {Mizo}\\
	10.&Dem $\succ$ Adj $\succ$ N $\succ$ Cl $\succ$ Num & {Mising}, perhaps {Nishi}\\ 
	\lspbottomrule
\end{tabularx}\vspace{.5\baselineskip}}}\tss{,}\footnote{See \citet[\S5, especially \S5.4.1]{Hall2015} for an explanation of the absence of the Cl $\succ$ \# $\succ$ N order.} It is clear that this is completely expected if the numeral and the classifier form a constituent to the exclusion of the \is{nouns}noun, but remains mysterious if we posit the kind of structure proposed by \citet{Simpson2005}. In the next subsection I will explicitly show why.

\subsection{Deriving word order variation}

Recent work on cross-linguistic variation in the relative order of \is{determiner phrases}DP internal elements has suggested that we can make sense of gaps in the typology in systematic ways, under certain assumptions about the nature of DP internal roll-up movements \citep{Cinque1996,Cinque2005}, or with a flexible approach to the linearization of the unordered sets produced by Merge \citep{AbelsNeeleman2012}. I give a brief summary here of the two related approaches, and then show what predictions they would produce with respect to word order variation in classifier languages, on the assumption that the classifier takes a NP complement. 

\subsubsection{Cinque (2005): Universal 20}\is{Universal 20 (Greenberg's)|(}
\citet{Cinque2005} shows that each of the 14 attested orders of \is{demonstratives}Demonstrative, Numeral, \is{adjectives}Adjective and \is{nouns}Noun can be generated, while ruling out each of the 10 unattested orders, if the following constraints on movement operations are applied:

\ea \label{ex:hall:55}
\ea Merge order: [ . . . [\sub{WP} Dem . . . [\sub{XP} Num . . . [\sub{YP} A [\sub{NP} N]]]]]
\ex Parameters of movement
\ea No movement, or
\ex Movement of NP plus pied-piping of the \textit{whose picture} type (movement of [NP[XP]]), or  \label{ex:hall:55bii}
\ex Movement of NP without pied-piping, or \label{ex:hall:55biii}
\ex Movement of NP plus pied-piping of the \textit{picture of who} type (movement of [XP[NP]]). \label{ex:hall:55biv}
\ex \textit{Total} versus \textit{partial} movement of the NP with or without pied-piping (either NP moves all the way up or only partially)
\ex Neither \isi{head movement} nor movement of a phrase not containing the (overt) NP is possible.
\z
\z
\z

The first assumption of a fixed universal hierarchical order of elements in the DP gives us the underlying structure in \figref{ex:hall:56}.

\begin{figure}
\caption{Proposed universal base structure of the DP from \citet{Cinque2005}\label{ex:hall:56}} %\resizebox{.9\textwidth}{!}{%
%\Tree [.Agr\sub{w}P {} !{\qbalance} [.{} Agr\sub{w} [.WP  \textbf{DemP} [.{} W [.Agr\sub{x}P {} !{\qbalance} [.{} Agr\sub{x} [.XP \textbf{NumP}  [.{} X [.Agr\sub{y}P {} !{\qbalance} [.{} Agr\sub{y} [.YP \textbf{AP} [.{} Y \textbf{NP} !{\qbalance} ] ] ] ] ] ] ] ] ] ] ] ]}
\resizebox{\textwidth}{!}{%
\begin{forest}
for tree={my nice empty nodes}
	[Agr\sub{w}P [{}] [{} [Agr\sub{w} ]	[WP [\textbf{DemP} ] [{} [W ] [Agr\sub{x}P [{}] [{} [Agr\sub{x} ]	[XP [\textbf{NumP}	] [{} [X ] [Agr\sub{y}P [{}] [{} [Agr\sub{y} ] [YP [\textbf{AP} ] [{} [Y ] [\textbf{NP} ] ] ] ] ] ] ] ] ] ] ] ] ]
\end{forest}}
\end{figure}

\citeauthor{Cinque2005} assumes that modifiers are merged in the specifiers of functional heads in the xNP, and that antisymmetry (i.e. the LCA of \citealt{Kayne1994})  rules out symmetric base generation of modifiers, meaning that all postnominal modifiers must be generated through movement of the NP, or some constituent containing the NP. Each of the elements \is{demonstratives}demonstrative, numeral and \is{adjectives}adjective are taken to be phrasal elements which merge in the specifier of a functional head. In each case of movement, the NP, or pied-piped constituent containing the NP, moves to the specifier of an Agr head above the contentful phrasal element. The \is{noun phrases}noun phrase can move to any of the Spec Agr positions \REF{ex:hall:55biii}, and can pied-pipe any constituent either in the form [NP[XP]] \REF{ex:hall:55bii} or [XP[NP]] \REF{ex:hall:55biv}. This movement can be partial (to one of the intermediate Agr positions), or complete (all the way to the highest Agr projection). Through a combination of movement steps, which must follow the constraints in \REF{ex:hall:55}, each of the attested orders can be derived.\is{Universal 20 (Greenberg's)|)}

\subsubsection{Abels \& Neeleman (2012)}

\citet{AbelsNeeleman2012} argue that all of the orders that are generated by Cin\hyp{}que's approach can in fact be produced without some of the assumptions that \citeauthor{Cinque2005} makes about phrase structure and movement. They show that a more constrained theory of movement, coupled with flexibility in the linearization of sister nodes (eschewing the LCA) generates the same results. 

\ea \label{ex:hall:57}
\ea The underlying hierarchy is Dem \textgreater\ Num \textgreater\ A \textgreater\ N (where \textgreater\ indicates C-command);
\ex there is cross-linguistic variation with respect to the linearization of sister nodes in this structure;
\ex all (relevant) movements move a subtree containing N;
\ex all movements target a c-commanding position;
\ex all movements are to the left.
\z 
\z

The idea is that, with the underlying structure shown in \REF{ex:hall:58}, eight different word orders can be generated if we assume that linearization of sisters is flexible.

\noindent\begin{minipage}[t]{.5\linewidth}
\ea \label{ex:hall:58}
%\Tree [.{} Dem [.{} Num [.{} A N ] ] ]
%\resizebox{0.5\textwidth}{!}
\z


\end{minipage}%
\begin{minipage}[t]{.5\linewidth}
\ea Base generated orders
\ea Dem Num A N
\ex N A Num Dem
\ex Dem Num N A
\ex A N Num Dem
\ex Dem A N Num
\ex Num N A Dem
\ex Dem N A Num
\ex Num A N Dem
\z
\z

\end{minipage}\vspace{0.4cm} 

The remaining six orders are generated through movement constrained in the ways noted in \REF{ex:hall:57}. Simply put, this approach produces the same results, but appeals to flexibile linearization of sisters instead of massive roll-up movement.

\subsubsection{Predictions}

For our purposes, either approach to cross-linguistic variation in word order will do, and I remain agnostic as to which is the preferred approach. Here we are trying to account for the gaps in classifier language word order typology: in particular, why the classifier and the numeral are never separated by the \is{nouns}noun. Whether we take a roll-up movement approach following Cinque, or a flexible linearisation approach following \citeauthor{AbelsNeeleman2012}, we would expect the noun to be able to appear between the numeral and the classifier under any analysis of \is{determiner phrases}DP internal structure which takes the classifier to be a head taking the noun as a complement, and which takes the numeral to appear in a specifier or adjunct position above the classifier (i.e. \ref{ex:hall:37}b--c above). If the numeral is merged in the specifier of Num, then, under the roll-up movement approach, both Cl $\succ$ N $\succ$ \# \REF{ex:hall:60} and \# $\succ$ N $\succ$ Cl \REF{ex:hall:61} can be generated.\footnote{I follow \citet{Cinque2005} in having the specifier of an Agr head as a landing site, but have left out irrelevant Agr positions (i.e. Agr positions which are not the landing site of movement).}\tss{,}\footnote{A reviewer points out that different assumptions about the numeral (it heads its own projection vs it is in a specifier of another head) would lead to different predictions about what word orders are possible. This is true, but under all approaches (except for where the numeral and classifier go together as a separate constituent) we still expect the numeral and classifier to be separable, with the \is{nouns}noun intervening.}\is{classifier phrases|(}

\noindent\begin{minipage}[t]{.5\linewidth}
\ea \label{ex:hall:60} \small
\begin{forest}sn edges
[AgrP, nice empty nodes [ClP,name=ClP [Cl] [NP]] [{} [Agr] [NumP [\#] [{} [Num] [\textit{t}\sub{ClP},name=trace] ] ] ] ] 
\draw[->,overlay] (trace) to[out=south west, in=south west, looseness=1.5] (ClP);
\end{forest} 
\z
\end{minipage}%
\begin{minipage}[t]{.5\linewidth}
\ea \label{ex:hall:61} \small \begin{forest}sn edges
[NumP, nice empty nodes [\#] [{} [Num] [AgrP [NP,name=NP] [{} [Agr] [ClP [Cl] [\textit{t}\sub{NP},name=tra]] ] ] ] ]
\draw[->,overlay] (tra) to[out=south west, in=south west, looseness=1.2] (NP);
\end{forest} 
\z
\end{minipage}\vspace{\baselineskip}

Under the flexible linearization approach too, both Cl $\succ$ N $\succ$ \# \REF{ex:hall:62} and \# $\succ$ N $\succ$ Cl \REF{ex:hall:63} can be generated:

\noindent\begin{minipage}[t]{.5\linewidth}
\ea \label{ex:hall:62} \small
%\Tree [.NumP [.{} Num [.ClP Cl NP ]  ] \# ]
\begin{forest}
	[NumP 
		[{},nice empty nodes 
			[Num] 
			[ClP 
				[Cl] 
				[NP]  
			]
		] 
		[\#]
	]
\end{forest}
\z
\end{minipage}%
\begin{minipage}[t]{.5\linewidth}
\ea \label{ex:hall:63} \small
%\Tree [.NumP \# [.{} Num [.ClP NP Cl ] ] ]
\begin{forest}
	[NumP
		[\#]
		[{},nice empty nodes
			[Num]
			[ClP
				[NP]
				[Cl]
			]
		]
	]
\end{forest}
\z
\end{minipage}\\ \is{classifier phrases|)}\largerpage

If, on the other hand, the numeral and classifier form a constituent to the exclusion of the \is{nouns}noun, as I have proposed, then we predict that the numeral and classifier should not be separated by the noun, and get the typological result for free. This is not a knockdown argument against an alternative, but it is something that would require explanation if we accept that the classifier takes N as its complement, and requires no explanation at all if Cl and \# form a constituent.

\section{Conclusion}

In this paper I have argued that a traditional account of the ``\isi{numeral blocking}'' effect in \is{numeral classifier languages}classifier languages, which appeals to the \isi{Head Movement Constraint}, should be revised in light of new empirical evidence from classifier languages with overt number and \is{definiteness marking}definiteness morphology on the classifier. I have suggested that a revised account, which can capture all of the empirical facts, leads us to the conclusion that there must be two separate syntactic structures for \#--Cl--N phrases and Cl--N phrases in these languages, and that when a numeral is present, the numeral and the classifier form a constituent to the exclusion of the noun. This conclusion is supported by typological evidence: there are no languages attested which exhibit a \is{determiner phrases}DP internal word order where the classifier and the numeral are separated by the noun, which would be mysterious under standard approaches to cross-linguistic word order variation in the DP, but which falls out naturally under the account proposed here.\is{classifiers|)}\is{numerals|)}

\section*{Acknowledgements}\largerpage
This work is developed from part of my PhD thesis, \citet{Hall2015}, and so first and foremost I thank my supervisors David Adger and Hagit Borer for their support and helpful ideas. I would like to thank Fryni Panayidou, Fangfang Niu, Panpan Yao, Annette Zhao, Christina Liu, Coppe van Urk, Tom Stanton, Klaus Abels, Peter Svenonius and Hazel Pearson for discussion of ideas (and in some cases judgments). I would also like to thank the audience at the Definiteness Across Languages conference at UNAM and El Colegio de México for their input and insightful questions, and two anonymous reviewers for very helpful and constructive criticism.

\section*{Abbreviations}
\begin{multicols}{3}
	\begin{tabbing}
		Num\hspace{.5em} \= agreement morpheme\kill
        \# \> numeral \\
		A \> \is{adjectives}adjective \\
		\textsc{agr} \> agreement \\ \> morpheme \\
		\textsc{aug} \> augmentative \\
		Cl \> classifier \\
		\textsc{def} \> definite \\
		\textsc{dim} \> diminutive \\
		N \> noun \\
		Num \> number \\
		\textsc{med} \> medial \\
		\textsc{pl} \> plural \\
		\textsc{prf} \> perfective \\
		\textsc{sg} \> singular \\
	\end{tabbing}
\end{multicols}

{\sloppy\printbibliography[heading=subbibliography,notkeyword=this]}
\end{document}
