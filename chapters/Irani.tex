\documentclass[output=paper,
modfonts
]{langscibook} 



\title{On (in)definite expressions in American Sign Language} 
\author{%
Ava Irani\affiliation{University of Pennsylvania}
}
% \chapterDOI{} %will be filled in at production

% \epigram{}

\abstract{
This paper provides an analysis of the properties and distribution of the pointing sign IX and \is{bare nominals}bare NPs in American Sign Language\il{American Sign Language}. We argue that IX followed by an NP when referring to a previously established locus is a \is{strong definite articles}strong definite article along the lines of \citet{Schwarz2009,Schwarz2013}. This claim goes contra previous analyses that draw parallels between IX and \isi{demonstratives} \citep{KoulidobrovaLilloMartin2016}. The data presented here also show that both bare NPs and IX+NPs double as definites and \isi{indefinites}, which suggests that definiteness is not semantically \is{definiteness marking}encoded in the language. We further illustrate that the interaction of the use of bare NPs and IX+NPs indicates that the specification of a locus has an impact on the interpretation of an expression as being definite or indefinite. An IX+NP cannot refer back to a bare NP in the \isi{discourse} due to the underspecification of a locus feature that characterizes bare NPs. These findings allow us to reanalyze the properties of the two kinds of nominals in the language.
}

\begin{document}

\maketitle
\section{Introduction} \is{bare noun phrases|(}\is{loci|(}\is{indefinites|(}

Definite and indefinite expressions in natural language are two widespread components of communication. Despite their ubiquitous presence, the way in which each language conveys these expressions can vary. For instance, \ili{English} indefinites are typically viewed as being introduced by the \is{indefinite articles}article \textit{a}, while \textit{the} precedes \is{definite noun phrases}definite NPs. The distinction doesn't stop there. \citet{Schwarz2009} observes that languages can further divide categories of definite expressions into those that\is{definiteness marking} encode \isi{uniqueness} and those that are \is{anaphora}anaphoric and \is{familiarity}familiar. There are also languages like \ili{Hindi}, which lack overt \isi{determiners} altogether. These type of languages have ensued a claim that their \is{bare nominals}bare nominal expressions lack a \is{determiner phrases}DP layer, as they do not encode pure indefinite readings \citep{Dayal2004}. And finally, there has been a plethora of research at least since the late 1800s on the properties of definite and indefinite expressions in \isi{discourse} (\citealt{Frege1892, Russell1905, Kamp1981, Heim1982}, i.a.). In this paper, we investigate a language that contributes to the discussion on \isi{definiteness} in varying respects, while simultaneously allowing us to examine natural language expressed via a different modality. 

American Sign Language\il{American Sign Language} (ASL) is generally claimed to be a language without overt \isi{determiners}, but it signifies the relationship between nominal expressions in more than one manner. Nominal phrases can be expressed as \is{bare noun phrases}bare NPs, or they can also be set up at locations in signing space through the use of loci. A language with more than one way of conveying nominals introduces another dimension in our goal to understand the realization of definite and indefinite \isi{reference} in language. 

Sign languages have been of interest in examining various linguistic phenomena due to their use of a different medium of communication and the visibility that signs provide to language through the use of this modality. Despite sign language research gaining momentum since \citeauthor{Stokoe1969}'s initial work in the 1960s, much work is left to be done in terms of thoroughly describing fundamental aspects of these languages. This paper aims to deepen our knowledge of the array of possible alternatives through which definite and indefinite referents can be expressed.

Although recent work has shown interest in \is{definite noun phrases}definite NPs in American Sign Language\il{American Sign Language}, there has been some disagreement in the literature in determining their status \citep{Bahanetal1995,KoulidobrovaLilloMartin2016}. Definiteness in ASL\il{American Sign Language} has been said to be expressed via the \is{definiteness marking}index marker, glossed as IX\footnote{Throughout this paper, I refer to the pointing sign, i.e. the index marker, in ASL\il{American Sign Language} as IX. When referencing indices or an index, I am referring to the formal semantic indices introduced by NPs in the \isi{discourse}.} \citep{Bahanetal1995}, despite indexing and IX having been described as performing multiple functions \citep[e.g.][]{LilloMartinKlima1990}. In the sections to follow, we discuss the nature of \isi{definiteness}, and explicate the behavior of IX in definite environments. Our proposal is compatible with the analysis of loci as being composed of \is{morphosyntax}morpho-syntactic features. Previous work has focused on loci as overt manifestations of indices \Citep{LilloMartinKlima1990,Schlenker2010}. The analysis argued for here follows that line of work, while also focusing on bare NPs introducing indices. We show that ASL\il{American Sign Language} has two types of indices: one type that is introduced by NPs specified for a locus, and the other set of indices introduced by bare NPs, which are underspecified for loci. The interaction of these systems has consequences for the definite or indefinite interpretation of expressions. Our proposal that loci are composed of features is motivated by previous work on locus re-use \citep{Kuhn2015}, but follows \citet{Schlenker2014} in adopting the \is{featural variables}featural variable view of loci, which ties in with our claims about \isi{definiteness} in the language.\is{bare noun phrases|)}

The ASL\il{American Sign Language} judgments provided in this paper are from three native signers who have been exposed to the language from birth. The consultants were presented with the target ASL\il{American Sign Language} sentences in the target language, and asked for grammaticality judgments and whether or not any particular construction was felicitous in ASL\il{American Sign Language}. They were also asked to provide the possible interpretations of each data point. Judgment reports of the data were preferred over examining data from more naturalistic sources such as corpora for two reasons: i) the circumstances in which the particular kinds of examples investigated in this paper would be found naturally occur infrequently, and ii) \is{corpus study}corpora do not allow for a study of infelicitous linguistic environments, which are crucial to the central idea of the proposal. It cannot be certain whether a construction that occurred with low frequency in a \is{corpus study}corpus is impossible in a given language or whether the opportunity to use it was simply not present.


This paper is structured as follows: first, we present an overview of previous work on definiteness in ASL\il{American Sign Language}, which focuses on the use of the index marker IX. Next, we take what has been previously discussed on IX and reanalyze it to draw parallels between IX and the two types of definite articles noted for \isi{numeral classifier languages} \citep{Jenks2015}. Even though we see that IX can be a \is{strong definite articles}strong definite article in the sense of \citet{Schwarz2009}, we find that ASL\il{American Sign Language} does not canonically \is{definiteness marking}encode definiteness lexically. Instead, there appears to be a more pragmatic force involved. IX+NP can be a definite or \is{indefinites|)}indefinite expression depending on whether it refers to another already introduced IX+NP at the same locus.  

\section{Background}\label{sec:irani:2}

The subsections below first discuss the general properties of IX when introducing loci in order to set the stage for developing an analysis of IX. We then present arguments for analyzing IX as a \is{demonstratives}demonstrative (to be rejected). This background will be beneficial in discussing the behavior of IX and \isi{indefinites} in the language. We first provide a description of some commonly known uses of loci; then, we present and jettison previous work on IX that argues for it as a demonstrative. We see that IX behaves differently when it is referring to a previously established locus, as opposed to when it is not.

\subsection{Loci}

Before diving into the details of previous analyses of IX, we must first understand its typical uses. A common use of the index marker is to make \isi{reference} to entities. When an entity is first introduced in the \isi{discourse}, the index (IX) can be used to establish a locus for the entity, which can later be referred to in the discourse \citep{KlimaBellugi1979,LilloMartinKlima1990}. By establishing a locus as the point of reference, the signer can simply point back with IX to the locus to refer back to the entity that was previously introduced. \REF{ex:irani:1} is an example of such a use of IX.\footnote{Any examples without citation are elicited from my own fieldwork with native ASL\il{American Sign Language} signers.}

\begin{exe}
\ex \label{ex:irani:1} 
IX$_\text{a}$ SARA$_\text{a}$ IX$_\text{b}$ STACY$_\text{b}$ $_\text{a}$BOTH$_\text{b}$ FRIENDS. IX$_\text{a}$ likes IX$_\text{b}$.\footnote{Signs are glossed in upper case letters as is standard in the literature. Loci are uniformly indicated with IX and a subscript both on IX itself and the nominal that follows. All cited examples have been adapted to fit this format.}\\
`Sara$_\text{i}$ and Stacy$_\text{j}$ are friends. She$_\text{i}$ likes her$_\text{j}$.'
\end{exe}

The sentence above illustrates how each locus is associated with an entity. In \REF{ex:irani:1}, locus a is associated with SARA while locus b is associated with STACY. \REF{ex:irani:2} fleshes out the paradigm of loci uses. The examples also show that loci typically refer to the entities set up at that location.

\begin{exe}
\ex \label{ex:irani:2}
\begin{xlist}
	\ex  WHEN IX$_\text{a}$ SOMEONE$_\text{a}$ LIVE WITH IX$_\text{b}$ SOMEONE$_\text{b}$, \\
	`When someone lives with someone,’ 
	
	\ex IX$_\text{a}$ LOVE IX$_\text{b}$. \\
	`the former loves the latter.’\footnote{ When the loci refer to the same signing space as below, they are infelicitous:
		
		\begin{exe} 
			\exi{(i)}
			\begin{xlist}
				\ex[\#]{\normalfont
					IX$_\text{a}$ LOVE IX$_\text{a}$ \\
					`the former loves the former.' }
				\ex[\#]{\normalfont 
					IX$_\text{b}$ LOVE IX$_\text{b}$\\
					`the latter loves the latter.'}
			\end{xlist}
		\end{exe}
		
		The reason for the unacceptability of these judgments results from our standard assumptions about binding theory \citep{ReinhartReuland1993} and from the special reflexive morphology that is required for ASL\il{American Sign Language} in these cases \citep{Meir1998}.} \citep[adapted from][13]{Schlenker2010}
	
\end{xlist}
\end{exe}

We see in \REF{ex:irani:2} that the loci retain their referents giving a meaning that can be translated as `the former' and `the latter' in \ili{English}. Moreover, in addition to entities, IX can also be used to refer to VPs.

\begin{exe} \il{American Sign Language}
\setcounter{xnumi}{2}
\ex IX$_\text{a}$ GET$_\text{a}$ JOB$_\text{a}$ DISJ/shift IX$_\text{b}$ GO$_\text{b}$ GRADUATE-SCHOOL$_\text{b}$. IX$_\text{a}$ I CAN IX$_\text{b}$ IMPOSSIBLE.\\
`Get a job or go to graduate school? The former I can do, but the latter is impossible.’ \citep[5]{KoulidobrovaLilloMartin2016}
\end{exe}

Here we see that the use of IX is not restricted to entities. Once loci are established, we can use IX as many times as necessary in the \isi{discourse} to refer back to the entity or proposition assigned at the locus.

\subsection{Previous work}

The most recent work on IX has argued for it to be a \is{demonstratives}demonstrative \citep{KoulidobrovaLilloMartin2016}, as opposed to a \is{definite articles}definite article \citep{Bahanetal1995}. Although in this paper we show evidence in favor of IX as a definite article, we first present parts of Koulidobrova \& Lillo-Martin's analysis in order to discuss patterns in the language that our analysis aims to capture.

Koulidobrova and Lillo-Martin base their argument on the assumption that definite articles are licensed by \isi{uniqueness}; however, the use of IX appears to be infelicitous in these instances.

\begin{exe}\il{American Sign Language}
\ex \label{ex:irani:4}FRANCE (\#IX$_\text{a}$) CAPITAL$_\text{a}$ WHAT. \\
`What is the capital of France?’ \citep[12]{KoulidobrovaLilloMartin2016}
\end{exe}

\begin{exe}                                                   
\ex \label{ex:irani:5} TODAY SUNDAY. DO-DO? GO CHURCH, SEE (\#IX$_\text{a}$) PRIEST$_\text{a}$.\\
`Today is Sunday. What to do? I’ll go to church, see the priest.' \\ \citep[12]{KoulidobrovaLilloMartin2016}
\end{exe}

The above examples show that IX is not licensed by uniqueness. Although there is only one capital of France, IX in \REF{ex:irani:4} is ungrammatical. Similarly, \REF{ex:irani:5} disallows IX with PRIEST even when referring to a single priest in a church. This point will become relevant in the following sections when we propose our analysis. For now, we simply note that \is{bare nominals}bare NPs are required in these \is{situational uniqueness}uniqueness situations. 

Another common use of definite articles in many languages is an \is{anaphora}anaphoric one. When IX+NP is not referring to a locus that has been previously established in signing space, it is unacceptable in anaphoric environments. 

\begin{exe} 
\ex \label{ex:irani:6} TODAY SUNDAY. DO-DO. GO CHURCH, SEE PRIEST. (\#IX$_\text{a}$) PRIEST$_\text{a}$ NICE. \\
`Today is Sunday. What to do? I’ll go to church, see the priest. The priest is nice.’ \citep[12]{KoulidobrovaLilloMartin2016}
\end{exe}

In \REF{ex:irani:6}, IX is infelicitous with the second instance of PRIEST even when its first mention is present in the \isi{discourse}. The inability of IX to appear in these cases can be explained under their account of IX being a \is{demonstratives}demonstrative, since demonstratives are not licensed without a contrastive reading or a kind of demonstration. Based on the above examples with \isi{uniqueness} and \isi{anaphoricity}, it might be tempting to label the index marker as a demonstrative; however, in further sections, we show that although there are some similarities between IX and demonstratives, there are also differences between them. In foreshadowing the analysis described in this paper, we note that IX here attempts to make \isi{reference} to a referent introduced by a \is{bare nominals}bare NP, and not a referent that was previously established at a locus. We show in the following sections that the \is{anaphora}anaphoric cases of IX are indeed felicitous when referring to a previously mentioned NP with an associated locus. Moreover, we argue that IX when referring to previously used loci is best analyzed as a strong article definite along the lines of \citet{Schwarz2009,Schwarz2013}

\section{Two types of definites in ASL\il{American Sign Language}}\label{sec:irani:3} \is{strong definite articles|(}\is{weak definite articles|(}


This section presents the two types of definite articles described by Schwarz, the strong definite article and the weak definite article, which occur cross\hyp{}linguistically. We argue here that the ASL\il{American Sign Language} index preceding an NP when referring to previously introduced loci patterns with the strong definite article. IX is also shown to behave unlike other demonstratives in the language, which is additional evidence for the strong article definite analysis. Weak article definites are argued to be expressed by \is{bare nominals}bare NPs, similar to the kind noted for \isi{numeral classifier languages} (e.g. \citealt{Jenks2015}).

\subsection{Two types of definites cross-linguistically}\is{definiteness marking}

\citet{Schwarz2009,Schwarz2013} has observed two types of definite articles that are found in a host of unrelated languages: weak definite articles which encode uniqueness and strong definite articles that encode \is{familiarity|(}familiarity and anaphoricity. Before diving into the properties of these two kinds of definite articles, let us first consider some typical uses of \isi{definiteness} in natural language. The following are some examples from \citet{Hawkins1978} modelled after \citet{Schwarz2009}:

\begin{exe}
\ex \label{ex:irani:7} \is{anaphora}Anaphoric use\\
\textit{John bought a book and a magazine. \textbf{The book} was expensive.} 

\ex \label{ex:irani:8} \is{immediate situation}Immediate situation\\
\textit{\textbf{the table}} (uttered in a room with exactly one table)

\ex \label{ex:irani:9} \is{larger situation}Larger situation \\
\textit{\textbf{the president}} (uttered in the US) 

\ex \label{ex:irani:10} \is{bridging}Bridging \citep{Clark1975}
\begin{xlist}
	\ex\label{ex:irani:10a} \textit{John bought a book. \textbf{The author} is French.}
	\ex\label{ex:irani:10b} \textit{John’s hands were freezing as he was driving down the street. \\ \textbf{The steering wheel} was bitterly cold and he had forgotten his gloves.}
\end{xlist}

\end{exe}

The examples in (\ref{ex:irani:7}--\ref{ex:irani:10}) indicate the various flavors in which definites can appear. \REF{ex:irani:7} describes a use of definites which requires referring back to an already introduced linguistic referent in the \isi{discourse}. We see in \REF{ex:irani:8} and \REF{ex:irani:9} that the definite NP does not need a linguistic antecedent; it can also refer to a salient entity in the environment. Similarly, \REF{ex:irani:10} presents examples that can refer to a relation between the definite NP and its antecedent. \REF{ex:irani:10a} illustrates a \is{product-producer relationship}product-producer \isi{bridging} relationship between the book and the author while \REF{ex:irani:10b} shows a \is{part-whole relationship}part-whole relationship between the car described by the driving event and the steering wheel. The different types of \isi{definiteness} here are relevant for the discussions to follow.

The definite expressions above appear in two forms across languages. They are divided along the lines of definite articles that denote \isi{uniqueness} or familiarity \citep{Schwarz2009,Schwarz2013}. They are coined the weak article definite and the strong article definite respectively. The following is an instance of an environment in which a weak article definite is licensed:\footnote{\ili{English} lacks the strong and weak article definite distinction; we use the examples here for purely expository purposes.}

\begin{exe}
\ex\label{ex:irani:11} Context: There is only one blackboard in the classroom and\\
the professor says:\\
\textit{I won't be using \textbf{the blackboard} today.}
\end{exe}

The definite article \textit{the} is felicitous in the example above even though a referent has not been previously introduced. The presence of a \is{uniqueness}unique blackboard in the classroom is sufficient to make the use of the definite article possible. Part-whole \isi{bridging} is another situation in which weak definite articles are licensed. 

\begin{exe} 
\ex\label{ex:irani:12} \textit{The police stopped the car because \textbf{the rear-view mirror} was broken.}
\end{exe} 

In the example above, the rear-view mirror is a part of the car, and hence, the relationship between them is said to be part-whole. These cases also encode \isi{uniqueness}, and languages that show a distinction between the two types of definite articles employ a weak article definite here. 

Strong definite articles, on the other hand, are based on familiarity; i.e. they are linked \is{anaphora}anaphorically to an antecedent. \REF{ex:irani:13} illustrates definite articles in strong environments. 

\begin{exe}
\ex \label{ex:irani:13} \textit{I bought a book. \textbf{The book} was interesting.}
\end{exe}

The definite article in \REF{ex:irani:13} is used with the second occurrence of \textit{book}. This usage is licensed by the presence of a contextually salient linguistic referent in the first sentence, which in this instance, is an \is{indefinites}indefinite expression. Languages with both types of articles use a distinct strong article definite in these familiarity cases. 

This distinction was first observed in \ili{German} (\citealt{Heinrichs1954,Hartmann1982,Schwarz2009}, i.a.), which evokes two overt forms of a \is{definiteness marking}definite marker to indicate the two types of definiteness.

\begin{exe} 
\ex \label{ex:irani:14} \ili{German} \\
\gll Der {K{\"u}hlschrank} {war} {so} {gro{\ss}}, {dass} {der} {K{\"u}rbis} {problemlos}
{\textbf{im}} \textnormal{/} \textnormal{\#}{in} {dem} {Gem{\"u}sefach} {untergebracht} {werden} {konnte}.\\
the fridge was so big that the pumpkin without-a-problem in-the$_\text{weak}$ / \phantom{\#}in the$_\text{strong}$ crisper stowed be could\\
\trans `The fridge was so big that the pumpkin could easily be stowed in the crisper.’ \citep[52]{Schwarz2009} 


\ex \label{ex:irani:15} \ili{German} \\ \gll {Das} {Theaterst{\"u}ck} {missfiel} {dem} {Kritiker} {so} {sehr}, {dass} {er} {in} {seiner} {Besprechung} {kein} {gutes} {Haar} \textnormal{\#}{am} \textnormal{/} {\textbf{an}} {\textbf{dem}} {Autor} {ließ}.\\
the play displeased the critic so much that he in his review no good hair \phantom{\#}on-the$_\text{weak}$ / on the$_\text{strong}$ author left\\
\trans `The play displeased the critic so much that he tore the author to pieces in his review.’ \citep[53]{Schwarz2009}

\end{exe} 

Although two forms of the \is{definiteness marking}definite marker are available, \ili{German} obligatorily requires the contracted version in \REF{ex:irani:14} and the uncontracted version in \REF{ex:irani:15}. These facts arise due to the type of \isi{bridging} relations; \REF{ex:irani:14} includes a \is{part-whole relationship}part-whole relation, a weak article definite environment, while \REF{ex:irani:15} includes a \is{product-producer relationship}product-producer one, a strong article definite environment. With these \ili{German} facts in place, we can now see how the distinction plays out in other languages. Akan\il{Akan}, a \ili{Niger-Congo} language, shows a strikingly similar pattern of \isi{definiteness}: 

\begin{exe} 
\ex \label{ex:irani:16} Akan\il{Akan} \\ \gll {{\'A}mstr{\'ɔ}ŋ} {ny{\'i}} {ny{\'i}mp{\'a}} {{\'a}{\`a}} {{\'o}-dz{\'i}-{\`i}} {k{\'a}n} {tu-u} {k{\'ɔ}-{\'{\={ɔ}}}} {{\'{\={ɔ}}}s{\`i}r{\`a}n} {d{\'ʊ}}.\\
Armstrong is person \textsc{rel} 3\textsc{sg}.\textsc{sbj}-eat-\textsc{pst} first uproot-\textsc{pst} go-\textsc{pst} moon top\\
\trans `Armstrong was the first person to fly to the moon.’ \citep[39]{ArkohMatthewson2013}

\ex\label{ex:irani:17} Akan\il{Akan} \\ \gll {{\'A}m{\'a}} {t{\'ʊ}-{\'{\={ʊ}}}} {{\`n}s{\'a}} {fr{\'ɛ}-{\'{\={ɛ}}}} {{\`n}n{\`o}m{\`a}hw{\'{\={ɛ}}}f{\'ʊ}} {b{\'i}} {b{\'a}-{\`a}} {{\`n}ky{\`r}{\'{\={ɛ}}}ky{\'i}r{\'{\={ɛ}}}} {n{\'a}{\'a}s{\'i}}. {M{\`i}-n-gy{\'i}} {p{\`a}p{\'a}} {\textbf{n{\'ʊ}}} {`n-dz{\'i}} {k{\`i}ts{\`i}k{\`i}ts{\'i}}. \\
Ama throw-\textsc{pst} hand call-\textsc{pst} birds.observer \textsc{ref} came-\textsc{pst} teaching.\textsc{nom} \textsc{poss}.under 1\textsc{sg}.subject-\textsc{neg}-take man \textsc{fam} \textsc{neg}-eat small.\textsc{red}\\
\trans `Ama invited a (certain) ornithologist to the seminar. I don’t trust the man in the least.' \citep[52]{ArkohMatthewson2013}
\end{exe}

Exactly like what was observed for \ili{German} strong article definite, the Akan\il{Akan} familiarity marker \textit{n{\'ʊ}} must occur in strong article definite environments. \REF{ex:irani:16}, in contrast, refers to a unique moon which does not license the familiarity marker, and unlike \ili{German}, the weak article definite is expressed as a \is{bare noun phrases|(}bare NP. \ili{Thai}, a \is{numeral classifier languages}numeral classifier language, also does not license a \is{definiteness marking}definite marker in weak article definite cases, a bare NP is used instead. The Thai example below patterns exactly like the Akan\il{Akan} case in \REF{ex:irani:16} that encodes \isi{uniqueness}. 

\begin{exe}
\ex\label{ex:irani:18} \ili{Thai} \\ \gll {r{\'o}t} {khan} {n{\'a}n} {th{\`u}uk} {tamr{\`u}at} {s{\`a}k{\`a}t} {phr{\'ɔ}ʔ} {m{\^a}j.d{\^a}j} {t{\`i}t} {satikəə} {w{\'a}j} {th{\^i}i} {th{\'a}bian} \op{\textnormal{\#}\textbf{baj}} {\textbf{n{\'a}n}}\cp\\
car \textsc{clf} that \textsc{adv}.\textsc{pst} police intercept because \textsc{neg} attach sticker keep at license \phantom{\#\op}\textsc{clf} that\\
\trans `That car was stopped by police because there was no sticker on the license.' \citep[7]{Jenks2015}
\end{exe} 

The \is{part-whole relationship}part-whole relation between the sticker and the car results in a weak article definite environment, where a bare NP is used. However, \isi{anaphoricity} licenses the obligatory presence of a \is{classifiers}classifier, which is argued to be the strong definite article in \ili{Thai} \citep{Jenks2015}. \newpage

\begin{exe} 
\ex \ili{Thai} \\ \gll {ʔɔɔl} {kh{\'i}t} {w{\^a}a} {klɔn} {b{\`o}t} {n{\'a}n} {pr{\'ɔ}ʔ} {m{\^a}ak} {m{\^ɛ}{ɛ}-w{\^a}a} {kh{\'a}w} {c{\`a}} {m{\^a}j} {ch{\^ɔ}{ɔ}p} {n{\'a}kt{\'{\={ɛ}}}{ɛ}ŋklɔɔn} \#\op{\textbf{khon}} {\textbf{n{\'a}n}}\cp\\
Paul thinks \textsc{comp} poem \textsc{clf} that melodious very although 3 \textsc{irr} \textsc{neg} like poet \phantom{\#\op}\textsc{clf} that\\
\trans `Paul thinks that poem is beautiful, though he doesn’t really like the poet.' \citep[7]{Jenks2015}
\end{exe}

Now that we have discussed the patterns to be expected of strong and weak definite articles across languages, we can examine the occurrences of the ASL\il{American Sign Language} IX in exactly these circumstances.\footnote{\ia{de Sá, Thaís Maíra Machado}\ia{de Souza, Guilherme Lourenço}\ia{Cunha Lima, Maria Luiza}\ia{Almeida Bernardino, Elidéa Lúcia}De Sá et al. (\citeyear{SaEtAlii2012}) find a \is{morphosyntax}morphosyntactic distinction between weak and strong definites in \ili{Brazilian Sign Language} (LIBRAS); however, this distinction follows \citegen{CarlsonSussman2005} line of work where weak definites or \is{indefinites}indefinite definites in instances such as \textit{John went to the store} does not have a \isi{uniqueness} requirement. We do not discuss this work any further here, but the reader is referred to \citet{CarlsonSussman2005} and \citet{CarlsonEtAlii2006} for more detail. The relevant distinction in the \isi{definiteness} domain here is that based on \is{familiarity|)}familiarity and uniqueness between what \citet{Schwarz2009} calls the strong article definite and weak article definite.} In the following section, we apply the above tests to IX in ASL\il{American Sign Language} and show that it indeed behaves like a strong definite article.

\subsection{IX as a strong definite article}\is{demonstratives|(}

Previous work \citep{KoulidobrovaLilloMartin2016} has claimed that IX is a demonstrative as it apparently fails to occur felicitously in definite environments and displays behavior typically expected of demonstratives. In this section, we address the first part of the argument and show that IX is obligatorily used in strong definite environments when referring to loci already established in the \isi{discourse}, thus indicating that IX can play the role of a strong definite article. 

It has been claimed that IX cannot occur in certain definite environments, like in \REF{ex:irani:6} repeated below as \REF{ex:irani:20}:

\begin{exe} \il{American Sign Language}
\ex \label{ex:irani:20} TODAY SUNDAY. DO-DO. GO CHURCH, SEE PRIEST. (\#IX$_\text{a}$) PRIEST$_\text{a}$ NICE. \\
`Today is Sunday. What to do? I’ll go to church, see the priest. The priest is nice.’ \citep[adapted from][12]{KoulidobrovaLilloMartin2016}
\end{exe} 

The example above suggests that IX with an NP cannot have a bare NP as its antecedent, but it is not informative regarding the overall status of IX or its interpretation in the given utterance. As stated earlier, IX can be used as a locus to establish referents in signing space. Once a locus for IX has been introduced, a different pattern emerges. This is illustrated in \REF{ex:irani:21} below: 

\begin{exe}\il{American Sign Language}
\ex \label{ex:irani:21} JOHN BUY IX$_\text{a}$ MAGAZINE$_\text{a}$, IX$_\text{b}$ BOOK$_\text{b}$. IX$_\text{b}$ BOOK$_\text{b}$ EXPENSIVE.\\
`John bought a magazine and a book. The book was expensive.'
\end{exe}

The occurrence of IX in \REF{ex:irani:21} is surprising if it were a demonstrative. For instance, \ili{English} does not permit demonstratives in these \is{anaphora}anaphoric cases.

\begin{exe}
\ex\label{ex:irani:22} \textit{John bought a book and a magazine. The\textnormal{/}\textnormal{\#}That book was expensive.}\footnote{This sentence becomes more acceptable if \textit{that} is pronounced with some exclamation. This gives the utterance an emphatic meaning. On the other hand, this emotive reading is not as available if the predicate was relatively more mundane; for instance, \textit{John bought a magazine and a book. That book was red.} is much worse than a definite article use even with an emphasis on \textit{that}.}
\end{exe} 

In addition to these examples where IX is possible in environments that only permit definite articles and not demonstratives, IX also occurs in instances of \is{product-producer relationship}product-producer \isi{bridging}. 

\begin{exe}\il{American Sign Language}
\ex \label{ex:irani:23} JOHN BUY IX$_\text{a}$ BOOK$_\text{a}$. \#(IX$_\text{a}$) AUTHOR$_\text{a}$ SELF FRENCH.\\
`John bought a book. The author is French.'\footnote{The possessive in ASL\il{American Sign Language} has a different form, the flat 5 handshape. The example here does not indicate a possessive like \textit{book's author} since the index finger with the 1 handshape is used instead, without the NP \textit{book}.}
\end{exe}

The examples in \REF{ex:irani:21} and \REF{ex:irani:23} are parallel to the \ili{German}, Akan\il{Akan}, and \ili{Thai} cases seen earlier. \is{anaphoricity}Anaphoricity licenses the occurrences of IX, which is exactly true for the strong definite article. Moreover, it is non-trivial for an IX as a demonstrative approach that the index is possible above. Although definite articles are possible in the environment in \REF{ex:irani:23}, demonstratives are not, as seen from \ili{English} in \REF{ex:irani:24}. 

\begin{exe}
\ex \label{ex:irani:24} \textit{John bought a book. The\textnormal{/}\textnormal{\#}That author is French.} 
\end{exe}

This section served to illustrate three things. First, bare NPs cannot serve as antecedents for IX.\footnote{A reviewer asks whether it is too strong a claim to argue that IX+NP cannot refer back to bare NPs. The consultants whose judgments are reported here did not allow it. However, it is possible that \is{language internal variation}some variation can be found in this area. For instance, \citet{SereikaiteToAppear} (in this volume) finds variation in the \is{product-producer relationship}product-producer \isi{bridging} cases in \ili{Lithuanian}.} Second, IX is possible in definite environments when referring back to previously established loci and patterns with the strong definite article. And third, IX and can appear in environments where demonstratives are infelicitous. The following section elaborates on this last point. 

\subsection{IX versus demonstratives}

We have seen evidence for IX as a strong definite article, but in this section, we also see arguments for IX behaving distinctly from demonstratives. ASL\il{American Sign Language} is already known to have a demonstrative THAT in the language, which is signed with a y-handshape.\footnote{The sign THAT is also used as a relative pronoun, but other than bearing the same phonological realization as the demonstrative, it is unclear that the two usages show any syntactic or semantic overlap.} Therefore, an easy test for the IX as a demonstrative hypothesis is to place IX in the same environment as THAT and observe their behavior. This sign was not examined by \citet{KoulidobrovaLilloMartin2016} in their investigation of IX. 

Although demonstratives and definite articles both contain presuppositions of \isi{familiarity} and \isi{uniqueness}, demonstratives carry with them an accompanying demonstration \citep{Roberts2002}. It is a known property of demonstratives that they enforce a contrastive reading. This property renders sentences like the following infelicitous with \textit{that}:

\begin{exe}
\ex\label{ex:irani:25} \textit{A car drove by. The\textnormal{/}\textnormal{\#}That horn was honking loudly.} \citep[70]{Wolter2006}

\ex\label{ex:irani:26} \textit{I met a doctor and a banker. The\textnormal{/}\textnormal{\#}That banker was full of himself}.
\end{exe} 

The sentences above are infelicitous with the demonstrative due to the lack of a contrastive reading. On the other hand, we have already seen that a sentence like \REF{ex:irani:26} in ASL\il{American Sign Language} permits IX, which would be surprising if IX is a demonstrative that requires a contrastive interpretation. The example in \REF{ex:irani:21} is repeated below in \REF{ex:irani:27}. 

\begin{exe}\il{American Sign Language}
\ex \label{ex:irani:27} JOHN BUY IX$_\text{a}$ MAGAZINE$_\text{a}$, IX$_\text{b}$ BOOK$_\text{b}$. IX$_\text{b}$ BOOK$_\text{b}$ EXPENSIVE.\\
`John bought a magazine and a book. The book was expensive.'
\end{exe}

\noindent The counterpart of the sentence with the demonstrative THAT; however, is infelicitous. 

\begin{exe}\il{American Sign Language}
\ex \label{ex:irani:28} JOHN BUY IX$_\text{a}$ MAGAZINE$_\text{a}$, IX$_\text{b}$ BOOK$_\text{b}$. \#THAT$_\text{b}$ BOOK$_\text{b}$ EXPENSIVE.\\
`John bought a magazine and a book. The book was expensive.'
\end{exe}

\noindent Even when THAT is signed aligned with the locus associated with the book, the demonstrative in this \is{anaphora}anaphoric situation is unavailable. Another situation where demonstratives and definite articles can be distinguished is when referring to a contextually salient referent out of the blue. Firstly, we note that it is not essential that demonstratives require physical pointing to the referent, as it is neither a sufficient nor a necessary condition.

\begin{exe}
\ex Context: Policeman, pointing in the direction of a man running through a crowd:\\ \textit{Stop that man!} \citep[121]{Roberts2002}
\end{exe}

\noindent The example above from \citet{Roberts2002} describes a situation in which a policeman is chasing a man through a crowd of several people. It is not obvious who he's pointing to, but the context makes the referent clear. A deictic gesture is also unnecessary in making out the \is{discourse}discourse referent. Roberts describes a situation in which two friends are sitting in a coffee shop when a man enters and begins to noisily harass the employee behind the counter. In this case, without pointing and drawing attention to herself, one friend can say to the other:

\begin{exe}
\ex \textit{That guy is really obnoxious.} \citep[121]{Roberts2002} 
\end{exe} 

Such an example can be tested in ASL\il{American Sign Language} as well. We expect demonstratives to be possible in this environment, but definite articles are predicted to be infelicitous. 

\begin{exe}\il{American Sign Language}
\ex \label{ex:irani:31} [out of the blue] (\#IX-neu) MAN ANNOYING.\\
`That man's annoying.'

\ex \label{ex:irani:32} [out of the blue] THAT-neu MAN ANNOYING.\\
`That man's annoying.'
\end{exe}

\REF{ex:irani:31} shows that IX pointing to a neutral location\footnote{We do not make any claims in regards to IX in neutral position and its featural specifications. We are simply pointing out here that IX-neu MAN is prohibited in this case is due to the presence of a salient individual.} cannot be used to refer to the contextually salient individual. We show this example with a neutral point in order to avoid any confound of assigning an arbitrary locus to an individual present in the environment; under normal circumstances, we would use a deictic locus in these cases. Even with a neutral point before MAN, the utterance is infelicitous. However, the same statement becomes acceptable with THAT or even as a bare NP. The use of the bare NP in \REF{ex:irani:31} becomes relevant in the discussion on weak definite articles; for the present argument, we are only concerned with the contrast between \REF{ex:irani:31} and \REF{ex:irani:32}. The situation described here is perfectly acceptable with the demonstrative THAT. We see that the two signs THAT and IX pattern differently, and furthermore, THAT in ASL\il{American Sign Language} behaves just like \textit{that} in \ili{English}.   

The instances of ASL\il{American Sign Language} THAT, IX, and the \ili{English} \textit{that} presented in this section force us to conclude that IX does not have much in common with the \ili{English} \textit{that}, and moreover, it does not align with our theory of demonstratives generally. In contrast, we find that THAT in ASL\il{American Sign Language} and \textit{that} in \ili{English} behave alike in the situations presented in this section.

Up to this point, we have seen arguments for a strong definite article in ASL\il{American Sign Language}. Its counterpart, the weak definite article, also exists in the language. The next section argues that bare NPs can play the role of weak article definites.\is{strong definite articles|)} 

\subsection{Bare NPs as weak definite articles}

In the previous two sections, we have seen evidence that the ASL\il{American Sign Language} index IX behaves like the strong definite article as opposed to a demonstrative. Here, we discuss evidence for the presence of weak article definites in the language. 

If we recall the examples from \ili{German}, \ili{Thai}, and Akan\il{Akan}, we know that weak definite articles can appear across languages in two varieties: overtly or as a bare NP. We have already argued that IX in ASL\il{American Sign Language} is a strong definite article, and by examining bare NPs, we find that they behave like weak definite articles similar to those in \ili{Thai} and Akan\il{Akan}. \REF{ex:irani:33} and \REF{ex:irani:34} illustrate this.  

\begin{exe}\il{American Sign Language}
\ex \label{ex:irani:33}FRANCE (\#IX$_\text{a}$) CAPITAL$_\text{a}$ WHAT \\
`What is the capital of France?’ \citep{KoulidobrovaLilloMartin2016}

\ex \label{ex:irani:34} TODAY SUNDAY. DO-DO? GO CHURCH, SEE (\#IX$_\text{a}$) PRIEST$_\text{a}$\\
`Today is Sunday. What to do? I’ll go to church, see the priest.' \\ \citep{KoulidobrovaLilloMartin2016}
\end{exe}

The sentences in \REF{ex:irani:4} and \REF{ex:irani:5} from \citet{KoulidobrovaLilloMartin2016} are repeated above in \REF{ex:irani:33} and \REF{ex:irani:34} respectively. These examples were aimed at indicating the incompatibility of IX with \is{uniqueness}unique NPs. In \REF{ex:irani:33}, IX is impossible even though there is only one capital of the country. Similarly, in \REF{ex:irani:34}, using IX with the NP PRIEST is unacceptable even when there is a unique priest at the church. The infelicity of these cases is expected if weak article definites have to be expressed by bare NPs.\footnote{In \sectref{sec:irani:5}, we present examples of where \isi{uniqueness} restrictions on IX are not as strong. These are cases with two unique referents in the \isi{discourse}. Such examples warrant further investigation, but they do not detract from the argument here, which indicates that under general circumstances, unique referents are unable to be associated with a locus. Moreover, the reason behind the prohibition of IX in these cases is still not an artifact of IX as a \is{demonstratives|)}demonstrative.} \is{weak definite articles|)}

\section{Reanalyzing IX}

Now that we have established IX as a strong article definite when it refers to previously established loci and bare NPs as weak definite articles, we can proceed to lay out the precise nature of \isi{definiteness} in ASL\il{American Sign Language} in relation to IX, loci, and bare NPs.\footnote{Some sign languages have been noted to express \is{definiteness marking}definiteness via non-manual markers; for example, a wrinkled nose co-articulated with an NP in \ili{Russian Sign Language} and in the \ili{Sign Language of the Netherlands} signals a known \is{discourse}discourse referent \citep{Kimmelman2015}. The use of non-manual markers to convey definiteness has yet to be observed in ASL\il{American Sign Language}; however, future work would benefit from examining the potential role of non-manual markers or the location of the referent in signing space. The latter has been noted to play a role in \ili{Catalan Sign Language} \citep{Barbera2014}. Thanks are due to an anonymous reviewer for bringing cross-linguistic work on \is{definiteness marking}definiteness and non-manual marking to my attention.} The present analysis also leads us to the question of why bare NPs cannot serve as antecedents to ASL\il{American Sign Language} strong definite articles. We address that question in this section. 

The key difference between the weak and strong definite articles manifests itself in the presence or absence of an extra individual argument and identity relation. This difference is encoded in the definitions of the weak and strong definite articles below, as formulated by \citet{Schwarz2009}.

\begin{exe} 
\ex Weak definite article\\ \is{weak definite articles}
\(\lambda\)s$_r$\(\lambda\)P$_{<e,st>}$:\(\exists\)!xP(x)(s$_r$).\(\iota\)x.P(x)(s$_r$) \citep[148]{Schwarz2009}
\end{exe} 

\begin{exe}
\ex Strong definite article \is{strong definite articles}

\begin{xlist}
	\ex \(\lambda\)s$_r$\(\lambda\)P.\(\lambda\)y:\(\exists\)!x(P(x)(s$_r$) \& x = y).\(\iota\)x[P(x)(s$_r$) \& x = y]
	\ex \label{ex:irani:36b} {[}$_{DP}$1 [[the s$_r$] \emph{NP}]{]}
	\ex ${[}$[\ref{ex:irani:36b}]${]}$$^{g}$ = \(\iota\)x.NP(x)(s$_r$) \& x = g(1) \citep[260]{Schwarz2009}
\end{xlist}

\end{exe}

In the formulations above, \textit{s$_r$} represents resource situation pronouns in DPs, which is essentially a variant of a standard indexed variable \citep[95]{Schwarz2009}. The difference between the two types of articles is that the weak article definite does not contain an individual argument. The strong definite article, on the other hand, is made up of the weak definite article, which expresses \isi{situational uniqueness}, and has a phonologically null pronominal element -- the \is{anaphora}anaphoric index argument -- built into it \citep[258]{Schwarz2009}. We adopt the above representations of the weak and strong definite articles for IX+NPs and bare NPs, as their properties align with the aforementioned distinctions. As per the discussion, weak article definites do not generally introduce an index, but under our proposal, we will see that both bare NPs and IX+NPs can introduce indices. The data presented in this paper do not allow us to make a claim regarding the introduction of indices for weak article definites more generally, although it is possible that they exhibit different behaviors when the conditions for the weak article definite is met. 


Bare NPs in ASL\il{American Sign Language}, moreover, are ambiguous between definites and \is{indefinites|(}indefinites. Similar to bare NPs, IX+NPs in ASL\il{American Sign Language} double as indefinite and definite expressions as well. These facts lead us back to wonder why indefinite bare NPs cannot serve as antecedents for the \is{strong definite articles}strong definite article. In order to answer this question, we first show in the consequent sections that both bare NPs and IX+NP have a bona fide indefinite reading. Then we discuss the properties of the strong article definite that require an antecedent which has been introduced through a locus. Bare NPs cannot serve as antecedents to IX+NPs precisely because they are not specified at a locus. We propose that bare NPs are underspecified for a locus feature, which creates a discordance between the two nominal types in the \isi{discourse} due to the types of indices they introduce. \sectref{sec:irani:4.2} provides evidence and expands on this idea. Support for our argument that IX is composed of features comes from work showing that features on loci can be uninterpreted under \isi{focus} \citep{Kuhn2015}, which we discuss in \sectref{sec:irani:4.3} In order to account for all the patterns of the language we see in this paper, we follow \citet{Schlenker2014} in adopting a \is{featural variables}featural variable analysis of loci. 

\subsection{ASL indefinites}\il{American Sign Language}

We provide evidence below for both bare NPs and IX+NPs as also having true indefinite readings. ASL\il{American Sign Language} is a \is{determiners}determinerless language, and it has been argued that these languages lack a true indefinite interpretation \citep{Dayal2004}. \ili{Hindi} has been shown to fit this description; however, we illustrate that ASL\il{American Sign Language} and \ili{Hindi} diverge in this respect.\footnote{ If true, this claim would be in contrast to \citet{Dayal2004}, who argues that bare NP languages without \isi{determiners} do not have a pure indefinite reading. }

Bare NPs in ASL\il{American Sign Language} are ambiguous between definites and indefinites. We have already seen definite readings of ASL\il{American Sign Language} bare NPs, and we can apply standard diagnostics to test their behavior as indefinites. In this section, we take a look at \isi{narrow scope} indefinite readings of bare NPs in subject position to illustrate that bare NPs can have a true indefinite reading. Moreover, IX+NPs can also have such an interpretation, a fact illustrated through their use in donkey sentences. 

\ili{Hindi}, a language without overt determiners, has been argued by \citet{Dayal2004} as having bare NPs that lack a pure indefinite reading. Let us consider the sentence\largerpage below:

\begin{exe} 
\ex[]{\label{ex:irani:37} \ili{Hindi} \\ \gll \llap{~\#~}Charon taraf baccha khel rahaa thaa.\\
	four	    ways child       play \textsc{prog.sg} be.\textsc{sg.pst}\\
	\trans `A (different) child was playing everywhere.’ \citep[adapted from][406]{Dayal2004}} 
\end{exe} 

\textit{Baccha} `child' in the sentence in \REF{ex:irani:37} above cannot have the interpretation where a different child is playing everywhere; the only reading available is that of a single child. This fact does not hold in ASL\il{American Sign Language}. The following example illustrates that ASL\il{American Sign Language} and \ili{Hindi} must be analyzed differently, as bare NPs in subject position in the language can be interpreted with a \isi{narrow scope} indefinite reading. 

\begin{exe} \il{American Sign Language}
\ex \label{ex:irani:38} CHILD PLAY EVERYWHERE.\\
`Same child/a different child was playing everywhere.’
\end{exe} 

The example in \REF{ex:irani:38} can either have the reading where only one child is playing everywhere, or the reading where different children are present. If a narrow scope indefinite reading were impossible, then we would only expect to get the former interpretation. Here we see that ASL\il{American Sign Language} bare NPs have passed this test for indefinite readings. The example in \REF{ex:irani:38} is similar to \ili{English} \REF{ex:irani:39}, a language with overt determiners, in this respect. 

\begin{exe}
\ex \label{ex:irani:39} \textit{A child was playing everywhere.}
\end{exe}

As the \ili{English} example illustrates, a narrow scope indefinite reading is possible with \textit{a child}, where both interpretations of a single child or different children are available. ASL\il{American Sign Language} and \ili{English} do not appear to differ in this regard, and it seems that bare NPs in ASL\il{American Sign Language} pattern with \ili{English} indefinites. 

Another test of a true indefinite is its use in donkey sentences. We know from decades of research on the topic (\citealt{Geach1962,Lewis1975}, i.a.) that indefinites allow for \isi{donkey anaphora}. \ili{English} indefinites show this property. 

\begin{exe}
\ex \label{ex:irani:40} \textit{Every time I meet a student, me and him get into a fight.}
\end{exe}

In \REF{ex:irani:40}, the encounters can refer to a different student each time, which is expected for true indefinites. The facts for IX+NPs in ASL\il{American Sign Language} are the same as in \ili{English}, again indicating that they are ambiguous between definites and indefinites. In the example below, we have set up a locus for STUDENT and the pronominal forms in the utterance make use of \isi{reference} to both, the space of the person uttering the sentence, and the locus for STUDENT.

\begin{exe}\il{American Sign Language}
\ex \label{ex:irani:41} EVERY-TIME I MEET IX$_\text{a}$ STUDENT$_\text{a}$, ME-IX$_\text{a}$ FIGHT.\\
`Every time I meet a student, me and him get into a fight.'
\end{exe}

Like the \ili{English} example, the sentence in \REF{ex:irani:41} can also refer to different encounters with students, which illustrates that donkey readings are possible with IX+NPs. Given the facts of bare NPs and IX+NPs in this section, we conclude that both bare NPs and IX+NPs have a true indefinite reading. We can now build on this fact and encapsulate it within our proposal. 

\subsection{The basic proposal}\label{sec:irani:4.2} \is{File Change Semantics|(}

In this section, we follow the file card semantics of \citet{Heim1983} to capture the patterns in the language observed earlier. Under this theory, information within an utterance can be metaphorically viewed as being stored in files. Each \isi{logical form} of a sentence is also assigned a file change potential, which is a function from the file that obtains prior to an utterance to the file obtained after the utterance. The truth of the file is determined by the sequence of individuals that satisfy the file. This sequence is a function from a subset of natural numbers N into the domain of all individuals, for instance, for the pair of members a$_{1}$ and a$_{2}$, <a$_{1}$, a$_{2}$> is the function which maps 1 to a$_{1}$ and 2 to a$_{2}$ \cite[228]{Heim1983}. 


Definites and indefinites in natural language, under this system, can be understood through the \isi{Novelty/Familiarity Condition}, as given in \REF{ex:irani:42}, where definites are \is{familiarity|(}familiar referents and indefinites are novel.

\begin{exe}
\ex \label{ex:irani:42} \textbf{The Novelty/Familiarity Condition}

``Let F be a file, p an atomic proposition. Then p is appropriate with respect to F only if, for every \is{noun phrases}noun phrase NP$_\text{i}$ with index i that p contains:\\
\indent 	If NP$_\text{i}$ is \is{definites}definite, then i \(\in\) Dom(F), and \\
\indent 	If NP$_\text{i}$ is \is{indefinites}indefinite, then i \(\notin\) Dom(F)'' \cite[233]{Heim1983} 
\end{exe}

The Novelty/Familiarity Condition simply states that definites are familiar referents whose index is already in the domain of the file F, whereas indefinites are novel referents whose index is not in the domain of the file. Taking this basic notion of definites and \is{indefinites|)}indefinites into account, we can now proceed to analyze the ASL\il{American Sign Language} patterns discussed throughout. Our basic proposal is this: IX introduces a locus, which can be viewed as the introduction of a locus feature on the NP to follow. Bare NPs lack such a feature as they are not signed at a locus, i.e., a particular point in signing space. Only bare NPs can refer back to bare NPs, while only NPs specified for a locus feature can refer back to loci because bare NPs are unspecified for them. What the specification of a locus feature in essence translates to is that bare NPs and IX+NPs introduce different types of indices: one specified for loci and the other which is underspecified for a locus feature. These distinct indices would force an IX+NP to be interpreted as a new referent even if there is a bare NP that could potentially serve as an antecedent.\footnote{The data could potentially be accounted for by proposing that bare NPs do not introduce an index at all, although then one would have to propose an additional mechanism by which bare NPs can refer to each other as in (\ref{ex:irani:43}). More data along these lines may allow us to distinguish between the two alternatives.} 

Let us illustrate this idea with some examples:\footnote{I leave out the loci for JOHN in (\ref{ex:irani:43}) and (\ref{ex:irani:45}) for expository purposes. This does not affect the readings of the sentences in any relevant way.} 

\begin{exe}\il{American Sign Language}
\ex \label{ex:irani:43} a) JOHN BOUGHT BOOK. b) BOOK INTERESTING. \\
`John bought a book. The book was interesting.'

\ex \label{ex:irani:44} a) IX$_\text{a}$ JOHN$_\text{a}$ BOUGHT IX$_\text{b}$ BOOK$_\text{b}$. b) IX$_\text{b}$ BOOK$_\text{b}$ INTERESTING. \\
`John bought a book. The book was interesting.'

\ex \label{ex:irani:45} a) JOHN BOUGHT BOOK. b) \#IX$_\text{b}$ BOOK$_\text{b}$ INTERESTING. \\
`John bought a book$_\text{i}$. \#A book$_\text{i}$ was interesting.'  
\end{exe}

We take each of the above examples in turn and explain how they are interpreted in accordance with our analysis. In \REF{ex:irani:43}, neither of the bare NPs BOOK are specified for a locus feature; therefore, the second instance of BOOK does not introduce an \is{indefinites|(}indefinite and it is interpreted as familiar. In \REF{ex:irani:44}, the first instance of BOOK with a locus feature introduces an indefinite index. The second instance of book, however, is signed at the same locus, referring back to the same index. Instead, BOOK in (\ref{ex:irani:44}b) is necessarily interpreted as familiar. Finally, the example in \REF{ex:irani:45} is key in understanding the proposed analysis. BOOK in (\ref{ex:irani:45}b) is specified for a locus feature, while the bare NP BOOK is not. In that case, the second instance of book is interpreted as an \is{indefinites}indefinite, and the sentence is infelicitous under the reading that the same book is under discussion.\footnote{The sentence is perfectly acceptable with the reading that there is a novel book that is interesting; i.e. when the two books do not corefer.  
	
	\begin{exe} \il{American Sign Language}
	\exi{(i)}{\normalfont JOHN BUY BOOK. IX$_\text{a}$ BOOK$_\text{a}$ INTERESTING.\\
		`John bought a book$_\text{i}$. A book$_\text{j}$ is interesting.'}
	\end{exe} 

The extent to which the above sentence is infelicitous in ASL\il{American Sign Language} may be compared to the \ili{English} translation provided.} 

Earlier in the paper, we showed that bare NPs and IX+NPs are ambiguous between definite and indefinite readings. Therefore, as per the \isi{Novelty/Familiarity Condition}, both bare NPs and IX+NPs can either introduce an indefinite or refer to a familiar expression. This rule for both bare NPs and IX+NPs, given a file F, the domain of F Dom(F), and the set of sequences that satisfy F Sat(F), and an index i, is summarized below: 

\begin{exe}
\ex \label{ex:irani:46} If i \(\in\) Dom(F), then Sat(F') = Sat(F+b$_\text{i}$ \(\in\) Ext(``NP''));\\ else, if i is \(\notin\) Dom(F), then Dom(F') = Dom(F) \(\cup\) \{i\}. \par  
\end{exe}

The analysis we have proposed here follows from the building blocks of Heim's system; every NP in \isi{logical form} carries an index, and the only distinction between the two types of nominal expressions in ASL\il{American Sign Language} is their association with a locus. Let us now see how the mechanisms of this analysis emerge under the workings of file card semantics. There are two basic requirements for \is{indefinites|)}indefinite expressions as stated in \REF{ex:irani:47}: i) the index must not be in the domain of the file (Dom(F)), and ii) the satisfaction set of the file (Sat(F)) plus an atomic formula p must not be empty. 

\begin{exe}
\ex \label{ex:irani:47} i \(\notin\) Dom(F) \& Sat(F+p) \(\neq\) \(\emptyset\)
\end{exe}

In ASL\il{American Sign Language}, when IX+NP is introduced, we get a new file card if the index is not in Dom(F). 

When introducing an \is{indefinites}indefinite, the sequences in Sat(F+p) have to be longer than those in Sat(F). With these principles in place, we can work through the examples in \REF{ex:irani:43}-\REF{ex:irani:45}. Below, we have the interpretation for \REF{ex:irani:43}.

\begin{exe}

\ex Sat(F$_0$+(\ref{ex:irani:43}a)) =
= Sat((F$_0$+[$_{\text{NP}_{1}}$ John] + [$_{\text{NP}_{2}}$ a book] + [e$_1$ bought e$_2$])\par 
= \{<b$_1$,b$_2$>: b$_1$ \(\in\) Ext (``John''), b$_2$ \(\in\) Ext (``book'') and <b$_1$,b$_2$> \(\in\) Ext (``bought'')\} \par 

\end{exe}

Here, we have thus far simply introduced extensions of sequences that were not in Dom(F), but whose sub-sequences satisfy F and p, by allowing for cases where F+p has a larger domain than F. We have not yet had to deal with cases with a familiar referent. (\ref{ex:irani:43}b) is such a case, and we account for it as shown below: 

\begin{exe}

\ex Dom(F$_1$) = \{1,2\} \par 
Sat(F$_2$) = \{<b$_1$,b$_2$>: b$_2$ \(\in\) Sat(F$_1$) and b$_2$ \(\in\) Ext (``interesting'')\} \par 

\end{exe}

We already have the two file cards for 1 and 2 at this point. When (\ref{ex:irani:43}b) is uttered, the file cards are updated accordingly. No new index is introduced as both instances of BOOK in this case are bare NPs unspecified for a locus feature, and BOOK in (\ref{ex:irani:43}b) is understood as a familiar referent. Both instances of BOOK introduce the same index; thus, \REF{ex:irani:43} can be summarized as follows: 

\begin{exe}
\ex John(x) \& book(y) \& bought(x,y) \& interesting(y)
\end{exe}


\REF{ex:irani:44} is interpreted in the same way as \REF{ex:irani:43}, even though both instances of BOOK here are specified for a locus feature. (\ref{ex:irani:44}a) is shown below:

\begin{exe}
  
\ex Sat(F$_0$+(\ref{ex:irani:44}a)) = \par 
= Sat((F$_0$+[$_{\text{NP}_{1}}$ John] + [$_{\text{NP}_{2}}$ a book] + [e$_1$ bought e$_2$])\par 
= \{<b$_1$,b$_2$>: b$_1$ \(\in\) Ext (``John''), b$_2$ \(\in\) Ext (``book'') and <b$_1$,b$_2$> \(\in\) Ext (``bought'')\} \par 

\end{exe}


As shown above, the interpretation for (\ref{ex:irani:44}a) is not different from (\ref{ex:irani:43}a). Similarly, we do not introduce a novel index when the second instance of BOOK is uttered in (\ref{ex:irani:44}b), as it is also specified for the same locus feature.  

\begin{exe}

\ex Dom(F$_1$) = \{1,2\} \par 
Sat(F$_2$) = \{<b$_1$,b$_2$>: b$_2$ \(\in\) Sat(F$_1$) and b$_2$ \(\in\) Ext (``interesting'')\} \par 

\end{exe}

Therefore, in sum, for \REF{ex:irani:44}) we also get: 

\begin{exe}

\ex John(x) \& book(y) \& bought(x,y) \& interesting(y)

\end{exe}

The interpretation for \REF{ex:irani:43} and \REF{ex:irani:44} does not work out differently as the second instance of BOOK in both cases is familiar, as both NPs for BOOK are either bare NPs or IX+NPs. We get a different result when the first NP for BOOK is a bare NP and the second NP has a locus feature. 

For \REF{ex:irani:45}, part (a), which contains novel expressions, is the same as the interpretations for \REF{ex:irani:43} and \REF{ex:irani:44} as no decision about the familiarity or novelty of the referent has to be made. 

\begin{exe}

\ex Sat(F$_0$+(\ref{ex:irani:45}a)) = \par 
= Sat((F$_0$+[$_{\text{NP}_{1}}$ John] + [$_{\text{NP}_{2}}$ a book] + [e$_1$ bought e$_2$])\par 
= \{<b$_1$,b$_2$>: b$_1$ \(\in\) Ext (``John''), b$_2$ \(\in\) Ext (``book'') and <b$_1$,b$_2$> \(\in\) Ext (``bought'')\} \par 

\end{exe} 

(\ref{ex:irani:45}b), however, is different. The first instance of BOOK in this case was a bare NP, one not specified for a locus feature. On the other hand, BOOK in (\ref{ex:irani:45}b) is specified for a locus feature. Since the index for the bare NP BOOK was underspecified for a locus feature, it cannot be the same one as IX+NP BOOK, and hence, a distinct index for the second instance is introduced. 

\begin{exe}

\ex Dom(F$_1$) = \{1,2,3\}\par 

Sat(F$_1$+(\ref{ex:irani:45}b)) = \par 
= Sat((F$_1$ + [$_{\text{NP}_{3}}$ a book]) + [e$_3$ interesting]) \par 
= \{<b$_1$,b$_2$,b$_3$>: b$_3$ \(\in\) Ext (``book'') and b$_3$ \(\in\) Ext (``interesting'')\} \par 

\end{exe}

Thus, for (\ref{ex:irani:45}), we get the following interpretation, which is unlike (\ref{ex:irani:43}) and (\ref{ex:irani:44}): 

\begin{exe}

\ex John(x) \& book(y) \& bought(x,y) \& book(z) \& interesting(z) \par 

\end{exe}

We see above that the second instance of BOOK is interpreted as an \is{indefinites}indefinite, which renders the pair of sentences infelicitous under the reading where the two books refer to the same entity. The BOOK in (b) cannot refer to the one in (a) as (\ref{ex:irani:45}a) is unspecified for a locus feature. 

Now that we have seen how the analysis plays out, we need to explicate the relationship between loci, bare NPs and indices. We have already stated that both IX+NPs and bare NPs introduce indices, but what kind of indices does a locus and a bare NP introduce? From the analysis we've laid out so far, we are led to say that bare NPs are underspecfied for a locus as the language allows for a locus feature to be associated with NPs. This locus feature is specified according to the index they take. The following section elaborates further on the final point, but for now we can formalize the two types of indices as those underspecified for a locus feature, and those specified for it. Bare NPs take the former kind, which we can denote using Greek letters, \(\alpha\), \(\beta\), etc. IX+NPs take indices of the type a,b,c, etc., the kind which is specified for a locus feature. Thus, for the sentences in \REF{ex:irani:43}-\REF{ex:irani:45}, we get a particular kind of index depending on whether the NP is associated with a locus or is a bare NP.\footnote{The underspecification of indices for a feature is not \is{uniqueness}unique to ASL\il{American Sign Language}. \ili{Persian} \is{pseudo-incorporation}pseudo-incorporated nominals are argued to display a similar property \citep{KrifkaModarresi2016}, where the \is{discourse}discourse referents introduced by these NPs are underspecified for \isi{number}. Covert pronouns are also said to lack number features, while overt ones are marked for number. Krifka \& Modarresi show that overt pronouns require number marked NPs, whereas covert pronouns do not. This analysis is parallel to what we propose here for ASL\il{American Sign Language} NPs with a locus feature.} With this updated proposal, let us revisit the example in \REF{ex:irani:43}, and illustrate its updated representation under this system. Below we have (\ref{ex:irani:43}a):  

\begin{exe}

\ex Sat(F$_0$+(\ref{ex:irani:43}a)) = \par 
= Sat(F$_0$+[$_{\text{NP}_\alpha}$ John ] + [$_{\text{NP}_\beta}$ a book] + [e$_{\alpha}$ bought e$_{\beta}$]) \par 
= \{< b$_{\alpha}$,b$_{\beta}$ >: b$_{\alpha}$ \(\in\) Ext (``John''), b$_{\beta}$ \(\in\) Ext (``book'') and <b$_{\alpha}$,b$_{\beta}$> \(\in\) Ext (``bought'')\} \par 

\end{exe}

We immediately notice in the interpretation of (\ref{ex:irani:43}a) that the numerical indices are now represented by \(\alpha\) and \(\beta\) to illustrate the underspecification of the locus feature. The type of indices we are dealing with is now transparent. Since (\ref{ex:irani:43}b) also makes use of bare NPs, no new file card is introduced and the utterance is interpreted as familiar. This is shown below: \par 

\begin{exe}

\ex Dom(F$_1$) = {$\alpha$,$\beta$} \par 
Sat(F$_2$) = \{<b$_{\alpha}$, b$_{\beta}$>: b$_{\beta}$ \(\in\) Sat(F$_1$) and b$_{\beta}$ \(\in\) Ext (``interesting'')\} \par 

\end{exe}

Thus, for \REF{ex:irani:43} we get:

\begin{exe}

\ex John(x) \& book(y) \& bought(x,y) \& interesting(y)

\end{exe}

Now that we have seen bare NPs introducing indices of the type \(\alpha\) and \(\beta\), we can account for \REF{ex:irani:44} in a similar manner by evoking indices of the type a and b, which are specified for a locus feature.  The interpretation for (\ref{ex:irani:44}a) is provided below.

\begin{exe}

\ex Sat(F$_0$+(\ref{ex:irani:44})) = \par 
= Sat(F$_0$+[$_{\text{NP}_\text{a}}$ John] + [$_{\text{NP}_\text{b}}$ a book] + [e$_\text{a}$ bought e$_\text{b}$])\par 
= \{<b$_\text{a}$,b$_\text{b}$>: b$_1$ \(\in\) Ext (``John''), b$_\text{b}$ \(\in\) Ext (``book'') and <b$_\text{a}$,b$_\text{b}$> \(\in\) Ext \\ (``bought'')\}  

\end{exe}

\REF{ex:irani:44} is understood in the same way as \REF{ex:irani:43}, except with the use of NPs that are associated with a locus. BOOK in (\ref{ex:irani:44}b) is also interpreted as a definite expression. 

\begin{exe}

\ex Dom(F$_1$) = \{a,b\} \par 
Sat(F$_2$) = \{<b$_\text{a}$,b$_\text{b}$>: b$_\text{b}$ \(\in\) Sat(F$_1$) and b$_\text{b}$ \(\in\) Ext (``interesting'')\} \par 

\end{exe}

In sum, for \REF{ex:irani:44} we get:

\begin{exe}

\ex John(x) \& book(y) \& bought(x,y) \& interesting(y)

\end{exe}

We now see an interaction between the two systems in \REF{ex:irani:45}, which ultimately does not result in the desired interpretation. The bare NPs in (\ref{ex:irani:45}a) introduce an index unspecified for loci, but IX+NP in (\ref{ex:irani:45}b) introduces an index with a locus feature. First, we see that the interpretation of (\ref{ex:irani:45}a), which contains novel expressions, simply introduces \isi{indefinites} like in (\ref{ex:irani:43}a). 

\begin{exe}

\ex Sat(F$_0$+(\ref{ex:irani:45}a)) = \par 
= Sat(F$_0$+[$_{\text{NP}_{\alpha}}$ John] + [$_{\text{NP}_\beta}$ a book] + [e$_{\alpha}$ bought e$_{\beta}$])\par 
= \{<b$_{\alpha}$,b$_{\beta}$>: b$_{\alpha}$ \(\in\) Ext (``John''), b$_{\beta}$ \(\in\) Ext (``book'') and <b$_{\alpha}$,b$_{\beta}$> \(\in\) Ext \\ (``bought'')\}  

\end{exe}

(\ref{ex:irani:45}b), in contrast, is different. Here we do not get a \is{familiarity|)}familiar reading of BOOK as this NP is associated with a locus. It introduces an index X, which is not an index of a type underspecified for a locus feature. Thus, it introduces a new file card and the second instance of BOOK is understood as an \is{indefinites}indefinite expression. \is{File Change Semantics|)}

\begin{exe}

\ex Dom(F$_1$) = \{\(\alpha\),\(\beta\),a\}\par 

Sat(F$_1$+(\ref{ex:irani:45}b)) = \par 
= Sat((F$_1$ + [$_{\text{NP}_{a}}$ a book]) + [e$_\text{a}$ interesting]) \par 
= \{<b$_{\alpha}$,b$_{\beta}$,b$_\text{a}$>: b$_\text{a}$ \(\in\) Ext (``book'') and b$_\text{a}$ \(\in\) Ext (``interesting'')\} \par 

\end{exe}

As a result, the interpretation for \REF{ex:irani:45} is the following:

\begin{exe}

\ex John(x) \& book(y) \& bought(x,y) \& book(z) \& interesting(z) 

\end{exe}

The analysis presented above illustrates two main points: one, NPs in ASL\il{American Sign Language} can be either specified or underspecified for a locus feature, and two, an NP specified for a locus feature cannot refer to an NP that is underspecified for them. Given this system, we can now predict the infelicity of a definite reading with IX in expressions like (\ref{ex:irani:45}b). 

Finally, our proposal allows us to explain some examples presented in the literature regarding IX without an NP. \citet{KoulidobrovaLilloMartin2016} also argue that IX without an NP is not a pronoun, against previous claims in the literature \citep{Kuhn2015}. Our proposal now allows us to decide between the two sides of the debate, as we can lay out the arguments against IX as a pronoun, and show that they do not hold under the current analysis. We have already established that IX+NPs and bare NPs introduce two flavors of indices that do not interact with each other. An IX+NP will be interpreted as an \is{indefinites}indefinite expression unless it has an IX+NP antecedent with the same specified locus feature. The argument against IX as pronoun is based on evidence like the following: 

\begin{exe} \il{American Sign Language}
	\ex \label{ex:irani:66} PETER THINK IX$_\text{a}$ / IX-neu SMART. \\
	`Peter$_\text{i}$ thinks he$_\text{*i/j}$ is smart.' \citep[18]{KoulidobrovaLilloMartin2016}
\end{exe} 

\begin{exe} \il{American Sign Language}
	\ex \label{ex:irani:67}
	\begin{xlist} 
		\ex \label{ex:irani:67a} WHEN ONE $_\text{a}$CL STUDENT COME PARTY, $_\text{a}$IX HAVE-FUN.\\
		`When a student$_\text{i}$ comes to the party, he$_\text{i/*j}$ has fun.’  
		
		\ex  WHEN ONE STUDENT$_\text{i}$ COME PARTY, $_\text{a}$IX/neu-[CL IX] HAVE-FUN.\\
		`When a student comes to the party, he$_\text{*i/j}$ has fun.’ (\citealt{Schlenker2010}, as cited by \citealt{KoulidobrovaLilloMartin2016})
	\end{xlist}     
\end{exe} 

\noindent The line of reasoning here is that IX cannot refer back to the bare NP as in \REF{ex:irani:66}, which would be odd given the pronominal nature of IX. The mystery absolves itself under the present approach, wherein the bare NP and IX+NP introduce indices of different types. The example in \REF{ex:irani:66} shows that the first instance of IX cannot refer back to Peter, but to another individual, which is completely predictable if we assume that IX, similar to IX+NPs, cannot refer back to bare NPs as they are specified for a locus feature. 

The system of NPs being specified or unspecified for a locus feature allows us to view the function of loci differently. They are not merely the realization of indices in the language; they also allow us to keep track of \is{discourse}discourse referents. Specifying an NP for a locus feature is, then, simply more efficient than using bare NPs. Certainly, we do not wish to make a strong functional claim here in which ease of processing drives the use of loci. We are only stating that a signed language has the option of using loci, and ASL\il{American Sign Language} makes use of this option.  

Throughout this section, we have underlyingly assumed that loci are features, a fact that has been proposed previously for ASL\il{American Sign Language} \citep{Kuhn2015, Schlenker2014}. Since this assumption is non-trivial, we discuss it further in detail in the following section. 

\subsection{Loci as featural variables}\label{sec:irani:4.3}\is{featural variables|(}

The notion that IX consists of a locus feature and \is{bare noun phrases|)}bare NPs are underspecified for them integrates previous proposals, namely that of featural variables \citep{Schlenker2014}. A featural variable analysis of loci accounts for the ability of loci to be reused and shared, and for features to be uninterpreted under \textit{only}, a fact that has been noted for the language \citep{Kuhn2015}. Below, we discuss the arguments for a featural variable analysis, and then show how our analysis fits in with this approach to ASL\il{American Sign Language}. 

\subsubsection{Arguments for loci as features}

The motivation for a featural variable approach consists of two parts: arguments for loci as \is{morphosyntax}morpho-syntactic features and arguments for loci as variables. We discuss both aspects of the analysis so that we can examine how our proposal relates to the other facts of the language. We start with arguments for loci as features in this section. 

There are several crucial facts that illustrate the need for ASL\il{American Sign Language} loci to be analyzed in part as morpho-syntactic features. Loci can be reused, shared, and the features of the NP associated with the locus can be uninterpreted under \textit{only}. We illustrate each of the above facts below in turn. 

Prima facie, we know that loci can be reused since loci do not remain associated with a particular entity for longer than a conversation. Moreover, loci can be reused even within the same conversation.

\begin{exe} \il{American Sign Language}
	\ex \label{ex:irani:68} \makebox[\linewidth][s]{KINDERGARTEN CLASS STUDENTS IX-arc\textsubscript{ab} STUDENTS PRACTICE} \\ \makebox[\linewidth][s]{DIFFERENT COMPLIMENTS. FIRST, IX$_\text{a}$ ALAN$_\text{a}$ TELL IX$_\text{b}$ BILL$_\text{b}$ IX$_\text{a}$} \\ ADMIRES IX$_\text{b}$. SECOND, IX$_\text{a}$ CHARLES$_\text{a}$ TELL IX$_\text{b}$ DANIELLE IX$_\text{a}$ LIKES POSS$_\text{b}$ STYLE. THIRD, IX$_\text{a}$ EVE$_\text{a}$ TELL IX$_\text{b}$ FRANCIS$_\text{b}$ IX$_\text{a}$ THINK IX$_\text{b}$ HANDSOME.\\
	`In a kindergarten class, the students were practicing different compliments. First, Alan$_\text{i}$ told Bill$_\text{j}$ that he$_\text{i}$ admires him$_\text{j}$. Second, Charles told Danielle that he likes her style. Third, Eve told Francis that she thinks he's handsome.' \citep[adapted from][462]{Kuhn2015}
\end{exe} 

\REF{ex:irani:68} demonstrates how the loci a and b can be reused for every pair \is{reference}referenced in the sentences. Therefore, we see that there is no one-to-one correspondence between loci and \is{discourse}discourse referents throughout  single discourse. Under our approach, the introduction of a distinct NP even with the same locus feature associated with it, would introduce a new index, and thus, the loci get reused. 

The argument that there is no one-to-one correspondence between loci and variables is, furthermore, bolstered by the fact that loci can be shared. This is illustrated below: 

\begin{exe} \il{American Sign Language}
	\ex \label{ex:irani:69} EVERY-DAY, IX$_\text{a}$ JOHN$_\text{a}$ TELL IX$_\text{a}$ MARY$_\text{a}$ IX$_\text{a}$ LOVE IX$_\text{a}$. BILL$_\text{b}$ NEVER TELL SUZY$_\text{b}$ IX$_\text{b}$ LOVE IX$_\text{b}$. \\
	`Every day, John$_\text{i}$ tells Mary$_\text{j}$ that he$_\text{i}$ loves her$_\text{j}$. Bill$_\text{x}$ never tells Suzy$_\text{y}$ that he$_\text{x}$ loves her$_\text{y}$.’
\end{exe} 

In this example, we see that two referents can be situated at one locus; therefore, it appears that loci can be shared. This property further undermines the strong one-to-one correspondence between loci and variables.  

Another argument that shows the need to evoke features on loci arises from the uninterpreted phi-features on pronouns under \isi{focus}-sensitive operators like \textit{only}. Let us first consider the following \ili{English} sentences: 

\begin{exe} 
	\ex \label{ex:irani:70} \textit{Only Mary did her homework.}
	
	\ex \label{ex:irani:71} \textit{Only I did my homework.}
\end{exe} 

\REF{ex:irani:70} entails that John did not do his homework even though he is male, and \REF{ex:irani:71} entails that John did not do his homework even though he is not the speaker. Thus, we can see in \ili{English} that both gender and person features can be uninterpreted under \textit{only}. These facts are paralleled by the ASL\il{American Sign Language} loci examples as well: \newpage

\begin{exe} \il{American Sign Language}
	\ex IX$_\text{a}$ JESSICA$_\text{a}$ TELL-ME IX$_\text{b}$ [BILLY ONLY-ONE]$_\text{b}$ FINISH POSS$_\text{b}$ \\ HOMEWORK.\\
	Bound reading: Jessica$_\text{x}$ told me [only Billy$_\text{y}$] \(\lambda\)z.z did z’s homework. \citep[9]{Kuhn2015}
\end{exe}

If there was a one to one relationship between the locus and the index associated with it, then it is unexpected that the gender feature can be deleted such that it is able to refer to persons not associated at that locus; in other words, BILLY at locus b should be impossible to consider JESSICA, signed at locus a, as a value for the index associated with locus b. The fact that the sentence signed at locus b can refer to entities outside that set indicates that some features at the locus can be uninterpreted. In this case, the locus feature is uninterpreted and \isi{reference} can be made to both BILLY and JESSICA. 

In this section, we have seen arguments to abandon the view that there is an absolute one-to-one correspondence between loci and variables. We have also seen that the ASL\il{American Sign Language} data presented here are compatible with an analysis that analyzes loci as features. The following section presents an overview of the argument that variables are not obsolete in analyzing loci. 

\subsubsection{Arguments for loci as variables}

The evidence for loci being composed of features is convincing, but there are also reasons for which we would not want to opt for a completely variable-free analysis. In addition to the fact that loci generally refer to the individual they are associated with, as we saw in \sectref{sec:irani:2}, \citet{Schlenker2014} argues for another reason to retain variables: iconic bound loci, which refer to an individual's importance, height, or position. Loci in such instances can be set up high or low to indicate the aforementioned aspects, which makes them iconic. It appears that in these cases not all features under \textit{only} get deleted and the iconic height feature on the locus remains intact. 

Iconic bound loci in ASL\il{American Sign Language} can be easily captured in a variable account of loci, but the account for iconic bound loci under a variable-free analysis is not straightforward. The examples below illustrate that in ASL\il{American Sign Language}, high loci can be used to refer to tall, powerful, or important individuals, and the height of the loci is still interpreted under binding and under \textit{only} \citep{Schlenker2014}.

\newpage
\begin{exe} \il{American Sign Language}
		\ex GYMNAST COMPETITION MUST STAND BAR FINISH STAND \\ HANG.\\
		`In a gymnastics competition one must stand on a bar and then go from standing to hanging position.'
	\begin{xlist}	
		\ex  ALL GYMNAST IX$_\text{a}$-{neutral} WANT IX-1 LOOK$_\text{a}$-{high} FINISH FILM IX$_\text{a}$-{low}.\\
		`All the gymnasts want me to look at them while they are up before filming them while they are down.'
		
		\ex  ONLY-CL GYMNAST IX$_\text{a}$-{neutral} WANT IX-1 LOOK$_\text{a}$-{high} FINISH \\ FILM IX$_\text{a}$-{low}. \\
		`Only one of the gymnasts wants me to watch her while standing before filming her while hanging.' \citep[1081]{Schlenker2014}
	\end{xlist}     
\end{exe}

We see from this example that although phi-features under \textit{only} can be uninterpreted, the height feature must necessarily keep its positional association intact. Therefore, iconic bound loci lend evidence to an analysis of loci that also make use of variables. These facts now lead us to a featural variable analysis of ASL\il{American Sign Language} loci. Combining both aspects of loci, \citet{Schlenker2014} proposes a featural variables analysis, which we expand on in the next section. 

\subsubsection{Featural variables}

The facts noted earlier in the paper show the need for an approach of loci that accounts for them as both features and variables. A featural variable analysis \citep{Schlenker2014} provides a platform to do exactly that. Below, we discuss how the cases of locus reuse, locus sharing, and interpretation under \textit{only} are accounted for under Schlenker's analysis.\footnote{See \citet{Schlenker2014} for a complete account of how a featural variable system can incorporate the various properties of loci.}  

\is{focus|(}
Let us first lay out the tools needed to address the observed patterns. We saw that features can be deleted under focus operators; therefore, a deletion rule is needed. Below we have rules that result under a semantic or a \is{morphosyntax}morpho-syntactic approach. The following rule under a semantic analysis allows a feature \textit{F} on a pronoun to remain uninterpreted under focus. For expository purposes, we look at Schlenker's illustration of the deletion of a potential feminine feature.  

\begin{exe}
	\ex \label{ex:irani:74}``Let E be an expression of type \textit{e} and \textit{f} a feminine feature, \textit{F} a focus marker, and {[}[\(\alpha\){]}]$^\text{{O,c,s,w}}$ the ordinary and focus values of \(\alpha\) under a context \textit{c}, an assignment function \textit{s} and a world \textit{w}.
	
	\begin{xlist}
		\ex {[}[E$^\text{{f}}${]}]$^\text{{O,c,s,w}}$ = \# iff {[}[E{]}]$^\text{{O,c,s,w}}$ = \# {[}[E{]}]$^\text{{O,c,s,w}}$ is not female in the world of c. If {[}[E$^\text{\textit{f}}${]}]$^\text{{O,c,s,w}}$ \(\neq\) \#, {[}[E$^\text{{f}}${]}]$^\text{{O,c,s,w}}$ = {[}[E{]}]$^\text{{O,c,s,w}}$
		
		\ex {[}[E$^\text{{f}}${]}]$^\text{{F,c,s,w}}$ = {[}[E{]}]$^\text{{O,c,s,w}}$ (i.e. the feature \textit{f} plays no role in the focus dimension.)
		
		\ex {[}[E$^\text{{f}}_\text{{F}}${]}]$^\text{{F,c,s,w}}$ = {[}[E$_\text{{F}}${]}]$^\text{{F,c,s,w}}$ = E, the set of individuals.'' \citep[1070]{Schlenker2014}
	\end{xlist}
	
\end{exe}

The above rule states that an expression with a feminine feature \textit{f} results in a presupposition failure if and only if the expression itself results in a presupposition failure or if the expression is not female in the world with context \textit{c}. If the expression does not result in a presupposition failure, then the feminine feature plays no role in the focus dimension. Another alternative to feature deletion under focus is the deletion under agreement rule, which tethers to a \is{morphosyntax}morpho-syntactic approach. The rule below optionally requires a feature \textit{F} to be uninterpreted if a pronoun is bound by an element with feature \textit{F}; i.e. when the features agree.

\begin{exe}
	\ex 
	\begin{xlist} 
		
		\ex \label{ex:irani:75a}``Optionally delete feature \textit{F} of a variable \textit{v$^\text{\textit{F}}$} if (i) \textit{v$^\text{\textit{F}}$} appears next to a \(\lambda\)-abstractor \(\lambda\)v$^\text{\textit{F}}$ and the appearance of \(\lambda\)v$^\text{\textit{F}}$ is triggered by an expression with feature \textit{F}, or (ii) v$^\text{\textit{F}}$ is bound by \(\lambda\)v$^\text{\textit{F}}$.
		
		\ex \(\lambda\)-abstractors inherit the features of the expressions that trigger their appearance.'' \citep[1071]{Schlenker2014}
	\end{xlist}
	
\end{exe} 

As opposed to the rule in \REF{ex:irani:74}, \REF{ex:irani:75a} provides us with a deletion under agreement approach. \REF{ex:irani:75a} simply states that a feature on a variable gets deleted when the variable appears next to a \(\lambda\)-abstractor, whose occurrence is triggered by an expression with that feature, or if the variable is bound by the \(\lambda\)-abstractor. The rules above allow us to account for cases where the features of an entity associated with a loci are uninterpreted. 

Although these rules can straightforwardly account for the deletion or uninterpreted features under focus operators, there is another option available for locus sharing cases. Below is the relevant example in \REF{ex:irani:69} originally discussed by \citet{Kuhn2015} repeated below as \REF{ex:irani:76}. Here, JOHN and MARY share locus a and BILL and SUZY share locus b. 

\begin{exe} \il{American Sign Language}
	\ex \label{ex:irani:76} EVERY-DAY, IX$_\text{a}$ JOHN$_\text{a}$ TELL IX$_\text{a}$ MARY$_\text{a}$ IX$_\text{a}$ LOVE IX$_\text{a}$. IX$_\text{b}$ BILL$_\text{b}$ NEVER TELL IX$_\text{b}$ SUZY$_\text{b}$ IX$_\text{b}$ LOVE IX$_\text{b}$ \\
	`Every day, John$_\text{i}$ tells Mary$_\text{j}$ that he$_\text{i}$ loves her$_\text{j}$. Bill$_\text{x}$ never tells Suzy$_\text{y}$ that he$_\text{x}$ loves her$_\text{y}$.' \citep[1073]{Schlenker2014}
\end{exe} 

The pattern noted above can be captured via deletion under agreement \REF{ex:irani:75a}. For a deletion analysis, we can simply say that the a locus feature get deleted under agreement as shown below. 

\begin{exe}
	\ex \textit{John}$_\text{a}$ $\lambda$i$^{\text{\sout{a}}}$ \textit{Mary} $\lambda$k$^{\text{\sout{a}}}$ t$_\text{i}^{\text{\sout{a}}}$ \textit{tell} t$_\text{k}^{\text{\sout{a}}}$ [pro$_{\text{i}}^{\text{\sout{a}}}$ \textit{love} pro$_{\text{k}}^{\text{\sout{a}}}$] \citep[1079]{Schlenker2014}
\end{exe}

However, it does seem a bit odd that one would be able to refer back to a locus after its features have been deleted.\footnote{\citet{Schlenker2014} does not provide any further details on how a deletion analysis captures cases like \REF{ex:irani:76}. Without this supplementary information, the merits of appealing to feature deletion here are yet to be seen.} Schlenker also proposes another alternative where perhaps in the example above, \textit{John} and \textit{Mary} form a plurality of individuals, and IX only refers to a part of this plurality of individuals. Given that the contribution of loci is sensitive to the assignment function \textit{s}, and an expression \textit{E} associated with a locus \textit{a}, we can say that it is required that \textit{E} in these cases denotes a part of what \textit{a} denotes. A general part-denoting rule for loci can thus be spelled out as follows:  

\begin{exe}
	\ex \label{ex:irani:78} ``For every locus a \(\neq\) 1,2, if \textit{E} is an expression of type e, {[}[E$^\text{a}$]{]}$^\text{{c,s,w}}$ = \# iff {[}[E]{]}$^\text{{c,s,w}}$ =  \# or {[}[E]{]}$^\text{{c,s,w}}$ isn't a mereological part of s(a) or {[}[E]{]}$^\text{{c,s,w}}$ is present in the situation of utterance in c and 1, {[}[E]{]}$^\text{{c,s,w}}$ and \textit{a} are not roughly aligned. If {[}[E$^\text{a}$]{]}$^\text{{c,s,w}}$ \(\neq\) \#, {[}[E$^\text{a}$]{]}$^\text{{c,s,w}}$ = {[}[E]{]}$^\text{{c,s,w}}$'' \citep[1080]{Schlenker2014}
\end{exe}

This rule proposes that the locus denotes the plurality \textit{John}\(\oplus\)\textit{Mary}, and we're referring back to a part of that expression. The expression E has to be a mereological part of the the assignment function that maps on to the locus. Hence, we now have two options of dealing with the locus sharing examples: via deletion under agreement \REF{ex:irani:75a} or via a denotation of parts (rule \REF{ex:irani:78}). 

Schlenker's rules allow us to capture the properties of loci observed by Kuhn. The deletion rule can be evoked for the breakdown of the one-to-one correspondence under a focus operator like \textit{only}. Moreover, we must modify our own rule stated in \REF{ex:irani:46} in order to account for the locus sharing instances. First, we note as \citeauthor{Kuhn2015} did that these examples, like the one in \REF{ex:irani:76}, are heavily dependent on the right context. They become possible when the \isi{discourse} facilitates its use using parallelism between the two sentences or a similar mechanism, but they are not ordinarily judged as unexceptional. Taking that into consideration, we can accordingly modify our rule in \REF{ex:irani:46}, repeated below in \REF{ex:irani:79}. \is{focus|)}

\begin{exe}
	\ex \label{ex:irani:79} If i \(\in\) Dom(F), then Sat(F') = Sat(F+b$_\text{i}$ \(\in\) Ext(``NP''));\\ else, if i is \(\notin\) Dom(F), then Dom(F') = Dom(F) \(\cup\) \{i\}. 
\end{exe}

The loci sharing cases now require us to add the following condition: 

\begin{exe}
	\ex \label{ex:irani:80} If i \(\in\) Dom(F), \textit{and b$_\text{i}$ \(\in\) Ext(``NP'') is consistent with the context}, then Sat(F') = Sat(F+b$_i$ \(\in\) Ext(``NP''));\\ 
	else, if i is \(\notin\) Dom(F), then Dom(F') = Dom(F) \(\cup\) \{i\}. \par  
\end{exe}

By adding the consistency with the context requirement in \REF{ex:irani:80}, we can now have more than one NP associated with the same locus. When a second NP is signed at the same locus as a previous NP, it is considered a novel referent once context has determined that the second NP is not equal to the first. In other words, when MARY is signed at the same locus as JOHN, the inconsistency in the context that John is not Mary, leads us to conclude that the index is not in the domain of the file. There are scenarios that can push this claim further. For instance, if an individual is both a linguist and a student, the interpretation of signing the two at different loci or at the same locus can be informative. We do not address this point in more detail here, but we note that our rule does not allow us to distinguish between the two alternatives of dealing with loci-reuse and sharing cases proposed by \citeauthor{Schlenker2014}. Our formulation is compatible with either a feature deletion account or a \is{part-whole relationship}part-whole account of the phenomenon. Below, we dwell on these possibilities a little longer. 

For the purposes of our analysis of IX, we need to say nothing further. The examples noted by \citeauthor{Kuhn2015} suggesting that IX is composed of features is successfully integrated into our approach by adopting the rules proposed by Schlenker that are described in this section. We now have a more complete picture of the nature of the ASL\il{American Sign Language} IX. Even so, we can attempt to disambiguate between these two options of feature deletion or part-denotation by using the \is{product-producer relationship}product-producer \isi{bridging} examples. \citet{Schwarz2009} proposes that these cases require the representation of a null pronoun in the structure; thus, they behave like regular \is{anaphora}anaphoric strong definites \citep[268]{Schwarz2009}. Therefore, the sentences in \REF{ex:irani:81a} are structurally understood as \REF{ex:irani81b}. 

\begin{exe}
	\ex 
	\begin{xlist} 
		\ex \label{ex:irani:81a} \textit{I bought a book the other day. The author is French.}
		\ex \label{ex:irani81b} \textit{I bought a book the other day. The author (of it) is French.}
	\end{xlist} 
\end{exe}

Such a proposal leads us to consider that \textit{the author} in such cases was never introduced as a referent by itself, and it only exists in relation to the pronoun. We can employ a similar example in ASL\il{American Sign Language}, and by attempting to refer back to the locus associated with BOOK and AUTHOR with IX (without an NP), we can determine whether AUTHOR was introduced in the \isi{discourse} if IX can refer to it. The example is shown below:


\begin{exe} \il{American Sign Language}
	\ex \label{ex:irani:82} IX$_\text{a}$ JOHN$_\text{a}$ BUY IX$_\text{b}$ BOOK$_\text{b}$. IX$_\text{b}$ AUTHOR$_\text{b}$ SELF FRENCH. IX$_\text{a}$ JOHN$_\text{a}$ TIRED TODAY. SLEEP. TWO HOURS LATER, WOKE-UP. THEN, REMEMBERED IX$_\text{b}$. \\
	`John bought a book. The author was French. John's tired today. He fell asleep. Two hours later, he woke up and recalled it.'
\end{exe} 

Our consultants maintain that the final pronoun IX in the example above can refer to either BOOK or AUTHOR. This example indicates that an index for each of these entities was introduced in the utterance. It seems that even though the AUTHOR in \REF{ex:irani:82} was mentioned in relation to BOOK, ASL\il{American Sign Language} introduces a new index for it. This data points us towards the direction of the denotation of parts analysis of locus sharing and reuse cases since AUTHOR was separately introduced in the \isi{discourse} at the same locus. It appears that BOOK and AUTHOR form a plurality of individuals associated with the same locus, and we can refer back to either part of the plurality using IX and the rule in \REF{ex:irani:78}. Under a deletion analysis, capturing these facts is not straightforward. 

The example presented in \REF{ex:irani:82} does not completely allow us to differentiate between the two alternatives; however, we do learn something about these \is{product-producer relationship}product-producer \isi{bridging} cases. Even in such examples, IX allows us to set up a new referent for both the product and the producer, and we can return back to the locus associated with them later on in the discourse. For present purposes, we do not expand on this data further, but leave it open for future work. 

Throughout this section, we have seen evidence for loci being composed of features, and we have adopted a system of featural\is{featural variables|)} variables that allows us to capture the full range of locus properties. These aspects are important for the analysis at hand as we crucially assume that \is{bare noun phrases|(}bare NPs, unlike IX+NPs, are underspecified for a locus feature. The difference between the two nominal types is not that one introduces an index and the other doesn't, but that the type of indices introduced by the bare NPs and IX+NPs differ precisely in their specification of these features. 

\subsection{Final points}

We have seen that the analysis here accounts for the distribution of IX in definite and \is{indefinites}indefinite environments. Although we have discussed our proposal in detail, some judgments presented in the literature are not in line with those of our consultants and may need further investigation. We describe those examples in this section. \newpage

\citet{Bahanetal1995} argue that IX before NPs is a \is{definiteness marking}definite marker, but they do so on the basis of data that are incompatible with ours, at least as they stand. They claim that IX+NP must necessarily be \is{definites}definite, which is at odds with the IX+NPs in donkey sentences seen earlier. They provide the example below: 

\begin{exe}\il{American Sign Language}
	\ex{\label{ex:irani:83} \#~JOHN LOOK-FOR IX$_\text{a}$ MAN$_\text{a}$ FIX GARAGE. \\
	\#~`John is looking for a man to fix the garage.’ \citep[4]{Bahanetal1995}}
\end{exe}

\REF{ex:irani:83} is taken to show that the \is{indefinites}indefinite reading is unavailable with the use of IX, as John is only looking for a particular man to fix the garage, not any man. We do not agree with their argumentation here for two reasons: one, we have shown that IX+NPs have an indefinite reading, and two, it is unclear what effects we expect when a locus is set up for an entity that is not used further in the \is{discourse}discourse. In other words, we cannot rule out that the IX+NP MAN in this case is truly not \is{indefinites}indefinite, or if the infelicity is simply a result of introducing an entity that is set up to be continually referred to throughout the discourse. Moreover, our consultants do not agree with this judgement. Hence, we leave this example open for further investigation.\footnote{One way of resolving this example would be to continue the discourse on the man, and checking to see whether the \is{specificity}non-specific interpretation is available, but we do not have the relevant example at hand.} 


Returning to the view arguing for IX as a \is{demonstratives}demonstrative, \citet{KoulidobrovaLilloMartin2016} also present a pair of examples that our consultants do not agree with; therefore, we describe them here in order to address them in more detail. Taking into consideration that definite articles are known to carry covarying readings while demonstratives do not, Koulidobrova \& Lillo-Martin argue that co-varying readings are unavailable with IX. Let us take a look at the \ili{English} examples first:

\begin{exe}
	\ex \label{ex:irani:84} \textit{That guy in the red shirt always wins.} = referential / *covarying \\ (\citealt{Nowak2013}, as cited by \citealt{KoulidobrovaLilloMartin2016}) 
	
	\ex \label{ex:irani:85} \textit{The guy in the red shirt always wins.} = referential / covarying \\ (\citealt{Nowak2013}, as cited by \citealt{KoulidobrovaLilloMartin2016}) 
\end{exe}

The above examples describes two situations, one in which any unspecified individual wins, i.e. the covarying reading, and another in which one specified person wins, which is the referential reading. Both of the above examples allow for referential readings; however, only \REF{ex:irani:85} allows for the covarying interpretation. When the \is{demonstratives}demonstrative \textit{that} is used in \REF{ex:irani:84}, we do not get the reading for the rigged race where any person wearing red is the winner. This diagnostic is now applied to ASL\il{American Sign Language} to indicate that IX behaves more like a \is{demonstratives}demonstrative than a \is{definite articles}definite article. 

\begin{exe}\il{American Sign Language}
	\ex \label{ex:irani:86} {IX$_\text{a}$ PERSON$_\text{a}$ / IX$_\text{a}$} RED SHIRT SELF TEND WIN. = referential / *covarying\\ 
	`IX person / IX in the red shirt tends to win' \citep[14]{KoulidobrovaLilloMartin2016}.
	
	\ex \label{ex:irani:87} PERSON HAVE RED SHIRT TEND WIN. = referential / covarying \\
	`The person in the red shirt tends to win’ \citep[14]{KoulidobrovaLilloMartin2016}.
\end{exe}

\noindent It appears at first glance that these examples are problematic for our proposal. However, we have already noted that IX+NPs are perfectly compatible with donkey readings. Moreover, our consultants find a covarying reading acceptable in \REF{ex:irani:86}. Since there is a discrepancy in the judgments between consultants, it would be useful to retest these sentences with different contexts in order to clarify whether a covarying reading is truly unavailable in these cases. In retesting these cases, we should also be careful to test sentences that are only minimally different; \REF{ex:irani:86} and \REF{ex:irani:87} are not minimal pairs. 

The above examples, at least on the surface, are points of contention between the different analyses. We do allow for the possibility that there is true inter-speaker variation in the language as the ASL\il{American Sign Language} signing community is extremely spread out. Nevertheless, as we have discussed, these matters are not immediately problematic for the analysis at hand without further investigation. 

\subsection{Summary}

Before moving on to the implications of our analysis, let us summarize our findings thus far. After we present an overview of the various discussions in this paper, we contemplate the theoretical implications of our proposal in the following section. 

Previous work on ASL\il{American Sign Language} assumed that loci were the overt realization of an index introduced by \is{discourse}discourse referents, and that IX+NPs were \isi{demonstratives}. In this paper, we saw that both bare NPs and IX+NPs introduce an index, but these indices are of different types based on their specification or underspecification of a locus feature. In doing so, we also saw that both nominal types double as definite and \is{indefinites}indefinite expressions. This fact results in the nominals having the ability to either set up a new referent, or refer back to a \is{familiarity|(}familiar one if they have the same index. The ability to set up a new referent when the index is not in the domain of the file signifies that ASL\il{American Sign Language} definite expressions do not have a familiarity restriction. 

In spite of the lack of a familiarity restriction, we also saw that the two kinds of definite articles observed by \citet{Schwarz2009,Schwarz2013} corresponds to bare NP and IX+NP in ASL\il{American Sign Language} when they are not indefinite. This is telling that perhaps \isi{definiteness} is not completely semantically void, and that it does hold in ASL\il{American Sign Language}, albeit only to an extent. The next section discusses the implications of the analysis provided \largerpage in this paper. 



\section{Discussion}\label{sec:irani:5}



We have seen throughout that the choice between bare NPs and IX+NPs appears to be more or less unrestricted, barring the \is{uniqueness}unique definite environment cases, which is the only instance where IX is not permitted. The examples we saw in \sectref{sec:irani:3} indicate that there is some restriction on locus association with unique referents. However, one can imagine a scenario in which there are two unique referents under discussion. It appears that in these cases, the locus association is not completely ruled out. Let us consider the following example of a unique priest and a unique principal at a school. 

\begin{exe} \il{American Sign Language}
	
	\ex[?]{I VISIT SCHOOL. MET IX$_\text{a}$ PRINCIPAL$_\text{a}$, IX$_\text{b}$ PRIEST$_\text{b}$. IX$_\text{a}$ PRINCIPAL$_\text{a}$ NICE LADY. \\  
	`I visited the school and met the priest and the principal. The principal is a nice lady.'}
	
\end{exe} 

\noindent This example suggests that context can at least sometimes play a role in making IX felicitous with unique referents. Without delving into further detail, we leave open the possibility that \isi{uniqueness} restrictions on IX may or may not consistently hold, although future work on such cases is necessary to determine whether definiteness in the language is semantically \is{definiteness marking}encoded.  

\section{Conclusion}

The pattern of definite expressions in ASL\il{American Sign Language} and the proposal that resulted from it, can potentially pave the way to a new prespective on \isi{definiteness} in the language. We have already seen that there is no familiarity restriction on definite expressions as a new referent can be set up if its index has not already been introduced. This tells us that definiteness might not be lexically \is{definiteness marking}encoded in ASL\il{American Sign Language}. IX was previously assumed to be an overt index, which might have taken up a special status. Given that both bare NPs and IX+NPs introduce indices and can either be \is{definites}definite or \is{indefinites}indefinite, we may be led to rethink the nature of definiteness in ASL\il{American Sign Language}, and perhaps, in sign languages overall. 

Examining ASL\il{American Sign Language} indices and bare NPs has unveiled many aspects of the language in particular, and languages in general. We first saw that the index IX when referring to a locus is a \is{strong definite articles}strong definite article, and bare NPs are weak definite articles that do not permit IX. This pattern indicates that the language distinguishes between \isi{anaphoricity} and familiarity on one hand, and \isi{uniqueness} on the other. On the flip side, we saw that the language does not have a restriction on familiarity; we can introduce a new referent if it is not already present in the \isi{discourse}.  

In the literature, only ASL\il{American Sign Language} loci were typically viewed as indices. Here, reanalyzing definite and \is{indefinites}indefinite expressions allows us to view things a bit differently, as we were lead to propose that bare NPs introduce indices as well. The double life of IX+NPs and \is{bare noun phrases|)}bare NPs as definite and indefinite expressions, which do not have a \is{familiarity|)}familiarity restriction imposed on them, suggest that we are not dealing with a system that lexically \is{definiteness marking}encodes definiteness. Instead, we find that pragmatics might play a huge role in facilitating conversation, and in a language that has the option of using loci, the specification of a locus feature can play a role in determining whether or not an expression has been introduced.\is{loci|)}

Finally, the data reported in this paper are the judgments of three ASL\il{American Sign Language} signers. Future work on the topic would greatly benefit from \is{experimental study}experimental work investigating native speaker intuitions on a greater scale. There is known to be significant interspeaker variation in the community, and any such variation could be captured by surveying a larger group of ASL\il{American Sign Language} signers. 

\section*{Acknowledgements}

Many thanks are due to Florian Schwarz for invaluable input on the project. Thanks are also due to my consultants Scott Bradley, Maggie Hoyt, and Sophia Hu for their judgments. I would also like to thank Julie Anne Legate, Ava Creemers, Luke Adamson, Nattanun Chanchochai, Kajsa Dj{\"a}rv, Milena \v{S}ereikait\.{e}, two anonymous reviewers, and the audience at the Mid-Atlantic Colloquium of Studies in Meaning V and the Definiteness across Languages workshop for their comments and feedback on various drafts of the paper. All errors are my own.

{\sloppy
\printbibliography[heading=subbibliography,notkeyword=this]
}

\end{document}