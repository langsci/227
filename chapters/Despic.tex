\documentclass[output=paper,
modfonts
]{langscibook} 

\title{On kinds and anaphoricity in languages without definite articles} 
\author{Miloje Despić\affiliation{Cornell University}}

% \chapterDOI{} %will be filled in at production

% \epigram{}

\abstract{
	This paper investigates the availability of \is{anaphora}anaphoric readings with \isi{bare nouns} in \is{article-less languages}languages without \isi{definite articles}, with a special focus on kind-level interpretation. Various facts from \il{Serbo-Croatian!Serbian}Serbian, \ili{Turkish}, \ili{Japanese}, \ili{Mandarin}, and \ili{Hindi} shows that the anaphoric reading of bare nouns is constrained by two general factors: (i) \isi{number} morphology; in particular, whether the language in question has number morphology to begin with, and if it does, whether the bare noun in question is \is{mass nouns}\is{mass/count distinction}mass or count, and (ii) kind interpretation. It seems that mass and plural nouns can have anaphoric readings only if they are not interpreted as kinds. \is{count nouns}Singular count bare nouns, on the other hand, do not seem to be restricted in this way: they can have anaphoric readings regardless of whether or not they are interpreted as kinds. I argue that this state of affairs naturally follows from the system developed in \citet{Dayal2004}, which is based on a limited set of type-shifting operations and a particular analysis of number morphology. Alternative approaches to interpretation of bare nouns, on the other hand, do not seem to directly predict this sort of variation and require additional assumptions to account for it. 
}

\begin{document}
	\maketitle
	\section{Introduction}\is{kinds|(}\is{bare nouns|(}\largerpage[-2]
	In this paper, I explore the \is{anaphoric definites}anaphoric definite interpretation of bare nouns in languages without definite articles. Evidence presented here reveals an interesting generalization about the availability of anaphoric readings with bare nouns, which requires an adequate explanation. In particular, it seems that the anaphoric interpretation of a bare noun depends on (i) whether or not the noun in question is singular or \is{mass nouns}mass/plural and (ii) whether or not it is interpreted as kind-denoting. I will present data from \il{Serbo-Croatian!Serbian}Serbian, \ili{Turkish}, \ili{Japanese}, \ili{Mandarin} and \il{Hindi}Hin\-di to illustrate this phenomenon. Before introducing the main empirical puzzle, it is useful to go over two major types of approaches to the structure and interpretation of NPs in languages without definite articles.
	
	A theoretical challenge for anyone dealing with bare nouns in languages without articles is how to formally treat the absence of the \is{definite determiners}definite determiner.\footnote{This is part of a more general question of how to treat a construction/language which lacks a particular morpheme that is otherwise present in other constructions/languages.}  On the one hand, there is what we may call the \is{Universal DP (Approach)}Universal DP Approach (UDP), on which \is{determiner phrases}DP is present in all languages, regardless of whether they have a \is{definite articles}definite article or not  \citep[e.g.][]{Longobardi1994,Cinque1994,Scott2002,Pereltsvaig2007} etc.). The central claim of this line of research is that even \is{article-less languages}article-less languages have a definite article (i.e. a D head) in syntax, but unlike in languages like \ili{English}, the article is unpronounced/covert. In some versions of it, a fixed layer of functional projections is present in the nominal domain of all languages:
	
	\ea \label{ex:despic:1} \is{number}\is{compounding}
	{Determiner} $>$ {Ordinal Number} $>$ {Cardinal Number} $>$ {Subjective Comment} $>$ {?Evidential} $>$ {Size} $>$ {Length} $>$ {Height} $>$ {Speed} $>$ {?Depth} $>$ {Width} $>$ {Weight} $>$ {Temperature} $>$ {?Wetness} $>$ {Age} $>$ {Shape} $>$ {Color} $>$ {Nationality/Origin} $>$ {Material} $>$ {Compound Element} $>$ {NP}	\citep[114]{Scott2002} 
	\z\largerpage[-3]
	
	The idea here is that the structure of the nominal domain of all languages is underlyingly identical and involves a functional spine in \REF{ex:despic:1}, which is very similar to the adverbial functional spine proposed in \citet{Cinque1999}, for example.
	On the other hand, the \is{DP/NP Approach}DP/NP approach assumes that DP is present only in languages with articles. In this kind of approach, the lack of (overt) articles actually indicates a simpler syntactic structure, i.e. NP \citep{Baker2003, Boskovic2008, Boskovic2012, Despic2011, Despic2013, Despic2015}. The contrast between the two types of languages in the DP/NP approach is illustrated in \REF{ex:despic:2}.\pagebreak
	
	\ea \label{ex:despic:2} 
	\ea \label{ex:despic:2a}
	{\normalsize Languages \textit{with} definite articles} \\
	\begin{forest}
		[\textbf{DP}
		[\textbf{D}\\\textit{\textbf{the}}]
		[FP$_{1}$
		[F$_{1}$
		] 
		[FP$_{2}$[F$_{2}$
		][NP[{},roof]]]
		]
		]
		%]
	\end{forest}
	\\\vspace{-5pt}(F$_1$ and F$_2$: potential functional projections)
	
	\ex\label{ex:despic:2b} {\normalsize Languages \textit{without} definite articles} \\
	\begin{forest}
		[\textit{(DP projection absent)}%,tikz={\node [draw,circle,fit=()] {};}
		[\textbf{}%,tikz={\node [draw,circle,fit=()()] {};}
		]
		[FP$_{1}$
		[F$_{1}$%,tikz={\node [draw,dotted,fit=()()] {};}
		] %{\draw[-] () .. (!us);}
		[FP$_{2}$[F$_{2}$%,tikz={\node [draw,dotted,fit=()()] {};}
		][NP[{},roof]]]
		]
		]
		]
	\end{forest}
	\\ \vspace{-5pt} (F$_1$ and F$_2$: potential functional projections)
	\z
	\z
	
	There seems to be a number of cross-linguistic (and language-specific) syntactic patterns which are strongly correlated with whether or not \is{definiteness marking|(}definiteness marking is overtly present \citep[e.g.][]{Boskovic2008}. Two such generalizations are given in \REF{ex:despic:3} (see \citealt{Boskovic2008} for more):
	
	\ea \label{ex:despic:3}
	\ea 
	Only languages without articles may allow \is{Left Branch Extraction}Left Branch Extraction
	\citep{Boskovic2008,Boskovic2012}.
	\label{ex:despic:3a}
	
	\ex \label{ex:despic:3b}
	Reflexive possessives are available only in languages which lack definiteness marking, or which encode definiteness postnominally. Languages which have prenominal \is{articles}(article-like) definiteness marking, on the other hand, systematically lack reflexive possessives \citep{Reuland2011, Despic2015}.
	\z
	\z 
	
	Correlations like these are expected on the \is{DP/NP Approach}DP/NP approach, since the presence of the \is{definite articles}definite article in a language indicates a richer syntactic structure in the nominal domain. 
	For example, to explain \REF{ex:despic:3b}, \citet{Despic2015} proposes that \is{determiner phrases}DP is a binding domain, in contrast to NP, which is not (see \citealt{Boskovic2012} and \citealt{Despic2015} for discussion of \ref{ex:despic:3a}).\footnote{\is{Left Branch Extraction}\textsc{Left branch extraction} (LBE) refers to situations in which a nominal modifier can be syntactically moved/fronted to the exclusion of the \is{nouns}noun it modifies. \citet{Boskovic2008, Boskovic2012} observes that LBE is possible only in languages without articles. For example, while a construction like (i.a) is grammatical in \il{Serbo-Croatian!Serbian}Serbian, an \is{article-less languages}article-less language, its \ili{English} counterpart is ungrammatical (see i.b). 

		\ea \label{ex:despic:n3}
		\ea {\label{ex:despic:n3a}
		\il{Serbo-Croatian!Serbian}\textnormal{Serbian} \\
		\gll Lepe\textnormal{\textsubscript{i}} je vidio {\ob}\textnormal{t\textsubscript{i}} ku\'ce{\cb}. \\ {beautiful} {is} {seen} {} {houses} \\ 
		\glt {`Beautiful houses, he saw.'}}
		
		\ex {\label{ex:despic:n3b}
		\textnormal{\ili{English}} \\ \largerpage
		\textnormal{*}\textit{Beautiful\textnormal{\textsubscript{i}} he saw {\ob}\textnormal{t\textsubscript{i}} houses{\cb}}.}
	    \z
	    \z 
	
	This strongly suggests that languages with and without definite articles have different nominal structures; e.g. while languages with articles project \is{determiner phrases}DP, which can block movement/\is{Left Branch Extraction}LBE, languages without articles seem to lack this projection (i.e. their nominal structure is simpler; see \ref{ex:despic:2b}).}
	Then in languages with prenominal definite articles, illustrated with \ili{English} in \REF{ex:despic:4}, the reflexive possessive is not bound in its binding domain.
	
	\ea \label{ex:despic:4}
	\ea 
	\begin{forest}
		[\textbf{DP}
		[\textbf{D}\\\textit{\textbf{the}}]
		[PossP
		[*Reflexive
		] 
		[Poss'[Poss
		][NP[{},roof]]]
		]
		]
		]
	\end{forest}
	
	\ex \textit{John\textnormal{$_\text{i}$} likes his\textnormal{$_\text{i}$}\textnormal{/*}himself\textnormal{$_\text{i}$}'s dog.}
	\z
	\z\largerpage
	
	In \is{article-less languages}languages without definite articles, on the other hand, the nominal domain lacks \is{determiner phrases}DP and a binding domain by assumption and reflexive possessives are, therefore, in principle ruled in. Finally, for languages with postnominal \is{definiteness marking|)}definiteness marking, it can be assumed that PossP moves out of DP (as indicated by the word order), which again rules in reflexive possessives.  The general point is that, in the \is{DP/NP Approach}DP/NP approach, it is expected that at least some syntactic patterns would be directly sensitive to the overt presence/absence of the \is{definite articles}definite article.  
	
	In the \is{Universal DP (Approach)}UDP, such correlations appear \textit{accidental}, since the presence of DP in the syntactic structure is independent of its morpho-phonologi\hyp{}cal manifestation.  To be clear, they are not strictly incompatible with the UDP, but additional assumptions are necessary to account for them. The question is, of course, whether these additional assumptions would simply re-describe the facts or actually provide true insight and be independently motivated.  At the same time, one may wonder about the predictive power of the UDP; i.e. what kind of facts would ultimately be able to falsify it?\largerpage
	
	On the semantic side, it is clear that bare nouns in languages without articles can have definite, \is{anaphora}anaphoric readings, unlike in languages like \ili{English}. The question is then what is responsible for the availability of this anaphoric reading, given that the anaphoric reading in languages like \ili{English} requires the \is{definite articles}definite article. In the \is{Universal DP (Approach)}UDP, the presence of a phonologically \is{null determiners}null determiner creates this interpretation \citep[e.g.][]{Longobardi1994}. There is ultimately very little difference between \ili{English} and an \is{article-less languages}article-less language like \il{Serbo-Croatian!Serbian}Serbian: the definite, anaphoric reading in both of them is created by a \is{definite determiners}definite D head. The only difference is that, in contrast to \ili{English}, D is not overtly realized in Serbian. On the other hand, approaches that do not assume null D heads argue that a limited set of type-shifting operations is responsible for the  general interpretation of bare nouns, including the anaphoric reading \citep[e.g.][]{Chierchia1998,Dayal2004}.  
	
	In this paper, I focus on \is{anaphora}anaphoric, definite readings of bare nouns in languages without definite articles.\footnote{For an overview of different aspects of the meaning of definite descriptions see \citet{Schwarz2009} and references
		therein.}  I show that their availability crucially depends on two factors (among other things): (i) \isi{number} morphology and (ii) kind interpretation. I argue that the particular cross-linguistic variation discussed here is expected in the system developed in \citet{Dayal2004}, which employs type-shifting operations and a specific view of number morphology. 
	As discussed in \S\ref{sec:despic:3}--\ref{sec:despic:5}, the system based on type-shifting operations developed in \citet{Chierchia1998} and \citet{Dayal2004} is far from being unconstrained. That is, type-shifting operations do not apply arbitrarily. For example, the so-called \is{Blocking Principle}\textsc{blocking principle} regulates the availability of covert type-shifting operations by making sure that if a language has a lexical item whose meaning is a particular type-shifting operation, then that item must be used instead of the covert version. For this reason, for example, bare nouns in \ili{English} (\is{mass nouns}mass or plural) cannot have definite meaning -- the covert type-shifting operation that would create this meaning is blocked by the existence of the overt lexical item \textit{the}. Also, covert type-shifting operations that are not excluded by the Blocking Principle are not equally available, but are rather ranked in terms of \is{Meaning Preservation}meaning preservation/simplicity; e.g. the operation responsible for \isi{kind reference} $^\cap$ is more highly ranked than $\exists$, and the latter may apply only~if $^\cap$ is undefined for some argument (see \S\ref{sec:despic:3}). Both of these principles are independently motivated; e.g. the Blocking Principle follows the general logic of the \textsc{elsewhere condition} (language particular choices win over universal tendencies).  
	
	At the same time, the data discussed in this paper raise certain questions for the UDP, which seems to require extra assumptions to explain them and it is not clear to which extent these assumptions could be independently motivated. In the remainder of the paper, I will therefore focus on demonstrating how th\largerpage facts presented in the next section follow from \citegen{Dayal2004} proposal. 
	
	The paper is organized as follows. In \sectref{sec:despic:2} I present the main empirical puzzle, while in \sectref{sec:despic:3} I show how it can be explained under \citegen{Dayal2004} approach. In \sectref{sec:despic:4} I discuss some predictions and consequences of the data and analysis introduced in \sectref{sec:despic:2} and \sectref{sec:despic:3}. Finally, a summary and concluding remarks are offered in \sectref{sec:despic:5}.
	Here I also offer some thoughts on how the generalizations presented in this paper and \citet{Dayal2004} can be connected to the distinction between \is{weak definiteness}weak and strong definiteness \citep[e.g.][]{Schwarz2009}. 
	
	\section{The puzzle: Anaphoricity and kinds } \label{sec:despic:2}\is{anaphoricity}
	
	In this section, I present the central empirical problem of the paper. \is{bare singulars}Bare singular \isi{count nouns} in languages without articles can be used anaphorically to refer to a previously introduced individual. Thus, the bare noun \textit{book} in both \il{Serbo-Croatian!Serbian}Serbian (see \ref{ex:despic:5}) and \ili{Turkish} (see \ref{ex:despic:6}) can refer to \textit{Crime and Punishment} in the antecedent clause. \ili{English}, on the other hand, must use the \is{definite articles}definite article (or \is{demonstratives}demonstrative) in the same situation. 
	
	\ea \label{ex:despic:5}
	\il{Serbo-Croatian!Serbian}Serbian \\
	\gll  {{Ju\v ce}}  {sam} {pro\v citao} {``Zlo\v cin   i      Kaznu''} --     {knjiga}       {mi}        {se} {zaista} {svidela}. \\
	yesterday am   read  \phantom{``}{Crime and Punishment}  {} book-\textsc{nom} me-\textsc{dat} \textsc{refl} really liked\\ 
	\glt `Yesterday I read \textit{Crime and Punishment} -- I really liked the book.'
	\z 
	\vspace{-.5\baselineskip}\largerpage[2]%
	\ea \label{ex:despic:6}
	\ili{Turkish} \\
	\gll {D\"un}         {``Su\c c      ve   Ceza''}          {okudum} --   {kitap} {harikayd\i}.\\
	yesteday \phantom{``}{Crime and Punishment} read-\textsc{pst} {} book terrific-\textsc{pst}\\ 
	\glt `Yesterday I read \textit{Crime and Punishment}. The book was terrific.'
	\z 
	
	
	As shown in (\ref{ex:despic:7}--\ref{ex:despic:11}), similar holds for \ili{Mandarin}, \ili{Japanese} and \ili{Hindi}, also languages without \isi{definite articles} (note that \ili{Mandarin} and \ili{Japanese} do not mark \isi{number}, which will become relevant in  \sectref{sec:despic:3} and \sectref{sec:despic:4}). In \ili{Mandarin} examples in \REF{ex:despic:7}, bare nouns \textit{shu} `book' and \textit{ta} `tower' are used to refer anaphorically to \textit{Crime and Punishment} and \textit{Oriental Pearl}, respectively. In \REF{ex:despic:8}, the bare noun \textit{mao} `cat' is referring to the NP in the antecedent clause. Japanese examples in \REF{ex:despic:9} illustrate the same point: \textit{hon} `book' in \REF{ex:despic:9a} refers to \textit{Crime and Punishment}, while \textit{roojin} `old man' in \REF{ex:despic:9b} refers to the \is{proper names}proper name \textit{Yahachi}. Examples from \ili{Hindi} are given in \REF{ex:despic:10} and \REF{ex:despic:11}. Now, although \is{anaphora}anaphoric readings with bare nouns are available in these languages, it should be noted that nouns with \isi{demonstratives} or simple pronouns are preferred in many contexts, for a number of pragmatic and \isi{discourse} reasons, which I will not discuss here. What is crucial is that such use of bare nouns in languages like \ili{English} is disallowed regardless of discourse/context properties (that is, \is{bare singulars}bare singular nouns are in general ungrammatical in \ili{English}).
	
	
	\ea \label{ex:despic:7} 
	\ili{Mandarin}
	\ea \label{ex:despic:7a}
	\gll 
	{Wo} {kan} {le} {Zuiyufa} {Shu} {zai} {zhuo} {zi-shang.} \\
	I read \textsc{asp} {Crime and Punishment} book be at table-\textsc{top} \\
	\glt `I read \textit{Crime and Punishment.} The book is on the table.'
	
	\ex \label{ex:despic:7b}
	\gll 
	{Wo} {canguan} {le} {dongfangmingzhu}. {Ta} {hen} {gao.} \\
	I visit \textsc{ptcp} {Oriental Pearl} tower very tall \\
	\glt `I visited the Oriental Pearl. The tower is high.'
	\z 
	\z\vspace{-.5\baselineskip}
	
	\ea \label{ex:despic:8}
	\ili{Mandarin} \\
	\gll 
	{Wo} {kanjian} {yi-zhi} {mao}. {Mao} {zai} {huayuan-li.} \\
	I see one-\textsc{clf} cat cat at garden-inside \\ 
	\glt `I see a cat. The cat is in the garden.' \citep[403]{Dayal2004}  
	\z\vspace{-.5\baselineskip} 
	\largerpage
	\ea \label{ex:despic:9}
	\ili{Japanese} \\
	\ea \label{ex:despic:9a}
	\gll
	{Kinou} {``Tsumi to Batsu''-o} {yonda}. {Hon-wa} {subarashikatta}. \\
	yesterday \phantom{``}{Crime and Punishment}-\textsc{acc} read-\textsc{pst} book-\textsc{top} fantastic-\textsc{pst} \\ 
	\glt `Yesterday I read \textit{Crime and Punishment}. The book was fantastic.' 
	
	\ex \label{ex:despic:9b}
	\gll 
	{Yahachi-o} {miru-to}, {roojin-wa} {damatte} {unazuita}. \\
	Yahachi-\textsc{acc} see-when {old man-\textsc{top}} silently nodded \\ 
	\glt `When I saw Yahachi, the old man silently nodded.' \citep[14]{Fujisawa1992}
	\z
	\z 
	
	\ea \label{ex:despic:10}
	\ili{Hindi} \\ 
	\gll 
	{Kal} {mei-ne} {Crime and Punishment} {pari} {aur} {kitaab} {bariya} {hai.} \\
	yesterday I-\textsc{erg} {Crime and Punishment} read and book excellent is \\ 
	\glt `Yesterday I read \textit{Crime and Punishment} and the book is excellent.'
	\z
	
	\ea \label{ex:despic:11}
	\ili{Hindi} \\
	\gll
	{Kuch} {bacce} {andar} {aaye}. {Bacce} {bahut} {khush} {the.} \\
	some children inside came children very happy were \\ 
	\glt `Some children came in. The children were very happy.' \citep[403]{Dayal2004}
	\z 
	
	Consider now bare \isi{mass nouns}. When they are used in a kind-denoting context they \textit{cannot} be
	used anaphorically in these languages. For example, \textit{meyve} `fruit' in \REF{ex:despic:12} cannot pick out \textit{\"uz\"um} `grapes' in the antecedent clause, just like \textit{vo\'ce} `fruit' cannot refer to \textit{gro\v z\dj e} `grapes' in \REF{ex:despic:13}. They only have the implausible general meaning -- the second clause in these examples can be interpreted only as a statement about fruit in general, not about a particular kind of fruit (grape) introduced in the antecedent clause. 
	
	
	\ea \label{ex:despic:12}
	\ili{Turkish} \\
	\gll 
	{\"Omr\"um} {boyunca} {\"uz\"um} {yeti\c stirdim}. {\textnormal{\#}{\op}Bu{\cp}} {meyve} {her\c seyim} {oldu}. \\
	{my life} throughout grape produce \phantom{(\#}this fruit {my everything} became \\ 
	\glt `I have been producing grapes my whole life. (This) fruit is everything to me.' \\ 
	$\rightarrow$ * if \textit{meyve} `fruit' is anteceded by \textit{\"uz\"um} `grapes' \\
	$\rightarrow$ OK if \textit{bu meyve} `that fruit' is anteceded by \textit{\"uz\"um} `grapes'
	\z 
	
	
	\ea \label{ex:despic:13}
	\il{Serbo-Croatian!Serbian}Serbian \\
	\ea \label{ex:despic:13a}
	\gll
	{Na\v se} {mesto} {ve\'c} {generacijama} {proizvodi} {belo} {gro\v z\dj e}. {Sve} {dugujemo} {\textnormal{\#}{\op}tom{\cp}} {vo\'cu}. \\
	our town already generations produces white grape everything owe \phantom{\#}(that) fruit-\textsc{dat} \\ 
	\glt `Our town has been producing white grapes for generations. We owe everything to (that) fruit.' \\ 
	$\rightarrow$ * if \textit{vo\'cu} `fruit' is anteceded by \textit{gro\v z\dj e} `grapes' \\
	$\rightarrow$ OK if \textit{tom vo\'cu} `that fruit' is anteceded by \textit{gro\v z\dj e} `grapes'
	
	\ex \label{ex:despic:13b}
	\gll 
	{\ldots} {\textnormal{\#}{\op}To{\cp}} {vo\'ce} {je} {jako} {ukusno}.  \\
	{} \phantom{\#{\op}}that fruit is very tasty \\
	\glt `\ldots (That) fruit is very tasty.' \\
	$\rightarrow$ * if \textit{vo\'ce} `fruit' is anteceded by \textit{gro\v z\dj e} `grapes' \\
	$\rightarrow$ OK if \textit{to vo\'ce} `that fruit' is anteceded by \textit{gro\v z\dj e} `grapes' 
	\z
	\z 
	
	In order to get the \is{anaphora}anaphoric reading, a \is{demonstratives}demonstrative must be used. These examples are minimally different from those in (\ref{ex:despic:5}--\ref{ex:despic:6}), which in contrast do allow anaphoric interpretation of the bare noun. Also note that whether \textit{vo\'ce} `fruit' in \il{Serbo-Croatian!Serbian}Serbian is in the subject or object position is irrelevant for \isi{anaphoricity}.\footnote{\ili{Turkish}, however, has differential object marking and in accusative case makes a morphological distinction between \is{specificity}specific and non-specific objects \citep[e.g.][]{Enc1991}.}\textsuperscript{,}\footnote{Other \isi{mass nouns} behave in a similar way; e.g. \textit{vino} `wine' in (i.b) below cannot be anteceded by \textit{Vranac} (a special type of wine) in (i.a) without the \is{demonstratives}demonstrative. Both \textit{vo\'ce} `fruit' and \textit{vino} `wine' in Serbian in general require a \is{classifier phrases}classifier phrase (like truckload of  or glass of) or a measure phrase (like lot of) for counting, which is typical of mass nouns. At the same time, they are very useful here because they have well-established subclasses/subtypes (in contrast to, say, \textit{sand}), which could in principle serve as pragmatically plausible antecedents. The fact that the \is{anaphora}anaphoric relationship cannot be formed in these examples, thus, cannot be due to pragmatic factors.
		
		\ea \label{ex:despic:n5}
		\il{Serbo-Croatian!Serbian}\textnormal{Serbian}
		\ea \label{ex:despic:n5a}
		\gll 
		{Na\v se} {mesto} {ve\'c} {generacijama} {proizvodi} {``Vranac''}. \\
		{our} {town} {already} {generations} s{produces} \phantom{``}{Vranac} \\
		\glt {`Our town has been producing Vranac for generations.'}
		
		\ex  \label{ex:despic:n5b}
		\gll 
		{Sve} {dugujemo} \textnormal{\#}{\op}tom{\cp} {vinu}. \\   
		{everything} {owe} \phantom{{\#}}{(that)} {wine} \\
		\glt {`We owe everything to (that) wine.'}
		\z
		\z 
	}
	
We see a similar pattern in \ili{Mandarin}, \ili{Japanese} and \ili{Hindi}, as illustrated with some examples below. All of my informants find a strong contrast in the availability of \is{anaphora}anaphoric reading between examples (\ref{ex:despic:7}--\ref{ex:despic:11}), on the one hand, and the ones in (\ref{ex:despic:12}--\ref{ex:despic:16}), on the other. Just like in (\ref{ex:despic:12}--\ref{ex:despic:13}), the second clause in (\ref{ex:despic:14}--\ref{ex:despic:16}) below can be interpreted only as a general statement about fruit, not as a statement about a particular kind of fruit mentioned in the antecedent clause; i.e. `Fruit is our life' in \REF{ex:despic:14} cannot be interpreted as `Apples are our life'.
	
	\ea \label{ex:despic:14}
	\ili{Mandarin} \\ 
	\gll
	{Women} {shidai} {zhong} {pingguo} {shuiguo} {jiu} {shi} {women} {de} {ming}. \\
	we generation grow apple fruit \textsc{ptcp} is we \textsc{gen} life \\
	\glt `We have been growing apples for generations. Fruit is our life.'
	\z 
	
	\ea \label{ex:despic:15} \largerpage[4]
	\ili{Japanese} \\ 
	\gll
	{Watashitachi-wa} {daidai} {budou-o} {sodatetekita}. {\textnormal{\#}{\op}Kono{\cp}} {Kudamono-wa} {subarashi}. \\
	we-\textsc{top} for-generations grapes-\textsc{acc} {have grown} \phantom{\#(}this fruit-\textsc{top} fantastic \\ 
	\glt `We have been growing grape for generations. This fruit is fantastic.'
	\z 
	
	\ea \label{ex:despic:16}
	\ili{Hindi} \\
	\gll 
	{Mei-ne} {angur} {ki} {kheti} {mei} {saari} {jeevan} {biaayi} {hai} {aur} {\textnormal{\#}{\op}ye{\cp}} {phal-ne} {mujh-ko} {ameer} {bana} {dija} {hai.} \\
	I-\textsc{erg} grapes of farming in all life spend is and \phantom{\#(}this fruit-\textsc{erg} me-\textsc{acc} rich make-\textsc{pst} give-\textsc{pst} is \\ 
	\glt `I have been growing grapes all my life and the fruit has made me rich.'
	\z 
	
	Now, a \is{mass nouns}mass noun with a kind reading can be used anaphorically in \ili{English}, if it is accompanied
	by the \is{definite articles}definite article. Consider, for instance, \REF{ex:despic:17} in which `the fruit' is anteceded by `grapes'.
	Many speakers I have consulted find the \is{anaphora}anaphoric reading in \REF{ex:despic:17} perfectly possible, although
	some of them would still prefer the \is{demonstratives}demonstrative `that' instead of `the', presumably for the
	same type of reasons mentioned in the discussion of (\ref{ex:despic:5}--\ref{ex:despic:11}).\footnote{What seems to be clear is that the bare noun \textit{fruit} in (i) has no anaphoric potential; i.e. the second clause in (i) is interpreted as a general statement about fruit, which is exactly the kind of judgment speakers of languages without
		articles discussed here have for (\ref{ex:despic:12}--\ref{ex:despic:16}).
		\ea{\label{ex:despic:n6}
			\textit{We have been growing grapes for generations -- and you know, we have made millions on fruit.}}
		\z\vspace*{-\baselineskip}}\textsuperscript{,}\footnote{Similar facts about \isi{anaphoricity} of \isi{mass nouns} interpreted as kinds have also been observed by \citet[ft. 43, 435--436]{Dayal2004}, who points out that ``\ldots mass terms can occur with a \is{definites}definite if anaphorically linked to an antecedent, even if such anaphoricity leads to \isi{kind reference}, as in (i).''
		\ea \textit{Patients need medicine and food. {\op}The{\cp} medicine fights the disease and {\op}the{\cp} food builds up strength.}
		\label{ex:despic:n7}
		\z
		See \sectref{sec:despic:5} for a discussion of kinds in connection with the distinction between \is{uniqueness}unique and \is{familiarity}familiar definites.}\largerpage[-2]
	
	
	\ea \label{ex:despic:17}
	\textit{We have been growing grapes for generations -- and you know, we have made millions on
		the fruit.}
	\z
	
	Why would this be the case? Why would the existence of \is{kind reference}kind-reference affect the
	\is{anaphora}anaphoric potential of a bare noun in \is{article-less languages}article-less languages in such a way? This state of affairs
	seems to raise some non-trivial questions for the basic\pagebreak{} version of the \is{Universal DP (Approach)}UDP approach. In
	particular, if the covert version of the definite article, which is overt in \ili{English}, is responsible for
	the definite reading of the bare nouns in (\ref{ex:despic:5}--\ref{ex:despic:11}) (e.g. \textit{knjiga} `book'), why cannot it produce the
	same effect in (\ref{ex:despic:12}--\ref{ex:despic:16}) (with the bare noun \textit{gro\v z\dj e} `fruit') given that `the fruit' in \ili{English} \REF{ex:despic:17} has the definite article? In the UDP all languages have identical underlying structure in the
	nominal domain, and the phonologically \is{null determiners}null/covert D in \il{Serbo-Croatian!Serbian}Serbian or \ili{Turkish} should in principle
	perform the same function as its overt version in languages like \ili{English}; e.g. it assigns the
	definite/\is{anaphora}anaphoric interpretation to, say, \textit{knjiga} or \textit{kitap} `book' in (\ref{ex:despic:5}--\ref{ex:despic:6}), just like the overt
	\is{definite articles}article \textit{the} does in \ili{English}. One could assume that, for some reason, covert versions of D are more limited in meaning, and cannot combine with, for instance, kind-denoting nouns, but this would have to be independently supported. That is, these additional assumptions would have to explain why the opposite situation does not arise.
	
	Note that the real culprit here is the presence of \is{kind reference}kind-reference. In other
	words, \is{bare nouns}\is{mass nouns}bare mass nouns in \is{article-less languages}languages without \isi{definite articles} \textit{can} have anaphoric readings in the
	absence of kind interpretation. This is shown in (\ref{ex:despic:18}--\ref{ex:despic:22}): in all of these examples the antecedent
	clause describes a particular object-level entity, and the bare mass nouns in the second clause
	(`fruit' or `wine') can be anaphorically anteceded by it. This is true even though these examples
	are overall very similar to those in (\ref{ex:despic:12}--\ref{ex:despic:16}) -- the only difference is that the latter force the
	kind-level interpretation.\largerpage
	That is, bare mass nouns can have both kind-level and object-level interpretation, but the anaphoric reading is possible only in the latter case (see \citealt{Chierchia1998}: \S4 and references therein) for the kind vs. object level distinction). Compare (\ref{ex:despic:18}a--b) with \REF{ex:despic:13}, for instance. As discussed in \citet{Chierchia1998}, from an intuitive, pretheoretical point of view, kinds are seen as regularities that occur in nature -- although they are similar to individuals, ``their spatiotemporal manifestations are typically ``discontinuous'''' \citep[348]{Chierchia1998}. That is, a kind can be identified in any given world with the totality or sum of its instances. It may lack instances in a world/situation (e.g. \textit{dodo}), but something that is necessarily instantiated by just one individual (e.g. \textit{Noam Chomsky}), would not qualify as a kind (this contrast will in fact play one of the central roles in the explanation offered in the next section). So in \REF{ex:despic:13}, for example, we interpret the mass noun as an idealized sum of its instances with discontinuous spatiotemporal manifestations, which is highlighted by the use of the expression `for generations' -- we clearly do not interpret it as a particular object-level instantiation of the mass noun (e.g. \textit{a bowl of fruit}). In \REF{ex:despic:18b}, on the other hand, we have exactly that -- a \is{specificity}specific, object-level interpretation of the mass noun, with a specific quantity, at a specific time/situation. And exactly in this case the anaphoric relationship can be established. 
	
	Also, as in the case of examples in (\ref{ex:despic:5}--\ref{ex:despic:11}), an NP with a \is{demonstratives}demonstrative or a simple pronoun might be preferred in (\ref{ex:despic:18}--\ref{ex:despic:22}), but the bare noun is nevertheless quite possible. What is important is that there is a substantial contrast between this set of examples and those in (\ref{ex:despic:12}--\ref{ex:despic:16}), in which the anaphoric reading is not\largerpage available without the demonstrative. 
	
	\ea \label{ex:despic:18} 
	\il{Serbo-Croatian!Serbian}Serbian \\
	\ea \label{ex:despic:18a} 
	\gll
	{Ju\v ce} {sam} {po} {prvi} {put} {pojeo} {nekoliko} {braziliskih} {papaja}. {Vo\'ce} {je} {zaista} {fantasti\v cno!} \\
	yesterday am at first time ate {a few} Brazilian papaya fruit is truly fantastic \\ 
	\glt `Yesterday I ate a few Brazilian papayas for the first time. The fruit is fantastic!'
	
	\ex \label{ex:despic:18b} 
	\gll 
	{Danas} {sam} {kupio} {malo} {gro\v z\dj a}, {hleb} {i} {mleko}. {Vo\'ce} {sam} {stavio} {un} {fri\v zider} {a} {sve} {ostalo} {na} {sto}. \\
	today am bought bit grapes bread and milk fruit am put in fridge and all else on table \\ 
	\glt `Today I bought some grapes, bread and milk. I put the fruit in the fridge and the rest on the table.' \\
	$\rightarrow$ OK if \textit{vo\'ce} `fruit' is anteceded by \textit{gro\v z\dj e} `grapes' 
	
	\ex \label{ex:despic:18c} 
	\gll
	{Sa} {prijateljima} {sam} {ju\v ce} {popio} {tri} {fla\v se} {Dom Perinjon-a}. {Vino} {je} {zaista} {fantasti\v cno.}\\
	with friends am yesterday drank three bottles {Dom Perignon} wine is truly fantastic \\ 
	\glt `I drank three bottles of Dom P\'erignon yesterday. The wine is truly fantastic.' \\
	$\rightarrow$ OK if \textit{vino} `wine' is anteceded by \textit{Dom P\'erignon} 
	\z
	\z
	
	The examples below behave the same way:
	
	\ea \label{ex:despic:19}
	\ili{Turkish} \\
	\gll
	{D\"un} {\"uz\"um}, {peynir} {ve} {s\"ut} {ald\i m}. {Meyve} {pahal\i yd\i} {ama} {di\u gerleri} {hesapl\i yd\i}. \\
	yesterday grape cheese and milk buy-1.\textsc{pst} fruit expensive-\textsc{pst} but rest affordable-\textsc{pst} \\ 
	\glt `I bought grapes, cheese and milk yesterday. The fruit was expensive but the rest was affordable.'
	\z 
	
	\ea \label{ex:despic:20} 
	\ili{Mandarin} \\ 
	\ea \label{ex:despic:20a} 
	\gll
	{Wo} {ba} {na} {dai} {pinguo} {fang} {dao} {zhuozi-shang}, {danshi} {shuiguo} {yixia zi} {jiu} {diao-chulai} {le.}  \\
	I ba that packet apple put towards table-\textsc{top} but fruit all-of-a-sudden \textsc{ptcp} fall-out \textsc{asp} \\ 
	\glt `I put the packet with apples on the table, but the fruit immediately\largerpage fell out of it.'
	
	\ex \label{ex:despic:20b} 
	\gll
	{Wo} {mai} {le} {san} {ge} {pingguo} {niunai} {he} {baozhi} {shuiguo} {hen} {gui,} {qita} {dongxi} {dou} {hen} {pianyi.} \\ 
	I bought \textsc{asp} three \textsc{clf} apple milk and newspaper fruit very expensive other things all very cheap \\
	\glt `I bought three apples, milk and newspapers. The fruit was expensive; the other things were cheap.'\footnote{\label{ft8}Contrastive particle \textit{jiu} before `fruit' in \REF{ex:despic:20b} makes the \is{anaphora}anaphoric relation clearer, but it is not necessary -- \REF{ex:despic:20b} is fine without it. Also, \citet{Jenkstoappear} observes that \ili{Mandarin} seems to make a principled distinction between \is{unique definites}unique and \isi{anaphoric definites} (e.g. \citealt{Schwarz2009}); while unique definites are realized as bare nouns, anaphoric definites are realized with a \is{demonstratives}demonstrative, except in subject positions, where bare nouns can also be interpreted anaphorically. For this reason, in all \ili{Mandarin} examples in this paper bare nouns are located in subject positions.} 
	\z 
	\z 
	
	\ea \label{ex:despic:21}
	\ili{Japanese} \\
	\ea 
	\gll 
	{Tana}-{no} {ue}-{no} {ringo-o} {miruto}, {kudamono}-{wa} {sudeni} {kusatte} {ita}. \\
	shelf-\textsc{gen} {top}-\textsc{gen} apple-\textsc{acc} {saw time} fruit-\textsc{top} already rotten was \\ 
	\glt `When I saw the apple on the shelf, the fruit was already rotten.'
	
	\ex 
	\gll 
	{Kinou} {budou} {to} {chiizu} {to} {gyuunyuu-o} {katta}. {Kudamono-wa} {teeburu-ni} {oite}, {hoka-wa} {reizouku-ni} {ireta}. \\
	yesterday grape and cheese and milk-\textsc{acc} bought fruit-\textsc{top} table-at put-and rest-\textsc{top} fridge-in insert-\textsc{pst} \\
	\glt `Yesterday I bought grapes, cheese and milk. I put the fruit on the table and the rest in the fridge.'
	\z 
	\z\largerpage[-2]\pagebreak
	
	\ea \label{ex:despic:22}
	\ili{Hindi} \\
	\gll 
	{Aaj} {mei-ne} {angur}, {dudh}, {aur} {paneer} {kharidi} {aur} {phal} {mehenga} {tha} {par} {baki} {sab} {theek-thak} {tha.} \\
	today I-\textsc{erg} grapes milk and cheese bought and fruit expensive was but rest all okay was \\ 
	\glt `I bought grapes, milk, and cheese today and the fruit was expensive but the rest was okay.'
	\z 
	
	I argue in the next section that this contrast follows from \citegen{Dayal2004} approach.
	
	\section{Solution: Dayal (2004)} \label{sec:despic:3} 
	
	\citegen{Dayal2004} work is based on \citet{Chierchia1998} and \citet{Carlson1977}, who take \ili{English} \isi{bare plurals} to \is{kind reference}refer to kinds (as opposed to \citealt{Wilkinson1991, Diesing1992, KrifkaGerstnerLink1993, Kratzer1995}, who take bare plurals as ambiguous between kind terms and \isi{indefinites}). \citet{Chierchia1998}, in particular, attempts to derive the typology and distribution of bare nominals across different types of languages. \citet{Chierchia1998} focuses on two parameters: (i) presence vs. absence of \isi{determiners}, and (ii) presence vs. absence of \isi{number} morphology. \citet{Dayal2004}
	modifies \citegen{Chierchia1998} theory, most importantly in the way languages with number
	morphology but without determiners should be analyzed (see \sectref{sec:despic:4}), but many core assumptions are adopted from \citet{Chierchia1998}. I will provide a brief overview of two assumptions of \citegen{Chierchia1998} system that are most important for the purposes of this paper. The first assumption is that languages may employ a number of \is{type shifting}type-shifting operations, a subset of
	which is given in \REF{ex:despic:23}:
	
	\ea \label{ex:despic:23}
	\ea \label{ex:despic:23a}
	$\langle e,t\rangle=(^\cap,\iota,\exists)\Rightarrow\langle e\rangle/\langle\langle e,t\rangle t\rangle$ 
	
	\ex \label{ex:despic:23b}
	$\iota$: \hfill $\lambda P\;\iota x[Ps(x)]$\hspace{75pt}
	
	\ex \label{ex:despic:23c}
	$^\cap$: \hfill $\lambda P\lambda s\;\iota x[Ps(x)]$\hspace{75pt}
	
	\ex \label{ex:despic:23d}
	$\exists$: \hfill $\lambda P\lambda Q\exists x[Ps(x) \ni Qs(x)]$\hspace{75pt}
	
	\z
	\sn \citep[413]{Dayal2004}
	\z 
	
	The main idea is that \ili{English} \isi{bare plurals} are derived via a nominalization operation (`down') $^\cap$ , defined as in \REF{ex:despic:23c} (like other \isi{common nouns}, they start life as type $\langle s, \langle e, t\rangle\rangle$). $^\cap$ is a function from properties to functions from situations to the \is{maximality}maximal entity that satisfies that property in
	that situation. The function is partial in that it requires the kind term to pick out distinct maximal individuals across situations, thereby capturing the inherently intensional nature of the term. As shown in \REF{ex:despic:24}, this term can be a direct argument of a \is{kind-level predicates}kind-level predicate:
	
	\ea \label{ex:despic:24}
	\textit{Dodos are extinct.} 
	\z 
	
	\is{Derived Kind Predication|(}
	In object-level contexts, however, further operations (see \ref{ex:despic:25a}) come into play to repair the sort mismatch. This repair (\textsc{derived kind predication} -- DKP; see \citealt[364]{Chierchia1998}, \citealt[399]{Dayal2004}) involves the introduction of \is{existential quantifier}existential quantification over the instantiations of the kind in a given situation. It draws on the inverse of $^\cap$, the predicativizer or `up', operation $^\cup$ (see \ref{ex:despic:25b}) to take kinds and return their instantiation sets in a given situation:
	
	\ea \label{ex:despic:25}
	\ea \label{ex:despic:25a}
	DKP: If $P$ applies to objects and $k$ denotes a kind, then $P(k)=\exists x[^\cup k(x)\wedge P(x)]$
	
	\ex \label{ex:despic:25b}
	$^\cup: \lambda k_{\langle s,e\rangle}\lambda x[x\leq k_s]$ 
	
	\ex \label{ex:despic:25c}
	Dogs didn't bark $=\neg\textrm{bark}(^\cap \textrm{dogs})=\textrm{DKP}\Rightarrow\neg\exists x[^{\cup\cap}\textrm{dogs}(x)\wedge\textrm{bark}(x)]$
	
	\z 
	\z 
	
	The source of \is{existential quantifier}existential quantification over instances of the kind in episodic sentences is an automatic, local adjustment triggered by a type mismatch. \is{bare plurals}Bare plurals are in many ways different from \is{indefinites}indefinite singulars (e.g. \citealt{Carlson1977}), for instance in scope:
	
	\ea \label{ex:despic:26}
	\ea \label{ex:despic:26a}
	\textit{John didn't read a book.} \hfill$\neg\exists$ and $\exists\neg$\hspace{75pt}
	
	\ex \label{ex:despic:26b}
	\textit{John didn't read books.} \hfill\textit{only:} $\neg\exists$\hspace{75pt} 
	
	\z 
	\z 
	
	The \is{indefinites}indefinite denotes a \is{generalized quantifiers}generalized quantifier, and it can therefore take wide or \isi{narrow scope} with respect to negation, as shown in \REF{ex:despic:26a}. The bare plural, on the other hand, is a kind term, which is a direct argument of the \is{kind-level predicates}predicate (see \ref{ex:despic:25c}). Thus, whenever a kind (in an episodic
	frame) fills an object-level slot, the type of the element in question is automatically adjusted by introducing a local existential quantification over instances of the kind. The existential introduced by DKP therefore necessarily takes scope below negation. One prediction of this system is that non-kind denoting bare plurals should behave like regular existentially quantified
	NPs. For instance, they could take different scope with respect to negation: this prediction appears to be borne out \citep{Carlson1977, Chierchia1998}:\is{Derived Kind Predication|)}
	
	\ea \label{ex:despic:27}
	\ea[*]{ \label{ex:despic:27a}
		\textit{Parts of this machine are widespread.}}
	\ex[]{ \label{ex:despic:27b}
		\textit{John didn't see parts of this machine.} \hfill $\neg\exists$ and $\exists\neg$\hspace{75pt}}
	\z 
	\sn\citep[419]{Dayal2004}
	\z 
	
	\textit{Parts of this machine} in \REF{ex:despic:27a} is not compatible with true kind predication, presumably because the \is{definites}definite inside the NP would force the extension of the \is{noun phrases}noun phrase to be constant across worlds. But, as shown in \REF{ex:despic:27b}, this bare plural can now interact with negation, a diagnostic that separates \isi{indefinites} from kind terms. Compare then \REF{ex:despic:27} to \REF{ex:despic:28}:
	
	\ea \label{ex:despic:28} 
	\ea \label{ex:despic:28a} 
	\textit{Spots on the floor are a common sight.}
	\ex \label{ex:despic:28b} 
	\textit{John didn't see spots on the floor.} \hfill\textit{only:} $\neg\exists$\hspace{75pt}
	\z 
	\z 
	
	In \REF{ex:despic:28}, possibility of \isi{kind reference} results in the loss of scope interaction. The \is{bare plurals}bare plural \textit{spots on the floor} in \REF{ex:despic:28a} is compatible with the \is{kind-level predicates}kind-level predicate, which indicates that it has a kind reference. As a result, it can only have the low scope in \REF{ex:despic:28b}. Thus, this sort of
	system neatly explains this state of affairs. What needs to be assumed then is that $^\cap$ (see \ref{ex:despic:23c}) should apply whenever it can; i.e. it should take precedence over $\exists$ (see \ref{ex:despic:23d}). In \REF{ex:despic:27b} $^\cap$ is unavailable, and therefore $\exists$ applies, as confirmed by the scope ambiguity. \citet{Chierchia1998} thus ranks $^\cap$ above $\exists$ arguing that the former is simpler, since it does not introduce \is{quantifiers}quantificational force (see \ref{ex:despic:29}).
	
	\ea \label{ex:despic:29}\is{Meaning Preservation}
	\textbf{Meaning Preservation:} $^\cap>\{\iota,\exists\}$ \citep[419]{Dayal2004}
	\z 
	
	The immediate question that arises here concerns the availability of \is{iota operator|(}$\iota$. In particular, if $^\cap$ is not available in \REF{ex:despic:27} and $\iota$ (see \ref{ex:despic:23b}) is an available type-shifting operation, why cannot \textit{parts of this machine} be interpreted as definite? This brings us to the second important component of the \citet{Chierchia1998}/\citet{Dayal2004} system called \is{Blocking Principle}\textsc{blocking principle}, which is given in \REF{ex:despic:30}: 
	
	\ea \label{ex:despic:30}
	\textbf{Blocking Principle (Type Shifting as Last Resort)} \\
	For any \is{type shifting}type-shifting operation $\phi$ and any $X$: $^{*}\phi(X)$ if there is a \is{determiners}determiner D such that for any set $X$ in its domain, D$(X)=\phi(X)$.  \citep[216]{Dayal2004}
	\z
	
	The intuition behind this principle is that for considerations of economy lexical items must be exploited to the fullest before covert type-shifting operations can be used. So, since \ili{English} has \textit{the}, which is the lexical version of $\iota$, it will always block $\iota$. Thus, in \ili{English}, bare plurals can
	avail of $^\cap$ (or $\exists$ when $^\cap$ is blocked for independent reasons, as in \ref{ex:despic:27b}), but not $\iota$, because of the
	presence of the lexical determiner \textit{the}. This in turn also explains the following contrast between \ili{Hindi} (a determiner-less language) and \ili{English} \citep[417]{Dayal2004}:
	
	\ea \label{ex:despic:31}
	\ea \label{ex:despic:31a}
	\ili{English} \\ 
	\textit{Some children came in. \textnormal{*}{\op}The{\cp} children were happy.}
	\ex \label{ex:despic:31b}
	\ili{Hindi} \\
	\gll 
	{Kuch} {bacce$_\text{\textnormal{i}}$} {aaye}. {Bacce$_\text{\textnormal{i}}$} {bahut} {khush} {lage.} \\
	some children came children very happy seemed \\ 
	\glt `Some children came. The children seemed very happy.' 
	\z 
	\z 
	
	While bare nouns in \ili{Hindi} can be used anaphorically, as shown in \REF{ex:despic:31b}, this is not possible in \ili{English} (see \ref{ex:despic:31a}). This is because there is no lexical \is{definite determiners}definite determiner in \ili{Hindi}, which makes $\iota$ as well as $^\cap$ available options for \isi{bare nominals}. For this reason, \textit{bacce} `children' in \REF{ex:despic:31b} can be interpreted as definite. In \ili{English}, on the other hand, \isi{bare plurals} can avail of $^\cap$ but not $\iota$. $^\cap$ is a function whose extension varies from situation to situation, while $\iota$ is a constant function to a
	contextually anchored entity. Thus, the bare noun \textit{children} in \REF{ex:despic:31a} cannot be interpreted as definite/anaphorically.
	In other words, the underlying assumption of \citet{Chierchia1998} and \citet{Dayal2004} about $^\cap$ is that it manufactures a kind out of a property (i.e. an intensional entity) by taking the largest member of its extension at any given world; it creates a saturated object with concrete, but possibly spatiotemporally discontinuous manifestations. But $^\cap$ cannot establish an \is{anaphora}anaphoric relationship with a contextually anchored entity. Only $\iota$, which selects the greatest element from the \textit{extension} of the predicate, can do this. That is, even though $^\cap$ (\textit{nom}) is simply an intensional counterpart of $\iota$, ``\ldots \textit{nom} cannot be used referentially'' \citep[1103]{Dayal2011}. In \sectref{sec:despic:5} I offer some remarks on how \citegen{Dayal2004} typological observations about the relationship between $^\cap$ and $\iota$ relate to Schwarz's (\citeyear{Schwarz2009,Schwarz2013}) typology of \isi{definiteness marking} (i.e. \textit{strong} vs. \textit{weak} definite articles). 
	
	Now, since in \citet{Dayal2004} mass kinds are treated on a par with plural kinds, we have the solution to the puzzle introduced in \sectref{sec:despic:2}. Recall first that a \is{bare singulars}bare singular noun in an \is{article-less languages}article-less language like \il{Serbo-Croatian!Serbian}Serbian can be interpreted as definite. This is expected: $\iota$ is allowed, since there is no lexical \is{articles}article to block it. This is illustrated by \REF{ex:despic:5}, repeated below as \REF{ex:despic:32}:
	
	\ea \label{ex:despic:32}
	\il{Serbo-Croatian!Serbian}Serbian \\
	\gll 
	{Ju\v ce} {sam} {pro\v citao} {{Zlo\v cin i Kaznu}} {--} {knjiga} {mi} {se} {zaista} {svidela.} \\
	yesterday am read {Crime and Punishment} {} book-\textsc{nom} me \textsc{refl} really liked \\ 
	\glt `Yesterday I read \textit{Crime and Punishment} -- I really liked the book.'
	\z 
	
	However, a bare \is{mass nouns}mass noun in a kind-denoting context cannot be interpreted as \is{definites}definite in
	language like \il{Serbo-Croatian!Serbian}Serbian, as shown in \REF{ex:despic:33} (=\ref{ex:despic:13a}) below. 
	
	\ea \label{ex:despic:33}
	\il{Serbo-Croatian!Serbian}Serbian \\
	\gll 
	{Na\v se} {mesto} {ve\'c} {generacijama} {proizvodi} {belo} {gro\v z\dj e}. {Sve} {dugujemo} {\textnormal{\#}{\op}tom{\cp}} {vo\'cu}. \\
	our town already generations produces white grape everything owe \phantom{\#}(that) fruit \\ 
	\glt `Our town has been producing white grapes for generations. We owe everything to (that) fruit.' \\
	$\rightarrow$ * if \textit{vo\'cu} `fruit' is anteceded by \textit{gro\v z\dj e} `grapes' \\
	$\rightarrow$ OK if \textit{tom vo\'cu} `that fruit' is anteceded by \textit{gro\v z\dj e} `grapes' 
	\z 
	
	This is exactly expected on this approach since kind-denoting terms must be derived via $^\cap$; thus, the bare noun \textit{vo\'ce} `fruit' in \REF{ex:despic:33} behaves similarly to the bare noun \textit{children} in \REF{ex:despic:31a} with respect to \isi{anaphoricity}/definiteness. But bare \isi{mass nouns} which do not denote kinds can avail of $\iota$ in languages like \il{Serbo-Croatian!Serbian}Serbian, because there is no lexical \is{determiners}determiner to block it. Therefore they can be interpreted as \is{definites}definite, as illustrated in \REF{ex:despic:34} (=\ref{ex:despic:18b}):
	
	\ea \label{ex:despic:34}
	\il{Serbo-Croatian!Serbian}Serbian \\
	\gll 
	{Danas} {sam} {kupio} {malo} {gro\v z\dj a}, {hleb} {i} {mleko}. {Vo\'ce} {sam} {stavio} {un} {fri\v zider} {a} {sve} {ostalo} {na} {sto}. \\
	today am bought bit grapes bread and milk fruit am put in fridge and all else on table \\ 
	\glt `Today I bought some grapes, bread and milk. I put the fruit in the fridge and the rest on the table.' \\
	$\rightarrow$ OK if \textit{vo\'ce} `fruit' is anteceded by \textit{gro\v z\dj e} `grapes' 
	\z 
	
	\citegen{Dayal2004} approach also makes some interesting predictions about the availability of definite interpretations for \is{bare plurals}\is{bare singulars}bare singular and plural (i.e. non-mass) kinds in languages without determiners. I discuss these predictions in \sectref{sec:despic:4} and show that they are borne out.
	
	\section{Predictions and consequences}  \label{sec:despic:4}\largerpage[-1]
	
	An important observation about languages with \isi{number} marking but no determiners, which is
	central to \citegen{Dayal2004} modification of \citegen{Chierchia1998} system, is that bare plurals in such languages behave more or less like \ili{English} bare plurals, but bare singulars are substantially different. Although bare singulars and bare plurals in such languages allow for kind as well as \is{anaphora}anaphoric readings, their existential reading, however, is distinct from that of regular \isi{indefinites}
	in two respects: (i) they cannot take wide scope over negation or other operators, and (ii) they cannot refer non-maximally. Thus, \is{bare nominals}bare NPs cannot be used in translating \REF{ex:despic:35b} or \REF{ex:despic:35c} to refer
	to a subset of the children mentioned in \REF{ex:despic:35a} \citep[1100]{Dayal2011}:
	
	\ea \label{ex:despic:35}
	\ea \label{ex:despic:35a}
	\textit{There were several children in the park.}
	\ex \label{ex:despic:35b}
	\textit{A child was sitting on the bench and another was standing near him.}
	\ex \label{ex:despic:35c}
	\textit{Some children were sitting on the bench, and others were standing nearby.} 
	\z 
	\z 
	
	So, even though there are no \is{definite determiners}definite or \is{indefinites}indefinite determiners in these languages, only readings associated with \isi{definites} are available to bare NPs. Dayal argues that this shows that the availability of covert \is{type shifting}type shifts is constrained, as proposed by \citet{Chierchia1998}, but that the correct ranking is as in \REF{ex:despic:36} not \REF{ex:despic:29} (note that both $^\cap$ and $\iota$ are simpler than $\exists$):
	
	\ea \label{ex:despic:36}\is{Meaning Preservation}
	\textbf{Revised Meaning Preservation}: $\{^\cap,\iota\}>\exists$ \citep[219]{Dayal2004} 
	\z 
	
	This is also motivated by the fact that the \ili{Hindi} version of \ref{ex:despic:27b} (i.e. \ref{ex:despic:37b}) does not allow a
	wide scope reading of \textit{parts of this machine}, even though this \is{bare plurals}bare plural is not compatible with true kind predication, as shown in \REF{ex:despic:37a}.
	
	\ea \label{ex:despic:37}
	\ili{Hindi} \\
	\ea[*]{ \label{ex:despic:37a}
	\gll {Is} {mashin} {ke} {TukRe} {aam} {haiN.} \\
	this machine of parts common are \\ 
	\glt `Parts of this machine are common.' }
	\ex[]{ \label{ex:despic:37b}
	\gll
	{Anu-ne} {is} {mashiin} {ke} {TukRe} {nahiiN} {dekhe.} \\
	Anu-\textsc{erg} this machine of parts not see \\
	\glt `Anu didn't see any/the parts of this machine.'} \citep[420]{Dayal2004} 
	\z 
	\z 
	
	Thus, given the revised ranking in \REF{ex:despic:36}, in the absence of $^\cap$ , the availability of \is{iota operator|)}$\iota$ blocks $\exists$. What one might take to be the frozen existential reading in \REF{ex:despic:37b} is, in fact, the (non-\is{familiarity}familiar) definite reading of a sentence with negation.\footnote{It seems rather clear that \is{bare nominals}bare NPs in languages like \ili{Hindi} are not true \isi{indefinites}, but there are cases for which the most natural translation into \ili{English} uses an indefinite \citep[1101]{Dayal2011}: 
		\ea \label{ex:despic:n9}
		\textnormal{\ili{Hindi}} \\
		\gll Lagtaa hai kamre meN cuhaa hai. \\
		{seems} {be} {room} {in} {mouse} {be} \\ 
		\glt {`There seems to be a mouse in the room.'}
		\z
		Dayal argues that covert and overt \is{type shifting}type shifts agree on semantic operations but not on presuppositions. So, \ili{English} \is{definite articles}article \textit{the} \is{definiteness marking}encodes the \is{iota operator}operation $\iota$, which \ili{Hindi} bare NPs use to shift to type $\langle e\rangle$ covertly. Both of these variants entail \isi{maximality}/\isi{uniqueness}. In addition, the lexical definite article \textit{the} has a \isi{familiarity} requirement that \ili{Hindi} bare NPs do not. The assumption is that familiarity presuppositions are attached to lexical items, and that a language that does not have a lexical \is{definite determiners}definite determiner will not enforce familiarity presuppositions. This non-familiar maximal
		reading can then be confused with a true existential reading (see also \citealt{Heim2011}).} \citet{Dayal2004} also observes that \isi{bare singulars} are not trivial variants of \isi{bare plurals} in languages like \ili{Hindi}, and that these languages raise important questions about the connection between singular \isi{number} and \isi{kind reference}. For example, the \ili{Hindi} example in \REF{ex:despic:38a} has only the implausible reading whereby the same child is assumed to be playing everywhere. Its plural counterpart in \REF{ex:despic:38b}, however, readily allows for a plausible reading:
	
	\ea \label{ex:despic:38}
	\ili{Hindi} \\
	\ea[\#]{\label{ex:despic:38a}
		\gll 
		{CaaroN} {taraf} {bacca} {khel} {rahaa thaa.} \\
		four ways child was playing \\
		\glt `The (same) child was playing everywhere.'}
	\ex[]{ \label{ex:despic:38b}
		\gll 
		{CaroN} {taraf} {bacce} {khel} {rahe the.} \\
		four ways children were playing \\
		\glt `Children (different ones) were playing everywhere.' \citep[406]{Dayal2004}}
	\z 
	\z 
	
	In order to explain this contrast, Dayal argues that singular and plural kind terms differ in the way they relate to their instantiations, as illustrated by the following quote:\largerpage
	
	\begin{quote} 
		An analogy can be drawn with ordinary sum individuals \textit{the players} whose atomic parts are available for predication, and collective \isi{nouns} or groups like \textit{the team} which are closed in this respect: \textit{The players live in different cities} vs. \textit{*The team lives in different cities} \citep{Barker1992, Schwarzschild1996}. $^\cap$ applies only to plural nouns and yields a kind term that allows semantic access to its instantiations, analogously to sums. A singular kind term restricts such access and is analogous to collective nouns. \citep[1100]{Dayal2011}
	\end{quote} 
	
	Thus, $^\cap$ is taken to be undefined for singular terms, which makes a prediction and raises a question. The prediction is that in \isi{article-less languages} without singular-plural distinction (e.g. \ili{Mandarin}) a sentence like \REF{ex:despic:38a} should be fine. This is because a language that does not mark \isi{number} on kind terms should not impose any constraints on the size accessibility of their instantiation sets, effectively aligning it with \isi{bare plurals}. The prediction is borne out:
	
	\ea \label{ex:despic:39}
	\ili{Mandarin} \\
	\gll
	{Gou} {zai} {meigeren-de} {houyuan-li} {jiao.} \\
	dog at everyone-\textsc{ptcp} backyard-inside bark \\
	\glt `Dogs (different ones) are barking in everyone's backyard.' \citep[413]{Dayal2004}
	\z 
	
	The question is how to characterize singular kind formation. Dayal argues that in these cases, the \is{common nouns}common noun has a taxonomic reading and denotes a set of taxonomic kinds. It can then combine with any determiner and yield the relevant reading.
	
	\ea \label{ex:despic:40}
	\ea \label{ex:despic:40a}
	\textit{Every dinosaur is extinct.} 
	\ex \label{ex:despic:40b}
	\textit{The dinosaur is extinct.}
	\z
	\z 
	
	In \REF{ex:despic:40a}, the presupposition that \textit{every} ranges over a plural domain is satisfied if the
	\is{quantifiers}quantificational domain is the set of \is{subkinds}sub-kinds of dinosaurs. The \isi{uniqueness} requirement of \textit{the} with a singular noun in \REF{ex:despic:40b} is satisfied if the quantificational domain is the set of sub-kinds of animals. There is, therefore, nothing special about the \is{definite articles}definite article in \is{definite kinds}definite singular kinds like \REF{ex:despic:41}, according to Dayal. The \is{definite generics}definite singular generic is derived compositionally from the regular \is{definite determiners}definite determiner plus a \is{common nouns}common noun under its taxonomic guise:
	
	\ea \label{ex:despic:41}
	\textit{The lion comes in several varietis, the African lion, the Asian lion \ldots}
	\z 
	
	Specifically, in the case of kind formation out of singular nouns, there is a clash between singular morphology and plurality associated with kinds, which is repaired as in \REF{ex:despic:42}, where $X$ ranges over entities in the taxonomic domain. \REF{ex:despic:42} then forces the application of \is{iota operator|(}$\iota$, which in \ili{English} comes out/is lexicalized as \textit{the}.
	
	\ea \label{ex:despic:42}
	$\textrm{PredK}(^\cap\textrm{lion}=$*$^\cap(\textrm{SING})\Rightarrow\textrm{PredK}\;(\iota X[\textrm{LION}(X)])$ \citep[435]{Dayal2004}
	\z 
	
	At the same time, mass kinds must be bare in \ili{English} \REF{ex:despic:43}, which is expected given that $^\cap$ is defined for them. Mass kinds thus behave like plural kinds.
	
	\ea \label{ex:despic:43}
	\textit{{\op}\textnormal{*}The{\cp} wine comes in several varieties, {\op}\textnormal{*}the{\cp} red wine, {\op}\textnormal{*}the{\cp} white wine and {\op}\textnormal{*}the{\cp} ros\'e.}
	\z 
	
	We expect then that plural kinds and singular kinds in \ili{English} should differ in their ability to be interpreted as \is{definites}definite, i.e. only the latter could be interpreted anaphorically.
	This is because in the case of singular kinds $^\cap$ cannot apply (it clashes with the singular \isi{number} morphology), and \textit{the} (lexical realization of $\iota$ in \ili{English}) is introduced via \REF{ex:despic:38}.
	This appears to be true, as the contrast between \REF{ex:despic:44} and \REF{ex:despic:45} illustrates. The definite singular \textit{the bird} can be anteceded by \textit{the dodo} in \REF{ex:despic:45}, while establishing the \is{anaphora}anaphoric relationship between \isi{bare plurals} \textit{birds} and \textit{dodos} in \REF{ex:despic:44} does not seem to be possible.
	
	\ea \label{ex:despic:44} 
	\textit{Only dodos and gorillas survived on the continent.} \\
	\textit{After the humans arrived birds were wiped out.} \\
	$\rightarrow$ ?* if \textit{birds} is anteceded by \textit{dodos}
	\z 
	
	\ea \label{ex:despic:45}
	\textit{Only the dodo and the gorilla survived on the continent.} \\
	\textit{After the humans arrived the bird was wiped out.} \\
	$\rightarrow$ OK if \textit{the bird} is anteceded by \textit{the dodo}
	\z
	
	Crucially, the same kind of contrast should in principle appear in \is{article-less languages}article-less languages with \isi{number}
	morphology. $^\cap$ should not be defined for singular terms, and \is{iota operator|)}$\iota$ should be available for them via \REF{ex:despic:42} -- thus, the \is{definites}definite/anaphoric interpretation should be available for singular kinds in languages without articles. However, since $^\cap$ is defined for plural kinds, they should pattern with mass kinds in terms of the availability of definite interpretation; i.e. they should lack the anaphoric interpretation. I believe that the following contrasts from \il{Serbo-Croatian!Serbian}Serbian and \ili{Turkish} are clear enough to confirm this prediction. For example, Serbian examples in \REF{ex:despic:46} and \REF{ex:despic:47} differ only in
	terms of number. However, there is a noticeable contrast between them in the availability of anaphoric interpretation, similar to (\ref{ex:despic:44}--\ref{ex:despic:45}). \ili{Turkish} examples in (\ref{ex:despic:48}--\ref{ex:despic:51}) illustrate the same point.\footnote{As indicated in the translation of \REF{ex:despic:47}, the object here can be modified with the expression `as a kind', which shows that what we are dealing with here is not an object-level but a kind-level expression. This is true for previous examples involving \isi{kind reference} as well. Also, the object in \REF{ex:despic:46} can be replaced with `the kind of bird known as `bald eagle'' (e.g. \textit{My whole life, I have been studying the kind of bird known as bald eagle}). Similar can be done to other relevant examples. Moreover, one can dedicate one's entire career to studying the work of Abraham Lincoln, and use (i.a) to express that, but `as a kind' cannot modify the object in this particular case; e.g. (i.b) is clearly more marked than (i.c). This follows from the fact that something that is necessarily instantiated by just one individual (Abraham Lincoln) does not qualify as a kind. All of this shows that these examples truly involve \isi{kind reference}.
		\ea \label{ex:despic:n10}
		\ea[]{ \label{ex:despic:n10a}
			{\textit{I have been studying Abraham Lincoln my whole life.}}}
		\ex[{\#}]{ \label{ex:despic:n10b}
			{\textit{I have been studying Abraham Lincoln, as a kind, my whole life.}}}
		\ex[]{ \label{ex:despic:n10c} 
			\textit{{I have been studying the bald eagle, as a kind, my whole life.}}}
		\z 
		\z\vspace*{-\baselineskip} 
	}\textsuperscript{,}\footnote{Recall that due to the \is{Blocking Principle}Blocking Principle, \is{iota operator}$\iota$ is never available for bare nouns in \ili{English}, singular or plural (the existence of the \is{definite articles}definite article blocks it); for this reason, bare nouns can never be interpreted anaphorically in \ili{English}. On the other hand, $\iota$ is in principle available to both singular and plural bare nouns in languages like \il{Serbo-Croatian!Serbian}Serbian and \ili{Turkish}. In the case of \isi{bare plurals}, both $^\cap$ and $\iota$ are available depending on whether the noun in question has a kind or object-level interpretation, respectively. In such languages, the context and the type of predicate could play a crucial role: a \is{kind-level predicates}kind-selecting predicate (\textit{rare}, \textit{widespread}, \textit{extinct}\ldots) could, for instance, make the contrast clearer for some speakers; compare (i--ii) with (\ref{ex:despic:46}--\ref{ex:despic:47}) respectively. In general, it is not unexpected that this contrast would be somewhat subtler in languages like \il{Serbo-Croatian!Serbian}Serbian or \ili{Turkish} than in \ili{English}.   
		\ea \label{ex:despic:n11i}
		\il{Serbo-Croatian!Serbian}\textnormal{Serbian} \\
		\gll Ceo {\v zivot} {prou\v cavam}      beloglavog     orla  {---}   {na \v zalost},         pre     deset godina ptica je  istrebljena. 	 \\
		{whole} {life}    {study-1.\textsc{prs}} {white-headed} {eagle}  {}   {unfortunately} {before} {ten} {years} {bird}  {is} {exterminated} \\
		\glt {`I have been studying the bald eagle my whole life. Unfortunately, ten years ago the bird was exterminated.'} \\
		{$\rightarrow$ OK if \textit{ptica} `bird' is anteceded by \textit{beloglavog orla} `bald eagle'}
		\z
		
		\ea  \label{ex:despic:n11ii}
		\il{Serbo-Croatian!Serbian}\textnormal{Serbian} \\
		\gll    Ceo     {\v zivot} {prou\v cavam}     beloglave        orlove --- {na \v zalost},       pre      deset godina ptice  su   istrebljene. 	 \\
		{whole} {life}    {study-1.\textsc{prs}} {white-headed} {eagles}  {}    {unfortunately} {before} {ten}      {years}  {birds}  {are} {exterminated} \\
		\glt  {`I have been studying bald eagles my whole life. Unfortunately, ten years ago birds were exterminated.'}  \\ 
		\fts{$\rightarrow$ ?* if \textit{ptice} `birds' is anteceded by \textit{beloglave orlove} `bald eagles'}
		\z
	} 
	
	\ea \label{ex:despic:46}
	\il{Serbo-Croatian!Serbian}Serbian (singular) \\
	\gll 
	{Ceo} {\v zivot} {prou\v cavam} {beloglavog} {orla} {--} {ptica} {je} {fantasti\v cna}. \\
	Whole life study-\textsc{prs} white-headed eagle {} bird is fantastic \\
	\glt `I have been studying the bald eagle (as a kind) my whole life. The bird is fantastic.' \\
	$\rightarrow$ OK if \textit{ptica} `bird' is anteceded by \textit{beloglavog orla} `bald eagle'
	\z 
	
	\ea \label{ex:despic:47} 
	\il{Serbo-Croatian!Serbian}Serbian (plural) \\
	\gll 
	{Ceo} {\v zivot} {prou\v cavam} {beloglave} {orlove} {--} {ptice} {su} {fantasti\v cne.} \\
	Whole life study-\textsc{prs} white-headed eagles {} birds are fantastic \\
	\glt `I have been studying bald eagles (as a kind) my whole life. Birds are fantastic.' \\
	$\rightarrow$ ?* if \textit{ptice} `birds' is anteceded by \textit{beloglave orlove} `bald eagles'
	\z 
	
	\ea \label{ex:despic:48}
	\ili{Turkish} (singular) \\
	\gll 
	{Kel} {kartal}, {Kuzey} {Amerika'da} {bulunur}. {G\"u\c c} {ve} {h\i z-\i n} {sembol\"u} {olarak} {tan\i n\i r.} {Ancak}, {k\"uresel} {\i s\i nma} {nedeniyle}, {ku\c s} {yak\i nda} {tamamen} {yok} {olabilir}. \\
	bald eagle North America-\textsc{loc} {is found} strength and speed-\textsc{gen} symbol as recognized however global warming because bird soon completely may disappear \\
	\glt `The bald eagle is found in North America. It is the symbol of strength and speed. However, because of the global warming, the bird may soon completely disappear.' \\
	$\rightarrow$ OK$^?$ if \textit{ku\c s} `bird' is anteceded by \textit{kel kartal} `bald eagle'
	\z 
	
	
	\ea \label{ex:despic:49} 
	\ili{Turkish} (plural) \\
	\gll 
	{Kel} {kartallar}, {Kuzey} {Amerika'da} {bulunurlar}. {G\"u\c c} {ve} {h\i z-\i n} {sembol\"u} {olarak} {tan\i n\i rlar}. {Ancak}, {k\"uresel} {\i s\i nma} {nedeniyle}, {ku\c slar} {yak\i nda} {tamamen} {yok} {olabilir}. \\
	bald eagles North America-\textsc{loc} {are found} strength and speed-\textsc{gen} symbol as recognized however global warming because birds soon completely may disappear \\
	\glt `Bald eagles are found in North America. They are the symbol of strength and speed. However, because of the global warming, birds may soon completely disappear.' \\
	$\rightarrow$ * if \textit{ku\c slar} `birds' is anteceded by \textit{kel kartallar} `bald eagles' 
	\z 
	
	\ea \label{ex:despic:50}
	\ili{Turkish} (singular) \\
	\gll {Kel} {kartal}, {Kuzey} {Amerika'da} {bulunur}. {G\"u\c c} {ve} {h\i z-\i n}{ sembol\"u} {olarak} {tan\i n\i r}. {Ayerica}, {ku\c sun} {g\"ozleri} {olduk\c ca} {keskindir}. \\
	bald eagle North America-\textsc{loc} {is found} strength and speed-\textsc{gen} symbol as recognized also bird-\textsc{gen} eyes quite sharp \\
	\glt `The bald eagle is found in North America. It is the symbol of strength and speed. Also, the bird's eyes are quite sharp.' \\
	$\rightarrow$ OK if \textit{ku\c s} `bird' is anteceded by \textit{kel kartal} `bald eagle'
	\z 
	
	\ea \label{ex:despic:51}
	\ili{Turkish} (plural) \\
	\gll 
	{Kel} {kartallar}, {Kuzey} {Amerika'da} {bulunurlar}. {G\"u\c c} {ve} {h\i z-\i n} {sembol\"u} {olarak} {tan\i n\i rlar.} {Ayerica}, {ku\c slar\i n} {g\"ozleri} {olduk\c ca} {keskindir}. \\
	bald eagles North America-\textsc{loc} {are found} strength and speed-\textsc{gen} symbol as recognized Also birds-\textsc{gen} eyes quite sharp \\
	\glt `Bald eagles are found in North America. They are the symbol of strength and speed. Also, birds' eyes are quite sharp.' \\
	$\rightarrow$ * if \textit{ku\c slar} `birds' is anteceded by \textit{kel kartallar} `bald eagles'
	\z 
	
	Finally, bare non-mass kinds in \is{article-less languages}article-less languages without \isi{number} morphology (e.g. \ili{Mandarin}, \ili{Japanese}) are expected \textit{not to} have definite/anaphoric\linebreak\noindent interpretations. $^\cap$ is defined for such nouns, since these languages do not have singular morphology that would clash with plurality associated with kind formation (recall also \ref{ex:despic:39}; see \citealt[411-413]{Dayal2004}). In terms of \isi{anaphoricity}/definiteness, bare non-mass kinds in these languages should pattern with plural kinds (and mass kinds) in languages like \il{Serbo-Croatian!Serbian}Serbian and \ili{Turkish}. This also appears to be borne out, as shown in \REF{ex:despic:52} and \REF{ex:despic:53}. The non-mass \is{nouns}noun \textit{tori} `bird' in \REF{ex:despic:52} cannot be anteceded by \textit{hagetaka} `bald eagle', in contrast to (\ref{ex:despic:46}--\ref{ex:despic:48}).
	As already mentioned in footnote \ref{ft8}, \citet{Jenkstoappear} shows that \ili{Mandarin} makes a systematic distinction between unique and \isi{anaphoric definites} (e.g. \citealt{Schwarz2009}); while \isi{unique definites} are realized as bare nouns, anaphoric definites are realized with a \is{demonstratives}demonstrative, except in subject positions, where bare nouns can also be interpreted anaphorically. Examples in \REF{ex:despic:20} which involve object-level interpretation are consistent with Jenks' observations in that bare nouns in subject positions can be used anaphorically. Bare nouns in \REF{ex:despic:14} and \REF{ex:despic:53}, on the other hand, lack anaphoric readings precisely because they are derived by $^\cap$, which is responsible for the kind-level interpretation.\pagebreak
	
	\ea \label{ex:despic:52} 
	\ili{Japanese} \\
	\gll 
	{Watashi}-{wa} {nagai} {aida} {hagetaka-o} {kenkyu shitekita}. {Tori}-{wa} {subarashi}. \\
	I-\textsc{top} long time {bald eagle-\textsc{acc}} studied bird-\textsc{top} fantastic \\ 
	\glt `I have been studying the bald eagle for a long time. The bird is fantastic.' \\
	$\rightarrow$ * if \textit{tori} `bird' is anteceded by \textit{hagetaka} `bald eagle'
	\z 
	
	\ea \label{ex:despic:53}
	\ili{Mandarin} \\
	\gll 
	{Zhiyou} {gezi} {he} {daxingxing} {xingcun} {zai} {zhe} {pian} {dalu} {shang}. {Danshi} {hen} {kuai} {niao} {jiu} {miejue} {le.} \\
	only pigeon and gorilla survive \textsc{loc} this \textsc{clf} continent on but very quickly bird \textsc{ptcp} exinct \textsc{asp} \\
	\glt `Only the pigeon and the gorilla survived on the continent. But very quickly the bird went extinct.' \\
	$\rightarrow$ * if \textit{niao} `bird' is anteceded by \textit{gezi} `pigeon'
	\z 
	
	\section{Summary and further questions} \label{sec:despic:5} 
	
	The initial contrast in interpretation between mass kinds in \ili{English} and languages without
	definite articles led us to an analysis from which some rather systematic patterns appear to emerge.
	
	\begin{table} \is{mass nouns}\is{mass/count distinction}\is{count nouns}\is{anaphora}\is{number}\is{article-less languages}
		\caption{Languages without definite articles: Bare nouns}
		\begin{tabular}{c|c|c|c|c|c|c|c|c|c|c|}
			\cline{2-11}
			& \multicolumn{6}{|c|}{$+${\footnotesize Number}} & \multicolumn{4}{|c|}{$-${\footnotesize Number}} \\
			\cline{2-11}
			& 
			\multicolumn{3}{|c|}{{{\footnotesize Kind-level}}} & \multicolumn{3}{|c|}{{{\footnotesize Object-level}}} & \multicolumn{2}{|c|}{{{\footnotesize Kind-level}}} & \multicolumn{2}{|c|}{{{\footnotesize Object-level}}} \\
			\cline{2-11}
			& \footnotesize Mass & \multicolumn{2}{|c|}{\footnotesize Count} & \footnotesize Mass & \multicolumn{2}{|c|}{\footnotesize Count} & \footnotesize Mass & \footnotesize Count & \footnotesize Mass & \footnotesize Count \\
			\cline{2-11}
			& \cellcolor{lightgray} & \cellcolor{gray}\textsc{sg} & \textsc{pl} & \cellcolor{lightgray} & \textsc{sg} & \textsc{pl} & \cellcolor{lightgray} & \cellcolor{lightgray} & \cellcolor{lightgray} & \cellcolor{lightgray} \\
			
			\hline
			\multicolumn{1}{|c|}{\footnotesize {Anaphoric}} & * & \cellcolor{gray}\checkmark & * & \checkmark & \checkmark & \checkmark & * & * & \checkmark & \checkmark \\
			
			\hline
			\multicolumn{1}{|c|}{\footnotesize {Type-shift}} & $\cap$ & \cellcolor{gray}$\iota$ & $\cap$ & $\iota$  & $\iota$ & $\iota$ & $\cap$ & $\cap$ & $\iota$ & $\iota$ \\ \hline
			\multicolumn{2}{c}{} & \multicolumn{9}{l}{\scalebox{0.8}[0.7]{$\uparrow$} \footnotesize $\cap$ undefined for singular nouns; $\iota$ applies to the taxonomic domain
			}
		\end{tabular}
		\label{tab:despic:1}
	\end{table}
	
	As \tabref{tab:despic:1} above shows, the availability of \is{anaphora}anaphoric/definite readings of \isi{bare nominals} in languages without definite articles correlates with the availability of $^\cap$ and \is{iota operator}$\iota$. More specifically, whenever $^\cap$ applies, the anaphoric/definite reading is missing. We see that object-level and kind-level readings are available both in languages with \isi{number} marking (e.g. \il{Serbo-Croatian!Serbian}Serbian) and in languages without number-marking (e.g. \ili{Japanese}). $\iota$ is responsible for anaphoric interpretation of object-level bare nouns in both types of languages. Where the two language types differ is how they manufacture kinds. In languages without number marking, all kinds are created via $^\cap$, which means that bare kind-level nouns in these languages cannot be interpreted anaphorically. In other words, since \isi{count nouns} in these languages do not mark number (and are used with \isi{classifiers} etc.), they pattern with \isi{mass nouns} and are accessible to $^\cap$. But in languages with number marking, kind-level singular count bare nouns cannot be formed via $^\cap$, due to a clash with singular number morphology. This is repaired by \REF{ex:despic:42}, which introduces $\iota$. As a result, only this type of bare kind-level noun will have anaphoric potential. For bare mass and plural nouns, both $\iota$ and $^\cap$ are available, given the modified ranking of operations in \REF{ex:despic:36}, according to which they are both more highly ranked than $\exists$. 
	Which one of them applies will depend on the context (among other things). In contexts like \REF{ex:despic:31b}, $\iota$ applies and creates the anaphoric reading. But if a kind-level interpretation of the antecedent \is{nouns}noun is forced by the context (as in \ref{ex:despic:33}), the anaphoric relation will be missing; $\iota$ maps property extension to individuals, and a kind is identified with the totality of its instances in any given world (or situation). If, on the other hand, $^\cap$ applies, the anaphoric relation will still be absent, since $^\cap$ is a function whose extension varies from world/situation to world/situation (while \is{iota operator}$\iota$ is a constant function to a contextually anchored individual).
	
	Now, as already noted, $^\cap$ is the intensional counterpart of $\iota$, and \citet{Dayal2004} takes the latter to be the canonical meaning of the \is{definite determiners}definite determiner. One of significant cross-linguistic patterns discussed in \citet{Dayal2004} is the absence of dedicated kind determiners in natural language. That is, plural kind terms are either bare (e.g. \ili{English}, \ili{Hindi}), or \is{definite kinds}definite (e.g. \ili{Italian}, \ili{Spanish}). A simple explanation for this robust generalization is that $^\cap$ is the intensional counterpart of $\iota$ and that languages do not lexically mark extensional/intensional distinctions. There are additional systematic restrictions: for example, if a language uses \isi{bare nominals} for \is{anaphora}anaphoric readings, then it also uses them as plural kind terms. Also, if a language uses \isi{definites} as plural kind terms, it also uses them for anaphoric readings. Thus, correlations are not completely arbitrary; e.g. there are no attested languages in which \isi{bare plurals} could be used anaphorically and at the same time definite plurals could \is{kind reference}refer to kinds. To account for these facts, Dayal proposes a universal principle of lexicalization in which $\iota$ (which is canonically used for \isi{anaphoric reference}) and $^\cap$ (which is canonically used for \isi{generic reference}) are mapped along a scale of diminishing identifiability: $\iota>^\cap$. Languages can then lexicalize at distinct points on this scale, proceeding from $\iota$ to $^\cap$. Languages without determiners like \il{Serbo-Croatian!Serbian}Serbian use the extreme left as the cut-off for lexicalization -- in such languages both $\iota$ and $^\cap$ are covert \is{type shifting}type shifts. The cut-off point for mixed languages like \ili{English} is in the middle -- here $\iota$ is lexicalized (\textit{the}) and $^\cap$ is a covert type-shift. $\iota$ and $^\cap$ are both encoded lexically in obligatory determiner languages like \ili{Italian}, where the cut-off point is at the extreme right. So if a language has a lexical determiner for plural kind formation, this automatically means that its cut-off point is at the extreme right. The principle of lexicalization above therefore entails that such a language could not have a covert $\iota$. The unattested language type mentioned above would then not conform to the proposed direction of lexicalization.\footnote{Languages like \il{Portuguese!Brazilian Portuguese}Brazilian Portuguese and \ili{German} are particularly interesting because they allow a certain degree of optionality. Brazilian Portugese admits \isi{bare singulars} while some dialects of \ili{German} allow both \is{bare plurals}bare and definite plurals/\is{mass nouns}mass terms for \isi{kind reference}, but the variation in available meanings is still quite limited. For detailed discussion of these languages see \citet{Dayal2004,Dayal2011}, \citet{Krifka1995}, \citet{Muller2002}, \citet{MunnSchmitt2005}, \citet{Cyrino2015} and references therein.} 
	
	
	We can also view the relationship between $\iota$ and $^\cap$ from the perspective of \citegen{Schwarz2009} account of strong/weak definites. Schwarz discusses a distinction between \textit{strong} and \textit{weak} definite articles in \ili{German}: strong articles are used in \is{familiar definites}familiar definite environments and are \is{anaphora}anaphoric to a previously introduced referent, while weak articles occur in \is{unique definites}unique definite contexts. \citeauthor{Schwarz2009} proposes that strong (anaphoric) definites take an index as an argument, while unique definites do not \citep[see also][]{Jenkstoappear}. That is, anaphoric articles are more complex than their unique counterparts since they take one extra argument. At the same time, both types of \isi{articles} presuppose the existence of a unique individual. \citet{Jenkstoappear} shows that different languages lexicalize\is{definiteness marking}/mark these two types of definites differently. Languages like \ili{German} and \ili{Lakhota} \citep[see][]{Schwarz2013} have two separate lexical items/markers to \is{definiteness marking}encode unique definites \is{iota operator}(i.e. $\iota$) or \isi{anaphoric definites} (i.e. $\iota^x$). There are also languages like \il{Akan!Fante Akan}Fante Akan and \ili{Mandarin} (see footnote \ref{ft8}) which have a lexical \is{definiteness marking}definite marker for definite anaphoric environments (i.e. $\iota^x$), but no marker for unique definite contexts (covert \is{type shifting}type shift is used). And finally there are languages like \ili{English} that use a single lexical item for both types of definites. We could add to this list languages like \il{Serbo-Croatian!Serbian}Serbian which can use covert type shifts for both environments. But if \citeauthor{Schwarz2009} and \citeauthor{Jenkstoappear} are right in making a distinction between the unique $\iota$ and the anaphoric $\iota^x$ (which I believe they are), then the facts discussed here strongly suggest that $^\cap$ is the intensional counterpart of the unique $\iota$ and not the anaphoric $\iota^x$. This is further supported by the fact that in \ili{German} it is the \is{weak definite articles}weak (unique definite) article that is used for \isi{kind reference} (e.g. \citealt[65-66]{Schwarz2009}). That is, if languages do not lexically mark extensional/intensional distinctions and if $^\cap$ is the intensional counterpart of the unique $\iota$, then it follows that in languages which use two separate markers for unique and anaphoric definites, the unique definite marker will also be used for kind reference. 
	
	I have to leave some questions for future work, since they are outside of the scope of this study. For example, I showed that if a \is{demonstratives}demonstrative is added to the constructions with kind-level context, the anaphoric reading becomes possible. The question is, of course, how this should be formalized. At this point I have to assume that this is due to some specific property of this lexical\largerpage[2] element.\footnote{Similar questions can be raised with respect to \is{kind reference}kind-referring pronouns that can be anteceded by non-kind NPs. In (i) below, for example, the antecedent \textit{Martians} refers to some Martians, while \textit{themselves} refers to the kind (see \citealt{Rooth1985} and \citealt{Krifka2003} for details). So the next step would be to check whether constructions like (i) are allowed in languages discussed here (in particular, whether both coreference and anaphoric binding are possible) and then what kind of implications would such facts have for the analysis presented here. I have to leave this for future work. 
		\ea \textit{At the meeting, \textbf{Martians} presented \textbf{themselves} as almost extinct.}
		\label{ex:despic:n13i}
		\z \label{ft13}
	} For instance, \citet[353]{Chierchia1998} proposes (for independent reasons) that \is{determiners}determiners may semantically come in two variants: those that apply to predicates and those that apply to kinds. One possibility is that a \is{demonstratives}demonstrative like \il{Serbo-Croatian!Serbian}Serbian \textit{to} `that' has both types of interpretations and can therefore combine with kinds.\footnote{This line of reasoning would be supported by a language which makes some kind of morphological distinction between the two determiner variants. This seems to be true for \il{Serbo-Croatian!Serbian}Serbian (and some other \ili{Slavic} languages), at least to a first approximation: in addition to \textit{taj} `that', which seems to be ambiguous as noted above, there are also determiners like \textit{takav} which are best translated as `that kind' (also \textit{kakav} `what kind', \textit{onakav} `that kind', etc.). This, however, requires a more careful examination, which I leave for future work.}\textsuperscript{,}\footnote{It needs to be clarified that the presence of demonstratives does not necessarily indicate the presence of \is{determiner phrases}DP (or some other functional projection) in languages without articles. For example, as discussed in \citet{Boskovic2005}, \citet{Despic2011,Despic2013}, \citet{Zlatic1997}, etc., it is much more plausible to analyze demonstratives (and possessives) in \il{Serbo-Croatian!Serbian}Serbian as NP-adjuncts. A number of \is{morphosyntax}morpho-syntactic arguments support this claim: the availability of \is{Left Branch Extraction}LBE, the appearance of Serbian possessives and demonstratives in adjectival positions (and \is{adjectives}adjective-like agreement), stacking up, impossibility of modification, \isi{specificity} effects, etc. This is based on syntactic evidence, and as long as the demonstrative is assigned appropriate meaning, semantic composition is not affected.} Another question which should be more directly investigated is what kind of discourse factors facilitate or inhibit the anaphoric reading of bare nouns and how they can be distinguished from those discussed in this paper. 
	It is clear that, in terms of \isi{anaphoricity}, \is{iota operator}$\iota$ (i.e. a bare noun) is less potent than demonstratives and pronouns (see Footnote \ref{ft13}). The question is then whether this contrast can ultimately be reduced to some version of blocking (elsewhere) condition that governs the distribution of covert and overt elements (e.g. use overt \isi{demonstratives}/pronouns wherever you can and avoid the covert $\iota$), or whether the \is{anaphora}anaphoric potential of $\iota$ is truly impoverished compared to that of demonstratives/pronouns. \is{bare nouns|)}
	
	Overall I hope to have shown that the general pattern of cross-linguistic variation given in \tabref{tab:despic:1} follows from \citegen{Dayal2004} approach, which is based on a limited set of type-shifting operations constrained by the \is{Blocking Principle}Blocking Principle, and which incorporates an appropriate analysis of \isi{number} morphology.\is{kinds|)}
	
		\section*{Acknowledgements}
	For helpful discussion of the material presented here (and related ideas), I would like to thank Greg Carlson, Gennaro Chierchia, Amy Rose Deal, Jeff Runner, Neda Todorovi\'c, John Whitman, the participants of the Definiteness across Languages conference (Mexico City, June 2016) and the Dimensions of D workshop (Rochester, September 2016). For their generous help with data, I am very grateful to Shohini Bhattasali, Sachiko Komuro, Yanyu Long, Hasan Sezer, Deniz \"Ozy\i ld\i z, Hao Yi and Lingzi Zhuang. Finally, I also want to thank anonymous reviewers and the editors for their careful and helpful suggestions. All errors are my own\largerpage responsibility.	
	
	\section*{Abbreviations}
	\begin{multicols}{2}
		\begin{tabbing}
			1\hspace{2em} \= first person \\ \kill 
			\textsc{acc} \> accusative \\ 
			\textsc{clf} \> classifier \\
			\textsc{dkp} \> Derived Kind Predication \\ 
			\textsc{erg} \> ergative \\ 
			\textsc{gen} \> genitive \\ 
			\textsc{lbe} \> Left Branch Extraction \\ 
			\textsc{loc} \> locative \\
			\textsc{ptcp} \> particle \\
			\textsc{pst} \> past \\ 
			\textsc{prs} \> present \\ 
			\textsc{refl} \> reflexive \\ 
			\textsc{top} \> topic \\ 
			\textsc{udp} \> Universal DP (Approach) \\
		\end{tabbing}
	\end{multicols}
	
	{\sloppy
		\printbibliography[heading=subbibliography,notkeyword=this]
	}
\end{document}
