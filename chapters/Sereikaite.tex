\documentclass[output=paper,
modfonts
]{langscibook} 

\title{Strong vs. weak definites: Evidence from Lithuanian adjectives}

\author{Milena Šereikaitė\affiliation{University of Pennsylvania}}

\ChapterDOI{10.5281/zenodo.3252016}

\abstract{While \ili{Lithuanian} (a \ili{Baltic} language) \is{article-less languages}lacks definite articles, it can use an adjectival system to encode definiteness. \is{adjectives}Adjectives can appear in a bare short form as in \textit{graži} `beautiful.\textsc{nom.f.sg}' and a long form with the definite morpheme \textit{-ji(s)} as in \textit{gražio-ji} `beautiful.\textsc{nom.f.sg}-\textsc{def}'. In this paper, I explore definiteness properties of \ili{Lithuanian} nominals with long and short form adjectives. Recent cross-linguistic work identifies two kinds of definites: strong definites based on \isi{familiarity} and weak definites licensed by \isi{uniqueness} (\citealt{Schwarz2009,Schwarz2013,ArkohMatthewson2013,Jenks2015}; i.a.). Following this line of work, I argue that short form adjectives, in addition to being \is{indefinites}indefinite, are also compatible with situations licensed by uniqueness, and in this way resemble weak article definites. Long form adjectives pattern with strong article definites, as evidenced by familiar definite uses and certain \isi{bridging} contexts parallel to the \ili{German} data \citep{Schwarz2009}. This study provides novel evidence for the distinction between strong versus weak definites showing that this distinction is not necessarily reflected in \is{determiners}determiner patterns, but it can also be detected in the adjectival system.}

\begin{document}
\maketitle

\section{Introduction} \is{strong definite articles|(}\is{weak definite articles|(}
There is a tradition in the literature to define definiteness either in terms of \isi{uniqueness} \citep{Russell1905,Strawson1950,Frege1892,} or in terms of \isi{anaphoricity} (\isi{familiarity}) \citep{Christophersen1939,Kamp1981,Heim1982}. Nevertheless, a detailed study of \ili{German} articles by Schwarz (\citeyear{Schwarz2009}) demonstrates that both familiarity and uniqueness are necessary tools to capture definite uses. Specifically, Schwarz provides empirical evidence showing that there are two semantically distinct definites in \ili{German}: a strong article definite licensed by familiarity and a weak definite licensed by \isi{uniqueness}. The distinction between the two articles is visible not only in \is{anaphora}anaphoric and uniqueness-based contexts, but also in bridging contexts where a \is{part-whole relationship}part-whole relation is licensed by the weak definite article, and the \is{product-producer relationship}product-producer context is compatible with the strong definite article. The dichotomy of strong and weak definites has been supported by a number of other studies from different languages including: Akan\il{Akan} \citep{ArkohMatthewson2013}, ASL\il{American Sign Language} \citepv{chapters/Irani}, Austro-Bavarian\il{Austro-Bavarian} \citep{Simonenko2014}, and \ili{Icelandic} \citep{Ingason2016}.

This paper is the first attempt to bring into the discussion of strong versus weak definites \is{article-less languages}articleless languages like Lithuanian, which uses the adjectival system as one of the means to \is{definiteness marking}express definiteness. While \ili{Lithuanian} lacks definite articles, it has the suffix \textit{-ji(s)} associated with definiteness \citep{Ambrazas1997}. This definite morpheme appears on a variety of non-NP categories, but for present purposes I focus on adjectives. \is{adjectives}Adjectives can appear in a bare short form as in \REF{ex:sereikaite:1a} and a long form with a definite morpheme \textit{-ji(s)} as in \REF{ex:sereikaite:1b}. \citet{GillonArmoskaite2015} report that the nominals with short adjectives can be definite or \is{indefinites}indefinite depending on the context, while nominals with long adjectives are necessarily interpreted as \isi{definites}, as reflected in the glosses in \REF{ex:sereikaite:1}.  

\begin{exe}
\ex \label{ex:sereikaite:1}\il{Lithuanian} 
\begin{xlist}
\ex 
\gll {graž-i} {mergin-a} \\
beautiful-\textsc{nom.f.sg}  girl-\textsc{nom.f.sg}\\
\trans `a/the beautiful girl'  \label{ex:sereikaite:1a} 

\ex 
\gll {graž-io-\textbf{ji}} {mergin-a}\\
beautiful-\textsc{nom.f.sg}-{\textsc{def}} girl-\textsc{nom.f.sg}\\
\trans `the beautiful girl' \label{ex:sereikaite:1b}
\end{xlist}
\end{exe}

In this study, I provide novel evidence for the distinction between strong versus weak article definites \citep{Schwarz2009} by exploring definiteness properties of \ili{Lithuanian} nominals with short and \is{adjectives}long adjectives. In particular, I demonstrate that long form adjectives function like \isi{familiar definites}, and are equivalent to the \ili{German} strong article, as they emerge in \is{anaphora}anaphoric expressions that refer back to linguistic antecedents \REF{ex:sereikaite:2}. This \isi{reference} otherwise is not possible with short form adjectives. The long forms pattern with the strong article in \ili{German} not only in standard anaphoric cases, but also in \is{product-producer relationship}product-producer \isi{bridging} contexts as will be illustrated in \sectref{sec:sereikaite:4}.\largerpage[2]

\begin{exe}\il{Lithuanian}
\ex\label{ex:sereikaite:2} 
\gll {Marija} {pristatė} {mane} {savo} \textbf{pusbroliui} {iš} {Vilniaus}. \textbf{Gražus-is} \textnormal{/} \textnormal{\#}\textbf{gražus} {pusbrolis} {galantiškai} {nusilenkė} {ir} {pabučiavo} {man} {į} {ranką}.\\
Marija introduced me self \textbf{cousin} from Vilnius {beautiful-\textsc{def}} / {\phantom{\#}beautiful} cousin gallantly bowed and kissed me to hand \\
\trans `Marija introduced me to \textbf{her cousin} from Vilnius. \textbf{The beautiful cousin} gallantly bowed and kissed my hand.'
\end{exe}


While the nominals with short form \is{adjectives}adjectives can indeed function like \isi{indefinites} by introducing a new \is{discourse}discourse referent, I provide new data showing that they can also occur in situations licensed by \isi{uniqueness} as evidenced by \is{larger situation}larger situations based on general world knowledge, e.g. \is{genericity}generic rules as in \REF{ex:sereikaite:3}. This observation suggests that short adjectives pattern in a similar way to the weak definite that is associated with uniqueness. 
The similarity of short adjectives with weak definites is further supported by the felicity of short forms in \is{part-whole relationship}part-whole \isi{bridging} contexts, which in \ili{German} also require the weak article (see \sectref{sec:sereikaite:4}). \is{strong definite articles|)}\is{weak definite articles|)}

\begin{exe}
\ex\label{ex:sereikaite:3}\il{Lithuanian}
\gll {Praėjus} {dviem} {savaitėm} {po} {rinkimų}, {prezidentas} {turi} {teisę} {atleisti} \textbf{naują} \textnormal{/} \textnormal{\#}\textbf{naują-jį} \textbf{ministrą} \textbf{{pirmininką}} {tik} {išskirtiniais} {atvejais}.\\
passed two weeks after elections president has right fire new / \phantom{\#}new-\textsc{def} {minister} {prime} only exceptional cases\\
\trans `Two weeks after the election, the president has a right to fire \textbf{the new prime minister} only in exceptional cases.' 
\end{exe}

Nevertheless, a difference between \ili{Lithuanian} and \ili{German} occurs in larger situations that include \is{specificity}specific \is{uniqueness}unique individuals. \ili{German} permits only the weak article in such a context, whereas \ili{Lithuanian} uses the long form \is{adjectives}adjective as in \REF{ex:sereikaite:4}. A similar type of distinction is also observed by \citet{Jenks2015} between \isi{bare nouns} versus definite \isi{demonstratives} and pronouns in \ili{Thai}.\largerpage[2]

\begin{exe} 
\ex \label{ex:sereikaite:4} \il{Lithuanian}
\gll {Po} {rinkimų} \textbf{naujas-is} \textnormal{/} \textnormal{\#}\textbf{naujas} {prezidentas} {paskambino} {miestelio} {merui}.\\
after elections \phantom{\#}new-\textsc{def} / \phantom{\#}{new} president called city mayor\\
\trans `After the elections, \textbf{the new president} called the city mayor.' 
\end{exe}

Overall, the \ili{Lithuanian} data provide additional support for \citegen{Schwarz2009} proposal that definiteness is a two-fold phenomenon consisting of uniqueness and \isi{anaphoricity} that can be expressed by two separate forms/articles in a language. The \is{adjectives}adjective-based definite expressions presented here broaden the typological landscape on how languages encode strong vs. weak article distinction by demonstrating that this distinction is not necessarily reflected in \is{determiners}determiner patterns, but it can also be detected in the adjectival system.
The \ili{Lithuanian} data included in this paper have been tested with 7 informants who worked with the author, who is also a native speaker of \ili{Lithuanian}. In addition to that, an online survey with 20 additional native speakers has been carried out. This was a questionnaire study on Google Forms where the speakers had to read a sentence and select an appropriate \is{adjectives}adjective that sounded the most felicitous in a given context. While a number of instances show a very clear semantic contrast between long and short adjectives, the results from other examples exhibit a certain degree of \is{language internal variation}variation. Particularly, this arises in the contexts that are compatible with both \is{familiarity}familiar and \isi{uniqueness} uses. Indeed, \citetv{chapters/Schwarz} notes that there exist contexts where strong versus weak distinction can be blurry and languages show some variation with respect to which definite form is used. I will review the variation patterns exhibited by the data and discuss what consequences they have for the theory.

This paper is structured as follows. In \sectref{sec:sereikaite:2}, the main typological facts of nominals with short and long adjectives will be presented. In \sectref{sec:sereikaite:3}, I review different approaches that have been used to capture definite uses with a particular focus on \citegen{Schwarz2009} proposal and studies supporting it. \sectref{sec:sereikaite:4} compares the definite use of short and long adjectives with strong and weak articles in \ili{German} illustrating the parallels between the two languages. It is demonstrated that the long form enforces familiarity just like the strong article does in \ili{German}, and the short form is compatible with uniqueness in a similar way to the weak article in \ili{German}. \sectref{sec:sereikaite:5} concludes.

\section{Typological background} \label{sec:sereikaite:2} 

This section describes the basic patterns of the way \ili{Lithuanian} \is{definiteness marking}marks definiteness in relation to other languages. 
\is{article-less languages}\ili{Lithuanian} lacks (in)definite articles, and thereby a \is{bare nouns}bare noun is ambiguous between definite and \is{indefinites}indefinite readings as in \REF{ex:sereikaite:5}. Article-less languages, like, for example, most \ili{Slavic} languages, have been argued to have a \is{determiner phrases}DP layer with an \is{null determiners}empty D category (\citealt{Rappaport1998,Leko1999,Pereltsvaig2007}; i.a.). However, this proposal has been challenged by a number of researchers (\citealt{Boskovic2009,Boskovic2012,BoskovicGajewski2011,Despic2011}; i.a.) claiming that nominals in these languages are simply NPs. The recent work on \ili{Lithuanian} indicates that even though no overt article is present within a nominal, at least definite expressions are always DPs \citep{GillonArmoskaite2015}.

\begin{exe}
\ex[]{\label{ex:sereikaite:5}\il{Lithuanian}
\gll {mergin-a}\\
girl-\textsc{nom.f.sg}\\
\trans `a/the girl'}
\end{exe}

Nevertheless, \ili{Lithuanian} has some morphological means to \is{definiteness marking}mark definiteness, namely the suffix \textit{-ji(s)}. I will call this suffix a \textit{definite form}. The definite form cannot be attached to \isi{nouns} as shown in \REF{ex:sereikaite:6}. 

\begin{exe}
\ex[*]{ \label{ex:sereikaite:6}\il{Lithuanian}
	\gll mergin-a-\textbf{ji} \\
	girl-\textsc{nom.f.sg}-{\textsc{def}} \\
	\trans Int. `the girl'}
\end{exe}

The suffix \textit{-ji(s)} occurs with non-NP categories,\footnote{Other categories that can take the definite form are: pronouns like \textit{mana} `mine' vs. \textit{mano-ji} `mine-\textsc{def}', 
\isi{demonstratives} \textit{ta}  `that' vs. \textit{to-ji} `that-\textsc{def}', relative pronouns \textit{kuri} `who/which' vs. \textit{kurio-ji} `who/which-\textsc{def}', etc. For a full list see \citet[223--224]{Stolz2008}.} e.g. \is{adjectives}adjectives, recall our minimal pairs from \REF{ex:sereikaite:1} repeated here in \REF{ex:sereikaite:7}.\footnote{The definite form \textit{-ji(s)} is subject to elision. The glide \textit{j} is omitted before the sibilant consonant  /s/ as in e.g. \textit{graž-us} `beautiful-\textsc{nom.sg.m}' + \textit{jis} = \textit{gražus-is} `the beautiful'.} The traditional \ili{Lithuanian} Grammar \citep[142]{Ambrazas1997} defines the short form as \is{indefinites}indefinite, ``unmarked'', and the long form as \is{definites}definite, ``marked''. \citet{GillonArmoskaite2015} show that both forms can in fact be definite.

\begin{exe}
\ex \label{ex:sereikaite:7}\il{Lithuanian}
\begin{xlist}
\ex \gll graž-i mergin-a \\
beautiful-\textsc{nom.f.sg} girl-\textsc{nom.f.sg}\\
\trans `a/the beautiful girl'

\ex \gll graž-io-\textbf{ji} mergin-a\\
beautiful-\textsc{nom.f.sg}-\textsc{def} girl-\textsc{nom.f.sg}\\
\trans `the beautiful girl' 
\end{xlist}
\end{exe}

\ili{Lithuanian}, at least typologically, is different from some \ili{Slavic} languages that have a definite suffix. For example, \ili{Bulgarian}, unlike \ili{Lithuanian}, has an option to attach the definite suffix \textit{-ta} to a noun \REF{ex:sereikaite:8a} as well as to an \is{adjectives}adjective \REF{ex:sereikaite:8b}.

\begin{exe}
\ex\label{ex:sereikaite:8} 
\ili{Bulgarian}
\begin{xlist}
\ex \label{ex:sereikaite:8a}
\gll kniga-\textbf{ta} \\
book-\textsc{def}\\
\trans `the book' \\
\ex \label{ex:sereikaite:8b}
\gll {xubava-\textbf{ta}} {kniga}\\
nice-\textsc{def} book\\
\trans `the nice book'  
\end{xlist}
\end{exe}

The \ili{Lithuanian} short vs. long adjective pairs are cognate with short and long \is{adjectives}adjective forms found in \ili{Serbo-Croatian} (see \citealt{Aljovic2010} and references therein) and \ili{Old Church Slavonic} \citep{Sereikaite2015}. The definite suffix \textit{-ji(s)} is originally a pronominal form \citep{Ulvydas1965,Stolz2008} where \textit{`jis'} stands for `he' and \textit{`ji'} stands for`she'.\footnote{There are several theories about the origin of the definite form \textit{-ji(s)}. \citet{Stolz2008} argues that the \is{definiteness marking}definite marker used to function as a relative pronoun in preliterate times, while \citet{Rosinas1988} suggests that this definite marker is a ``postposed deictic pronoun''. In \citet{Valeckiene1986}, definite forms are treated as apposition constructions where the definite form is the apposition proper.} Both short and long adjectives agree with the \is{nouns}noun as indicated in \REF{ex:sereikaite:7}. The definite form \textit{-ji(s)} also shows agreement in \isi{number}, gender and case with the noun as illustrated in \tabref{tab:sereikaite:1} for both singular and plural masculine forms. However, for the reader's convenience and for the matter of space, I gloss \textit{-ji(s)} as \textsc{def}.

\begin{table} 
	\begin{tabularx}{\textwidth}{p{0.8cm} p{2cm} p{3cm} p{1.9cm} p{2.85cm}}
		\lsptoprule
		{ } & \textit{jaun-as-}\textsc{m.sg} & \textit{jaunas-is-}\textsc{m.sg-def} & \textit{jaun-i-}\textsc{m.pl} & \textit{jaunie-ji-}\textsc{m.pl-def} \\
		\midrule
		\textsc{nom} & \textit{jaun-as}  & \textit{jaun-as-is} & \textit{jaun-i}  & \textit{jaun-ie-j-i} \\ 
		\textsc{gen} & \textit{jaun-o}  & \textit{jaun-o-j-o} & \textit{jaun-ų}  & \textit{jaun-ų-j-ų} \\ 
		\textsc{dat} & \textit{jaun-am} & \textit{jaun-a-j-am} & \textit{jaun-iems}  & \textit{jaun-ies-iems} \\ 
		\textsc{acc} & \textit{jaun-ą} & \textit{jaun-ą-j-į} & \textit{jaun-us} & \textit{jaun-uos-i-us} \\ 
		\textsc{inst} & \textit{jaun-u}  & \textit{jaun-uo-j-u} & \textit{jaun-ais}  & \textit{jaun-ais-i-ais} \\ 
		\textsc{loc} & \textit{jaun-ame} & \textit{jaun-a-j-ame} & \textit{jaun-uose}  & \textit{jaun-uos-i-uose} \\
		\lspbottomrule
	\end{tabularx}
	\caption{Inflectional paradigm of short and long adjectives of \textit{jaunas} `young' \citep[adapted from][]{Stolz2008}}
	\label{tab:sereikaite:1}
\end{table}

In this paper, I will be looking at the instances with a single \is{adjectives}adjective, be it a short form or a long form. For completeness, observe that the occurrence of two long adjectives with a definite meaning is judged as odd at least in default cases \REF{ex:sereikaite:9b}.\footnote{Note that in formal written contexts or contexts that require emphasis/exaggeration the occurrence of two long forms is acceptable. Not only the \is{discourse}discourse plays a role, but also prosody. The examples in \REF{ex:sereikaite:9b} are judged as grammatical when there is a pause between the two adjectives. I thank Solveiga Armoskaite (personal communication) for bringing this up to my attention.}

\begin{exe}
	\ex \label{ex:sereikaite:9}\il{Lithuanian}
	\begin{xlist}
		\ex[]{\label{ex:sereikaite:9a}
			\gll {graž-us} {sen-as} {lok-ys} \\
			beautiful-\textsc{nom.m.sg} old-\textsc{nom.m.sg}  bear-\textsc{nom.m.sg}  \\
			\trans `a beautiful old bear'} 
		\ex[??]{\label{ex:sereikaite:9b}
			\gll {gražus-\textbf{is}} {senas-\textbf{is}} {lokys} \\
			beautiful-\textsc{nom.m.sg}-{\textsc{def}} old-\textsc{nom.m.sg}-{\textsc{def}} bear-\textsc{nom.m.sg} \\
			\trans `the beautiful old bear' }
	\end{xlist}
\end{exe}

\pagebreak Thereby, \ili{Lithuanian}, at least in standard, discourse-neutral cases, does not permit \is{definite reduplication}multiple definite forms in the context of a \is{definite noun phrases}definite noun phrase,\footnote{Nevertheless, \citet{Stolz2008} gives the example in (i.a) and claims that two definite \is{adjectives}adjectives can in fact occur together. Note that this instance includes coordination. It might be that the first \is{adjectives}adjective has been accompanied by a \is{nouns}noun which then has been \is{nominal ellipsis}elided. Observe that the example becomes ungrammatical in default cases without the conjunct (i.b).
\ea   \il{Lithuanian}
\ea[]{\label{ex:sereikaite:i} 
\gll Trūksta greta nuostabių-\textbf{jų} ir gražių-\textbf{jų} atstovių\\ 
{lack.\textsc{prs.3}} {near} {wonderful.\textsc{gen.f.pl}-{\textsc{def}}} {and} {beautiful.\textsc{gen.f.pl}-{\textsc{def}}} {representatives.\textsc{gen.f.pl}}\\
\glt \normalfont `The wonderful and beautiful representatives are missing.' \citep[adapted from][226]{Stolz2008}} 

\ex[*]{\label{ex:sereikaite:ii} 
\gll Trūksta greta nuostabių-\textbf{jų} gražių-\textbf{jų} atstovių\\
{lack.\textsc{prs.3}} {near} {wonderful.\textsc{gen.f.pl}-{\textsc{def}}} {beautiful.\textsc{gen.f.pl}-{\textsc{def}}} {representatives.\textsc{gen.f.pl}}\\
\glt \normalfont `The wonderful and beautiful representatives are missing.' \citep[adapted from][226]{Stolz2008}}
\z 
\z\vspace*{-\baselineskip}} unlike for example \ili{Greek} (see \citet{Alexiadou2014} and references therein) which is known for multiple \is{definiteness marking}marking of definiteness \REF{ex:sereikaite:10}.

\begin{exe}
\ex \label{ex:sereikaite:10}
\ili{Greek} \citep[19]{Alexiadou2014} \\
\gll {to} {vivlio} {to} {kokino} {to} {megalo}\\
the book the red the big\\
\trans `the big red book' 
\end{exe}

The definite suffix can also be used to refer to \isi{kinds} \citep{Rutkowski06}. The short \is{adjectives}adjective simply denotes a bear that happens to be white as in \REF{ex:sereikaite:11a}. In contrast, the long \is{adjectives}adjective is ambiguous between the definite reading and the \is{kinds}kind reading expressing a certain species of bears, namely the polar bear \textit{Ursus maritimus}, as in \REF{ex:sereikaite:11b}.\footnote{An anonymous reviewer asks how nominals without modifiers express \isi{kinds} in \ili{Lithuanian} in general. \is{bare nominals}Bare nominals can be kind-denoting. However, their use is restricted. \is{bare plurals}Bare plural nominals are compatible with kind-denoting predicates like \textit{extinct}, whereas \isi{bare singulars} are not as exemplified below.	
\ea \il{Lithuanian}
\ea[]{\footnotesize
	\gll Tigrai greitai išnyks.\\
\fts{Tigers.\textsc{nom.m.pl}} \fts{soon} \fts{extinct.\textsc{fut.3}}\\
\trans \normalfont `Tigers will extinct soon.'}

\ex[\#]{\footnotesize
	\gll Tigras greitai išnyks.\\
\fts{Tiger.\textsc{nom.m.sg}} \fts{soon} \fts{extinct.\textsc{fut.3}}\\
\trans \normalfont Int. `The tiger will extinct soon.'}
\z
\z\vspace*{-\baselineskip}}

\begin{exe}
\ex \label{ex:sereikaite:11}\il{Lithuanian}
\begin{xlist}
\ex \label{ex:sereikaite:11a}
\gll {balt-as} {lok-ys} \\
white-\textsc{nom.m.sg} bear-\textsc{nom.m.sg} \\
\trans `a/the white bear'  

\ex \label{ex:sereikaite:11b}
\gll {balt-as-is} {lok-ys}\\
white-\textsc{nom.m.sg}-\textsc{def} bear-\textsc{nom.m.sg}  \\
\trans 
\begin{enumerate}[(i)]
	\item `the white bear' \checkmark definite reading 
	\item `the polar bear' \checkmark kind reading\is{kinds}
\end{enumerate}
 
\end{xlist}
\end{exe}



Interestingly, a long \is{adjectives}adjective with a definite meaning and a long adjective with a \is{kinds}kind interpretation can be stacked together \REF{ex:sereikaite:12}. Observe that the definite meaning of `white' in default cases is disfavored. \citet{Sereikaite2017} argues that in \ili{Lithuanian} a combination of a kind-level adjective and a \is{nouns}noun syntactically is similar to a phrasal \is{compounding}compound, whereas a definite adjective and a nominal do not function like a single syntactic unit. Instead, the definite adjective behaves like a modifier of a nominal. 

\begin{exe}\il{Lithuanian}
\ex \label{ex:sereikaite:12}
\gll {graž-us-\textbf{is}} {balt-as-\textbf{is}} {lok-ys} \\
beautiful-\textsc{nom.m.sg}-{\textsc{def}} white-\textsc{nom.m.sg}-{\textsc{def}} bear-\textsc{nom.m.sg} \\
\trans 
\begin{enumerate}[(i)]
	\item `the beautiful polar bear'\\
	\item ?? `the beautiful white bear' 
\end{enumerate}
\end{exe}



Having presented the main typological facts on nominals with \is{adjectives}adjectives, I now turn to the theoretical discussion on two types of definites.


\section{Two types of definites} \label{sec:sereikaite:3}\is{uniqueness|(}

This section describes different approaches that have been used to define definiteness. There has been extensive debate in the literature whether definiteness should be characterized by uniqueness or by \is{familiarity|(}familiarity. On the one hand, \is{definite articles}definite articles in expressions like \textit{the moon} in \REF{ex:sereikaite:13} are argued to be licensed by uniqueness and no prior mention of the referent is necessary \citep{Russell1905,Strawson1950,Frege1892}. The earlier versions of this approach, e.g. \citegen{Strawson1950} work, that assume ``absolute'' uniqueness are problematic for instances that involve situational uniqueness. As mentioned by \citet{Schwarz2013}, there is a number of situations where the descriptive content of the definite expression holds true for more than one entity in the world. For example, the definite description \textit{the projector} is used in \REF{ex:sereikaite:14}, even though there is more than one projector existing in the world. 

\begin{exe}
\ex \label{ex:sereikaite:13}
\textit{Armstrong was the first man to walk on \textbf{the moon}}.  
\end{exe}

\begin{exe}
\ex \label{ex:sereikaite:14}
Context: Said in a lecture hall containing exactly one projector. \\
\textit{\textbf{The projector} is not being used today.} \citep[537]{Schwarz2013}
\end{exe}

On the other hand, definite articles can be viewed as expressing \isi{anaphoricity}, also often referred to as familiarity \citep{Christophersen1939,Kamp1981,Heim1982}. Under this approach, \is{definite nominals}definite nominals are \is{anaphora}anaphoric and need to be linked to a previously mentioned \is{discourse}discourse referent. This is the so-called strong familiarity in \citegen{Roberts2003} terms. While this anaphoricity-based analysis captures some of the uses of definite articles, it is still unclear how such an approach would account for cases as \REF{ex:sereikaite:15} that lack a prior mention of the definite description and instead include global familiarity.  

\begin{exe}
\ex \label{ex:sereikaite:15}
\textit{John bought \textbf{a book} and a magazine. \textbf{The book} was expensive.} \citep[537]{Schwarz2013}
\end{exe}

\is{strong definite articles|(}\is{weak definite articles|(}Several attempts have been made to propose a mixed view of both approaches that would use both uniqueness and familiarity to license definites \citep{Kadmon1990,Farkas2002,Roberts2003}. The hybrid view of definiteness requires different analyses for different uses of \isi{definites}, and thus conceptually is somewhat a less desirable outcome. Nevertheless, this approach has been empirically supported by recent cross-linguistic work suggesting that neither the purely uniqueness-based approach nor the anaphoricity-based analysis can fully account for the full paradigm of definite uses.

One of the main empirical studies that supports the hybrid approach comes from \citet{Schwarz2009,Schwarz2013}. Schwarz shows that \ili{German} has two types of definite articles that correspond to two semantically distinct definites. The weak definite contracts with a \is{prepositions}preposition in certain environments and the strong definite does not. Schwarz demonstrates that the weak definite is licensed by uniqueness and the strong definite is licensed by familiarity.\footnote{I gloss the weak article definite as D\textsubscript{weak} and the strong article definite as D\textsubscript{strong.}} \REF{ex:sereikaite:16} involves a \is{global uniqueness}globally unique situation, and the contracted form \textit{zum}, namely the weak definite, is felicitous. On the other hand, the non-contracted form \textit{in dem}, thus the strong definite, is used with nominals that are \is{anaphora}anaphoric with preceding expressions as in \REF{ex:sereikaite:17}. The strong vs. weak distinction has been shown to hold true in other environments that involve either \is{uniqueness}unique definites or \is{familiarity|)}familiar definites e.g., different cases of \isi{bridging}, \is{larger situation}larger situations or \is{immediate situation}immediate situations (see \sectref{sec:sereikaite:4} for some examples of these uses). 

\begin{exe}
	\ex \label{ex:sereikaite:16} 
	\ili{German} \citep[40]{Schwarz2009} \\
	\gll {Armstrong} {flog} {als} {erster} {\textbf{zum}} \textnormal{/} \textnormal{\#}\textbf{zu dem} \textbf{{Mond}}.\\
	Armstrong flew as {first one} {to-the\textsubscript{weak}} / \phantom{\#}{to-the\textsubscript{strong}} {moon}.\\
	\trans `Armstrong was the first one to fly to \textbf{the moon}.' 
\end{exe}

\begin{exe}
	\ex \label{ex:sereikaite:17} 
	\ili{German} \citep[30]{Schwarz2009} \\
	\gll {In} {der} {New} {Yorker} {Bibliothek} {gibt} {es} \textbf{{ein}} \textbf{{Buch}} {über} {Topinambur}. {Neulich} {war} {ich} {dort} {und} {habe} \textnormal{\#}\textbf{im} \textnormal{/} \textbf{in dem} \textbf{{Buch}} {nach} {einer} {Antwort} {auf} {die} {Frage} {gesucht}, {ob} {man} {Topinambur} {grillen} {kann}.\\
	in the New York library exists \textsc{expl} {a} {book} about topinambur recently was I there and have \phantom{\#}{in-the\textsubscript{weak}} / {in the\textsubscript{strong}} {book} for an answer to the question searched whether one topinambur grill can\\
	\trans `In the New York public library, there is \textbf{a book} about topinambur. Recently, I was there and looked in \textbf{the
		book} for an answer to the question of whether one can grill topinambur.’ 
\end{exe}


To encode these uses of \isi{definites}, \citet{Schwarz2009,Schwarz2013} proposes the following analysis. The denotation of the weak article introduces a unique referent in a given situation as in \REF{ex:sereikaite:18} thereby capturing the situational uniqueness, which has been problematic for the early proponents of the uniqueness approach. The strong article definite defined in \REF{ex:sereikaite:19} not only has a unique referent, but also includes an additional argument that is identical to previously introduced individual within a certain situation/context. Both the strong and weak articles are related: the strong article is a combination of the weak article plus the \is{anaphora}anaphoric link. 


\begin{exe} 
\ex \label{ex:sereikaite:18}
 {} [[D\textsubscript{weak}]] = $\lambda$s\textsubscript{r}.$\lambda$P.$\iota$x.P(x)(s\textsubscript{r})  \citep[264]{Schwarz2009}
\end{exe} 

\begin{exe} 
\ex \label{ex:sereikaite:19}
{} [[D\textsubscript{strong}]] = $\lambda$s\textsubscript{r}.$\lambda$P.$\lambda$y.$\iota$x.P(x)(s\textsubscript{r}) $\land$ x=y \citep[260]{Schwarz2009}
\end{exe} 

Schwarz's proposal that there are two semantically distinct articles in natural language has been supported by recent work. Note that \ili{English} does not show morphological distinction and uses \textit{the} for both types of definites as in \REF{ex:sereikaite:20}.

\begin{exe}
\ex \label{ex:sereikaite:20}
\textit{Amy bought a book about \textbf{the\textsubscript{weak} sun}. \textbf{The\textsubscript{strong} book} was expensive.} \citep[115]{Ingason2016}
\end{exe}

However, a number of other languages employ different types of \is{morphosyntax}morphosyntactic means to express different definite uses. For instance, \citet{Ingason2016} argues that \ili{Icelandic} parallels with \ili{German} in having two distinct phonological exponents for two semantically distinct definites. In general, the article in \ili{Icelandic} is usually expressed as a suffix attached to a \is{nouns}noun in both \is{anaphora}anaphoric and \is{uniqueness|)}uniqueness-based contexts. Nevertheless, the morphological distinction between two types of definite uses emerges in the presence of evaluative adjectives. In situations that include an evaluative \is{adjectives}adjective intervening between a \is{determiners}determiner and a noun, the free article 
\textit{HI} is used. Specifically, the free article functions as a \is{uniqueness}unique definite and corresponds to the weak article in \ili{German} as in \REF{ex:sereikaite:21}. This article cannot be used anaphorically, and instead the \is{demonstratives}demonstrative is used in this type of environment as illustrated in \REF{ex:sereikaite:22}. The demonstrative, thus, behaves like the strong definite in \ili{German}.  

\begin{exe}
	\ex \label{ex:sereikaite:21} 
	\ili{Icelandic} \citep[123]{Ingason2016} \\
	Context: First mention of the World Wide Web. \\ 
	\gll {Tim} {Barners} {Lee} {kynnti} {heiminn}  {fyrir} \textbf{hinum} \textnormal{/} \textnormal{\#}\textbf{þessum} {ótrúlega} {veraldarvef}.\\
	Tim Berners Lee introduced world.the to {{HI}-the\textsubscript{weak}} / \phantom{\#}{this\textsubscript{strong}} amazing world.wide.web\\
	\trans `Tim B. Lee introduced the world to \textbf{the amazing World Wide Web}.' 
\end{exe}

\begin{exe}
	\ex  \label{ex:sereikaite:22}
	\ili{Icelandic} \citep[133]{Ingason2016} \\
	\gll {Hún} {fékk} {engin} {góð}  {svör} {frá} \textnormal{\#}\textbf{hinum} \textnormal{/} \textbf{þessum} {hræðilega} {stjórnmálamanni}.\\
	she got no good answers from \phantom{\#}{{HI}-the\textsubscript{weak}} / {this\textsubscript{strong}} terrible politician\\
	\trans `She got no good answers from \textbf{the terrible politician}.'  
\end{exe}
 
 
In addition, \ili{Fering} Frisian \citep{Ebert1971} and Austro-Bavarian\il{Austro-Bavarian} \citep{Simonenko2014} have also been reported to have two distinct morphological forms to express both definites in this respect resembling \ili{German} and \ili{Icelandic}.

Another important case worth mentioning comes from Akan\il{Akan} (\ili{Kwa}, \ili{Niger-Congo}). Akan\il{Akan}, unlike \ili{German}, has only one overt form used for one of the definites. According to \citet{ArkohMatthewson2013}, the weak definite article is realized as zero, and thus \isi{bare nominals} are used in this context \REF{ex:sereikaite:23}. Nevertheless, Akan\il{Akan} employs an overt form for \is{anaphora}anaphoric uses, namely the demonstrative \textit{nʊ}, as in \REF{ex:sereikaite:24}, equivalent to the \ili{German} strong article.

\begin{exe}
	\ex \label{ex:sereikaite:23}
	Akan\il{Akan} \citep[2]{ArkohMatthewson2013} \\
	\gll {Ámstr\`{ɔ}{\ng}} {nyí} {nyímpá} {áà}  {ó-dzí-ì} {kán} {tú-ù} {k\'{ɔ}-\`{ɔ}} {\`{ɔ}s\`irán} ∅ {d\`{ʊ}}.\\
	Armstrong is person \textsc{rel} \textsc{3.sg.sbj}-eat-\textsc{pst} first uproot-\textsc{pst} go-\textsc{pst} moon {the\textsubscript{weak}} top\\
	\trans `Armstrong was the first person to fly to \textbf{the moon}.'  \\
\end{exe}

\begin{exe}
	\ex \label{ex:sereikaite:24}
	Akan\il{Akan} \citep[2]{ArkohMatthewson2013} \\
	\gll {M\`{ʊ}-t\'{ɔ}-\`{ɔ}} {èkùtú}. {Èkùtú} {\textbf{n\'{ʊ}}} {y\`{ɛ}} {d\`{ɛ}w}  {pápá}.\\
	\textsc{1.sg.sbj}-buy-\textsc{pst} orange orange the\textsubscript{strong} be nice good\\
	\trans `I bought \textbf{an orange}. \textbf{The orange} was really tasty.'  \\
\end{exe}

Similarly to Akan\il{Akan}, \isi{numeral classifier languages} like \ili{Thai} also have been shown to employ bare nominals to express weak definites as in \REF{ex:sereikaite:25}, whereas the strong definite expressions are \is{definiteness marking}encoded by \isi{demonstratives} or overt pronouns as in \REF{ex:sereikaite:26} \citep{Jenks2015}.


\begin{exe}
	\ex \label{ex:sereikaite:25} 
	\ili{Thai} \citep[106]{Jenks2015} \\
	\gll {dua{\ng}-can}  {\op}\#\textbf{dua\ng} \textbf{n\'an}{\cp}   {s\`aw\`aa\ng} {m\^aak}\\
	moon  {\phantom{(\#}\textsc{clf}} {that} bright very\\
	\trans `\textbf{The moon} is very bright.' 
\end{exe} 

\begin{exe}
	\ex \label{ex:sereikaite:26}
	\ili{Thai} \citep[112]{Jenks2015} \\
	Previous discourse: `Yesterday I met a student...\\
	\gll {\op}nákrian{\cp}  \textbf{khon} \textbf{nán} \textnormal{/} {\op}{\textbf{kháw}}{\cp} {chalàat} {m\^aak}\\
	\phantom{(}student {\textsc{clf}} {that} / \phantom{(}{\textsc{3p}} clever very\\
	\trans `\textbf{That student / (s)he} was clever.' 
\end{exe}

All in all, empirical evidence from these languages draws a new perspective on definiteness showing that definiteness is a two-fold phenomenon. Both \isi{uniqueness} and \isi{familiarity} are necessary tools to capture different uses of definite descriptions. These findings make the hybrid approach the most accurate account of all the existing approaches so far. This approach will also be supported by the \ili{Lithuanian} data presented in the subsequent section. 

\section{Strong vs. weak distinction in Lithuanian}\il{Lithuanian} \label{sec:sereikaite:4}

In this section, I explicitly discuss the occurrence of \ili{Lithuanian} nominals with long and short adjectives in familiar and \is{uniqueness}unique definite environments, and \isi{bridging} contexts based on the examples from \citet{Schwarz2009}. I demonstrate that the nominals with two distinct \is{adjectives}adjective forms correspond to the two distinct definite uses, namely familiar uses and unique uses. The long adjective with the definite morpheme \textit{-ji(s)} is analogous to the \ili{German} strong article and is licensed by familiarity -- recall our original example \REF{ex:sereikaite:2}, repeated here in \REF{ex:sereikaite:27}. The short form adjective, in addition to its \is{indefinites}indefinite use, is compatible with uniqueness \REF{ex:sereikaite:3}, repeated in \REF{ex:sereikaite:28}. From now on, the short form will be glossed as weak and the long form will be glossed as a strong definite. For the reader's convenience, I provide glosses only for expressions under the discussion. To draw clear parallels between nominals with long and short adjectives, and the strong and weak articles, the \ili{Lithuanian} data will be compared with \ili{German}. 

\begin{exe}
	\ex \label{ex:sereikaite:27}\il{Lithuanian}
	\gll {Marija} {pristatė} {mane} {savo} {\textbf{pusbroliui}} {iš} {Vilniaus}. \textbf{Gražus-is} \textnormal{/} \textnormal{\#}\textbf{gražus} {pusbrolis} {galantiškai} {nusilenkė} {ir} {pabučiavo} {man} {į} {ranką}.\\
	Marija introduced me self {cousin} from Vilnius {beautiful-\textsc{def}\textsubscript{strong}} / {\phantom{\#}beautiful\textsubscript{weak}} cousin gallantly bowed and kissed me to hand \\
	\trans `Marija introduced me to \textbf{her cousin} from Vilnius. \textbf{The beautiful cousin} gallantly bowed and kissed my hand.'
\end{exe}

\begin{exe}
	\ex \label{ex:sereikaite:28}\il{Lithuanian}
	\gll {Praėjus} {dviem} {savaitėm} {po} {rinkimų}, {prezidentas} {turi} {teisę} {atleisti} \textbf{naują} \textnormal{/} \textnormal{\#}\textbf{naują-jį} \textbf{{ministrą}} \textbf{{pirmininką}} {tik} {išskirtiniais} {atvejais}.\\
	passed two weeks after elections, president has right fire {new\textsubscript{weak}} / {\phantom{\#}new-\textsc{def}\textsubscript{strong}} {minister} {prime} only exceptional cases\\
	\trans `Two weeks after the election, the president has a right to fire \textbf{the new prime minister} only in exceptional cases.' 
\end{exe}

This study gives additional insights into the debate on how definiteness should be characterized, and also broadens the typological landscape of how languages express the two definites. The exploration of nominal expressions accompanied by adjectives shows that \ili{Lithuanian} typologically belongs to the group of languages like Akan\il{Akan} (cf. \ref{ex:sereikaite:23}--\ref{ex:sereikaite:24}) or \ili{Thai} (cf. \ref{ex:sereikaite:25}--\ref{ex:sereikaite:26}) since it uses a bare form, the short \is{adjectives}adjective, in situations with a \is{uniqueness}unique referent, and it has one marked form, namely the long adjective, that is equivalent to the \is{strong definite articles|)}strong article in \ili{German}. At the same time, \ili{Lithuanian} \is{definiteness marking}manifestation of definiteness through \is{adjectives}adjectival system resembles \ili{Icelandic} which also exhibits the strong vs. weak distinction whenever evaluative adjectives intervene between D/n categories (cf. \ref{ex:sereikaite:21}--\ref{ex:sereikaite:22}).

Before I proceed to our discussion of \isi{definites}, a couple of general remarks regarding definiteness in \ili{Lithuanian} should be kept in mind. As has been illustrated by \citet{GillonArmoskaite2015}, a number of factors can affect the definiteness of a nominal, e.g. word order or aspect. The basic word order in \ili{Lithuanian} is SVO. The syntactic position that has been reported to be mostly neutral with respect to definiteness is the initial subject position. Even though the definite interpretation is slightly preferred for the initial subject, both definite and \is{indefinites}indefinite readings are available depending on the context \REF{ex:sereikaite:29}.\is{weak definite articles|)}

\begin{exe} 
\ex \label{ex:sereikaite:29}\il{Lithuanian}
\gll {\v{Z}mog-us} {atvyk-o}.\\
human-\textsc{nom.m.sg} arrive-\textsc{pst.3}\\
\trans `The/a man arrived.' \citep[74]{GillonArmoskaite2015}
\end{exe} 

The interpretation of the object in SVO instances is dependent on the aspect. The imperfective aspect, which is unmarked, permits both definite or indefinite readings of the object depending on the context \REF{ex:sereikaite:30a}. In contrast, the perfective aspect, which is realized with a prefix on a verb, requires the object to be definite, \REF{ex:sereikaite:30b}. %\newpage


\begin{exe}
\ex \label{ex:sereikaite:30}\il{Lithuanian}
\begin{xlist}
	\ex \label{ex:sereikaite:30a}
	\gll {Jon-as} {valg-ė} {obuol-į}.\\
	Jonas-\textsc{nom.m.sg} eat-\textsc{pst.3} apple-\textsc{acc.m.sg}\\
	\trans `Jonas ate the/an apple.' \citep[75]{GillonArmoskaite2015}
	
	\ex \label{ex:sereikaite:30b}\il{Lithuanian}
	\gll {Jon-as} {su-valg-ė} {obuol-į}.\\
	Jonas-\textsc{nom.m.sg} \textsc{prf}-eat-\textsc{pst.3} apple-\textsc{acc.m.sg}\\
	\trans `Jonas ate up the/\#an apple.' \citep[76]{GillonArmoskaite2015}
\end{xlist}
\end{exe}

In order to ensure that the (in)definiteness of nominal expressions that we are testing is purely dependent on the context and is not influenced by the aforementioned factors, the examples are set up in such a way that the target nominal expression appears in a subject initial position. The cases where the tested nominals appear in the object position will include the imperfective aspect which does not reinforce the definite reading.
Lastly, recall from \sectref{sec:sereikaite:2} that nominals with long \is{adjectives}adjectives can have either definite or \is{kinds}kind-level interpretations \REF{ex:sereikaite:11b}, repeated here with the original glosses in \REF{ex:sereikaite:31}. The nominals in our examples will include evaluative adjectives like \textit{strange} or classifying adjectives such as \textit{young} which lack a kind-level interpretation and provide a good testing ground for (in)definite interpretation of nominals.\largerpage[-2]


\begin{exe} \il{Lithuanian}
\ex \label{ex:sereikaite:31}
\gll {balt-as-\textbf{is}} {lok-ys} \\
white-\textsc{nom.m.sg}-{\textsc{def}} bear-\textsc{nom.m.sg}  \\
\trans 
\begin{enumerate}[(i)]
	\item `the white bear' \checkmark definite reading  
	\item `the polar bear' \checkmark kind reading  
\end{enumerate}
\end{exe}

Having said that, I now review the basic descriptive facts that have been associated with short and long forms in the literature. 


\subsection{Definite vs. indefinite noun phrases with adjectives} \label{sec:sereikaite:4.1}\largerpage[-2]

In this sub-section, I show that nominals with short form adjectives can have an \is{indefinites}indefinite reading whereas those with long form adjectives cannot. The \ili{Lithuanian} Grammar \citep{Ambrazas1997} defines the short form \is{adjectives}adjective as indefinite/unmarked and the long form adjective with the definite suffix as definite/marked. Indeed, nominals accompanied by short adjectives can be used to introduce a new \is{discourse}discourse referent, a typical function of indefinites as in \REF{ex:sereikaite:32}. The nominal with short form \textit{strange} is used here to introduce a discourse-new information, i.e. the stranger that my friend has never heard about. Nominals with long adjectives, in contrast, are infelicitous in this context \REF{ex:sereikaite:32}.

\begin{exe}
\ex \label{ex:sereikaite:32}\il{Lithuanian}
Context: I am telling Mary for the first 
time about my evening at the bar where I have met a stranger that I have never seen before. \\
\gll {Vakar} {bare} {sutikau} \textbf{keistą} \textnormal{/} \textnormal{\#}\textbf{keistą-jį} {vaikiną}. \\
yesterday bar met {strange\textsubscript{weak}} / \phantom{\#}{strange-\textsc{def}\textsubscript{strong}} guy\\
\trans `Yesterday, at the bar, I met a strange guy.' \\
\end{exe}

The long form is acceptable in cases that include a prior mention of the linguistic antecedent \REF{ex:sereikaite:33}. This suggests that nominals with long \is{adjectives}adjectives enforce an \is{anaphora}anaphoric interpretation which is a common feature of definite expressions. 
 
\begin{exe}
\ex \label{ex:sereikaite:33}\il{Lithuanian}
Context: I have heard about a strange guy from Mary. Finally, yesterday I was able to meet that guy and now I am telling this story to Mary.\\
\gll {Vakar} {bare} {sutikau} {\textbf{keistą-jį}} {vaikiną}. \\
yesterday bar met {strange-\textsc{def}\textsubscript{strong}} guy\\
\trans `Yesterday, at the bar, I met the strange guy.' \\
\end{exe}

Another environment showing the same pattern is existential sentences with a post-verbal subject. The subject in this construction can only be indefinite \citep{GillonArmoskaite2015}. While nominals with short adjectives are possible in this environment, nominals with long adjectives are not \REF{ex:sereikaite:34}. This pattern is further evidence that short adjectives can behave like \isi{indefinites}, in contrast to long adjectives that lack this function. 

\begin{exe}
\ex \label{ex:sereikaite:34}\il{Lithuanian}
Context: I have heard a rustling sound in the bushes, I went closer and...\\
\gll {Ten} {buvo} \textbf{graži} \textnormal{/} \textnormal{\#}\textbf{gražio-ji} {katė.} \\
there was {beautiful\textsubscript{weak}} / \phantom{\#}{beautiful-\textsc{def}\textsubscript{strong}} cat\\
\trans `There was a beautiful cat.'  
\end{exe}

Taking these facts into account, at the first blush, there seems to be a sharp contrast between nominals with short and long form adjectives in terms of their (in)definite use. Nominals with short form adjectives occur in indefinite environments. In contrast, the presence of a long \is{adjectives}adjective in nominal expressions is incompatible with an \is{indefinites}indefinite context, and instead is licensed by linguistic antecedents exhibiting the behavior of strong, familiarity definites to which I now turn to.

\subsection{Familiarity}\is{familiarity|(}
Familiarity definites are referential expressions licensed by an \is{anaphora}anaphoric link to a preceding expression. In \ili{German}, as has already been discussed, the \is{strong definite articles}strong article, the non-contracted form, is used in such cases \REF{ex:sereikaite:17}, repeated here in \REF{ex:sereikaite:35}.

\begin{exe}
	\ex \label{ex:sereikaite:35}
	\ili{German} \citep[30]{Schwarz2009} \\
	\gll {In} {der} {New} {Yorker} {Bibliothek} {gibt} {es} \textbf{{ein}} \textbf{{Buch}} {über} {Topinambur}. {Neulich} {war} {ich} {dort} {und} {habe} \textnormal{\#}\textbf{im} \textnormal{/} \textbf{in} \textbf{dem} \textbf{{Buch}} {nach} {einer} {Antwort} {auf} {die} {Frage} {gesucht}, {ob} {man} {Topinambur} {grillen} {kann}.\\
	in the New York library exists \textsc{expl} {a} {book} about topinambur recently was I there and have \phantom{\#}{in-the\textsubscript{weak}} / {in} the\textsubscript{strong} {book} for an answer to the question searched whether one topinambur grill can\\
	\trans `In the New York public library, there is \textbf{a book} about topinambur. Recently, I was there and looked in \textbf{the	book} for an answer to the question of whether one can grill topinambur.’ 
\end{exe}

For the \isi{anaphoric reference}, \ili{Lithuanian} employs a nominal with a long form \is{adjectives}adjective. The first sentence in both examples in (\ref{ex:sereikaite:36}--\ref{ex:sereikaite:37}) introduces a new individual which is expressed by a \is{bare nominals}bare nominal. In the subsequent sentence in (\ref{ex:sereikaite:36}--\ref{ex:sereikaite:37}), that individual is mentioned for the second time and this time it is accompanied by an adjective. Only the long form adjective is possible in these situations and the short form adjective is infelicitous. The use of the long adjective in these examples is parallel to the use of the \is{strong definite articles}strong article in \ili{German} in the anaphoric context as in \REF{ex:sereikaite:35}. 

\begin{exe}
	\ex \label{ex:sereikaite:36}\il{Lithuanian}
	\gll {Neįtikėtina}, {vakar} {meno} {galerijoje} {vaizdo} {kameros} {užfiksavo} {\textbf{katiną}}. \textbf{Keistas-is} \textnormal{/} \textnormal{\#}\textbf{keistas} {\textbf{katinas}} {nepabūgo} {žmonių} {ir} {vaikščiojo} {po} {parodą} {it} {tikras} {meno} {žinovas}. \\
	incredible yesterday art gallery screen cameras captured {cat}. {strange-\textsc{def}\textsubscript{strong}} / \phantom{\#}{strange\textsubscript{weak}} {cat} not-scared people and walked through exhibition as real art connoisseur\\
	\trans `Incredible, yesterday in the art gallery, cameras captured \textbf{a cat}. \textbf{The strange cat} was not afraid of people and walked through the exhibition as a true art connoisseur.' 
\end{exe}

\begin{exe}
	\ex \label{ex:sereikaite:37}\il{Lithuanian}
	\gll {Marija} {pristatė} {mane} {savo} {\textbf{pusbroliui}} {iš} {Vilniaus}. \textbf{Gražus-is} \textnormal{/} \textnormal{\#}\textbf{gražus} {pusbrolis} {galantiškai} {nusilenkė} {ir} {pabučiavo} {man} {į} {ranką}.\\
	Marija introduced me self {cousin} from Vilnius {beautiful-\textsc{def}\textsubscript{strong}} / \phantom{\#}{beautiful\textsubscript{weak}} cousin gallantly bowed and kissed me to hand \\
	\trans `Marija introduced me to \textbf{her cousin} from Vilnius. \textbf{The beautiful cousin} gallantly bowed and kissed my hand.'
\end{exe}

Nevertheless, not all cases are that transparent. Examples like \REF{ex:sereikaite:38} present a situation where both the linguistic antecedent and its anaphoric expression are identical. The newly introduced antecedent in the first sentence in \REF{ex:sereikaite:38} takes the short form \is{adjectives}adjective, which, as discussed above, can function as \is{indefinites}indefinite. The \is{anaphora}anaphoric expression in the following sentence in \REF{ex:sereikaite:38} can appear in the long form as expected, given that the long form \is{definiteness marking}encodes \isi{anaphoricity}. However, the short form is not completely ruled out here as well. While 18 out of 27 speakers selected the long form, the rest of the speakers allowed the short form as well. It can be hypothesized that the short form is available in this situation because it is used as a \is{uniqueness}unique definite assuming that there is a unique famous writer that the speaker is referring to. I will come back to this type of use of short adjectives in \sectref{sec:sereikaite:4.3}.

\begin{exe}
	\ex \label{ex:sereikaite:38}\il{Lithuanian}
	\gll {Jonas} {pas} {save} {vakarienės} {pakvietė} \textbf{žymų} \textbf{rašytoją} {ir} {seną} {politiką}. \textbf{Žymus-is} \textnormal{/} \textbf{žymus} {\textbf{rašytojas}} {maloniai} {priėmė} {Jono} {kvietimą}. \\
	Jonas to his dinner invited {famous\textsubscript{weak}} {writer} and old\textsubscript{weak} politician {famous-\textsc{def}\textsubscript{strong}} / {famous\textsubscript{weak}} {writer} pleasantly accepted Jonas invitation.\\
	\trans `Jonas has invited \textbf{a famous writer} and an old politician for dinner. \textbf{The famous writer} pleasantly accepted Jonas' invitation.' 
\end{exe}

\is{anaphora}Anaphoric expressions can be more general than their antecedents. The more general \is{anaphoric definites}anaphoric definite in \ili{German} is expressed by the \is{strong definite articles}strong article \REF{ex:sereikaite:39} and the \is{weak definite articles}weak article definite is prohibited. The same behavior is observed in situations where the anaphoric phrase is an epithet as in \REF{ex:sereikaite:40}.


\begin{exe}
	\ex \label{ex:sereikaite:39}
	\ili{German} \citep[31]{Schwarz2009} \\
	\gll {Maria} {hat} {\textbf{einen}} {\textbf{Ornithologen}} {ins} {Seminar} {eingeladen}. {Ich} {halte} \textbf{von} \textbf{dem} \textnormal{/} \textnormal{\#}\textbf{vom} {\textbf{Mann}} {nicht} {sehr} {viel}. \\
	Maria has {an} {ornithologist} to-the seminar invited I hold \ {of} {the\textsubscript{strong}} / \phantom{\#}{of-the\textsubscript{weak}} {man} not very much\\
	\trans `Maria has invited \textbf{an ornithologist} to the seminar. I don't think very highly of \textbf{the man}.' 
\end{exe}

\begin{exe}
	\ex \label{ex:sereikaite:40}
	\ili{German} \citep[31]{Schwarz2009} \\
	\gll {\textbf{Hans}} {hat} {schon} {wieder} {angerufen}. {Ich} {will} \textbf{von} \textbf{dem} \textnormal{/} \textnormal{\#}\textbf{vom} \textbf{Idioten} {nichts} {mehr} {hören}.\\
	\textbf{Hans} has already again called I want of {the\textsubscript{strong}} / \phantom{\#}of-the\textsubscript{weak} \textbf{idiot} not hear\\
	\trans `\textbf{Hans} has called again. I don’t want to hear anything anymore from \textbf{that} \textbf{idiot}.’  
\end{exe}


Similarly, long adjectives can appear with \is{anaphora}anaphoric nominals that do not completely match their antecedents. For example, the \is{proper names}proper name \textit{Darius} in the second mention is referred to as `clingy guy' with the \is{adjectives}adjective in the long form, rather than short as illustrated in \REF{ex:sereikaite:41}. Additionally, the long form is also preferred over the short one with anaphoric epithets \REF{ex:sereikaite:42}.\largerpage[-2]

\begin{exe}
	\ex \label{ex:sereikaite:41}\il{Lithuanian}
	\gll {\textbf{Darius}} {man} {šiandiena} {skambino} {net} {dešimt} {kartų}. \textbf{Įkyrus-is} \textnormal{/} \textnormal{\#}\textbf{įkyrus} {vaikinas} {visiškai} {pamišo}. \\
	\textbf{Darius} me today called even ten times {clingy-\textsc{def}\textsubscript{strong}} / \phantom{\#}{clingy\textsubscript{weak}} guy totally went.mad\\
	\trans `\textbf{Darius} called me today at least ten times. \textbf{The clingy guy} went totally mad.' 
\end{exe}

\begin{exe}
	\ex \label{ex:sereikaite:42}\il{Lithuanian}
	\gll {\textbf{Darius}}, {būdamas} {vos} {penkerių} {metų}, {laimėjo} {matematikos} {olimpiadą}. \textbf{Jaunas-is} \textnormal{/} \textnormal{\#}\textbf{jaunas} {genijus} {labai} {didžiuojasi} {savo} {pasiekimais}.\\
	\textbf{Darius} being only five years won math olympiad {young-\textsc{def}\textsubscript{strong}} / \phantom{\#}{young\textsubscript{weak}} genius very proud self achievements\\
	\trans `When being only five years old, Darius won the math olympiad. The young genius is very proud of his achievements.' 
\end{exe}

\is{weak definite articles}\is{strong definite articles}Lastly, the strong vs. weak distinction can be captured in covarying uses where the value of the \is{quantifiers}quantifier determines the value of the definite. \ili{German} co-vary\-ing \is{anaphora}anaphoric uses are incompatible with the weak article and select the strong article instead \REF{ex:sereikaite:43}. 

\begin{exe}
	\ex \label{ex:sereikaite:43}
	\ili{German} \citep[33]{Schwarz2009} \\
	\gll {Jedes} {Mal}, {wenn} {\textbf{ein}} {\textbf{Onithologe}} {im} {Seminar} {einen} {Vortrag} {h\"alt}, {wollen} {die} {Studenten} \textbf{von} \textbf{dem} {\textbf{Mann}} {wissen} {ob} {Vogelgesang} {grammatischen} {Regeln} {folgt}. \\
	every time when {an} {ornithologist} in-the seminar a lecture holds want the students {of} {the\textsubscript{strong}} {man} know whether bird.singing grammatical rules follows\\
	\trans `Every time \textbf{an ornithologist} gives a lecture in the seminar, the students want to know from \textbf{the man} whether bird songs follow grammatical rules.' 
\end{exe}

Again, the long form \is{adjectives}adjective seems to be equivalent to the \ili{German} strong article and surfaces in covarying uses as a part of the \is{anaphora}anaphoric expression \REF{ex:sereikaite:44}.\footnote{This example is modeled on the basis of \citegen[134]{Ingason2016} example from \ili{Icelandic}.} In addition, the nominal with short form is felicitous for 12 speakers out of 27. Indeed, this context suffices to identify a unique famous artist. The speakers selecting the short form might be accessing this reading given that the short form, as will be demonstrated below, is compatible with \isi{uniqueness}.


\begin{exe}
	\ex \label{ex:sereikaite:44}\il{Lithuanian}
	\gll {Kiekvieną} {kartą} {kai} \textbf{{kino}} \textbf{{žvaigždė}} {aplanko} {mokyklą}, {studentai} {visuomet} {klausia} \textbf{žymio-jo} \textnormal{/} \textbf{žymaus} \textbf{{artisto}} {ar} {aktoriai} {gerai} {uždirba}. \\
	every time when {movie} {star} visits school students always ask {famous-\textsc{def}\textsubscript{strong}} / {famous}\textsubscript{weak} {artist} whether actors earn well\\
	\trans `Every time \textbf{a movie star} visits the school, students always ask \textbf{the famous artist} if actors earn well.' 
\end{exe}

To summarize, I have examined the behavior of nominals with short and long adjectives in \is{anaphora}anaphoric environments that include identical and non-identical linguistic antecedents, more general anaphoric phrases and anaphoric expressions in covarying uses. It has been demonstrated that \ili{Lithuanian}, similarly to \ili{German}, has one form that functions like a \is{familiar definites}familiar definite, namely the long form \is{adjectives}adjective with the definite suffix \textit{-ji(s)}. Nominals with short form adjectives lack anaphoric properties. However, they arise in contexts where there is a possibility of a referent to count as being unique. \is{familiarity|)}

\subsection{Uniqueness} \label{sec:sereikaite:4.3}\is{uniqueness|(}

The fact that nominals with short adjectives can be \is{indefinites}indefinite, as illustrated in \sectref{sec:sereikaite:4.1}, is only one part of the story. \citet{GillonArmoskaite2015} point out that, depending on the context, the short form adjectives can also have a definite reading. I now investigate this possibility by showing that nominal expressions with short forms can occur in situations that are licensed by uniqueness.

\subsubsection{Larger situation environments}\is{larger situation|(}

Larger situation environments \citep{Hawkins1978} license weak definites and permit only \is{weak definite articles}weak articles in \ili{German} as illustrated in \REF{ex:sereikaite:45}.

\begin{exe}
	\ex \label{ex:sereikaite:45}
	\ili{German} \citep[31]{Schwarz2009} \\
	\gll {Der} {Empfang} {wurde} \textbf{vom} \textnormal{/} \textnormal{\#}\textbf{von} \textbf{dem} \textbf{{Bürgermeister}} {eröffnet}.\\
	The reception was {by-the\textsubscript{weak}} / \phantom{\#}{by} {the\textsubscript{strong}} {mayor} opened\\
	\trans `The reception was opened by \textbf{the mayor}.’ 
\end{exe}

Interestingly, both types of \is{adjectives}adjectives are available in \ili{Lithuanian}, but are associated with different readings. The nominal with a short form stands for a \is{uniqueness}unique individual licensed by general world knowledge as exemplified in \REF{ex:sereikaite:46}. \REF{ex:sereikaite:46} is a general rule where following the law the president can fire anyone who occupies the role of the new prime minister.

\begin{exe}
	\ex \label{ex:sereikaite:46}\il{Lithuanian}
	\gll {Praėjus} {dviem} {savaitėm} {po} {rinkimų}, {prezidentas} {turi} {teisę} {atleisti} \textbf{naują} \textnormal{/} \textnormal{\#}\textbf{naują-jį} \textbf{{ministrą}} \textbf{{pirmininką}} {tik} {išskirtiniais} {atvejais}.\\
	passed two weeks after elections president has right fire {new\textsubscript{weak}} / \phantom{\#}{new-\textsc{def}\textsubscript{strong}} {minister} {prime} only exceptional cases\\
	\trans `Two weeks after the election, the president has a right to fire \textbf{the new prime minister} only in exceptional cases.' 
\end{exe}

In contrast, the long form denotes \is{specificity}context-specific unique individuals. For example, once the election happened, everyone knows who is the new president. Thus, there is a specific unique individual, and to \is{definiteness marking}encode such a reading the long form is used as in \REF{ex:sereikaite:47}.
 
\begin{exe} 
	\ex \label{ex:sereikaite:47}\il{Lithuanian}
	\gll {Po} {rinkimų} \textbf{naujas-is} \textnormal{/} \textnormal{\#}\textbf{naujas} {prezidentas} {paskambino} {miestelio} {merui}.\\
	after elections {new-\textsc{def}\textsubscript{strong}} / \phantom{\#}{new\textsubscript{weak}} president called city mayor\\
	\trans `After the election, \textbf{the new president} called the city mayor.' 
\end{exe} 

Note that it is not uncommon to encode different types of uniqueness context by different forms. For instance, \ili{Thai} makes a distinction between unique individuals that are supported by the world knowledge and those that are not \citep{Jenks2015}. Generally, \ili{Thai} provinces elect one Senator and two Ministers of Parliament. In \REF{ex:sereikaite:48}, the \is{bare nominals}bare noun phrase, generally used for weak definites, denotes a unique senator and this referent is licensed by the world knowledge. To encode a reading that distinguishes a unique individual from another individual, the \is{demonstratives}demonstrative, typically used for \is{anaphoric reference}anaphoric references, is used \REF{ex:sereikaite:49}.

\begin{exe}
	\ex \label{ex:sereikaite:48}
	\ili{Thai} \citep[107]{Jenks2015} \\
	\gll {s\v{ɔ}{ɔ}-w{ɔ}{ɔ}} {chiaŋ-mày} {\op}\textnormal{\#}\textbf{khon} \textbf{nán}{\cp} {gròot} {mâak}\\
	senator Chiang.Mai \phantom{(\#}{\textsc{clf}} {that} angry very\\
	\trans `The/\#that Senator from Chiang Mai is very angry' 
\end{exe}

\begin{exe}
	\ex \label{ex:sereikaite:49}
	\ili{Thai} \citep[107]{Jenks2015} \\
	\gll {s\v{ɔ}{ɔ}-s\v{ɔ}{ɔ}} {chiaŋ-m\`ay} \textnormal{\#}{\op}\textbf{khon} \textbf{nán}{\cp} {gròot} {mâak}\\
	MP. Chiang.Mai \phantom{\#(}{\textsc{clf}} {that} angry very\\
	\trans `\#The/that M.P. from Chiang Mai is very angry.' 
\end{exe}

Additionally, \is{uniqueness}\is{definite nominals}unique definite nominals can also be based on social or cultural knowledge \citep{Hawkins1978}. Again both forms are possible in \ili{Lithuanian} yielding different interpretations. \ili{Lithuanian} \is{comparatives}comparative \is{adjectives}adjectives occur with the suffix \textit{-esn-}, which is equivalent to the \ili{English} \textit{-er} in cases like \textit{smarter}. Both short and long adjectives can have a comparative form. The short form with the comparative suffix as in \REF{ex:sereikaite:50} refers to a \is{genericity}generic set of children that is unique. Nevertheless, in contrastive sentences that include a \is{specificity}specific unique set of children both forms are available \REF{ex:sereikate:51}. 
 
\begin{exe}
	\ex \label{ex:sereikaite:50}\il{Lithuanian}
	\gll {Mokslo} {komitetas} {norėtų}, {kad} {mokyklą} {pradėtų} {lankyti} \textbf{jaun-esn-i} \textnormal{/} \textnormal{\#}\textbf{jaun-esn-ie-ji} {vaikai}.\\
	education committee would.want that school begin attend-\textsc{inf} {young-\textsc{comp}-\textsc{nom.m.pl}\textsubscript{weak}} / \phantom{\#}{young-\textsc{comp}-\textsc{nom.m.pl}-\textsc{def}\textsubscript{strong}} children\\
	\trans `The education committee wants \textbf{the younger children} to start attending the school.' (adapted from the Internet)
\end{exe}

\begin{exe}
	\ex \label{ex:sereikate:51}\il{Lithuanian}
	\gll \textbf{Jaun-esn-ie-ji} \textnormal{/} \textbf{jaun-esn-i} {vaikai} {žaidė} {smėlio} {dėžėje}, {o} \textbf{vyr-esn-ie-ji} \textnormal{/} \textbf{vyr-esn-i} {vaikai} {laipiojo} {medžiais}.\\
	{young-\textsc{comp}-\textsc{nom.m.pl}-\textsc{def}\textsubscript{strong}} / {young-\textsc{comp}-\textsc{nom.m.pl}\textsubscript{weak}} children played sand box, while {old-\textsc{comp}-\textsc{nom.m.pl}-\textsc{def}\textsubscript{strong}} / {old-\textsc{comp}-\textsc{nom.m.pl}\textsubscript{weak}} children climbed trees.\\
	\trans `\textbf{The younger children} were playing in the sand box, while \textbf{the older children} were climbing the trees.' 
\end{exe}
\is{larger situation|)}

\subsubsection{Bridging context}\is{bridging|(}

\is{weak definite articles}\is{strong definite articles}I establish a further  distinction between nominals with short and long \is{adjectives}adjectives by exploring bridging contexts \citep{Clark1975}. There are two types of bridging contexts: \is{part-whole relationship}part-whole and \is{product-producer relationship}product-producer. The latter licenses the \is{uniqueness}unique definite article, whereas the former is associated with the \is{familiarity}familiar definite. This contrast is reflected in \ili{German}: the weak article is permitted in the part-whole context \REF{ex:sereikaite:52} and the strong article is realized in the product-producer environment \REF{ex:sereikaite:53}. 
 
\begin{exe}
	\ex \label{ex:sereikaite:52}
	\ili{German} \citep[52]{Schwarz2009} \\
	\gll \textbf{Der} \textbf{Kühlschrank} {war} {so} {groß}, {dass} {der} {Kürbis} {problemlos} \textbf{im} \textnormal{/} \#\textbf{in} \textbf{dem} {Gemüsefach} {untergebracht} {werden} {konnte}.\\
	{the} {fridge} was so big that the pumpkin problem {in-the\textsubscript{weak}} / \phantom{\#}{in} {the\textsubscript{strong}} crisper stowed be could\\
	\trans `\textbf{The fridge} was so big that the pumpkin could easily be stowed in \textbf{the crisper}.' 
\end{exe} 

\begin{exe}
	\ex \label{ex:sereikaite:53}
	\ili{German} \citep[53]{Schwarz2009} \\
	\gll \textbf{Das} \textbf{Theaterstück} {missfiel} {dem} {Kritiker} {so} {sehr}, {dass} {er} {in} {seiner} {Besprechung} {kein} {gutes} {Haar} \textnormal{\#}\textbf{am} \textnormal{/} \textbf{an} \textbf{dem} \textbf{Autor} {ließ}.\\
	{the} {play} displeased the critic so much that he in his review no good hair \phantom{\#}{on-the\textsubscript{weak}} / {on} {the\textsubscript{strong}} {author} left\\
	\trans ‘\textbf{The play} displeased the critic so much that he tore \textbf{the author} to pieces in his review.’ 
\end{exe}


Placing the short form \is{adjectives}adjective in the \is{part-whole relationship}part-whole environment results in felicity. In the situation where I am telling my friend for the first time about my car breaking down, to refer to \textit{the old engine} which is part of my car, the short form is used \REF{ex:sereikaite:54}. This gives additional evidence for the short form being compatible with situations governed by uniqueness. In contrast, the long form becomes acceptable in bridging contexts if the listener has some prior knowledge about the old engine from before \REF{ex:sereikaite:55}. 


\begin{exe}
	\ex \label{ex:sereikaite:54}\il{Lithuanian}
	Context: I am telling my friend for the first time about what happened to my car yesterday. My friend has no prior knowledge about the car. \\
	\gll {Vakar} {sugedo} {mano} {\textbf{automobilis}}, {kurį} {vairavau} {ištisus} {dešimtmečius}! {Autoserviso} {darbuotojai} {dabar} {taiso} \textbf{seną} \textnormal{/} \textnormal{\#}\textbf{seną-jį} {\textbf{variklį}}. {Tikiuosi} {automobilis} {ir} {vėl} {važiuos} {puikiai}.\\
	yesterday broke.down my {car} that drove whole decades repair.shop employees now fix {old\textsubscript{weak}} / \phantom{\#}{old-\textsc{def}\textsubscript{strong}} {engine} hope car and again will.drive well \\
	\trans `Yesterday, my \textbf{car}, that I have been driving for entire decades, broke down. The mechanics now are changing \textbf{the old engine}. I hope that the car will work great again!' 
\end{exe} 

\begin{exe}
	\ex \label{ex:sereikaite:55}\il{Lithuanian}
	Context: I have told my friend before that my car kept on breaking down because the old engine was not working properly. Today, I met my friend and told him again about my problems with the old engine. \\ 
	\gll {Vakar} {sugedo} {mano} {\textbf{automobilis}}. {Autoserviso} {darbuotojai} {dabar} {taiso} \textnormal{\#}\textbf{seną} \textnormal{/} \textbf{seną-jį} {\textbf{variklį}}. {Tikiuosi} {automobilis} {ir} {vėl} {važiuos} {puikiai}.\\
	yesterday broke.down my {car} repair.shop employees now fix \phantom{\#}{old\textsubscript{weak}} / {old-\textsc{def}\textsubscript{strong}} engine hope car and again will.drive well \\ 
	\trans `Yesterday, my \textbf{car} broke down. The mechanics now are changing \textbf{the old engine}. I hope that the car will work great again!' 
\end{exe} 


If the long form indeed functions like a strong article, it should appear in \is{product-producer relationship}product-producer bridging. This prediction is borne out. Modifying the author of the book by a long form yields felicity as in \REF{ex:sereikaite:56}. 20 speakers prefered the long form, their judgment is illustrated in the example. 7 speakers selected the short form. While it is unclear why some speakers use the short form in this context, the contrast for the rest of 20 speakers is pretty robust. 

\begin{exe}
	\ex \label{ex:sereikaite:56}\il{Lithuanian}
	\gll \textbf{{Knyga}} ``{Lietus}" {sulaukė} {neįtikėtino} {populiarumo}, {nepaisant} {to}, {kad} \textbf{talentingas-is} \textnormal{/} \textnormal{\#}\textbf{talentingas} {rašytojas} {nusprendė} {likti} {anonimas}.\\
	{book} `Rain' received incredible popularity despite {} that {talented-\textsc{def}\textsubscript{strong}} / \phantom{\#}{talented\textsubscript{weak}} writer decided remain anonymous\\
	\trans `\textbf{The book `Rain'} became incredibly popular despite the fact that \textbf{the talented writer} decided to remain anonymous.' 
\end{exe} 

All in all, the examination of \is{larger situation}larger situations and bridging contexts provides us with some evidence showing that nominals with short form \is{adjectives}adjectives can have a definite reading. Short adjectives resemble weak definites given their acceptability in \is{part-whole relationship}part-whole bridging contexts and larger situations based on general world knowledge. The fact that nominals with long adjectives are allowed in larger situations, but do not emerge in part-whole bridging contexts tell us that this form lacks the properties of a true weak article definite. While a precise characterization of the conditions that govern the use of long forms in larger situations requires further research, it is rather intriguing that the similar split within this environment also exists in \isi{numeral classifier languages} like \ili{Thai}. \is{bridging|)}\is{uniqueness|)}

\subsection{Section summary and implications}

To summarize this section, I have provided additional arguments that nominals with long form adjectives lack \is{indefinites}indefinite uses and indeed function like definites as has been suggested by \citet{GillonArmoskaite2015}. Specifically, using different \isi{familiarity} environments and \is{product-producer relationship}product-producer \isi{bridging} contexts, it was demonstrated that nominals with long form \is{adjectives}adjectives resemble \ili{German} nominals with the \is{strong definite articles}strong article licensed by familiarity. Furthermore, while nominals with short adjectives seem to be \is{definiteness marking}unmarked for definiteness, as noted by \citet{Ambrazas1997}, definite contexts were presented that trigger the occurrence of the short form. The nominals with short form adjectives surface in \is{part-whole relationship}part-whole \isi{bridging} contexts and \is{larger situation}larger situations based on general world knowledge, and thereby function like weak definites.
 
Given that I argued for the presence of the two \is{adjectives}adjective forms in \ili{Lithuanian} that occur in definite environments, an anonymous reviewer asks what the basic structure of a \ili{Lithuanian} \is{noun phrases}noun phrase would be. Indeed, these findings provide important implications for how the structure of a noun phrase could look like. Following \citet{GillonArmoskaite2015}, I assume that definite phrases in \ili{Lithuanian} involve a D layer. The long form, which is the short form plus the definite suffix \textit{-ji(s)} expresses \isi{anaphoricity}. I take the D head to be \textit{-ji(s)}.\footnote{Note that the suffixation of the definite morpheme is subject to local adjacency. The suffix cannot be realized on the adjective if there is an adverb intervening between the D head and the \is{nouns}noun as shown in (i).

    \begin{exe}
    \exi{\fts{(i)}} \label{ex:sereikaite:n9}\il{Lithuanian}
    \begin{xlist}  
    \exi{\fts{a.}}[]{\footnotesize
    	\gll gražus-\textbf{is} lokys\\
    \fts{beautiful-{\textsc{def}}} \fts{bear}\\
    \trans \normalfont `the beautiful bear'}
 
    \exi{\fts{b.}}[\fts{*}]{\footnotesize
    	\gll labai gražus-\textbf{is} lokys\\
     \fts{very} \fts{beautiful-{\textsc{def}}} \fts{bear}\\
     \trans \normalfont `the very beautiful bear'}
     \end{xlist}
     \end{exe}
     } Recall that short form is compatible with \isi{uniqueness}, which suggests that in those cases there also should be a D head, but it is not overtly expressed. Therefore, the D head can be \is{definiteness marking}encoded either by the suffix \textit{-ji(s)} or be marked as \is{null determiners}null as illustrated in \REF{ex:sereikaite:57}.
\begin{exe}\il{Lithuanian}
\ex The basic structure of \ili{Lithuanian} \is{definite nominals}definite nominals \label{ex:sereikaite:57}\\
\begin{forest}
[DP [D [ \textit{-ji(s)}/∅]]  [AP [A [\textit{gra\v{z}us} \\ `beautiful', roof]] [NP [\textit{lokys} \\ `bear', roof]]
]
]
\end{forest} 
\end{exe}

\section{Conclusion} \label{sec:sereikaite:5}

This paper has intended to show that the distribution of short and long form \is{adjectives}adjectives in \ili{Lithuanian} supports Schwarz's (\citeyear{Schwarz2009,Schwarz2013}) claim that there exist two types of definites: \isi{familiar definites} and \isi{unique definites}. The detailed analysis of nominals with two kinds of adjectives has revealed interesting parallels between two distinct languages, \ili{Lithuanian} and \ili{German}. \ili{Lithuanian}, similarly to \ili{German}, can use two forms to \is{definiteness marking}encode definiteness: long form \is{adjectives}adjective are compatible with \isi{familiarity} and short from adjectives are compatible with \isi{uniqueness}. This distinction emphasizes the need to adopt the hybrid approach that includes both familiarity and uniqueness for the analysis of definite uses. The reality of strong vs. weak distinction is supported further by identifying genetically unrelated languages that uses similar means to encode this distinction. \ili{Lithuanian} patterns with languages like Akan\il{Akan} and \ili{Thai} since it uses a bare form, the short adjective, for uniqueness and it has one marked form, namely the long adjective, that is equivalent to the strong article in \ili{German}. 

Long and short form \isi{demonstratives} are also distinguished in \ili{Lithuanian}. Further research would be to see what the nature of the definite interpretation of these forms is, and how this can be related to short vs. long adjective variations in \ili{Slavic}.

\section*{Acknowledgements}
I would like to thank Florian Schwarz for invaluable comments  and suggestions while working on this project. I also thank the audience at `Definiteness Across Languages' Workshop and the anonymous reviewers. Many thanks to Ava Irani for her suggestions and Solveiga Armoskaite for brief comments on the data. I also thank my consultants who provided their judgments. 

\section*{Abbreviations}
\begin{tabbing}
	\textsc{comp}\hspace{1em} \= comparative morpheme\\ \kill 
	\textsc{def} \> the definite morpheme \textit{–ji(s)} \\
	D\textsubscript{weak} \> the weak article definite\\
	D\textsubscript{strong} \> the strong article definite \\
\end{tabbing}

{\sloppy
\printbibliography[heading=subbibliography,notkeyword=this]}

\end{document}
