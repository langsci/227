\documentclass[output=paper
,modfonts
,nonflat]{langsci/langscibook} 

\title{Definiteness in Russian bare nominal kinds}
\author{%
	Olga Borik\affiliation{UNED}\lastand 
	M.-Teresa Espinal\affiliation{UAB}
}
% \chapterDOI{} %will be filled in at production

% \epigram{}


\begin{document}
	
	\abstract{
		In the literature on \is{generic reference}generic nominal reference, it is usually pointed out that in \ili{Russian}, both singular and plural nominal expressions can have a generic reference \citep{Chierchia1998,Doron2003,Dayal2004}. The main contribution of this article is to propose an explicit analysis for composing definite kinds from \isi{bare nominals} in this language. We provide independent empirical support for the definiteness of apparent bare nominals in argument position of \is{kind-level predicates}kind-level predicates and argue that definiteness is to be associated with a \is{null determiners}null D(eterminer), interpreted as the \isi{iota operator}. The general hypothesis we defend is that definite kinds, even in a \is{article-less languages}language without \isi{articles} such as \ili{Russian}, \is{definiteness marking}encode definiteness semantically and syntactically.
	}
	
	
	\maketitle
	\section{Introduction} \label{sec:borik:1}\is{kinds|(}
	In the literature on generic nominal reference it is usually pointed out that in \ili{Russian}, a language without articles, both \is{bare singulars}bare singular and \is{bare plurals}bare plural \is{bare nominals}nominal expressions can have a generic reference \citep{Chierchia1998,Doron2003,Dayal2004}. This is exemplified in \REF{ex:borik:1}, where \isi{nouns} specified morphologically for singular \REF{ex:borik:1a} and for plural \REF{ex:borik:1b} occur in argument position of a k(ind)-level predicate.\footnote{In this paper, we assume a three-way classification of verbal predicates into \is{kind-level predicates}k(ind)-level, \is{individual-level predicates}i(ndividual)-level and \is{stage-level predicates}s(tage)-level \citep{Carlson1977}. While k-level predicates appear to form a scarce but stable class, it is well known that the division line between \is{individual-level predicates}i- and \is{stage-level predicates}s-level predicates is not clearly marked. For instance, \textit{fly} in (i.a) denotes an i-level property while in (i.b) it functions as an s-level predicate: 
		
	\ea\label{ex:borik:i}
		\ea\label{ex:borik:ia} \textit{Hummingbirds fly backwards.}
		\ex\label{ex:borik:ib} \textit{Hummingbirds are flying over the lake.}
		\z
	\z}
	\newpage
	In this context both \textit{panda} and \textit{pandy} can be said to refer to kinds. 
	\ea\label{ex:borik:1}\il{Russian}
	\ea\label{ex:borik:1a}{
		\gll Panda naxoditsja		na		grani			is\v{c}eznovenija.\\ 
		panda.\textsc{nom.sg} is.found on verge extinction.\textsc{gen}\\
	}
	\ex\label{ex:borik:1b}{\il{Russian}
		\gll Pandy		naxodjatsja	na	grani	is\v{c}eznovenija. \\
		panda.\textsc{nom.pl} 	are.found 	on	verge	extinction.\textsc{gen}\\
	}
	\z
	\z
	
	A common background assumption considers plural \isi{generics} as more natural and preferable, so in a significant part of literature on \isi{genericity} it is taken for granted that plurals (\isi{bare plurals} in \ili{English}) constitute the ``default'' way to refer to kinds.\footnote{See \citet{Ionin2011} for an \is{experimental study}experimental investigation on the expression of \isi{genericity} in \ili{English}, \ili{Spanish} and \il{Portuguese!Brazilian Portuguese}Brazilian Portuguese.} Setting aside the question of what is the ``default'' way to express genericity in the nominal domain in \ili{Russian}, we simply point out that, given that \REF{ex:borik:1a} is grammatical and natural, an analysis of it is needed in the theory of grammar in any case.
	
	In contrast to \ili{Russian}, in a language with overt \isi{determiners}, \ili{English} for instance, the subject of a sentence corresponding to \REF{ex:borik:1a} will be expressed by means of a \is{definite generics}definite generic \citep{Carlson1977} or the singular generic \citep{Chierchia1998} \textit{the} N construction (i.e. \textit{the panda}), as in \REF{ex:borik:2a}. On the other hand, \ili{English} also allows bare plurals to refer to kinds, as illustrated in \REF{ex:borik:2b}.
	
	\ea\label{ex:borik:2}
	\ea\label{ex:borik:2a}
	\textit{The panda is on the verge of extinction.}
	\ex\label{ex:borik:2b}
	\textit{Pandas are on the verge of extinction.}
	\z
	\z
	
	The correspondence between the so-called \ili{English} \is{definite generics}definite generic and the \ili{Russian} \is{bare nominals}bare nominal with a \isi{kind reference} interpretation in \REF{ex:borik:1a} is usually assumed to hold merely on the basis of their singular \isi{number} morphology \citep[cf.][]{Dayal2004}, so a reasonable expectation is that the analysis assumed for definite generics in \ili{English} can also be extended to the corresponding \ili{Russian} cases. This approach has to address at least the following issue. Any analysis of the \ili{English} definite generic includes the \isi{iota operator} ($\iota$) in the semantic represen\hyp{}\linebreak\newpage\noindent tation (cf. \citealt{Chierchia1998}, \citealt{Dayal2004}), which is quite indisputable for \ili{English}, given that these expressions appear with a \is{definite articles}definite article.\footnote{Although see \citet{CoppockBeaver2015}, who argue that definiteness as encoded by the \is{definite articles}definite article must be distinguished from determinacy, which consists in denoting an individual. Should this claim also be adopted for \ili{Russian}, it would need an independent motivation, since \ili{Russian} does not overtly \is{definiteness marking}express definiteness.}
	
	More generally, a number of questions arise with respect to \REF{ex:borik:2} if we take into account some cross-linguistic data. In \ili{Spanish}, for instance, \isi{bare plurals} do not have a \is{genericity}generic reading \citep{Laca1990,Dobrovie-Sorin1996, Dobrovie-Sorin2003}, making them different from bare plurals in \ili{English} (e.g. \ref{ex:borik:2b}), which are considered to be the genuine expression of kind reference in that language (\citealt{Longobardi1994, Longobardi2001, Longobardi2005,Chierchia1998,Dayal2004}, i.a.). By contrast, the default way to refer to kinds in \ili{Spanish} is by means of a (non-plural) \is{common nouns|(}common noun preceded by a \is{definite articles}definite article \citep{Borik2015}. The question is then how to derive a \isi{kind reference} for languages like \ili{Spanish} and \ili{English} keeping in mind these crucial differences concerning the interpretation of bare plurals. A look at languages like \ili{Russian} makes the issue even more complex: \ili{Russian}, does \is{article-less languages}not have any articles but clearly possesses the means to make reference to kinds, as shown in \REF{ex:borik:1}. Does this mean that the same type of analysis as for \ili{English} and \ili{Spanish} could or should be extended to \ili{Russian} despite the observed superficial differences in the syntax of nominal phrases?\footnote{See also \citet{Cyrino2015} for an analysis of \isi{definite kinds} and \is{definite generics}definite plural generics within the \is{NP/DP Approach}NP/DP debate in \il{Portuguese!Brazilian Portuguese}Brazilian Portuguese, a language that allows the omission of the article in all argument positions.}
	
	This paper aims at contributing to an understanding of kind expressions of the type exemplified in \REF{ex:borik:1a}. We provide independent empirical support for the definiteness of the subject in \REF{ex:borik:1a}, and argue that it is to be associated with a \is{null determiners}null D(eterminer), interpreted as \is{iota operator}{$\iota$}. We postulate the structure in \REF{ex:borik:3a} for \is{definite kinds}definite kind arguments in languages with and \is{article-less languages}without articles (e.g. \ili{Germanic}, \ili{Romance}, \ili{Slavic}), the meaning of which is represented in \REF{ex:borik:3b}. 
	
	\ea\label{ex:borik:3}
	\ea\label{ex:borik:3a}
	$[_{\mbox{\scriptsize DP}} \mbox{D} [_{\mbox{\scriptsize NP}} \mbox{N}]]$\\
	\ex\label{ex:borik:3b}
	$[\![ \mbox{Def N} ]\!] = \iota x^k [\mbox{P}(x^k)]$\\
	where P corresponds to the descriptive content of a noun N, and $x^k \in K$ (i.e. the domain of kinds)
	\z
	\z
	
	Although we do not deal with plural kind expressions exemplified in \REF{ex:borik:1b} in this paper, we would like to point out that they do not constitute a counterexample to our analysis for \REF{ex:borik:1a}. We assume that a different syntactic and semantic composition is to be associated with the \is{genericity}generic \is{bare plurals}(bare) plural in \REF{ex:borik:1b}. In particular, the analysis proposed in \citep{Chierchia1998}, in which plural kind nominals are semantically derived by the down operator $^{\scriptsize{\bigcap}}$ that applies to plural properties, could be adopted to account for plural \isi{generics} in \ili{Russian}. Our hypothesis (which we will not defend or justify further in this paper) with respect to plural kind nominals in \ili{Russian} is, therefore, that these expressions are, indeed, derived from pluralities and are specified for \is{number}Number, namely, for plural. Their structural representation would then look like in \REF{ex:borik:4}. 
	
	\ea\label{ex:borik:4}
	$^{\scriptsize{\bigcap}} [_{\mbox{\scriptsize NumP}}  \mbox{Num}_{\scriptsize+pl} [_{\mbox{\scriptsize NP}}  \mbox{N} ]]$
	\z
	
	The differences between \REF{ex:borik:3a}, the structure that we adopt for definite kinds, and \REF{ex:borik:4}, the structure that we would hypothesize for generic plurals, are obvious. First of all, definite kinds are syntactically and semantically definite and hence are structurally represented as full DPs, whereas there is no a priori evidence to suggest that the same holds for generic plurals.\footnote{This matter, however, deserves a full and thorough investigation, which falls outside the scope of this paper.}  Secondly, only in the structure for \is{genericity}generic plurals Number is present.\footnote{We differentiate between morphophonological \isi{number}, on the one hand, and syntactic Number, which is always interpreted semantically, on the other. In \ili{Russian}, any nominal expression is marked for number and case and these two specifications come as a cluster. In other words, it is impossible to determine which part of a cluster encodes number and which part encodes case, which is a standard feature of a language with synthetic morphology. We assume that this cluster does not necessarily correspond to a syntactic Number projection, which has to have a semantic effect, and yield either a singular or a plural interpretation for a nominal phrase (cf. \citealt{Ionin2006,Pereltsvaig2013} for similar claims).} We will not deal specifically with the syntax and semantics for Number in this paper, but in general, we assume that definite kinds are syntactically and semantically numberless, at least in those languages where nominals inflect for number (see \citealt{Borik2015} for details).
	
	The paper is organized as follows. \sectref{sec:borik:2}  presents the theoretical framework that constitutes the basis for our analysis. We will introduce the fundamental theoretical claims regarding the composition of definite kinds, focusing, in turn, on the meaning of Ns (properties of kinds) and the meaning of the \is{definite articles}definite article \is{iota operator}($\iota$). In \sectref{sec:borik:3} we will present our analysis of definite kinds in \ili{Russian}. With this aim in mind we will provide both semantic arguments for definiteness and syntactic arguments for a \is{determiner phrases}DP structure with a \is{null determiners}null D (translated as $\iota$). This section will close with an account of modified definite kinds. \sectref{sec:borik:4} will conclude the paper. 
	
	\section{Theoretical background} \label{sec:borik:2}
	
	In this section we will briefly summarize the theoretical assumptions or postulates underlying our account of definite kinds in natural languages.
	
	We assume that definite kinds express \isi{D-genericity} (cf. \citealt{KrifkaEtAlii1995}) and argue that they are composed by applying \is{iota operator}$\iota$, which is encoded by the \is{definite articles}definite article, to the denotation of a common noun, which denotes properties of kinds. This proposal is conceived as a universal principle, no matter whether the languages considered have overt articles (such as \ili{English}) or not (such as \ili{Russian}). 
	
	We start this section by discussing the meaning of common nouns. We argue that they denote properties of kinds (\citealt{Espinal2007,Espinal2007a}; \citealt{Dobrovie-Sorin2008}; \citealt{Espinal2010}; \citealt{Espinal2011}). Next, we discuss the meaning of the definite article, conceived as a \isi{maximality} operator \citep{Sharvy1980}, and the composition of a \is{definite kinds}definite kind reading. 
	
	\subsection{Theoretical postulate 1: Root common nouns denote properties of kinds}
	
	\is{kind reference}Kind reference in natural language is quite often assumed to be a special type of reference contrasted with the reference to objects. In other words, if objects are standard entities of the semantic ontology, so are kinds. This theoretical hypothesis can be traced back to at least \citet{Carlson1977}, who distinguished between three types of entities relevant for natural language semantics: kinds, that is, the denotation of \textit{the panda} and \textit{pandas} in \REF{ex:borik:2}; objects, that is, the denotation of \isi{proper names} and common noun phrases; and stages, i.e. the denotation of the last type of nominal expressions in combination with \is{stage-level predicates}stage-level predicates. Kinds and objects, in Carlson's typology, are abstract entities and together they form a class of ``individuals'', whereas stages are concrete spatio-temporal realizations of abstract entities. 
	
	In less fine-grained classifications of entities, only two types are recognized: kinds and objects (cf. \citealt{Zamparelli1995}).\footnote{In a different terminological tradition (e.g. \citealt{Vergnaud1992}) this distinction corresponds to types vs. tokens.} This is the ontology assumed here as well: we distinguish between kinds, or abstract entities, and objects, or particular entities, although we do not agree with \citet{Carlson1977}, \citet{Zamparelli1995}, and many others after them, for whom the denotation of a common noun is a kind entity. \newpage
	
	Under a different approach it is claimed in the semantic literature that common nouns denote properties, rather than entities (\citealt{Chierchia1984,Chierchia1998,Partee1987} among many others), that is, common nouns are lexical predicates. 
	
	In this paper, we adopt a third alternative and postulate that common nouns denote \is{kind properties}properties of kinds.\footnote{We adopt this hypothesis for all types of nouns, i.e. \is{count nouns}count, \is{mass nouns}mass and abstract nouns.} This alternative has been empirically motivated in a number of recent proposals, including \citegen{Dobrovie-Sorin2008} work on \isi{bare nouns} in \il{Portuguese!Brazilian Portuguese}Brazilian Portuguese, \citegen{McNally2004} analysis of relational \is{adjectives}adjectives, and \citegen{Espinal2010} and Espinal \& McNally's (\citeyear{Espinal2007a}; \citeyear{Espinal2011})  semantic description of the meaning of bare nouns in object position in \ili{Catalan} and \ili{Spanish}. The arguments supporting the hypothesis that common nouns denote descriptions of kinds are based on pronominalization, \is{number neutrality}number neutral interpretation and \is{adjectives}adjective modification. The reasoning is the following: 
	
	\begin{enumerate}[(i)]
	
	\item A common noun (a real \is{bare nominals}bare nominal) cannot be taken to refer to individual object-entities because the \is{anaphora}anaphoric pronoun that it licenses (in some \ili{Romance} languages) is not compatible with an object/token interpretation (cf. the difference between \ili{Catalan} \textit{en} lit. \is{numeral `one'}`one', referring to properties, and \textit{el/la/els/les} lit. 3\textsc{rd.acc.sg/pl.masc/fem} `it/them'); if it cannot denote an entity, it must denote a property. 
	
	\item If a common noun has a property denotation, it has no inherent \isi{number} information, and therefore it has a number neutral interpretation (i.e. it is compatible with atomicity and non-atomicity entailments, \citealt{Farkas2003}); by contrast, nouns specified syntactically for Number refer either to atomic or non-atomic sums. 
	
	\item If a common noun had an individual property denotation, it would be expected to easily combine with any kind of modifier, but this is not the case. Bare nouns in syntactic positions that allow bare nominals (e.g. in object position of a restricted class of predicates (Espinal \& McNally \citeyear{Espinal2007a}; \citeyear{Espinal2011}) and in predicate position of copular sentences (\citealt{deSwart2007}; \citealt{Zamparelli2008})) can only combine with classifying adjectives, and this restriction can be explained only if both expressions are taken to denote \is{kind properties}properties of kinds or if the appropriate adjectives are \is{modified kinds}kind modifiers.
	\end{enumerate} 
	
	We thus conclude that it is highly plausible to assume the denotation of a common noun to be a \is{kind properties}property of a kind.\footnote{This view should be contrasted with those in which the interpretation of a \is{nominal roots}nominal root is equivalent to that of a \is{mass nouns}mass noun \citep{Borer2005,Rothstein2010}, and with those that derive taxonomic kinds in the lexicon by a direct application of the MASS operation to a N\textsubscript{root} \citep{PiresdeOliveira2011,Trugman2013}.}
	
	Now, what precisely does it mean to say that common nouns denote \is{kind properties}properties of kinds? We assume that there are two domains in our semantic ontology, the domain of objects and the domain of kinds. Under a standard view, the denotation of the predicate with the descriptive content P is the set of objects that share property P. Thus, the denotation of the noun \textit{boy} in the domain of objects is a set of objects that have the boy-property. Note, however, that in our world some nouns can denote singleton sets (e.g. \textit{sun} or \textit{moon}). Without challenging the process described above, we propose that instead of the domain of objects, common nouns range over kinds, conceived as integral entities. Thus, the same noun \textit{boy} in our proposal looks for entities that share a boy-property but in the domain of kinds rather than objects. 
	
	In accordance with what we have just said the meaning of a common noun should have the logical representation in \REF{ex:borik:5}, where P stands for a property corresponding to the descriptive content of N, and $x^k$ a kind entity, such that the property P applies to $x^k$.
	
	\ea\label{ex:borik:5}
	$[\![N]\!] = \lambda{x}^k [\mbox{P}(x^k)]$		
	\z					 
	
	Having given a formal definition of the denotation of a \is{common nouns|)}common noun, we will now briefly clarify our more general assumptions about kinds, although we do not pretend to give a full justified answer to the question of what type of entities kinds essentially are. Following \citet{Borik2015}, we adopt the claim that kinds are not sets of \isi{subkinds}, but are instead perceived as integral, undivided entities with no internal structure, which means that kinds do not form part of a standard quantificational domain for individuals represented by a lattice structure \citep{Link1983}. We also share the view of \citet{Mueller-Reichau2011}, according to whom kinds are, in essence, abstract sortal concepts. Sortal concepts are mental representations that are used to ``categorize and individuate objects'' \citep[21]{Mueller-Reichau2011}. Thus, kinds are entities, but their (mental) representations are obtained by abstraction over a number of individual objects that share certain relevant properties. This, however, does not necessarily mean that linguistically, a kind should necessarily be construed as a set of representative objects, although conceptually it might be the case. 
	
	\subsection[Theoretical postulate 2: The definite article corresponds to ι and expresses maximality]{Theoretical postulate 2: The definite article corresponds to $\iota$ and expresses maximality}\is{definite articles}\is{maximality}\is{iota operator|(}
	
	In \citet{Partee1987}, it is proposed that \isi{definite noun phrases} are generated by a \is{type shifting}type shifting operator that maps a singleton property $\langle e, t\rangle$ onto an individual denotation of type $\langle e\rangle$. This type shifting operation is called \textit{iota}. In this sense, the meaning of the definite article is to map a property onto the maximal/\is{uniqueness}unique\largerpage individual having that property.\footnote{The terms \is{maximality}\textit{maximal} and \textit{unique} are used in this paper in the sense of \citet{Sharvy1980} and \citet{Link1983}, who provide a unified semantics for definiteness, independently of whether the definite article combines with a singular or a plural expression. Thus, these terms should not be confused or even associated with plural and singular \isi{number}, respectively.}
	
	\ea \label{ex:borik:6}
	$[\![\mbox{D}_{\mbox{\scriptsize{DEF}}}]\!] = \mbox{P} \rightarrow \iota{x} [\mbox{P}(x)]$
	\z
	
	When the \is{definite articles}definite article applies to a \is{nouns}noun that denotes a \is{kind properties}property of a kind, the iota operator yields a \is{maximality}maximal/\is{uniqueness}unique kind entity. This is how \is{definite kinds|(}definite kind expressions are derived. Crucially for our analysis, in the composition of definite kinds, there is no intervener between the iota operator, associated with the definite article (in languages with articles), and the noun. We illustrate this derivation in example \REF{ex:borik:7}. 
	
	\ea\label{ex:borik:7}
	\ea\label{ex:borik:7a}
	\textit{
		\textbf{The panda} is on the verge of extinction.} \\
	\ex\label{ex:borik:7b}
	$[_{\mbox{\scriptsize{DP}}} \mbox{\textit{ the }} [_{\mbox{\scriptsize{NP}}} \mbox{\textit{ panda}} ]]$\\
	\ex\label{ex:borik:7c}
	$[\![\mbox{\textit{the panda}}]\!] = \iota{x^k} [\mbox{\textit{panda}}(x^k)]$
	\z
	\z
	
	The subject of \REF{ex:borik:7a}, repeated from \REF{ex:borik:2a}, is a definite kind expression derived by applying the iota operator to the noun \textit{panda}. Its syntactic structure is given in \REF{ex:borik:7b}, and the semantic composition associated with this expression is provided in \REF{ex:borik:7c}.\footnote{Once again, we propose this derivation for all types of nouns, i.e. \is{count nouns}count, \is{mass nouns}mass and abstract nouns. See \citet{Borik2015} for details.} This is the essence of our analysis of definite kinds, which we would like to extend to \ili{Russian}. 
	In this section, we have presented the fundamental theoretical postulates on which we base our analysis of \is{kind reference}reference to kinds in natural languages. We now address the main issue of this paper, namely, the question of whether \ili{Russian} has definite kinds, in spite of the fact that it has no overt \isi{articles}, and which are the arguments that support the existence of definite kinds in this language. 
	
	\section{Definite kinds in Russian} \label{sec:borik:3}
	
	As we pointed out in \sectref{sec:borik:1}, the correspondence between the \ili{English} definite kind expression in \REF{ex:borik:2a} and the \ili{Russian} \is{bare nominals}bare nominal in \REF{ex:borik:1a} (repeated in \ref{ex:borik:8}) with a \isi{kind reference} is usually assumed to hold, and a reasonable expectation is that the analysis adopted for definite kinds in \ili{English} can also be extended to \ili{Russian} cases. 
	
	\ea\label{ex:borik:8}\il{Russian}
	\gll Panda naxoditsja		na		grani		is\v{c}eznovenija. \\
	panda.\textsc{nom.sg} 	is.found 			on		verge		extinction.\textsc{gen}\\
	\glt `The panda is on the verge of extinction.'
	\z
	
	However, any analysis of \ili{English} definite kinds includes at least the iota operator in the semantic representation (cf. \citealt{Chierchia1998}, \citealt{Dayal2004}). The iota operator is standardly assumed to correspond to the \is{definite articles}definite article, a claim that we do not want to challenge. However, in the \is{article-less languages}absence of articles in \ili{Russian}, we should be able to find other independent evidence that the iota operator is, indeed, present in the semantic representation of the subject argument in \REF{ex:borik:8} and not merely assume that it is there due to an interpretation that corresponds to the \ili{English} kind nominal. In \sectref{sec:borik:3.1} and \sectref{sec:borik:3.2} we provide independent empirical semantic and syntactic arguments for the definiteness of the subject in \REF{ex:borik:8} and argue that it is to be associated with a \is{null determiners}null D(eterminer), interpreted as the iota operator. \is{iota operator|)}
	
	\subsection{Semantic definiteness of kind referring expressions} \label{sec:borik:3.1}
	
	The core of the argument that we employ to prove that \ili{Russian} definite kinds are really semantically definite is based on the use and interpretation of these expressions in a context that requires definiteness. The following context can show that \is{kind reference}kind-referring expressions behave like proper \isi{definites}. 
	
	\ea\label{ex:borik:9}
	Context: In a biology lesson, the teacher explains various things about mammals. She explains that there are many endangered species in the world, then says the following:
	
	\textit{\textbf{The whale}, for instance, is on the verge of extinction.}
	\z
	
	Note first that in \ili{English}, the only morphologically singular expression that can refer to the species itself, and not to a \is{subkinds}subkind or an individual whale, is the definite one, i.e. \textit{the whale} \citep{Jespersen1927}, which we claim to be unspecified for \is{number}Number. A \is{determiner phrases}DP with a \is{demonstratives}demonstrative or a \is{numerals}numeral, as illustrated in \REF{ex:borik:10}, will not get the same interpretation as the definite kind expression in \REF{ex:borik:9}.
	
	\ea\label{ex:borik:10}
	\ea\label{ex:borik:10a}
	\textit{
		\textbf{This whale}, for instance, is on the verge of extinction. \\
	}
	\ex\label{ex:borik:10b}
	\textit{
		\textbf{One whale}, for instance, is on the verge of extinction.	
	}\z
	\z
	
	\REF{ex:borik:10a} with the demonstrative can only be acceptable if the teacher points directly to a picture of a representative instance of the corresponding type of whale (say, a blue whale), and thus, refers to a subkind via a representative, and \REF{ex:borik:10b} can only refer to a subkind of whale as well. 
	
	In \ili{Russian}, in the context of \REF{ex:borik:9}, the only expression that can be used is the \is{bare nouns}bare noun \textit{kit}, as illustrated in \REF{ex:borik:11}. \textit{Kit} in (11) has exactly the same interpretation as the overt DP \textit{the whale} in \ili{English}, and cannot get an interpretation comparable to \REF{ex:borik:10a} or \REF{ex:borik:10b}. This strongly suggests that \textit{kit} in \REF{ex:borik:11} corresponds to a \textit{definite} \is{kind reference}kind referring expression.  
	
	\ea\label{ex:borik:11}\il{Russian}
	\gll	\textbf{Kit}, 				naprimer, 		naxoditsja	 	na 	grani		is\v{c}eznovenija.\\
	whale.\textsc{nom.}	for.instance 	is.found 		on	verge		extinction.\textsc{gen.}\\
	\glt	`The whale, for instance, is on the verge of extinction.'
	\z
	
	Note, however, that theoretically, there could still be an option that while in \ili{English} the kind referring \is{determiner phrases}DP has to be definite, in \ili{Russian} it might be \is{indefinites}indefinite. Next, we will discuss why this is not the case.
	
	Even though it is commonly believed that with \is{kind-level predicates}k-level predicates indefinite DPs can only be interpreted taxonomically, i.e. as referring to a \is{subkinds}subkind rather than to a kind (see \citealt{Mueller-Reichau2011} and references therein), \citegen{Dayal2004} examples like \textit{to invent a pumpkin crusher} challenge this standard assumption. In this paper, we follow Mueller-Reichau who argues that there is a fundamental difference between k-level predicates like \textit{to be extinct} and the ones like \textit{to invent}. Only the latter allow for reference to novel (non-\is{familiarity}familiar) kinds, whereas the former impose a familiarity condition on the argument. This is why, by default, \textit{A blue whale is in danger of extinction} can only be interpreted as referring to a subkind of the blue whale, whereas \textit{Fred invented a pumpkin crusher} can be interpreted as referring to the kind pumpkin crusher, as well as to a subkind of crusher.\footnote{We thank an anonymous reviewer for the observation that \textit{Fred invented a pumpkin crusher} allows for two interpretations: the kind `pumpkin crusher' and a \is{subkinds}subkind of `crusher'. Our intuition is that this is due to the fact that the object NP contains a modified \is{nouns}noun. Thus, if we consider a non-modified NP, as in \textit{Steve Jobs invented an i-pod} only the subkind reading is salient.} This distinction between different types of k-level predicates is both empirically motivated by the examples just given and by our intuition: it is difficult for something that has not existed before to become extinct, therefore, \textit{to be extinct} requires familiar entities. By contrast, it is expected that if someone invents something, they will invent novel entities. 
	
	We observe similar effects in \ili{Russian} with the same type of \is{kind-level predicates}predicates: in \REF{ex:borik:12a} an indefinite description can only refer to a subkind of whale, but the nominal in object position in \REF{ex:borik:12b} can refer, indeed, to a new kind of artifact, a `mechanical calculator', as well as to a subkind of `calculator'.\footnote{There are overt indefinite markers in \ili{Russian}, although they are not \isi{articles}. In \REF{ex:borik:12} we use the unstressed version of \textit{odin} \is{numeral `one'}`one', which we take to be a \isi{specificity} marker for \isi{indefinites} in \ili{Russian} (cf. \citealt{Ionin2013}). If this marker bears stress, it is interpreted as a \is{numerals}numeral. 
		Note also that not all native speakers readily accept a \is{subkinds}subkind interpretation for examples like \REF{ex:borik:12a}. We have encountered judgments that vary from full rejection to full acceptance.}
	
	\ea\label{ex:borik:12}\il{Russian}
	\ea\label{ex:borik:12a}
	\gll \textbf{Odin} \textbf{kit} naxoditsja na grani is\v{c}eznovenija.\\
	One whale.\textsc{nom.sg} is.found on verge extinction.\textsc{gen}\\
	\glt	`One whale is in danger of extinction.'\\
	\ex\label{ex:borik:12b}\il{Russian}
	\gll Fred izobrel \textbf{odnu} \textbf{s\v{c}etnuju} \textbf{ma\v{s}inu}.\\
	Fred invented one.\textsc{acc.sg} calculating.\textsc{acc.sg} machine.\textsc{acc.sg}\\
	\glt `Fred invented a mechanical calculator.'
	\z
	\z
	
	Thus, we have all reasons to believe that the same distinction between different types of \is{kind-level predicates}k-level predicates that Mueller-Reichau postulates for \ili{English} also holds in \ili{Russian}. Crucially, according to this view, with predicates of the \textit{extinct}-type, ``the speaker presupposes the existence of instances of the kind X as \is{hearer knowledge}known to the hearer'' \citep[80]{Mueller-Reichau2011}. This lexical specification blocks \is{kind reference}reference to a kind for an \is{indefinites}indefinite expression in the context of \textit{extinct}-type predicates.\footnote{Similarly, \citet{Stankovic2016} postulates a complex \is{determiner phrases}DP structure for \ili{Serbo-Croatian}, which includes a \is{kind reference}kind-referring DP embedded under an individual referring DP. He argues that the kind-referring DP can only be definite, not \is{indefinites}indefinite in \ili{Serbo-Croatian}.}
	
	Let us now go back to our example \REF{ex:borik:11}. As has just been demonstrated in \REF{ex:borik:12a}, should the subject of \REF{ex:borik:11} be indefinite, it would necessarily yield a \is{subkinds}subkind reading, which it does not. This allows us to conclude that the subject argument in \REF{ex:borik:11} is indeed a definite expression and the semantic representation for this BN includes the \is{iota operator|(}iota operator, which ``supplies'' its definiteness, as shown in \REF{ex:borik:13}.
	
	\ea \label{ex:borik:13}
	$[\![\text{\textit{kit}}]\!] = \iota{x^k}[\mbox{\textit{kit}}(x^k)]$
	\z
	
	The iota operator simply selects the \is{uniqueness}unique entity that refers to the class itself (i.e. to the class described by the \is{nouns}noun \textit{kit}), but does not make the denotation restricted to a given world. 
	
	The next issue we need to address is what kind of syntactic structure corresponds to the semantic representation in \REF{ex:borik:13}. 
	
	\subsection{Syntactic arguments for a DP structure} \label{sec:borik:3.2}
	
	In example \REF{ex:borik:7b} of \sectref{sec:borik:2} we already gave a syntactic structure for the definite kind expression in \REF{ex:borik:7a}, so it should be clear by now that the general syntactic structure associated with definite kinds should look like \REF{ex:borik:14}. 
	
	\ea\label{ex:borik:14}
	$[_{\mbox{\scriptsize{DP}}} \mbox{D} [_{\mbox{\scriptsize{NP}}} \mbox{N}]]$
	\z
	
	Syntactically, we defend the claim that definite kinds in \ili{Russian} are DPs, that is, the D-layer is present in the syntactic representation of definite kind arguments even though there is no overt realization of the D-projection. 
	
	Before we discuss this analysis, let us point out that we assume a strict correspondence between syntactic and semantic representations at the syntax\hyp{}semantic interface as a null hypothesis. This view on the \isi{syntax-semantics interface} by default requires a consistent syntactic representation for each particular semantic operation. In the case of definite kinds, the operator that turns the meaning of a \is{common nouns}common noun (i.e. a \is{kind properties}property of kinds; see \sectref{sec:borik:2}) into a kind expression is the iota operator, which needs to be represented syntactically, unless we assume that all nouns are structurally ambiguous and one and the same expression can be associated with various syntactic structures. Since there is ample cross-linguistic evidence that the iota operator is syntactically represented by the \is{definite articles}definite article (consider, for example, the situation in \ili{Germanic} and \ili{Romance}), we should conclude that we need a D projection even for \is{article-less languages}article-less languages where \is{iota operator|)}iota is not lexicalized. Making this proposal, we follow the insights of \citet{Longobardi1994, Longobardi2001, Longobardi2005}, who claims that semantic referentiality (i.e. being a referring expression) is associated with a particular syntactic position, namely, the head of the \is{determiner phrases}DP. This claim could be considered one of the strongest mapping principles between the \is{syntax-semantics interface}syntax and semantics of natural languages, and it fits neatly with the syntax-semantics correspondence that we are assuming in this paper. 
	
	As for \ili{Russian}, proposals that provide a similar semantic motivation for the DP projection with a \is{null determiners}null D have been made, for instance, by \citet{Ramchand2008} who argue that the D head in \ili{Russian} is needed for reasons of semantic uniformity: this is the head that turns nominal expressions, which are originally of property-type $\langle e,t\rangle$, to arguments, i.e. expressions of type $\langle e\rangle$. They further suggest that the D head in \ili{Russian} should be underspecified for features like \is{definiteness marking}(in)definiteness, (un)\isi{specificity}, etc., which are determined contextually. This means that DPs in \ili{Russian} can represent definite or \is{indefinites}indefinite (specific and non-specific) arguments, the hypothesis that we adopt in here as well.
	
	However, the strict \is{syntax-semantics interface}syntax-semantic correspondence is a working hypothesis that, in and by itself, cannot be taken as an argument for the presence of the DP layer in the syntactic representation of definite kinds in \ili{Russian}. A well-known debate in the literature on languages with and \is{article-less languages}without articles is the discussion between the \is{Universal DP (Approach)}Universal-DP hypothesis \citep{Longobardi1994,Cinque2005,Pereltsvaig2007} and the \is{Parametrized DP Hypothesis}Parametrized\hyp{}DP hypothesis \citep{Boskovic2005, Boskovic2008,Boskovic2008a,Boskovic2009}. According to the former, languages with or without articles would have all nominal arguments projected as full DPs and would allow \is{null determiners}null Ds. According to the second hypothesis, however, there exist two types of languages, those with articles (like \ili{English} and Modern \ili{French}), which project arguments as DPs, and those \is{article-less languages}without articles (like \ili{Serbo-Croatian} and \ili{Russian}), which are postulated to project NPs.\footnote{The \is{Parametrized DP Hypothesis}Parametrized-DP hypothesis is given extensive empirical motivation in the literature. However, the arguments for the \is{DP/NP Approach}DP/NP split between languages, to the best of our knowledge, are purely syntactic (e.g. \is{Left Branch Extraction}left-branch extraction, negative raising, superiority effects, etc.; e.g. \citealt{Boskovic2008}). The proponents of the Parametrized-DP hypothesis usually do not take into account the semantic functions attributed to the \is{determiner phrases}DP projection as we do in this paper.}  
	
	We adopt the view advocated by \citet{Pereltsvaig2006}, according to which nominal arguments can differ in ``size'', i.e. have different types of syntactic structure in argument position, both across languages and \is{language internal variation}language internally. Thus, in both \ili{Russian} and, for instance, \ili{English} or \ili{Spanish}, we can find nominal arguments that syntactically correspond to either full DPs or smaller nominals: NPs, NumPs or \is{quantifier phrases}QPs.\footnote{For similar claims in \ili{Romance} languages see  \citet{Schmitt1999, Schmitt2003}, \citet{MunnSchmitt2005}, \citet{Dobrovie-Sorin2006}, \citet{Cyrino2015}, among others.} In \ili{Russian}, nominal arguments associated with different syntactic structures exhibit a number of different properties and have a different semantic interpretation as well. In particular, \is{determiner phrases}DP subjects obligatorily agree with the verbal predicate, whereas small nominals do not. Agreeing subjects allow an individuated / \is{specificity}specific interpretation, a non-isomorphic wide scope reading, they may control PRO and be antecedents of anaphors, whereas non-agreeing subjects do not.\footnote{For details, see \citet[447]{Pereltsvaig2006}.} To illustrate this difference between agreeing and non-agreeing nominal subjects, consider the minimal pair in \REF{ex:borik:15} (from \citealt[438--9, ex. 3]{Pereltsvaig2006}). Example \REF{ex:borik:15a} exhibits \isi{number} agreement between \textit{pjat' izvestnyx akt\"erov} `five famous actors' and the verb, and this agreement is supposed to correlate with the distributive individuated interpretation of the subject, in the sense that each one of the famous actors played a role in the film. By contrast, in example \REF{ex:borik:15b} there is no number agreement between the subject and the verb, the latter being in the third person singular neuter default form.\footnote{\citet{Pereltsvaig2006} does not indicate \textsc{sg}, but only \textsc{neut}, in the gloss for the verb in this example, because \isi{nouns}, verbs, \is{adjectives}adjectives and various agreeing elements can express gender only in singular. We modified the gloss to include the number specification on the verb plus the number and case on the noun for the sake of explicitness.} Lack of syntactic agreement correlates with a group interpretation of the nominal expression. This means that the subject argument \textit{pjat' izvestnyx akt\"erov} `five famous actors' is attributed a full DP structure with a \is{null determiners}null D in \REF{ex:borik:15a} but a QP with a \is{numerals}numeral in \REF{ex:borik:15b}.
	
	\ea\label{ex:borik:15}\il{Russian}
	\ea\label{ex:borik:15a}
	\gll V ètom fil'me igrali {\ob}pjat' izvestnyx akt\"erov{\cb}.\\
	in this  film played.\textsc{pl} {\db}five famous 	actors.\textsc{pl.gen}\\
	\glt `Five famous actors played in this film.'
	\ex\label{ex:borik:15b}\il{Russian}
	\gll V ètom fil'me igralo {\ob}pjat' izvestnyx akt\"erov{\cb}.\\
	in this  film played.\textsc{sg.neut} {\db}five famous 	actors.\textsc{pl.gen}\\
	\glt `Five famous actors played in this film.'
	\z
	\z
	We find Pereltsvaig's proposal that in \ili{Russian} some nominals are DPs but small nominals can be found in the same syntactic position as DPs very plausible, and thus we adopt the claim that in all languages, including \ili{Russian}, there can be nominal arguments of different ``size'', that is, involving a different ``amount'' of functional structure on top of the minimal NP projection, the highest projection that a nominal argument can have being a \is{determiner phrases}DP. 
	
	Let us now go back to definite kinds and test how arguments of \is{kind-level predicates}k- and \is{individual-level predicates}i-level predicates behave with respect to some properties listed in \citet{Pereltsvaig2006}. Note that only some of the properties this author lists can be tested for definite kinds. The reason for this is that the majority of Pereltsvaig's arguments are built for nominal phrases with various types of modifiers (\isi{numerals}, \isi{adjectives}, etc.), but kind expressions almost never accept regular modifiers.\footnote{See, however, \sectref{sec:borik:3.3} below.} We thus focus on the following properties that kind arguments can be tested for: control of PRO, licensing of \is{anaphora}anaphors, substitution by pronominal elements and presence of \is{relative clauses!non-restrictive}non-restrictive relative clauses. We show that all these properties support an analysis of definite kinds in \ili{Russian} as full DPs. 
	
	\subsubsection{Control of PRO}
	
	Non-agreeing subjects cannot be controllers for PRO in infinitival clauses, while agreeing subjects, being full DPs, can. The contrast is exemplified in \REF{ex:borik:16} (\citealt[444, ex. 10a]{Pereltsvaig2006}).
	
	\ea\label{ex:borik:16}\il{Russian}
	\gll {\ob}Pjat' 	banditov{\cb}\textnormal{$_\text{i}$} pytalis'  \textnormal{/} \textnormal{*}pytalos'  {\ob}\textnormal{PRO$_\text{i}$} 	ubit' 	D\v{z}emsa 	Bonda{\cb}.\\
	{\db}five 	thugs.\textsc{pl.gen} 	tried.\textsc{pl} / \phantom{*}tried.\textsc{sg.neut} {\db}PRO to.kill 	James 	Bond\\
	\glt `Five thugs tried to kill James Bond.'
	\z
	
	Let us now look at definite kinds. As shown in \REF{ex:borik:17}, definite kind subjects can control PRO of a purpose clause and, hence, pattern with agreeing subjects. Since agreeing subjects are argued to be full DPs, we can conclude that the same syntactic category should be attributed to definite kinds. 
	
	\ea\label{ex:borik:17}\il{Russian}
	\gll \textbf{Panda}\textnormal{$_\text{i}$} imeet neoby\v{c}nye perednije lapy 	\v{c}toby  \textnormal{PRO$_\text{i}$}	uder\v{z}ivat' stebli bambuka.\\
	panda.\textsc{sg.nom} has.\textsc{sg} unusual front 	paws in.order.to PRO hold stems bamboo\\
	\glt `The panda has unusual front paws to hold bamboo stems.' 
	\z
	
	\subsubsection{Antecedents of reflexive pronouns}
	
	Our next piece of evidence in favour of the \is{determiner phrases}DP status of definite kinds is that these expressions can be antecedents of a reflexive pronoun. We start by illustrating the contrast between agreeing and non-agreeing subjects with respect to their ability to license reflexive pronouns (\citealt[455, ex. 11a]{Pereltsvaig2006}): only agreeing subjects can license reflexive pronouns. 
	
	\ea \label{ex:borik:18}\il{Russian}
	\gll {\ob}Pjat 	banditov{\cb}\textnormal{$_\text{i}$} 	prikryvali \textnormal{/} \textnormal{*}prikryvalo sebja\textnormal{$_\text{i}$} ot pul' D\v{z}emsa 	Bonda.\\
	{\db}five 	thugs.\textsc{pl.gen} 	shielded.\textsc{pl} / \phantom{*}shielded.\textsc{sg.neut} self from bullets James Bond\\
	\glt		`Five thugs shielded themselves from James Bond's bullets.'
	\z
	
	As \REF{ex:borik:19} illustrates, definite kinds pattern likewise.
	
	\ea\label{ex:borik:19}\il{Russian}
	\gll \textbf{Tigr}\textnormal{$_\text{i}$} znaet 	kak za\v{s}\v{c}itit' 	\textbf{sebja}\textnormal{$_\text{i}$} ot napadenija.\\
	tiger.\textsc{sg.nom} knows.\textsc{sg} how defend self from attacks\\
	\glt `The/a tiger knows how to protect itself from being attacked.'
	\z
	
	This example shows that, according to the test, the antecedent of the reflexive must be a DP. This DP may be devoid of \is{number}Number, as in the structure \REF{ex:borik:14} above (i.e. the structure postulated for definite kinds), or may have Number. In the latter situation, the D can be either definite or \is{indefinites}indefinite, and either singular or plural. 
	
	\subsubsection{Pronominal substitution}
	
	Finally, a pronominal substitution test also shows that definite kinds behave like DPs rather than other, ``smaller'' types of arguments. The test as used in \citet{Pereltsvaig2006} shows that third person pronouns can be used to substitute full DPs, but not \is{quantifier phrases}QPs or NPs, which can only be substituted by other (\is{quantifiers}quantificational and/or pronominal) elements. The example below (based on \citealt[446, ex. 15a]{Pereltsvaig2006}) shows that the pronominal subject of \REF{ex:borik:20b} can only substitute the agreeing subject of \REF{ex:borik:20a}. 
	
	\ea\label{ex:borik:20}\il{Russian}
	\ea\label{ex:borik:20a}
	\gll Pjat' par tancevali \textnormal{/} tancevalo tango.\\
	five 	couples.\textsc{pl.gen} danced.\textsc{pl} \textnormal{/} danced.\textsc{sg.neut} 	tango\\
	\glt `Five couples danced tango.'
	\ex\label{ex:borik:20b}\il{Russian}
	\gll Oni tancevali \textnormal{/} \textnormal{*}tancevalo tango.\\
	they.\textsc{pl.nom} danced.\textsc{pl} \textnormal{/} \phantom{*}danced.\textsc{sg.neut} 	tango\\
	\glt `They danced a tango.'
	\z
	\z
	
	Coming back to definite kinds, it can be easily shown that the definite kind agreeing subject in \REF{ex:borik:21a} can only be replaced by a third person pronoun \textit{ona} `she', thus supporting the claim that definite kinds are DPs.
	
	\ea\label{ex:borik:21}\il{Russian}
	\ea\label{ex:borik:21a}	
	\gll \textbf{Panda}	naxoditsja		na		grani		is\v{c}eznovenija.\\
	panda.\textsc{sg.nom}		is.found.\textsc{sg} on		verge		extinction.\textsc{gen}\\
	\ex\label{ex:borik:21b}\il{Russian}
	\gll \textbf{Ona} 	naxoditsja		na		grani		is\v{c}eznovenija.\\
	she.\textsc{sg.nom} 		is.found.\textsc{sg} on		verge		extinction.\textsc{gen}	\\
	\glt `The panda/She is on the verge of extinction.'
	\z
	\z
	
	The three arguments just given, which are based on the syntactic tests proposed in \citet{Pereltsvaig2006} for differentiating between \is{determiner phrases}DP arguments and arguments associated with a ``smaller'' syntactic structure, all support the claim that definite kinds in \ili{Russian} are syntactically DPs. 
	
	Let us add one more observation to the arguments given above.
	
	\subsubsection{Distribution of relative clauses}\is{relative clauses|(}
	
	There is a limited number of constructions in \ili{Russian} where a nominal argument seems to have the status of a real \is{bare nominals}bare NP and be associated with a minimal possible NP structure with no additional functional layers. A couple of relevant examples from \ili{Russian} is given in \REF{ex:borik:22} (\ref{ex:borik:22b} is from \citealt[ex. 8]{Borik2012}).
	
	\ea\label{ex:borik:22}\il{Russian}
	\ea\label{ex:borik:22a}
	\gll Petja	 xodit v galstuke, {\op}\textnormal{*}kotoryj 	vsegda nravitsja ego \v{z}ene{\cp}.\\
	Petja	 goes in tie.\textsc{sg.obl}	{\db}{\phantom{*}}which always likes his wife\\
	\glt `Petja is a tie-wearer, (*which his wife always likes).'
	\ex\label{ex:borik:22b}\il{Russian}
	\gll Katya nosit jubku, {\op}\textnormal{*}kotoruju	 ona	vsegda pokupaet sama{\cp}. \\
	Katya wear.\textsc{imp} skirt.\textsc{sg.acc} {\db}{\phantom{*}}which she always buys.\textsc{imp} self\\
	\glt `Katya is a skirt-wearer, (*which she always buys).'
	\z
	\z
	
	The objects \textit{galstuke} `tie' and \textit{jubku} `skirt', despite being morphologically mark\hyp{}ed as singular, have a \is{number neutrality}number neutral interpretation (i.e. one or more tie, one or more skirt), that is, can denote either an atomic or a plural entity satisfying the description of the nominal.\footnote{See \citet{Kagan2011} and \citet{Pereltsvaig2013} for other types of \is{number neutrality}number neutral arguments in \ili{Russian}. In these papers, it is argued that semantically number neutral nominals are plural in \ili{Russian}. We agree with this claim, but we think that \ili{Russian} also has morphologically singular nominals with a number neutral interpretation.} Number neutrality is a hallmark of \isi{bare nominals} in various languages (cf. \citealt{Farkas2003} for \ili{Hungarian}; \citealt{Dayal2004} for \ili{Hindi}; \citealt{Espinal2011} for \ili{Spanish} and \ili{Catalan}, etc.), so this is a good reason to assume that the objects in \REF{ex:borik:22}, despite being morphologically singular, are ``true'' bare nominals unspecified for syntactic and semantic \is{number}Number. 
	
	Note, however, that neither \textit{galstuke} `tie' nor \textit{jubku} `skirt' in this interpretation can be modified by a relative clause.\footnote{This is also a property of bare nominals in the same syntactic position in \ili{Romance} languages, such as \ili{Catalan} and \ili{Spanish}. See \citet{Espinal2011}.} We suggest that a reason for blocking a relative clause in \REF{ex:borik:22} is that in a real NP structure there is no room for descriptive but only for classifying modifiers (which is in accordance with our theoretical postulate 1, see \sectref{sec:borik:2}.1). A classifying modifier but not a \is{relative clauses!restrictive}restrictive relative clause is allowed in \REF{ex:borik:23}, under the intended reading that Katya is a skirt-wearer. 
	
	\ea\label{ex:borik:23}\il{Russian}
	\gll Katya 	nosit mini-jubku, {\op}\textnormal{*}kotoruju	ona	vsegda		pokupaet sama{\cp}. \\
	Katya wear.\textsc{imp}	mini-skirt.\textsc{sg.acc} {\db}{\phantom{*}}which she always buys.\textsc{imp} self\\
	
	\glt `Katya is a mini-skirt wearer, (*which she always buys).'
	\z
	
	Consider now an example with a definite kind expression: 
	
	\ea\label{ex:borik:24}\il{Russian}
	\ea[\phantom{\#}]{\label{ex:borik:24a}
	\gll Amurskij tigr, kotoryj o\v{c}en' opasen, 	obitaet na 	jugo-vostoke Rossii.\\
	Siberian 	tiger 	which very 	dangerous lives 	on 	south-east Russia.\\
	\glt `The Siberian tiger, which is extremely dangerous, lives in the south-east part of Russia.'}
	\ex[\#]{\label{ex:borik:24b}\il{Russian}
		\gll Amurskij tigr, kotoryj 	rodilsja v 	na\v{s}em zooparke, obitaet na jugo-vostoke Rossii.\\
		Siberian 	tiger 	which was.born in our zoo live on south-east Russia\\
		\glt `The Siberian tiger that was born in our zoo lives in the south-east\largerpage[2] part of Russia.'
	}
	\z
	\z
	
	As can be seen in \REF{ex:borik:24a}, definite kinds allow subsequent modification by a \is{relative clauses!non-restrictive}non-restrictive relative clause. Non-restrictive (or appositive) relative clauses do not restrict the (set of) referents denoted by the nominal phrase, they just provide \textit{additional} information about an already established referent. By contrast, as the example \REF{ex:borik:24b} illustrates, a relative clause that can only be interpreted restrictively, imposes an individual (as opposed to a kind) interpretation on the subject of the clause, which is then difficult to combine with the verbal predicate \textit{obitaet} `to live' that normally selects for kinds.\footnote{Two notes are in order here. First of all, \ili{Russian} has several verbs that can be translated as `to live', and the one used in example \REF{ex:borik:24} is often used with kind nominals since its lexical meaning is closer to `to live permanently, to inhabit'. Secondly, the \# sign in front of \REF{ex:borik:24b} means that the subject can, in principle, be interpreted as referring to an individual tiger, although it takes a certain effort to get this interpretation, at least for one of the authors of this paper, and the intuition is that this interpretation is an effect of coercion.}
	
	Let us now go back to the claim that we made at the beginning of the section, namely, that the incompatibility of \is{relative clauses!restrictive}restrictive relative clauses with definite kinds can be seen as an additional argument for the \is{determiner phrases}DP status of the kind nominal. We now explain why it should be so. 
	
	Semantically, \is{relative clauses!non-restrictive}non-restrictive relative clauses are not interpreted in the scope of the determiner, as the following examples from \ili{English} illustrate: 
	
	\ea\label{ex:borik:25}
	\ea\label{ex:borik:25a} 
	\textit{
		{\ob}{\ob}The public transport{\cb}, {\ob}which is state-owned{\cb}{\cb}, is fast, clean and reliable.
	}
	\ex\label{ex:borik:25b}
	\textit{
		{\ob}The {\ob}public transport which is state-owned{\cb}{\cb} is fast, clean and reliable.
	}
	\z
	\z
	
	The example in \REF{ex:borik:25a}, which is interpreted \is{relative clauses!non-restrictive}non-restrictively, can be rephrased as a conjunction: `the public transport is fast, clean and reliable and it is state-owned'. It does not imply (in fact, it cannot imply) that there is any other public transport except for the state-owned. The example in \REF{ex:borik:25b}, on the other hand, implies that not all the public transport is owned by the state and it is clear that the \is{definite determiners}definite determiner the in \REF{ex:borik:25b} has the whole nominal phrase, including the relative clause, in its scope. 
	
	\citet{Jackendoff1977} suggested that the difference between \is{relative clauses!restrictive}restrictive and \is{relative clauses!non-restrictive}non-restrictive relative clauses should be reflected in their \is{syntax-semantics interface}syntactic configuration, in the sense that the latter adjoin higher in the structure than the former. \citet{Demirdache1991} specifically proposed that non-restrictive relatives are adjoined to \is{determiner phrases}DP, although only at \is{logical form}LF. \ia{de Vries, Mark}De Vries (\citeyear{deVries2006}) postulates that \is{relative clauses!non-restrictive}appositive relative clauses should be represented as a coordination of DPs, an appositive relative as a specifying conjunct to the visible antecedent. \citet{Arsenijevic2016} also argued that the configurational differences between restrictive and non-restric\hyp{}tive relative clauses should be reflected in overt syntax on the basis of agreement facts in \il{Serbo-Croatian}Bosnian/Serbian/Croatian. Generalizing over these and many more works on relative clauses, we can say that the main idea is that non-restrictive relatives can only have a DP as an antecedent. There is no a priori reason to believe that \ili{Russian} non-restrictive clauses would be different in their \is{syntax-semantics interface}syntax and semantics. Therefore, we take \REF{ex:borik:24a} to be another piece of evidence in favor of the DP status of definite kind expressions. 
	
	The discussion of relative clauses once again supports the point made by \citet{Pereltsvaig2006}: we should allow for different structures to be associated with nominals in argument position. \REF{ex:borik:24a} above indicates that definite kinds cannot be NPs, as we have seen that true \is{bare nominals}bare NPs do not take relative clauses, restrictive or non-restrictive. If we consider the empirical contrast between \REF{ex:borik:23} and \REF{ex:borik:24a}, together with Pereltsvaig's arguments discussed earlier in this section, the conclusion that we logically arrive at is the same: definite kinds in \ili{Russian} are DPs.
	
	This conclusion allows us to preserve the correspondence between the presence of D projection and the contribution of the \isi{iota operator}, which, as we have seen above, is realized as a \is{definite articles}definite article in languages with articles. Our claim for an \is{article-less languages}article-less language like \ili{Russian} is, thus, that the syntactic representation of definite kinds involves a \is{null determiners}null D, which is translated as the iota operator, too. \is{relative clauses|)}
	
	\subsection{Modified definite kinds} \label{sec:borik:3.3}\is{modified kinds|(}
	
	In \sectref{sec:borik:3.2} we have provided syntactic arguments for a \is{determiner phrases}DP structure. Still, a question that remains to be answered is whether definite kinds allow any sort of modification inside the DP. We think that the answer to this question is positive, and, following \citet{Borik2015} for \ili{Spanish}, we show in this section that \ili{Russian} has kind expressions with modifiers, which we call modified kinds. 
	
	Modified kinds are \is{kind reference}ind-referring expressions composed by a \is{nouns}noun and a modifier, normally expressed by an \is{adjectives}adjective, provide an additional semantic argument for the definiteness of \ili{Russian} \is{bare nominals}bare nominal kinds. Consider the data in\largerpage \REF{ex:borik:26}.
	
	\ea\label{ex:borik:26}\il{Russian}
	\ea\label{ex:borik:26a}{
		\gll \textbf{Amurskij} 	\textbf{tigr} 	zanesen	 v Krasnuju 	knigu.\\ 			
		Siberian 	tiger 	registered 	in Red 			book\\
		\glt `The Siberian tiger is registered in the IUCN Red list.'
	}
	
	\ex\il{Russian}\label{ex:borik:26b}{
		\gll \textbf{Mavrikijskij} \textbf{dront} izvesten tol'ko 	po izobra\v{z}enijam 	i pis'mennym isto\v{c}nikam XVII veka.\\
		Mauritius	dodo	known only	from	drawings 		and	written sources 	XVII century\\
		\glt `The dodo of the Mauritius island is only known from drawings and\largerpage written sources of the XVII century.'
	}
	\z
	\z
	
	The modified DPs in subject position in \REF{ex:borik:26}, similarly to the corresponding non\hyp{}modified versions, denote kinds. However, in comparison to the non-modi\hyp{}fied counterparts (e.g. \textit{tigr} `tiger'), modified kinds (e.g. \textit{amurskij tigr} `Siberian tiger') are semantically more restricted. We suggest that modified kinds, composed by a \is{nouns}noun preceded or followed by an \is{adjectives}adjective within a \is{determiner phrases}DP structure, are built by applying kind modifiers (of type $\langle \langle e^k, t\rangle, \langle e^k, t\rangle\rangle$) to  properties of kinds (of type $\langle e^k, t\rangle$). The formal representation for the modified kind in \REF{ex:borik:26} is given in \REF{ex:borik:27}.
	
	\ea\label{ex:borik:27}
	\ea
	$[_{\mbox{\scriptsize{DP}}} \mbox{D} [_{\mbox{\scriptsize{NP}}} \mbox{(A) N (A)}]]$
	\ex
	$[\![\mbox{\textit{amurskij tigr}}]\!] = \iota{x^k}[\mbox{(\textit{amurskij}(\textit{tigr}))}(x^k)]$
	\z
	\z
	
	A question that arises at this point is what kind of \is{adjectives}adjective can appear in a modified kind expression. We think that potentially any adjective can modify a kind although the whole expression is subject to an additional pragmatic constraint, known as the \isi{well-established kind restriction }(cf. \citealt{KrifkaEtAlii1995}). 
	
	The well-established kind restriction has been widely discussed in the literature for \ili{English} and other languages as applying to \isi{definite generics} (cf. \citealt{Vergnaud1992}, \citealt{KrifkaEtAlii1995}, \citealt{Dayal2004} and many others). If the well-established kind restriction is pragmatic in nature, it is expected that an appropriate contextual modification could make a definite kind reading in \REF{ex:borik:28a} plausible. This is, indeed, the case. If there are only two relevant classes of tigers, wounded tigers and hungry tigers, \REF{ex:borik:28b} becomes a perfectly acceptable characterization of the first class. In this case, the interpretation that should be attributed to the subject of \REF{ex:borik:28b} is the one characteristic of a definite kind.
	
	\ea\label{ex:borik:28}\il{Russian}
	\ea\label{ex:borik:28a}
	\gll Ranenyj 	tigr 	opasen. 	\\		 
	wounded 	tiger 	dangerous\\
	\glt `A wounded tiger is dangerous.'
	\ex\label{ex:borik:28b}\il{Russian}
	\gll Ranenyj 	tigr, kak 	vid, 	opasen. \\		
	wounded 	tiger as 	type	dangerous\\
	\glt `The wounded tiger, as a kind, is dangerous.'
	\z
	\z
	
	We propose that the well-established kind restriction can block a kind interpretation for modified nominal expressions at a pragmatic level, but this is not a grammatical constraint (for similar observations see \citealt{Dayal1992}; \citealt[69]{KrifkaEtAlii1995}; \citealt[footnote 30]{Dayal2004}). Rather, it is our world knowledge and accessible encyclopedic information that determines which expression can correspond to a known or established kind in the actual world. Note, furthermore, that this information can change, and hence, relevant contextual or extra-linguistic factors can have a strong influence on the interpretation of nominal expressions. \is{modified kinds|)}
	
	\section{Conclusions}\label{sec:borik:4}
	
	In this paper we have provided an analysis of definite kinds in \ili{Russian} at the \is{syntax-semantics interface}syntax-semantics interface. We have presented arguments for the semantic definiteness of \is{bare nominals}bare nominal kinds, and syntactic arguments for a \is{null determiners}null D. We have argued that definite kinds are compositionally built by applying the \isi{iota operator} corresponding to a (covert) \is{definite determiners}definite D to the property of kinds denoted by the N, and we have extended this analysis to modified definite kinds. The analysis we propose applies to one specific type of expressions which refer to kinds, the one that corresponds to \ili{English} definite kinds. In \ili{Russian}, as in many other languages, there is a range of other expressions which plausibly encode \isi{D-genericity}, notably, plural \isi{generics}. We see it as one of the main questions for future research to complement our proposal by an analysis of other types of nominal generics in \ili{Russian} and an account of similarities and differences in the meaning and use of various kind referring expressions.\is{kinds|)}\is{definite kinds|)}
	
		
	\section*{Acknowledgements}
	
	We would like to thank the editors of the book and the reviewers of this paper, as well as the audience of the conference \textit{Definiteness across languages} (Ciudad de M\'exico, 2016) for their comments. This research was supported by the Spanish MICINN (grants FFI2014-52015-P and FFI2017-82547-P) and the Catalan Government (grants 2014SGR2013 and 2017SGR634). The second author also acknowledges an ICREA Academia award.
	
	\section*{Abbreviations}
	\begin{multicols}{2}
		\begin{tabbing}
			\textsc{nom}\hspace{1em} \=  Nominative \kill 
			\textsc{gen} \>	genitive \\
			\textsc{acc} \>	accusative \\
			\textsc{obl} \>	oblique \\
			\textsc{sg} \>	singular \\
			\textsc{pl} \>	plural \\
			\textsc{masc} \> masculine \\
			\textsc{fem} \>	feminine \\ %\columnbreak
			\textsc{neut} \> neuter \\
			\textsc{imp} \>	imperfective \\
		\end{tabbing}
	\end{multicols}
	
	{\sloppy
		\printbibliography[heading=subbibliography,notkeyword=this]
	}
\end{document}