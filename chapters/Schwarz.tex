\documentclass[output=paper
,modfonts
,nonflat]{langscibook} 

\title{Weak vs.\ strong definite articles: Meaning and form across languages} 
\author{Florian Schwarz\affiliation{University of Pennsylvania}}
\ChapterDOI{10.5281/zenodo.3252012}

% \epigram{}

\abstract{One line of recent work on definite articles has been concerned with languages that utilize different forms for definite descriptions of different types. In the first part of this paper, I discuss the semantic analysis of the underlying distinction of \textit{weak} and \textit{strong} definite articles as proposed in \citet{Schwarz2009}, which formalizes the contrast in terms of \isi{uniqueness} (for \textit{weak} articles) vs. \isi{anaphoricity} (for \textit{strong} articles). I also review the empirical motivation for the analysis based on \ili{German} \is{prepositions}preposition-\is{determiners}determiner contraction and its implications for related semantic phenomena. The second part of the paper surveys recent advances in documenting contrasts between definites in various other languages. One issue here will be on assessing to what extent the cross-linguistic contrasts are uniform in terms of their semantics and pragmatics, and to what extent there is variation in the relevant patterns. A second issue is to evaluate how the obvious variation in the formal realization of the contrast across languages can contribute to a more refined implementation of the contrast in meaning.}

\begin{document}
\maketitle
\section{Introduction}\is{strong definite articles|(}\is{weak definite articles|(}
Definite descriptions have played a central role in the study of
meaning in natural language right from the start, going back to
early work by \citet{Frege1892}, and leading to the famous debate in
the philosophy of language between
\citet{Russell1905} and \citet{Strawson1950}, with continued interest
in related issues \citep[for an extensive collection, see][]{Reimer2004} . One central reason for
this would seem to be that they offer a particularly insightful
perspective on how (at least potentially) different dimensions of
meaning differ from one another and interact, as well as on the role
of context in interpreting linguistic utterances. Work
in linguistics has also been concerned with similar issues,
specifically with regards to related questions about the interplay
of contextual information and grammatical representations,
in particular concerning mechanisms for \is{quantifiers}quantificational
co-variation, starting most prominently
with \citet{Heim1982}.\footnote{For a comprehensive recent proposal from the perspective of situation semantics, see \citet{Elbourne2013}.}


One line of work on definite articles that has gained prominence in
recent years has been concerned with languages that utilize different
forms for definite descriptions of different types. While there is a
fairly rich tradition in the more descriptive literature, especially
on \ili{German} dialects, going back at least to \citet{Heinrichs1954}, the
notion that languages might have more than one type of definite
article (beyond mere inflectional variations), with different
semantic-pragmatic profiles, only received more wide-spread attention
in the formal semantics literature in the 2000s. The present paper
begins with a review of the analytical approach proposed in
\citet{Schwarz2009}. It characterizes the distinction between \textit{weak}
and \textit{strong} definite articles as in terms of
\is{uniqueness|(}uniqueness (for \textit{weak} articles) vs.\ \isi{anaphoricity} (for \textit{strong}
articles). The formal analysis is empirically motivated by  data on
\ili{German} \is{prepositions}preposition-\is{determiners}determiner contraction, and I briefly discuss the
main data points in its favor, as well as its implications for related
semantic phenomena. 

The second part of the paper surveys recent
advances in documenting contrasts between definites in various other
languages. One focus here will be on assessing to what extent the
cross-linguistic contrasts are uniform in terms of their semantics and
pragmatics, and to what extent there is variation in the relevant
patterns.  A second focus is to evaluate how the obvious variation in the formal
realization of the contrast across languages can contribute to a more
refined implementation of the contrast in meaning, and how this
relates to \is{noun phrases}noun phrase structure more generally. While a fair amount of the
cross-linguistic data supports the analytical contrast in terms of the
weak vs.\ strong article distinction, there certainly is variation
in definite contrasts beyond that. I briefly discuss one alternative family of
proposals for capturing such variation from the literature, and also
sketch some tentative analyses of additional points of variation. 

Before moving on, let me issue a few caveats concerning the
limitations in scope of
the present inquiry. First of all, I start from the theoretical
distinction I proposed in earlier work, and explore how it fares with
regards to a set of cross-linguistic data that considers relevant
phenomena and contrasts. This should not be taken to suggest that
other theoretical approaches, beyond the ones considered here, have no
role to play in the analysis of definite descriptions. Rather, it is
simply a decision grounded in a theory-driven approach to empirical data, within
which it makes sense to explore to what extent a particular analysis
can deal with empirical facts. Relatedly, a core part of the proposal under
consideration, as things stand, is that it makes a binary
distinction. This may well turn out to be too limited, as further
levels of distinction are likely to be relevant to capture all the
data. Another aspect of the theoretical approach is that it takes
notion(s) of definiteness developed on the basis of familiar languages
such as \ili{English} and \ili{German} to analyze a variety of other
languages. That may well come with its own pitfalls, but we have to start somewhere, and re-evaluate later to what
extent those notions are suitable for spelling out the broader
cross-linguistic picture. Finally, I limit my attention here to the
form and meaning of
definite descriptions alone, without consideration of
\isi{indefinites}. This, too, may be problematic in the long term, as
at least some key effects in a given language may relate to the system
of definite and indefinite expressions it has at its disposal. These
caveats notwithstanding, I hope that the following contributes to our
understanding of the typology of definiteness by evaluating a detailed
formal proposal in light of a broader range of cross-linguistic data.

\section{Two types of definite articles}

\subsection{Two semantic perspectives on definite descriptions}

Broadly speaking, there are two families of approaches to analyzing
definite descriptions that have been
predominant in the formal literature, namely ones based on the notion
of uniqueness, on the one hand, and ones based on the notion of
\is{familiarity|(}familiarity or \isi{anaphoricity} on the other hand. I provide a sketch of each of these here,
following the bulk of the literature in seeing them as comprehensive
proposals that aim to capture all data on definite descriptions, as is desirable
for reasons of theoretical parsimony (see below for some pointers to
mixed approaches in the literature).

Starting with uniqueness-based approaches, the intuitive motivation
is based on examples such as the following:

\begin{exe}
\ex\label{ex:schwarz:1}
Context: Speaker is standing in an office with {exactly one table}.\\
\textit{The table is covered with books.}

\end{exe}

The central idea here is that definite descriptions pick out an individual that
uniquely fits the provided description. Formally speaking, the analysis is usually cast in
terms of a definite description of the form \textit{the NP} \is{definiteness marking}encoding
that a) there is an entity in the extension of \textit{NP} (the
existence condition) and b)
that the number of such entities not exceed one (the uniqueness
condition). This is at the core of both the traditions following
\citeauthor{Russell1905} and \citeauthor{Frege1892}/\citeauthor{Strawson1950}, though they differ in the status they
accord these conditions. But they agree that in the end, \isi{reference} is
effectively established via uniqueness (though note that they need not see the
definite description itself as directly referential; Russell sees it
as \is{quantifiers}quantificational), so that the individual
that gets talked about is precisely the one uniquely satisfying the
nominal description.

For present purposes, a key point to note right away is
that any analysis grounded in uniqueness faces an obvious challenge --
namely that, taking \REF{ex:schwarz:1} as our example, there are many tables
in the world. The standard remedy, extensively spelled
out by \citet{Neale1990}, is to appeal to a general mechanism of
\isi{domain restriction}, which has to be assumed independently for other kinds of
\isi{noun phrases} (and likely for other constructions as well). While the
general idea of -- and need for -- such a mechanism is fairly
straightforward and intuitive,
its technical implementation is not, though we will not get into
further detail here for reasons of space.\footnote{For influential
proposals, see, e.g. \citet{Westerstahl1984}, \citet{Fintel1994}, \citet{StanleySzabo2000}, \citet{Elbourne2013}.}

One standard type of definite usage that constitutes a challenge for
uniqueness-based approaches is one involving a preceding \is{indefinites}indefinite
that introduces the intended referent of the \is{definites}definite:

\begin{exe}
\ex \label{ex:schwarz:2} 
\begin{xlist}
\ex\label{ex:schwarz:2a}
	\textit{I got {a table} and an armchair delivered to my office.}
\ex\label{ex:schwarz:2b}
	\textit{{The table} is already covered with books.} 
\end{xlist}
\end{exe}

Crucially, and unlike \REF{ex:schwarz:1} above, this example is perfectly
compatible with there being another table in the office, which both the
speaker and the addressee are aware of. The challenge for a
uniqueness-based account of domain restriction then is to formulate the
general purpose domain restriction machinery in such a way that the
previous mention of the indefinite can bring it about that the domain
only includes the newly delivered table, i.e. does not include
everything in the office, even though we may very well be talking about
the office as a whole in the larger conversation.

Examples like \REF{ex:schwarz:2} constitute the core intuitive
motivation for the second main approach to definite descriptions in
the formal literature. It sees definites as functioning in a way
rather parallel to pronouns (in a traditional view), and goes back to \citet{Christophersen1939}. The highly influential,
and first fully fleshed out modern account along these lines comes
from \citet{Heim1982} (with a similar perspective offered by
\citealt{Kamp1981}), who proposes that definite descriptions come with
an index, which has to be one that is already established, or familiar, in the
\isi{discourse}. The job of \isi{indefinites}, in contrast, is to introduce new
indices to the discourse, yielding a straightforward account of
\REF{ex:schwarz:2} as involving the establishment of an index mapped onto
the newly delivered table in \REF{ex:schwarz:2a}, which is then
anaphorically picked up by the definite in \REF{ex:schwarz:2b}. 

As may be obvious by now, the initial example in \REF{ex:schwarz:1} in turn
constitutes a challenge for accounts based on familiarity, as there is
no previous mention of the table there. The standard approach for
tackling this challenge is to detach the notion of familiarity from
the presence of a linguistic antecedent, e.g. by allowing entities
physically present in the utterance context to count as familiar as
well.\footnote{For extensive discussion of the pertinent distinction
between weak and strong \is{familiarity|)}familiarity, see \citet{Roberts2003}.} This needs to be further extended, however, to deal with
cases of so-called ``global uniques'', such as \textit{the sun} or
\textit{the pope}. 

Rather than diving further into the intricacies of how each of the two
accounts sketched above can deal with various challenging cases, we
now turn to another perspective, which bites the bullet and admits
that both analyses adequately capture how parts of natural language
work. While this may seem, from an a priori
perspective committed to theoretical parsimony, like admitting defeat,
such an approach gains empirical motivation once languages that
explicitly differentiate between different types of definite articles
are considered. This is precisely the perspective put forward in
\citet{Schwarz2009}, with a detailed empirical discussion of variation
in contraction of definite articles and \isi{prepositions}. The central
argument is that certain forms (namely the contracted ones) behave
exactly as expected from a \is{uniqueness|)}uniqueness-based approach, whereas others
(the non-contracted ones) exhibit the behavior we would expect from an
approach that sees definites as \is{anaphora}anaphoric. To the extent that parallel
patterns are found across other languages, the general empirical case
for a richer theoretical inventory gets strengthened further, and one
central aim of the present paper is to survey the evidence from a
variety of other languages in this regard. In addition, the richer
theoretical tool-box can also be put to use to deal with some of the
complexities in languages without any obvious contrast between different
definite articles, such as \ili{English}, though that part of the story will
not be pursued here, and it remains to be seen just how the \ili{English}
facts should be captured in light of this perspective.\footnote{For
previous discussion of \ili{English} data going beyond what can be captured
using just one of the two
approaches above, see, a.o. \citet{BirnerWard1994}, \citet{PoesioViera1998}.}

\subsection{Distinctions between definite
  articles in \ili{German} and \ili{Germanic dialects}}

Much early descriptive work on contrasts between definite articles focused on \ili{German} and \ili{Germanic dialects}.\footnote{Parts of this section are adapted from \citet{Schwarz2013}.} The first
detailed discussion of \ili{Germanic dialects} with two forms for definite articles that I am aware of dates back to\linebreak
\citet{Heinrichs1954}, who discusses dialects of the Rhineland
\citep[see also][]{Hartmann1967}. Other dialects for which this
phenomenon has been described include 
the \ili{Mönchengladbach dialect}
\citep{Hartmann1982}, 
the \ili{Cologne dialect} \citep{Himmelmann1997},
Bavarian \citep{Scheutz1988,Schwager2007} 
and Austro-Bavarian\il{Austro-Bavarian} \citep{BruggerPrinzhorn1996,Wiltschko2013},
\il{Austro-Bavarian!Viennese}Viennese \citep{SchusterSchikola1984},
\ili{Hessian} \citep{Schmitt2006},
and, perhaps the best
documented case, the Frisian dialect of \ili{Fering}
\citep{Ebert1971,Ebert1971b}.\footnote{\citet{Leu2008} discusses
  related matters in \il{Swiss German}Swiss German, although he focuses
  on syntactic issues.}  A parallel phenomenon also exists in
\il{German}Standard German, although here the contrast is only present in
particular morphological environments
(\citealt{Hartmann1978,Hartmann1980,Haberland1985,Cieschinger2006,Waldmuller2008};\linebreak \citealt{Schwarz2009}). I
will begin with some brief illustrations from \ili{Fering} as a
well-documented case with two fully distinct paradigms for definite
articles, and then introduce the basic contrast in Standard
\ili{German}. Somewhat more subtle \ili{German} data will be discussed in the
following section to flesh out the nature of the contrast in meaning
between the different articles.

The basic paradigm for what \citet{Ebert1971b} calls the A-article and
the D-article is presented in \tabref{tab:schwarz:1}. The
examples in \REF{ex:schwarz:3} illustrate the contrast between the two.

\begin{table}[h]
	\begin{tabular}{lllll} 
		\lsptoprule
		& {m.Sg.} & {f.Sg} & {n.Sg.} & {Pl}.\\
		\midrule
		{A-article} & \textit{a} & \textit{at} & \textit{at} & \textit{a}\\
		{D-article} & \textit{di} & \textit{det} & \textit{det} & \textit{d\"on}\\
		\lspbottomrule
	\end{tabular}
	\caption{The definite article paradigms in \ili{Fering} \citep[159]{Ebert1971b}\label{tab:schwarz:1}}
\end{table}


\begin{exe}
\ex\label{ex:schwarz:3}
\langinfo{Fering}{}{\citealt[161]{Ebert1971b}}
\begin{xlist}
  \ex \gll Ik skal deel tu \textbf{a} \textnormal{/} \textnormal{*}\textbf{di} kuupmaan.\\
           I  must down to {\theweak} / {\phantom{*}}{\thestrong} grocer\\
      \glt `I have to go down to the grocer.'
  \ex \gll Oki hee an hingst keeft. \textnormal{*}\textbf{A} \textnormal{/} \textbf{Di} hingst haaltet.\\
           Oki has a horse bought {\phantom{*}}{\theweak} / {\thestrong} horse limps\\
      \glt `Oki has bought a horse. The horse limps.' 
\end{xlist}
\end{exe}


A parallel contrast can be observed in \il{German}Standard German, where certain
combinations of \isi{prepositions} and \is{definite determiners}definite determiners can, but do not
have to, contract \citep[see, among others,][]{Hartmann1978,Haberland1985,Cieschinger2006}.


\begin{exe}
\ex\label{ex:schwarz:4}
\langinfo{German}{}{\citealt[7]{Schwarz2009}}
\begin{xlist}
\ex \gll Hans ging \textbf{zum} Haus.\\
        Hans went {to\_\theweak} house\\
    \glt `Hans went to the house.'
\ex \gll Hans ging \textbf{zu} \textbf{dem} Haus.\\
        Hans went to {\thestrong} house\\
    \glt `Hans went to the house.' 
\end{xlist}
\end{exe}

Descriptively, the two forms seem to correspond straightforwardly to the two distinct definite
articles in \ili{Fering}, and I will assume in what
follows that contraction reflects which article form is at play.\footnote{A word
  of caution is in order concerning variation in contraction: some
  contractions are more colloquial than others, and there are
  corresponding differences in frequencies in written texts. My
  discussion focuses on prescriptively fully recognized cases, to
  avoid prescriptive biases against contraction, but the full range of phenomena is
broader, and may even extend to differences of phonetic realization of
articles in environments where contraction is not available. See
\citet[\S2]{Schwarz2009} for further discussion.} \tabref{tab:schwarz:2} introduces the terminology I use to refer to the
different forms, with the weak article corresponding to Ebert's
A-article and the strong one to her D-article.\footnote{The notions \textit{weak} and \textit{strong} have
  been used to group \is{determiners}determiners in various other ways: \citet{Milsark1977}
  used the existential construction discussed in the introduction to
  identify ``weak'' determiners, while \citet{Herburger1997} makes yet
  another distinction. Finally, \citet{CarlsonEtAlii2006} introduce the
  notion of \is{weak definites}``weak definites'' \citep[with an earlier, related use
  by][]{Poesio1994}, briefly discussed below. To avoid confusion, I
  will generally use the terms \textit{weak article} and \textit{strong article
  (\isi{definites})} in talking about the distinction introduced here.}


\begin{table}[H]
 \begin{tabularx}{.5\textwidth}{lll} 
  \lsptoprule
Form&Article type &Gloss\\
  \midrule
\textit{zum}& weak & {P\_\theweak}\\
\textit{zu dem} &  strong  &  {P\_\thestrong}\\
  \lspbottomrule
 \end{tabularx}
\caption{Terminology for the \ili{German} article forms}
\label{tab:schwarz:2}
\end{table}


The
next section discusses the \ili{German}
contraction data in some detail to flesh out precisely what contrasts
in meaning and use are associated with the two forms.

\subsection{The contrast in meaning between weak and strong articles}

The key concern for our purposes is to what extent the two different
article forms differ in their meaning and conditions of use. As is the
case in \ili{Fering} \REF{ex:schwarz:3}, weak and strong article definites in
\ili{German} are not in free variation, but rather seem to be subject to
different contextual constraints:

\begin{exe}
\ex \ili{German}
\sn[]{\gll In der Kabinettsitzung heute wird ein neuer Vorschlag {vom}
{$\{$\checkmark Kanzler} \textnormal{/} \textnormal{\#}Minister$\}$ erwartet.\\
	in the {cabinet meeting} today is a new proposal {by\_\theweak}
        {{\phantom{$\{$\checkmark}}chancellor} / {\phantom{\#}}minister expected\\
	\glt `In today's cabinet meeting, a new proposal by the chancellor/minister is expected.'}\label{ex:schwarz:5}

\end{exe}

The minimal contrast in availability of the weak article, based on
whether the \is{nouns}noun is \textit{Kanzler} (`chancellor') or
\textit{Minister} (`minister') illustrates that the weak article
requires \is{uniqueness|(}uniqueness: in a given cabinet meeting, there is only one
chancellor, but several ministers, thus \is{uniqueness}unique \isi{reference} can only be
successful for the former. In contrast, the strong article does not
seem to benefit similarly from contextual uniqueness:

\begin{exe}
\ex\label{ex:schwarz:6} \ili{German}
\sn[\#]{
	\gll In der Kabinettsitzung heute wird ein neuer Vorschlag
  {von} dem {Kanzler} erwartet.\\
	in the {cabinet meeting} today is a new proposal by {\thestrong} chancellor expected\\
	\glt `In today's cabinet meeting, a new proposal by the chancellor is expected.'}
\end{exe}

Without further context, it is not available to refer to a minister,
either, but as soon as one minister has been introduced explicitly in
prior \isi{discourse}, this becomes perfectly straightforward: \newpage 


\begin{exe}
\ex \ili{German}\label{ex:schwarz:7}
\begin{xlist}
\ex[]{\gll Hans hat gestern {einen} {Minister} interviewt.\\
Hans has yesterday a minister interviewed\\
\glt `Hans interviewed a minister yesterday.'}
\ex[\checkmark]{\gll In der Kabinettsitzung heute wird ein neuer Vorschlag {von} {dem} {Minister} erwartet.\\
	in the {cabinet meeting} today is a new proposal by {\thestrong} minister expected\\
	\glt `In today's cabinet meeting, a new proposal by the minister is expected.'}\label{ex:schwarz:7b}
\end{xlist}
\end{exe}

Yet another example driving
home the contrast between weak and strong articles is provided in \REF{ex:schwarz:8}:

\protectedex{
\begin{exe}
\ex\label{ex:schwarz:8}
\langinfo{German}{}{\citealt[30]{Schwarz2009}} \\
\gll In der New Yorker Bibliothek gibt es {ein}
{Buch} \"uber Topinambur. Neulich war ich dort und habe
\textnormal{\#}im \textnormal{/} in {dem} {Buch} nach einer
Antwort auf die Frage gesucht, ob man Topinambur grillen kann.\\
in the New York library exists \textsc{expl} a book about topinambur recently
was I there and have {\phantom{\#}}{in-\theweak} / in {\thestrong} book for
an answer to the question searched whether one topinambur grill can\\
\glt`In the New York public library, there is a book about
topinambur. Recently, I was there and searched in the book for an answer
to the question of whether one can grill topinambur.' 
\end{exe}
}

Taken together, these facts suggest that uniqueness is neither
necessary or sufficient for \isi{reference} with the strong
article. Instead, it seems to require an antecedent, here the
\is{indefinites}indefinite, to refer to \is{anaphora}anaphorically. The two articles thus differ in
the way they relate to their context, and they do so in a way that
seems to line up rather naturally with the two main theoretical approaches
to definites. 

Consideration of further cases, which have been extensively discussed
in the literature, extends this perspective in interesting ways. So-called \isi{bridging} uses \citep{Clark1975,Hawkins1978,Prince1981} involve
definites that seem to relate back to the preceding context in more
indirect ways. \newpage

\begin{exe}
\ex\label{ex:schwarz:9}
\begin{xlist}
\ex \textit{John was driving down the street.} 
\ex \textit{The steering wheel was cold.}
\end{xlist}
\ex\label{ex:schwarz:10}
\begin{xlist}
\ex \textit{John bought a book today.} 
\ex \textit{The author is French.}
\end{xlist}
\end{exe}

The steering wheel in \REF{ex:schwarz:9} is of course understood as
belonging to the car involved in the driving event in the first
sentence. Similarly, the author in \REF{ex:schwarz:10} is understood to
be the one who authored the previously mentioned book. But how should
these relations to the preceding context be seen theoretically? As it turns out, the \ili{German} articles differentiate between these two
standard cases in a theoretically interesting way, such that the weak article is used in the former case,
but the strong article in the latter.

\begin{exe}
\ex\label{ex:schwarz:11} 
\langinfo{German}{}{\citealt[52--53]{Schwarz2009}}
\begin{xlist}\is{part-whole relationship}
\ex\label{ex:schwarz:11a} {Part-whole relation}\\\gll Der K\"uhlschrank war so gro\ss , dass der K\"urbis problemlos \textbf{im} \textnormal{/} \textnormal{\#}\textbf{in} \textbf{dem} \textbf{Gem\"usefach} untergebracht werden konnte.\\
the fridge was so big that the pumpkin {without a problem} {in\_\theweak} / {\phantom{\#}}{in} {\thestrong} {crisper} stowed be could\\
\glt`The fridge was so big that the pumpkin could easily be stowed
\textit{in the crisper}.'   
\ex\label{ex:schwarz:11b} \is{product-producer relationship}Producer relation\\\gll Das Theaterst\"uck missfiel dem Kritiker so sehr, dass er in seiner Besprechung kein gutes Haar \textnormal{\#}\textbf{am} \textnormal{/} \textbf{an} \textbf{dem} \textbf{Autor} lie\ss . \\
the play displeased the critic so much that he in his review no good hair {\phantom{\#}}{on\_\theweak} / {on} {\thestrong} {author} left\\
\glt `The play displeased the critic so much that he tore \textbf{the
  author} to pieces in his review.' 
\end{xlist}
\end{exe}

The first example is entirely unsurprising if we assume that the weak
article requires uniqueness (plus a suitable mechanism for domain
restriction, as needed for any \is{uniqueness|)}uniqueness-based account), assuming that there is a unique crisper
in the mentioned fridge. The second case is more interesting, and
arguably informs just what mechanisms are at play in relating the
interpretation of definites to the context. Taking the above
illustrations of the role of \isi{anaphoricity} for strong article definites
seriously, the most straightforward analysis here is that the
\is{relational nouns}relational noun can have its relatum slot filled by an \is{anaphora}anaphoric
index, which links the author directly back to the aforementioned
book.

Looking beyond simple referential cases, it is well known that
definites can also receive co-varying interpretations in
\is{quantifiers}quantificational contexts. Interestingly, both types of \isi{bridging}
examples (as well as ones parallel to the simple \is{uniqueness}unique and anaphoric
examples above) generalize to such environments:

\begin{exe}
\ex \ili{German}
\begin{xlist}
\ex \gll Jeder Student, der {ein} {Auto} parkte, brachte einen Parkschein {am} \textnormal{/} \textnormal{\#}an {dem} {R\"uckspiegel} an.\\
every student that a car parked attached a {parking-pass} {on\_\theweak} / {\phantom{\#}}on {\thestrong} {rear view mirror} \textsc{part}\\
\glt `Every student that parked a car attached a parking pass to the rearview mirror.'

\ex\gll Jeder, der {einen} {Roman} gekauft hat, hatte schon einmal eine Kurzgeschichte \textnormal{\#}vom \textnormal{/} {von} {dem} {Autor} gelesen.\label{author}\\
everyone that a novel bought has had already once a {short story} {\phantom{\#}}{by\_\theweak} / by {\thestrong} author read\\
\glt `Everyone that bought a novel had already once read a short story
by the author.'
\end{xlist}
\end{exe}

This is of substantial theoretical importance, as the analysis of
co-variation under \isi{quantifiers} is at the core of the interaction
between contextual information and grammatical machinery. Thus, any
analysis of the contrast between definite article forms must be rich
enough to extend to a broader framework that can account for
co-variation. A simple story in terms of purely pragmatic constraints on
\isi{reference} and contexts of use that is not tied into these more intricate aspects of
grammar would thus fall short.


\subsection{Sketch of the analysis in \citet{Schwarz2009}}

The core of the analysis of the two types of definites in
\citet{Schwarz2009} is that weak article definites are referential
expressions (of type $e$) that presuppose that there is a \is{uniqueness}unique
entity meeting the description of the \is{noun phrases}noun phrase (in the tradition of
\citeauthor{Frege1892} and \citeauthor{Strawson1950}). In contrast, strong article definites involve an
additional \is{anaphora}anaphoric component, captured by a (pronoun-like) index
introduced as
a syntactic argument of the strong article. The analysis
is couched in a broader framework to capture the \isi{bridging} data, as well
as the interplay of context
and grammatical mechanisms behind co-variation in different ways for
the two cases. 

Starting with the weak article, the analysis assumes that a
syntactically represented situation pronoun is an argument of the
\is{determiners}determiner, which provides the means for ensuring an appropriate
\isi{domain restriction} relative to which \isi{uniqueness} holds.\footnote{It also
accounts for the various interpretations of definites in the scope of
intensional operators; see \citep{Schwarz2009} for detailed
discussion.} Semantically, the weak article denotes a function that
takes a situation and a property as arguments, and returns the unique
entity that has the property in that situation, if there is one (else,
its denotation is undefined).

\begin{exe}
\ex\label{ex:schwarz:13}
\begin{xlist}
\ex\label{ex:schwarz:13a} $ [ _{\text{DP}}\ [\textit{the}_{\text{weak}}\ s]\ \text{NP}]$

\ex\label{ex:schwarz:13b} $\sem{\textit{the}_{\text{weak}}}^g = \la s_r\la P_{\pair{e,st}}.\iota x
[P(x)(s_r)]$ 
\end{xlist}
\end{exe}

The value of the situation pronoun is essentially determined in the
same way as that of regular pronouns: it can receive its value from
the assignment function, which captures the case where definites are
interpreted independently of the situation relative to which the
sentence as a whole is interpreted (i.e. relative to a resource
situation, following the terminology of
\citealt{Fintel1994}). Alternatively, it can be bound, either in such
a way that it is identified with the \isi{topic} situation (that the
sentence as a whole is about), or by a \is{quantifiers}quantificational expression, in
which case the denotation of the definite as a whole co-varies with
the situations quantified over.

The strong article minimally differs from the weak article in that it
takes an additional individual (type $e$) argument, which is
syntactically introduced by an index (that is semantically equivalent
to a pronoun). The referent of the definite as a whole is identified
with the value of this index (with the exception of \isi{bridging} cases,
discussed below). 

\begin{exe}
\ex\label{ex:schwarz:14}
\begin{xlist}
\ex\label{ex:schwarz:14a} $ [ _{\text{DP}}\ i\ [[\textit{the}_{\text{strong}}\ s]\ \text{NP}]]$

\ex\label{ex:schwarz:14b} $\sem{\textit{the}_{\text{strong}}}^g = \la s_r\la P_{\pair{e,st}}\la
y.\iota x [P(x)(s_r)\ \&\ x=y]$
\end{xlist}
\end{exe}

The additional index argument of the strong article essentially
introduces a \isi{familiarity} constraint, as the context has to provide a
value for the index via the assignment function. A preceding
\is{indefinites}indefinite is one standard way for ensuring that, though other options
may exist as well. While the issue of just how a referent for a strong
article definite can be made familiar in a suitable way in the context deserves more in-depth exploration (also in relation to prior
discussions of familiarity in the literature), I will limit
discussion here to the former case, because it is easiest to control for in
example contexts. 

In addition to receiving a value contextually, the
index can also be bound in various ways, rendering co-varying
readings. Fundamentally, once we subscribe to the above meanings for
the weak and strong articles, we are committed to allowing for both of
the standard mechanisms for introducing co-variation for \isi{definites},
namely via binding of the situation pronoun or of the
index.\footnote{Given the existence of so-called \isi{donkey anaphora} cases
  with strong article definites, the latter furthermore requires some
  version of dynamic binding.} Yet a
further key consequence for interpretation in context more generally
is that
the specific analysis in \citet{Schwarz2009} leaves no role to play
for \isi{domain restriction} via $C$-variables (basically, pronouns for
predicates; see \citealt{Fintel1994} and \citealt{StanleySzabo2000}). 


\subsection{Some additional theoretical issues}

While the main focus of the remainder of the paper is on
cross-linguistic empirical issues, there are some further theoretical
questions in relation to the analysis sketched above that should not go
unmentioned (though the discussion below is hardly exhaustive in this
regard). First, while the denotations in \REF{ex:schwarz:13b} and
\REF{ex:schwarz:14b} are clearly related, and in fact largely overlap,
this is not captured in any explanatory way as things stand -- there
simply are two lexical entries that happen to be very
similar. Recent work by \citet{Hanink2016} and \citet{Hanink2015}
proposes to address this issue by assuming just one definite article,
with a denotation like the one in \REF{ex:schwarz:13b}, which can be
compositionally extended to yield the strong article. In other words,
the lexical variation above is instead re-analyzed as purely
structural variation, all couched in a Distributed Morphology account
of the contraction phenomena. This seems like a very promising avenue,
though a few new questions also arise in light of it: first, given
that this account is directly tied into capturing contraction, how can
it be extended to languages with two full, independent paradigms for
weak and strong articles (such as \ili{Fering})? Relatedly, how does this
approach integrate languages where the correlate of weak article 
definites seems to be expressed by \isi{bare nouns}? Finally, some potential
evidence in favor of multiple lexical entries for different definite
articles comes from \citet{Grubic2016}, who presents data suggesting a
separate relational strong article variant being in play in \isi{bridging}
cases. Despite these further concerns, it is theoretically desirable to tie
together the analysis of weak and strong articles in a more
explanatory way, so reconciling these issues with a more explanatory\largerpage[1]
proposal should clearly be pursued in future work.

Another range of rather intricate issues arises in connection with
\isi{relative clauses}. It has commonly been claimed in the literature that \is{relative clauses!restrictive}restrictive relative clauses require the strong article in their
head. To the extent that this holds, it clearly requires an
explanation of the interaction between the structure and meaning of
the article and a \is{relative clauses}relative clause structure in a position that would
standardly be assumed to feature as part of its complement NP. But
complicating things further, various authors have pointed out
additional subtleties, potentially involving further distinctions
between types of \is{relative clauses}relative clauses \citep[see,
among others,][]{Hofherr2013,Wiltschko2013,Simonenko2014}. While the recent
literature (including a proposal for capturing the -- likely too --
simple generalization about \is{relative clauses!restrictive}restrictive relative clauses by
\citealt{Hanink2016}) has contributed real advances, this
area will require substantial further attention, especially cross-linguistically.


\section{The weak vs.\ strong contrast across languages}

\subsection{Key empirical and theoretical questions}

As we now turn to an overview of data from languages exhibiting
similar phenomena, let us begin by stating the key empirical questions
about the cross-linguistic data in relation to weak and strong article
definites. First, we need to determine what other languages exhibit
the same (or at least a highly similar) contrast in their \is{noun phrases}noun phrase
system. Secondly, what formal means do other languages utilize in
expressing it? Finally, to what extent do we find variation
in terms of its semantics/pragmatics, and how does this
relate to its formal expression on the one hand and
the noun phrase system of the language in question on the other?

To preview the perspective laid out below, I argue that there is
quite a broad set of unrelated languages that exhibit contrasts that
can arguably be modeled in a semantically uniform way, suggesting that the underlying
contrast between weak and strong article definites is generally
available as part of the inventory that natural languages can draw on. Within those languages, we
find a wide range of formal means for \is{definiteness marking}encoding it. Understanding this variation in form seems crucial for a
satisfactory analysis of the interplay of forms and meanings involved. In addition to this first set of languages with an essentially
uniform meaning contrast, other languages seem to diverge more
substantially from this pattern in that they display different types
of distinctions. One possibility is that these are simply revealing yet
another dimension of possible variation, that is in principle
independent of the weak vs.\ strong contrast. Alternatively, we can
consider a more gradient approach to variation, that allows languages
to fall into different places of a continuum of possible 
differences between types of definites. Ultimately, the key theoretical
questions are how many distinctions are needed to account for the range
of empirical variation, what is their nature (e.g. categorical or
gradient), and -- if there are multiple such distinctions -- how are
they related? We will naturally not be able to answer all these
questions conclusively, but will discuss pertinent data in relation to
these issues.

With regards to variation in form, one way in which languages clearly
differ is in whether they exhibit a contrast between two overt forms,
or whether the contrast is between the presence and the absence
of a given form (cf.\ the distinction between Type I and Type II splits
in \citealt{Ortmann2014}). The former situation clearly holds in the \ili{Germanic
dialects} and in \ili{Icelandic} \citep{Ingason2016}, and possibly also in
\ili{Hausa} and \ili{Lakhota} \citep[for discussion and references,
see][]{Schwarz2013}. The latter situation seems to hold
in  Akan\il{Akan} \citep{ArkohMatthewson2013},
 \ili{Korean} \citep{Cho2016,Ahn2016}, 
 \ili{Mauritian Creole} \citep{Wespel2008},
 \ili{Czech} \citep{Simek2016},
 \ili{Thai} and \ili{Mandarin} \citep{Jenks2015},
 \ili{Upper Silesian} \citep{Ortmann2014},
  \ili{Upper Sorbian} \citep{Ortmann2014},
 \ili{Ngamo} \citep{Grubic2016}, American Sign Language\il{American Sign Language} \citep{IraniSchwarz2016}
and \ili{Lithuanian} \citep{Sereikaite2016}. 

The following sections provide illustrative pairs of examples from
a fair number of these languages, selected to highlight cases where the contrast has been studied in some detail.
The core phenomenon I focus on is  \isi{bridging}, as this is
both in many ways the most subtle and perhaps most surprising aspect
of the article contrast, since the data themselves in no way intuitively
impose what analysis of definites would be the most obvious
candidate. But note that at least generally speaking, parallel
effects systematically occur for more standard \is{anaphora}anaphoric and \is{uniqueness}unique
  definite uses in all these cases, so the data discussed here for
  illustration should not be
  taken to suggest that the relevant distinction is only made for the
  \isi{bridging} cases.\footnote{A caveat before diving into the cross-linguistic data: not all
of the languages discussed below have been investigated at the same
level of empirical depth, and there thus may be more variation than
apparent here. But I tried to only include relatively well-documented cases
that so far have essentially yielded complete overlap with the \ili{German}
contrast.}


\subsection{Illustrations of weak and strong article definites across languages} 

The first illustration comes from Akan\il{Akan}. \citet{ArkohMatthewson2013}
discuss data parallel to that considered in \citet{Schwarz2009}, with a contrast between \is{bare nominals}bare \isi{noun phrases}, as in \REF{ex:schwarz:15a},
which presumably is a case of \isi{bridging} involving \isi{situational
uniqueness}, and the \is{familiarity}familiar form \textit{n\'ʊ} in
\REF{ex:schwarz:15b}, which they argue to be a case of anaphoric
\isi{bridging}.\footnote{For recent work offering a different perspective, which disagrees with
  the familiarity-based analysis by \citet{ArkohMatthewson2013}, see \citet{Bombi-Ferrer2017}.}

{\begin{exe}
\ex\label{ex:schwarz:15} Akan\il{Akan} \citep[14--15]{ArkohMatthewson2013}
\begin{xlist}
\ex\label{ex:schwarz:15a} Weak\\ \gll {Y\`e-h\'u-\`u} {d\`an} {d\'ad\'aw} {b\'i} {w\`ɔ}
{\`ek\`ur\'as\'i} {h\'ɔ} {\'nky\'ɛns\`id\'an}
{\op}\textnormal{\#}\textbf{n\'ʊ} \textnormal{/} \textnormal{\#}\textbf{bi}) {\'e-h\'odw\`ow}\\
1\textsc{pl.sbj}-see-\textsc{past} building old \textsc{indef} at village there roof {\phantom{\#}}\textsc{def} / {\phantom{\#}}{\textsc{indef}} \textsc{perf}-worn-out\\
\glt `We saw an old building in the village; \textbf{(\#the / \#a (certain))}
roof was worn out.'

\ex\label{ex:schwarz:15b} Strong\\ 
\gll {\`As\'aw} {n\'ʊ} {y\'ɛ-\`ɛ}
{\`ɔh\'in} {n\'ʊ} {f\`ɛw} {\'ar\'a}
{m\`a} {\`ɔ-ky\'ɛ-\`ɛ} {\`ɔky\`ir\'ɛf\'ʊ} \textbf{n\'ʊ} {\`adz\'i}\\
dance \textsc{def} do-\textsc{past} chief \textsc{def} beautiful just \textsc{comp} 3\textsc{sg.sbj}-give-\textsc{past}
trainer \textsc{fam} thing\\
\glt `The dance was so beautiful that the chief gave \textbf{the trainer} a gift.'
\end{xlist}
\end{exe}}

\hspace*{-0.55226pt}Similarly, \ili{Mauritian Creole}, discussed by \citet{Wespel2008}, distinguishes between a \is{null determiners}null form \REF{ex:schwarz:16a} and one clearly derived
from the \ili{French} definite article \textit{la}, but which seems to be
restricted to uses parallel to the strong article,
as illustrated by the \is{anaphora}anaphoric `book-author' \isi{bridging} case in \REF{ex:schwarz:16b}.

\begin{exe}
\ex\label{ex:schwarz:16}
\langinfo{Mauritian Creole}{}{\citealt[155--156]{Wespel2008}; source: O.M.2.8, O.M.22}
\begin{xlist}
\ex\label{ex:schwarz:16a} 
Weak\\ \gll Mo fin visite enn lavil dan provins. \textbf{Lameri} ti pli ot ki \textbf{legliz}.\\
 I \textsc{acc} visit one village in province {town-hall} \textsc{pst} more high than {church}\\
\glt `I visited a village in the province. \textbf{The town hall} was higher than \textbf{the church}.'
 
\ex\label{ex:schwarz:16b}
Strong\\ \gll Li fin kontan liv la ek aster li envi zwen \textbf{loter} \textbf{la}.\\
she \textsc{pst} love	book \textsc{def} and now she want meet {author} {\textsc{def}}\\
\glt `She was fond of the book and now she wants to meet \textbf{the author}.'

\end{xlist}
\end{exe}


American Sign Language\il{American Sign Language} features an expression resembling pointing
within the signing space, which has been much discussed in the recent literature with regards
to its pronominal uses 
\citep{Schlenker2017}. However, it also serves the role of a strong
definite article, as illustrated by its obligatory occurence in
\is{anaphora}anaphoric \isi{bridging} in \REF{ex:schwarz17b}.\footnote{Interestingly, this
  same form can also be used to introduce new \is{discourse}discourse referents, as
  can be seen in the first sentence of \REF{ex:schwarz17b}; see
  \citet{Irani2017} [in this volume] for a fuller analysis.} In contrast, cases
involving \isi{situational uniqueness} \isi{bridging}, as in \REF{ex:Schwarz:17a}, are
incompatible with this form.

\begin{exe}
\ex\label{ex:schwarz:17} \langinfo{American Sign Language}{}{\citealt{Irani2016}}
\begin{xlist}
\ex\label{ex:Schwarz:17a} Weak\\ \smc{IX}$_\text{a}$ \smc{CAR}, \smc{POLICE} \smc{STOPPED} \smc{WHY} (\#\textbf{\smc{IX}}$_\text{a}$) \smc{MIRROR} \smc{BROKEN}. \\
‘The car was stopped by the police because the mirror was broken.’ 
\ex\label{ex:schwarz17b} Strong\\ \smc{JOHN} \smc{BUY} \smc{IX}$_\text{a}$ \smc{BOOK}. \#(\textbf{\smc{IX}}$_\text{a}$) \smc{AUTHOR} \smc{FROM} \smc{FRANCE}.\\
   ‘John bought a book. The author is from France.’
\end{xlist}

\end{exe}

In yet another similar vein, recent discussion of \ili{Korean} suggests that
what had traditionally been considered a \is{demonstratives|(}demonstrative -- \textit{ku}
 -- seems to
function as a \is{familiarity}familiar definite marker, while \isi{uniqueness} based
definites are expressed with \isi{bare noun phrases}.\footnote{See
  \citet{Ahn2017} for a recent proposal that \ili{Korean} actually makes a
  three-way split, further extending the typological picture.}

\begin{exe}
\ex\label{ex:schwarz:18} \langinfo{Korean}{}{\citealt[6]{Cho2016}}
\begin{xlist}
\ex Weak\\ \gll Gyeolhonski-e gatda. {Sinbu-ga} \textnormal{/} \textnormal{\#}ku {sinbu-ga} paransek-ul
ipeotda.\\
 wedding-to went bride-\textsc{nom} / {\phantom{\#}}that bride-\textsc{nom} blue-\textsc{acc} wore \\
\glt `(I) went to a wedding. The bride / \#that bride wore blue.'

\ex Strong\\ \gll Jonathan-un eojebam-e sesigan dokseorul haetda. {ku}
{soseolchayk-i} \textnormal{/} \textnormal{\#}soseolchayk-i jaemi-itdago saengakhaetda.\\
Jonathan-\textsc{top} yesterday night-at {three hours} {reading did}. ku novel-\textsc{nom} / \phantom{\#}novel-\textsc{nom} interesting thought.\\
\glt `Jonathan read for three hours last night. (He) found the novel
interesting' 
\end{xlist}
\end{exe}



A final case (at least as far as the present discussion is concerned) of
a language that has been argued to feature an overt form, namely a \is{specificity}specific \is{classifiers}classifier
construction, that parallels strong article \isi{definites}, vs.\ \isi{bare nouns}
to express weak article definites, is that of \ili{Thai}.

\begin{exe}
\ex\label{ex:schwarz:19} \langinfo{Thai}{}{\citealt[109]{Jenks2015}} 
\begin{xlist}
\ex Weak\\ \gll r\'ot khan n\'an th\`uuk tamr\`uat s\`ak\`at
{phr\'ɔʔ} m\^aj.d\^aj t\`it
{satikəə} w\'aj th\^ii {th\'abian} ({\#baj} {n\'an}).\\
 car \textsc{clf} that \textsc{adv.pas} police intercept because \textsc{neg} attach sticker 
keep at license {\phantom{(\#}}\textsc{clf} that\\
\glt `That car was stopped by police because there was no sticker on
the license.'

 \ex Strong\\\gll ʔɔɔl kh\'it w\^aa
klɔɔn b\`ot n\'an {pr\'ɔʔ} m\^aak, m\^ɛɛ-w\^aa kh\'aw c\`a m\^aj
 ch\^ɔɔp {n\'akt\`ɛɛŋklɔɔ}  \#({khon} {n\'an}).\\
 Paul thinks \textsc{comp} poem \textsc{clf} that melodious very, although 3\textsc{sg} \textsc{irr} \textsc{neg} like poet {\phantom{\#(}}\textsc{clf} that\\
\glt`Paul thinks that poem is beautiful, though he doesn't really like the
poet.' 
\end{xlist}
\end{exe}

A rather different instantiation of the weak vs.\ strong article
contrast can be found in \ili{Icelandic}. While the definite article
generally appears as a suffix on the head \is{nouns}noun, this suffixation is
blocked by a certain class of evaluative
\is{adjectives}adjectives. \citet{Ingason2016} shows that the free form
\textit{hinum}, which had previously been considered as archaic, can
occur in such cases in the modern standard, but only if we are dealing
with a weak article definite. Strong article definites in such
circumstances can only be expressed by the demonstrative \textit{\th
  essum}. 

\begin{exe}
\ex\label{ex:schwarz:20} \langinfo{Icelandic}{}{\citealt[108, 131]{Ingason2016}}  
\begin{xlist}
\ex Weak \\ Context: The speaker is annoyed that she always loses. There is
   only one winner per round. \\
\gll Alltaf eftir hverja {umfer\dh} {eru} spilin gefin aftur af {{\ob}$_\text{\textnormal{DP} }$\textbf{hinum}} {\'o\th olandi} {sigurvegara}{\cb}.  \\
 always after each  round are cards.the given again by {\db}{\phantom{$_\text{DP }$}HI-{the}$_{\text{weak}}$} {intolerable$_{\text{evaluative}}$} winner \\
\glt `Always after each round, the cards are dealt again by the intolerable winner.'
   
\ex Strong\\ 
Previous discourse: Mary talked to a writer and a {terrible} politician.\\
  She got no interesting answers from\ldots \\
  \gll \ldots \th essum / \#{hinum} {hr\ae \dh ilega} stj\'ornm\'alamanni.\\
 \ldots this / \phantom{\#}HI-{the}$_\text{weak}$ terrible$_\text{evaluative}$ politician \\

\end{xlist}
\end{exe}

Another case where \is{adjectives}adjectives crucially feature in the expression
of the weak vs.\ strong contrast, though in a different way, is
\ili{Lithuanian} (\citealt{SereikaiteToAppear} {[}in this volume{]}). It exhibits a definite suffix that
appears on adjectives, but only when they are of the strong article
definite variety. In cases of \is{unique definites}uniqueness-based definites, the
\is{adjectives}adjective will form a \is{noun phrases}noun phrase with the \is{nouns}noun without this
suffix. Interestingly, such ``bare'' forms also have \is{indefinites}indefinite
uses. Furthermore, the suffix has a much wider distribution, and can
also appear on demonstratives and pronouns, among others. This wider
distribution, as well as more intricate variations in the range of
uses involving \isi{kind reference}, deserve much more detailed attention,
but at this point it seems safe to say that at least part of the
contrast between bare and definite-suffixed forms seems to track the
weak vs.\ strong article definite contrast. 

\begin{exe}
\ex\label{ex:schwarz:21} \langinfo{Lithuanian}{}{\citealt{SereikaiteToAppear} {[}in this volume{]}}
\begin{xlist}
\ex Weak\\ 
\gll {Praėjus} {dviem} {savaitėm} {po} {rinkimų}, {prezidentas} {turi} {teisę} {atleisti} \textbf{naują} \textnormal{/} \textnormal{\#}\textbf{naują-jį} \textbf{ministrą} \textbf{pirmininką} {tik} {išskirtiniais} {atvejais}.\\
Passed two weeks after elections president has right fire \textbf{new} / \phantom{\#}\textbf{new-\textsc{def}} \textbf{minister} \textbf{prime} only exceptional cases\\
\glt `Two weeks after the election, the president has a right to fire \textbf{the new prime minister} only in exceptional cases.'

\ex Strong\\ 
\gll \textbf{Knyga} ``{Lietus}" {sulaukė} {neįtikėtino} {populiarumo}, {nepaisant} {to}, {kad} \textbf{talentingas-is} \textnormal{/} \textnormal{\#}\textbf{talentingas} {rašytojas} {nusprendė} {likti} {anonimas}.\\
\textbf{Book} `Rain' received incredible popularity despite {} that \textbf{talented-\textsc{def}\textsubscript{strong}} / \phantom{\#}\textbf{talented\textsubscript{weak}} writer decided remain anonymous\\
\trans `\textbf{The book `Rain'} became incredibly popular despite the
fact that \textbf{the talented writer} decided to remain anonymous.' 
\end{xlist}
\end{exe}

While this overview can only be cursory, given space constraints, the
relatively minimal pairs of examples from this range of largely
unrelated languages should illustrate that key phenomena concerning the weak
vs.\ strong-article definite contrast are mirrored by formal
distinctions between different types of \isi{definite noun phrases}
cross-linguistically. There are two key questions, both from a theoretical
perspective and for pursuit in future research on definites across
languages: a) how does the formal expression of the contrast vary across
languages and how does this variation relate to the core meaning contrast?
b) to what extent is the contrast the same across languages, and
to what extent, and in what form, do we find variation in this regard. I
turn to some -- necessarily preliminary~-- considerations in the
following section.


\section{Variation in form and meaning}

\subsection{Variation in form}

Starting with variation in the form of how the contrast between
weak and strong article definites is expressed, an initial
generalization, from the perspective of the analysis of
\citet{Schwarz2009}, seems to be that a `more' in meaning is generally
reflected in a `more' in form: the weak article definites in \ili{German}
and related dialects all involve morpho-phonologically reduced forms,
e.g. contraction in \il{German}Standard German. In the \ili{Germanic dialects} with
two full article paradigms, weak article forms
also seem to be less complex than strong article ones. And in many
languages, of course, this situation descriptively holds in the
extreme, as weak article definites are expressed with \isi{bare noun phrases}. 

Two particularly interesting cases with regards to the formal
realization of the contrast are \ili{Icelandic} and \ili{Lithuanian}. In
\ili{Icelandic}, the same nominal suffix is used to express both types of
definites in most contexts. Only when, in the analysis of
\citet{Ingason2016}, suffixation is blocked by evaluative \is{adjectives}adjectives
do we find a distinction, such that an otherwise archaic free-form
article is used for weak article definites. While at first sight, this
seems perhaps at least in one sense more complex than the default
configuration, strong article definites cannot be realized by the
default form in that case either, but instead call for a
demonstrative (which \textit{is} more complex). 

Turning to \ili{Lithuanian}, the perhaps most notable point is that the
explicit \is{definiteness marking}indication of definiteness occurs neither on the \is{nouns}noun itself
or at the level of a (potential) D-head, but rather in the form of a suffix on
\is{adjectives}adjectives between these two. The formal relation between this
suffix and a potential \is{null determiners}null D-head of course constitutes one key
question in this regard, and there seem to be arguments in favor of a
\is{determiner phrases}DP-layer for both cases, contrary to what has been said about,
e.g. Serbo-Croatian, where the formal realization otherwise seems
somewhat similar \citep{Sereikaite2016}. In addition, it bears
repeating that the same suffixal form that we find on \is{adjectives}adjectives can
also appear in various other places, most relevantly pronouns and
demonstratives. While in principle, the effect there does not seem to
be dissimilar, the details are not obvious and require much more
extensive exploration.

Returning to the more general issue of meaning and form, the apparent
generalization about the formal realization of the distinction should be
taken seriously and relates to key choice points in the semantic
analysis of the article contrast: if we want to capture the
relationship between both the forms and meanings involved in such a
way that one is in some way derived from, or an extension of, the
other, then this would call for broader proposals of the sort put
forth by \citet{Hanink2016} and \citet{Hanink2015}, briefly
discussed above, which extend
to cases of languages with two full article paradigms. On the other
hand, if we assume two distinct lexical entries for weak and strong
articles, than the generalization about the forms involved would have
to be explained in another way, e.g. from the perspective of
historical development, which could see the morpho-phonologically less
complex forms as more \is{grammaticalization}grammaticalized or
bleached, perhaps in parallel to the relation between \is{demonstratives|)}demonstratives
and definite articles more generally \citep{Lyons1999}. 


The fact that many languages use \isi{bare noun phrases} for the weak
article also relates to this question, of course, as well as to key
issues in \is{determiner phrases}DP-syntax. In particular, the question arises of whether or
not a \is{determiners}determiner-level is present in these noun phrases in the first
place, and if so, why it is the weak article meaning that can
standardly be realized as phonologically \is{null determiners}null. Alternatively, a common
move is to assume that purely semantic \is{type shifting}type-shifters can do the job
of (both \is{definites}definite and \is{indefinites}indefinite) articles when overt forms are
lacking \citep{Partee1987,Chierchia1998,Dayal2004}. This then raises
questions about the interplay between the determiner-inventory in the
relevant languages and the constraints for the applications of such
type-shifters. Furthermore, since the null-hypothesis for such type-shifters clearly would be that their effect is universal across
languages, any variation in the interpretive options of bare noun
phrases that cannot be accounted for in terms of the determiner system
of the language in question, e.g. in terms of blocking effects from
available overt forms, would seem to support the notion that distinct
lexical determiners with the same phonologically \is{null determiners}null form can in
principle be available, in contrast to what is commonly argued by
proposals based on \is{type shifting}type-shifters \citep[for recent discussion,
see][]{Dayal2016}.\newpage

Of particular importance in this regard is the potential case of languages which
exhibit a genuine ambiguity between definite and \is{indefinites}indefinite
interpretations for bare noun phrases. Initial evidence in relevant
discussion of, e.g. Akan\il{Akan} \citep{ArkohMatthewson2013}, \ili{Lithuanian}
(\citealt{SereikaiteToAppear} [in this volume]), and ASL\il{American Sign Language} (\citealt{Koulidobrova2012,Irani2017} [in this volume])
suggests that this is a possibility, contra the type-shifter based
proposal by \citet{Dayal2016}, but further scrutiny is needed, both
empirically and in terms of integrating the article-contrast
issues into the broader theoretical picture.\footnote{One important question in this discussion
  is what counts as an ``article-less'' language for the purposes of
  generalizations made by such proposals: where do languages which
  express weak article definites with \isi{bare noun phrases}, but have an
  \is{determiners}explicit determiner form for strong article definites, fall?}






\subsection{Variation in meaning}


While in the data so far the semantic contrast arguably can be seen
as entirely
uniform, it is undeniable that there is some degree of variation in
this regard as well. Some of it consists of fairly detailed aspects,
including what forms are used in certain cases where the contextual
constraints for \is{anaphora}anaphoric uses or
situational \isi{uniqueness} are met, and in some cases additional
distinctions involving other features may be
at play as well. Generally speaking, these cases are consistent with
the semantic analysis of the contrast laid out above, but involve
differences in what form winds up being preferred given a certain type
of context. But there also seems to be more substantial
variation, which may require reconsidering the broader theoretical set
of options. Some illustrations of the former cases are provided in
the remainder of this section, while I turn to the latter in the next section.

One point of more subtle variation concerns anaphoric usage in longer
narrative texts. A central character of a story (e.g. a fisherman, as
in the \ili{Fering} story considered by  \citealt{Ebert1971b}) may be
introduced with an \is{indefinites}indefinite, and then initially picked back up by a
strong article definite. But as the central role of the character becomes
clear in the narrative, one may then switch to using weak article
definites for it. In contrast, according to intuitions reported by Anton Ingason (\pc), \ili{Icelandic} would keep using the form corresponding
to the strong article definite in this situation. But while the
conditions for anaphoric uses are met, the central role of the
character in question may also suffice to provide contextual
restriction to ensure \isi{uniqueness} of that entity.


Another point of variation concerns contexts involving entities which
are both \is{uniqueness}unique and \is{familiarity}familiar (at least in a weak sense) in the broader
non-linguistic context, e.g. with regards to a family dog. Akan\il{Akan} and
\ili{German} seem to differ here, in that the former chooses to use the
overt strong article, whereas \ili{German} prefers\largerpage[1] the weak article form.\footnote{\ili{Mauritian Creole} may be similar to Akan\il{Akan} in this regard; see \citet[189--190]{Wespel2008}.}

\protectedex{
\begin{exe}
\ex\label{ex:schwarz:22} Context: You and your spouse own one dog. While your spouse is away, someone breaks into your house and you are telling them about it on the phone. You say:

\begin{xlist}
\ex \langinfo{German}{}{\citealt[19]{ArkohMatthewson2013}}\\
\gll Der Einbrecher ist zum Gl\"uck {vom} \textnormal{/} \textnormal{\#}{von} {dem} {Hund} verjagt worden.\\
the burglar is {to\_the$_{\text{weak}}$} luck {by\_the$_\text{weak}$} / {\phantom{\#}by} {the$_\text{strong}$} dog chased been\\
\glt `Luckily, the burglar was chased away by the dog.'

\ex \langinfo{Akan}{}{\citealt[19]{ArkohMatthewson2013}}\\
\gll {\`Ow\`if\'ʊ}  {n\'ʊ},
{b\`ɔd\'ɔm} {n\'ʊ} {k\`a-\'a}
{n\'ʊ-d\'ʊ} {\'ar\'a} {m\'a} {\`o-g\'u\'an-\`i\`i}.\\
thief \textsc{def} {dog} \textsc{def} follow-\textsc{past} 3\textsc{sg-obj}-on just so \textsc{3sg.sbj}-run-\textsc{past}\\
\glt  ‘The thief, the dog chased away.’ 
\end{xlist}
\end{exe}}


But as before, the fact that conditions for \isi{situational uniqueness} are
met and an \is{anaphora}anaphoric
form is used is not
incompatible with the formal analysis. All that is required for a
strong article definite is that its index receives a value from the
assignment function. When an entity such as a family dog is \is{familiarity}familiar
in a context, that may suffice to establish that, parallel to how
personal pronouns can be used in similar situations, e.g. by parents
who have a single boy who can be referred to as \textit{he} without
any recent prior mention. But nonetheless, the question, of course,
needs to be addressed just why a language like Akan\il{Akan} should differ
precisely in that regard from other languages. One possibility is that
the availability of \is{indefinites}indefinite uses of \is{bare noun phrases} plays a role
here; this will need to be tested with regards to other languages with
similar properties.

Contexts of \isi{situational uniqueness} \isi{bridging} also seem to exhibit some
variation. For example, \citet{Wespel2008} cites Amern\il{Amern} data from \citet{Heinrichs1954}, showing
that the strong article is used in the following example for the \is{noun phrases}noun phrase headed by
\textit{altars}, even though it is clearly part of the aforementioned church.

\protectedex{
\begin{exe}
\ex\label{ex:schwarz:23} \langinfo{Amern}{}{\citealt[99]{Heinrichs1954}} \\
\gll {V\"or} {worən} {en} {də}
{n\"aldər} {kerək} {on} {wolən} {os} {\"ans} {di} {alt\"o\"ors}
{bekikə}.\\ we were in \textsc{def} of-N church and wanted us once {\textsc{def.pl}$_\text{strong}$} {altars} look-at\\
 \glt `We were in the church of Waldniel and wanted to have a look at
 the altars.'
\end{exe}}

The extent to which this is compatible with the formal analysis at
least in part depends on the properties of the nouns in question, in
particular with regards to the possibility of them receiving a
relational meaning, as \is{relational nouns}relational nouns in principle will open up to
\is{anaphora}anaphoric \isi{bridging} with the strong article, parallel to the
book-author cases considered above. Interestingly, other languages
have been argued to exhibit inter-speaker variation precisely in
this regard: \citet{Ortmann2014}  reports data from \ili{Upper Sorbian},
which seems to at least in part reflect generational
  variation such that, for some speakers, the strong article
  \textit{t\'on} is not obligatory in cases like the following, while
  it is obligatory across the board in cases parallel to the
  book-author examples. Additionally, Ortmann reports parallel judgment
patterns in \ili{Upper Silesian} to be extremely hard to ascertain empirically.

Yet another dimension of potential minor variation involves additional
distinctions. In particular, \citet{Ahn2016} reports a 3-way split in
\ili{Korean}, with an additional form specialized for genuinely deictic uses
(which are commonly available for strong article forms in other
languages as well).

In sum, there is clear evidence of what can be considered fairly minor
variation in the article contrast across languages, which in principle
is consistent with the semantic characterization provided, but calls
for further explanation of why languages should make different
pragmatic choices about which article to use in a given type of
context. Additionally, further and more fine-grained distinctions
extending beyond the weak-strong contrast seem to exist as
well. While much more needs to be explored, this data at least in
principle seems to be amenable to explanation within the general
approach outlined above.




\section{Beyond weak vs.\ strong}

\subsection{Different semantic contrasts}

In addition to what we saw in the previous section, there are other
languages that seem to diverge in more substantial ways in the way
that they exhibit a contrast between different types of definite
articles. For example, while \ili{Haitian Creole} is superficially similar to
\ili{Mauritian Creole}, and both have \ili{French} as their main source language,
the contrast between \isi{definite noun phrases} marked with \textit{la}
(derived from the \ili{French} definite article, as in \ili{Mauritian Creole}) and
\is{bare nominals}bare ones seems different from what we have seen
before.\footnote{Potential other candidate languages fitting this category
  include \ili{Bangla} \citep{SimpsonBiswas2016} and \ili{Jinyun} \citep{Simpson2017}, though further research is needed
  to compare these various cases in more detail.} First, parallel to
the \ili{Amern} data above, there seems to be no contrast between different
types of \isi{bridging}, and both situational and \is{anaphora}anaphoric \isi{bridging} use the
overt form (here realized as \textit{la} or \textit{a}):

\begin{exe}
\ex\label{ex:schwarz:24} \langinfo{Haitian Creole}{}{\citealt[114; source: E.F.32, E.F.36.9]{Wespel2008}}

\begin{xlist}
\ex {Weak article definite context}\\ 
\gll Y\`e,	mwen viste yon vil provens. \textbf{Meri}	\textbf{a}	pi
wo ke \textbf{legliz} \textbf{la}.\\
 yesterday I	visit	one	town province	{town-hall} {\textsc{def}} more high than {church} {\textsc{def}}\\
\glt`Yesterday I visited a town in the province. \textbf{The town hall} was higher than \textbf{the church}.' 

\ex {Strong article definite context}\\
\gll Eli te	renmen liv	la,	e	kounye a	li
vle	rankontre \textbf{ot\`e}	\textbf{a}.\\ 
Eli \textsc{pst}	love	book \textsc{def} and now	\textsc{def} she want meet	{author} {\textsc{def}}\\
\glt `Eli loved the book, and now she wants to meet \textbf{the
  author}.' 


\end{xlist}
\end{exe}


Similarly, \is{larger situation}larger or \is{immediate situation}immediate situation uses (in the
terminology of \citealt{Hawkins1978}), which in other languages call
for the weak article or equivalent, also generally call for the overt
form. The \is{bare nominals}bare form is only used for what Wespel calls complete
functional descriptions, i.e. cases where the head \is{nouns}noun denotes a
function and its relatum argument is explicitly introduced, as in
\REF{ex:schwarz:25}, which, as Wespel spells out in some detail, does not involve
a possessive construction of any sort.



\begin{exe}
\ex\label{ex:schwarz:25} \ili{Haitian Creole} \citep[98]{Wespel2008}\\
\textit{papa Mari}\\
`the father of Mary' 
\end{exe}

This situation seems very much at odds with the weak vs.\ strong
article contrast as spelled out above. To begin with, global uniques
(such as \textit{the sun}) are core cases for the analysis in
\citet{Schwarz2009}. The split between these and ``complete functional
descriptions'' is also rather puzzling from that perspective. One
sensible reaction might be to take this to reflect a fundamentally
different contrast, and I will explore some potential avenues for such
a move below. But even if this were successful, it would leave us with
vexing questions about how this state of affairs came about,
especially given the fairly minimal pair of two \ili{French-based creoles}
that both retain a form based on \ili{French} \textit{la}, but use it in
apparently very different ways.\footnote{Another interesting potential
  consequence of such a move, which I am not able to explore here in
  detail, is that this would seem like another case of genuine
  variation in the type of definiteness involved with \isi{bare noun
  phrases}, which would come as somewhat surprising for type-shifting
  based accounts of such noun phrases, again under the assumption that
  what type-shifters can do is universal.}

Turning to potential directions for alternative characterizations of
the \ili{Haitian Creole} contrast, some rather suggestive examples are
discussed by \citet{Wespel2008}. In particular, the presence or
absence of \textit{la} seems to
relate to the introduction of the domain of \textit{only}
(and parallel effects exist for \isi{superlatives}). In particular, when the
domain of \textit{only} is explicitly restricted by a post-nominal
\is{prepositions}prepositional phrase, such as `in his family', then no \textit{la} (or
allomorph) appears on the \is{noun phrases}noun phrase associated with \textit{only}
\REF{ex:schwarz:26a}. In contrast, when this prepositional phrase is used as a
framing adverbial, and not in the scope of \textit{only}, then the
overt article form does appear \REF{ex:schwarz:26b}.

\begin{exe}
\ex\label{ex:schwarz:26} 
\langinfo{Haitian Creole}{}{\citealt[118--119; source: E.F.76.20.a, E.F.76.20.b]{Wespel2008}}
\begin{xlist}
\ex \label{ex:schwarz:26a}\gll Py\'e se	\textbf{s\`el}	\textbf{gason} nan fanmi li.\\ 
\textsc{p} \textsc{cop} {only} {boy} in family his\\
\glt `Peter is \textbf{the only boy} in his family.'

\ex \label{ex:schwarz:26b}\gll 	Fanmi sa	a,	se	yon	gwo fami, men
Py\'e se	\textbf{s\`el} \textbf{gason} \textbf{an}.\\
 family \textsc{dem} \textsc{def} \textsc{cop} \textsc{indf} big family but \textsc{p}	\textsc{cop} only boy	\textsc{def}\\
\glt `This family is big, but Peter is \textbf{the only boy}.'

\end{xlist}
\end{exe}


Given this suggestive data, one potential avenue to explore, building
on the proposal by \citet{Wespel2008} that \textit{la} indicates the
use of a ``resource situation variable'', is that it is the overt
realization of a situation pronoun in the sense of
\citet{Percus2000}. Formally, a candidate requirement introduced by
this particular type of situation pronoun could be that it is not
identical to the \isi{topic} situation relative to which its clause is
evaluated.\footnote{Note that the analysis of \ili{English} \isi{demonstratives}
  by \citet{Wolter2006} develops some strikingly similar ideas for a
  different set of empirical facts.} The idea would then be that
(certain) overt phrases, such as the \is{prepositions}prepositional phrase `in the
family' in \REF{ex:schwarz:26a} as well as relatum DPs in functional
descriptions such as \REF{ex:schwarz:25}, are an alternative way of specifying the
value of this situation variable, making the overt article form
unnecessary. Interestingly, there also seems to be some variation in
the presence of the overt form
corresponding to the difference between \isi{situational uniqueness} through
common knowledge vs.\ \isi{anaphoricity} \REF{ex:schwarz:27}; however, much more
work is needed to flesh out the full empirical picture here.

\protectedex{
\begin{exe} 
\ex\label{ex:schwarz:27} \langinfo{Haitian Creole}{}{\citealt[116]{Valdman1977}} 
\begin{xlist}
\ex \gll Kote manje mwen? \\
              where meal my  \jambox{(interpreted relative to topic situation?)}\\
\ex \gll Kote manje mwen an?\\ 
             where meal my \textsc{def} \jambox{(based on previous mention)}\\
       \glt ‘Where is my meal?’ 
\end{xlist}
\end{exe}}

Theoretically, there are additional further implications of this type
of approach as well. For example, global uniques would have to be
assumed to require a situation pronoun (with a value distinct from the
topic situation). Potentially interesting predictions arise with
regards to \isi{intensional contexts}, where situation pronouns fill the
additional role of determining the intensional status of a
given \is{noun phrases}noun phrase (e.g. in terms of the \textit{de re}\slash\textit{de dicto}
contrast). In this regard, the fact that \textit{la} can occur on
entire clauses as well would also be of further interest. And as
already mentioned, the relationship between what happened to
\ili{French}-based \textit{la} over time in \il{Haitian Creole}Haitian and \ili{Mauritian Creole}
seems like a rich and important issue to explore. From the perspective
just sketched, we might be dealing with a situation where the two take
rather different paths to superficially similar but underlyingly
distinct systems, roughly corresponding to the difference between
representing \is{anaphora}anaphoric individual variables (as part of the strong
article meaning) and representing variables for situations in the form
of situation pronouns.

In sum, the case of \ili{Haitian Creole}, which likely is mirrored in other
languages as well, goes beyond what might be characterized as mere
pragmatic variations in how the same meanings are put to use in the
system of a given language, as reflected, e.g. in the lack of a
\isi{bridging} contrast in languages like Amern\il{Amern}. A striking observation, from the
present perspective, is that even global uniques come with the overt
form. The main question moving forward then will be to what extent the
pattern represented here by \ili{Haitian Creole} might reflect a
fundamentally different type of contrast, or whether there are other
languages that could be seen as further in-between cases, with a mix of
the properties of the languages discussed in previous sections and
cases like \ili{Haitian Creole}. If the latter were the case, this might
suggest that we are dealing with a more gradient spectrum after all,
which would require some fairly substantial reconsiderations for an
approach based on the formal article contrast as laid out above. I
briefly review and comment on such a more gradient account in the
following section.

\subsection{Semantic vs.\ pragmatic uniqueness}\is{pragmatic uniqueness|(}\is{semantic uniqueness|(}

A prominent alternative analysis goes back to \citet{Loebner1985},
with more recent developments in \citet{Loebner2011} and, of 
particular relevance for our purposes, a fairly extensive typological
discussion in \citet{Ortmann2014}. The core idea rests on a
distinction between semantic and pragmatic uniqueness, which crucially
rides on whether context has any role in establishing uniqueness. More
specifically, semantic uniqueness holds if a definite description
refers unambiguously based on the meaning of the \is{nouns}noun alone, in a
context-independent manner. In contrast, in cases of pragmatic
uniqueness, \isi{reference} is unambiguous only under consideration of
contextual information, which can be linguistic or
extra-linguistic. Crucially, this distinction is seen relative to a
gradient \isi{uniqueness scale}, which allows different languages to choose
different cut-off points for using one form as opposed to another. \citet{Ortmann2014} succinctly states the role of
these notions for article contrasts (or ``splits''):

\begin{quotation}
[\ldots] the distinction between semantic and pragmatic uniqueness is
the basis of all conceptually governed article splits, in that {a shift
towards an IC [Individual Concept] or FC [Functional Concept] is
overtly signaled}.\\\hbox{}\hfill{\hbox{\citep[296]{Ortmann2014}}}
\end{quotation}


The approach crucially rests on the assumption that nouns
differ lexically from one another with regards to their semantic
types. \tabref{tab:schwarz:3} provides an overview of the key
dimensions of variation, namely a) whether their meanings are at their
core referential (ending in type $e$) or predicative (functions from a
given number of individuals to truth values). 

\begin{table}[h]\is{uniqueness}\is{relational nouns}\is{associative anaphora}
 \begin{tabularx}{.75\textwidth}{lllll} 
  \lsptoprule
 & {Monadic} & {Polyadic}\\
  \midrule
{Non-unique} & {Sortal} nouns & {Relational} nouns\\
(pragmatic)& \textit{dog}, \textit{stone} & \textit{sister}, \textit{finger}\\
&\pair{e,t}& \pair{e, \pair{e,t}}\\[2ex]
{Unique} &  {Individual} nouns & {Functional} nouns\\
(semantic)&\textit{sun}, \textit{prime minister}& \textit{father},
                                                  \textit{head}\\
&\pair{e }& \pair{e,e}
  \\  \lspbottomrule
 \end{tabularx}
\caption{Semantic vs.\ pragmatic uniqueness \citep[adapted from][]{Ortmann2014}}
\label{tab:schwarz:3}
\end{table}

However, the type of nouns can be adjusted through (fairly standard)\largerpage[2]
type-shifting operations. \is{definite noun phrases}Definite noun phrases are generally analyzed as functional
concepts, in that they are assumed to refer unambiguously. However,
that status is attained in different ways, in that some \isi{nouns} require
a \is{type shifting}type-shifter, and others do not. The difference between two distinct
definite articles is then captured in terms of the signal they convey
about how \isi{uniqueness} was achieved. For example, the idea for \il{German}Standard
\ili{German} would be that the strong article indicates pragmatic
uniqueness, whereas the weak article indicates semantic uniqueness.

This idea is made more flexible by the notion that different types of
\isi{noun phrases} relate to the context in different ways. Based on
this, the approach assumes a \is{uniqueness scale}scale of uniqueness, ``defined according to the degree of invariance
of \isi{reference} of nominal expressions'' \citep{Ortmann2014}:

\begin{exe} \is{bridging}\is{relative clauses}\is{part-whole relationship}\is{anaphora}
\ex Scale of uniqueness (\citealt{Ortmann2014}: 314; adapted from \citealt{Loebner2011})\\
      deictic sortal noun < anaphoric sortal noun < SN with
      establishing relative clause < relational Definite Associative
      Anaphora\textsuperscript{*} < part-whole
      Definite Associative
      Anaphora, non-lexical functional nouns, < lexical individual
      nouns/functional nouns < \isi{proper names} < personal pronouns
\end{exe}

Essentially, a language with a contrast between definite articles
could then draw the line anywhere on this scale, \is{definiteness marking}marking expressions
to one side with a weak article and those to the other side with the
strong article. Intuitively, the
idea is that different \isi{nouns} require different amounts of lifting to
end up with the right semantic type for a definite description, and
the articles serve as indicators of whether a certain amount of lifting
had to occur. The approach naturally affords a substantially more fine-grained set of
typological options than any simple binary contrast.\largerpage[2] 

While not all relevant aspects of this proposal can be discussed here, let
us briefly assess both challenges and strengths of this general
approach. 

Starting with the former, there is a question at the level of the
general architecture of the \is{syntax-semantics interface}syntax-semantics interface with regards to
the mapping from syntactic categories to semantic types. While it is
clear that we have to allow for some flexibility, e.g. with regards
to the number of arguments a given predicate involves, sub-dividing
the space of lexical entries for \isi{nouns} into predicates and entities
gives rise to additional complications. These are by no means
insurmountable, but their repercussions have to be assessed carefully.
On the flipside of the coin, determining the availability of the
type-shifters that are standardly invoked for dealing with these
complications has to be carefully constrained. Another aspect that
requires further spelling out is the
nature of the measure on the \isi{uniqueness scale}, especially as new
potential contrasts are considered based on new data from additional
languages. On the semantic side, the question arises of how cases where there is a
clear overall meaning contrast based on which article is used are
captured in the formal derivation if the articles themselves do not
contribute any meaning. Finally, the specification of the key notions
of \isi{uniqueness} tries to characterize unambiguous \isi{reference} relative to
the denotation of the noun (since it is based on lexical properties),
rather than the full noun phrase. But this does not translate straightforwardly to
cases of more complex \isi{noun phrases}, where traditional uniqueness-based
analyses crucially rely on the compositional combination of the
\is{determiners}determiner with its complement noun phrase as a whole (e.g. including
modifying \is{adjectives}adjectives). Relatedly, it is
not obvious how the broader integration of this approach into a formal
semantic system that interacts with the grammar should proceed,
specifically with regards to the various mechanisms for co-variation
under \is{quantifiers}quantifying expressions briefly discussed above. 

There are empirical problems for this type of approach as well. In particular,
\is{nouns}sortal nouns of various kinds can be turned functional through
appropriate contexts -- as illustrated by the following variation on
\REF{ex:schwarz:7b} (where a strong article
was required):

\begin{exe}
\ex \ili{German} \\ Context: Hans, who works at a ministry, and his wife are talking about what has been going on at work.
\begin{xlist}
\ex What happened to the proposal you drafted?
\ex\gll Der Vorschlag wurde in der Kabinettssitzung gestern {vom$_\text{s}$} {Minister} vorgestellt, aber {7} {SPD-Minister} haben dagegen gestimmt.\label{minister1}\\
the proposal was in the {cabinet meeting} yesterday {by\_\theweak} minister introduced but 7 SPD-ministers have against voted\\
\glt `The proposal was introduced by the minister in yesterday's cabinet meeting, but 7 SPD-ministers voted against it.'
\end{xlist}
\end{exe}

Crucially, nothing about the \is{nouns}noun in such cases ensures \isi{uniqueness}
directly, and to the extent that uniqueness does hold, that only is so
based on a substantial amount of contextual information -- in essence,
the entire \is{definite noun phrases}definite noun phrase is interpreted relative to the
speaker's work place here. But surely such a contextual modulation
should not lead us to consider different lexical entries for the word
`minister'.\footnote{Note also that this is clearly a different
  contrast than that in the sketch of \ili{Haitian Creole} above, where
  resource situation would require a strong article.}


Let us now turn to some of the strengths of this proposal. First, as
already noted above, it allows for a substantial range of variation
between languages along a single dimension, and \citet{Ortmann2014}
applies the resulting prediction in interesting ways, both
synchronically and diachronically. But even as that success should be
registered, it is worth noting that the formal proposal on its own
predicts that languages should be able to choose a cut-off point anywhere
on the scale. In light of the variation present in existing data, it
seems that even though some flexibility is needed, the full range of
options goes beyond what is required (of course this could change with
additional data being brought under consideration). 

In relation to these concerns, it is also worth revisiting some aspects
of \ili{Haitian Creole} in light of the analysis in terms of semantic vs.\
pragmatic uniqueness. The \isi{uniqueness scale} has global uniques on par
with \is{nouns}functional nouns with explicit arguments. But \ili{Haitian Creole}
crucially draws  a line between these two, and any plausible
additional split of the uniqueness scale would predict an opposite
ordering from what is empirically attested in this
regard. Furthermore, the intriguing interaction of \textit{la} with
the domain of \textit{only} would not seem to be something that can be
explained in any straightforward way from this perspective. 

In sum, accounts based on the distinction between semantic and
pragmatic uniqueness do have some desirable empirical predictions going
for them, but they also face some challenges, both conceptually and
theoretically. In light of this, it should be clear that accounting
for the full range of article variation across languages requires
substantially more work, regardless of the theoretical approach one
starts out with. But the empirical picture overall is not
incompatible with a view where the core weak vs.\ strong contrast is
mirrored in properties of article contrasts across many languages, but
various other, potentially independent, factors can affect just what
form is thought to be ideally suited for the purposes at hand.\is{pragmatic uniqueness|)}\is{semantic uniqueness|)}

\section{Conclusion}

In this chapter, I have reviewed the key tenets of the contrast
between weak and strong article definites presented in
\citet{Schwarz2009}, and considered a range of data across various
languages in light of it. There seems to be a substantial number of
languages from entirely unrelated language families that use different
forms for different types of \isi{definite noun phrases} in a way that seems
to reflect the weak vs.\ strong article contrast found in
\ili{Germanic}. While there are some minor variations in the pragmatics of
which forms get used when both are available, the nature of the
semantic contrast in a large set of languages seems to be fairly
uniform and consistent with an analysis in terms of \isi{situational
uniqueness} and \isi{anaphoricity}. In addition, the formal realization of
the contrasts was considered, and there is at least preliminary
evidence from the languages discussed that there is real variation in
the interpretation of \isi{bare noun phrases}, in a way that suggests that
distinct \is{null determiners}null D-heads may be at play in at least some of them.

Additional languages enriched the picture further, as they exhibit
contrasts that clearly seem to go beyond the weak vs.\ strong contrast. There are two
possible approaches to tackling this. First, one can see these
languages in terms of orthogonal factors, providing insights into
potentially related, but ultimately separate dimensions of
variation. Alternatively, one can see them in terms of a more gradient
perspective on how different types of definites are signaled within a
grammar, as on the approach based on \is{semantic uniqueness}semantic vs.\ \is{pragmatic uniqueness}pragmatic
uniqueness. Both types of approaches require extensions and
elaborations, so
more work is needed both empirically and theoretically to achieve a
more conclusive assessment of the semantic typology of definiteness
across languages. However, the sharpening of key descriptive notions
and crucial contrasts goes a long way towards having more precise tools
that can help to get a more uniform and broad cross-linguistic
perspective on the nature and extent of variation.\is{weak definite articles|)}\is{strong definite articles|)}

{\sloppy
\printbibliography[heading=subbibliography,notkeyword=this]
}
\end{document}
